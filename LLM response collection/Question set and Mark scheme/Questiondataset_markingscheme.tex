\documentclass[a4paper,11pt]{article}
\setlength\parindent{0pt}
\usepackage{amsmath, amssymb}
\usepackage[margin=1in,letterpaper]
{geometry} % decreases margins

\usepackage{setspace}
\setstretch{1.2}

\begin{document}
\title{Question Dataset Markscheme}
\author{}
\date{}
\maketitle

1. Allow any equivalent expressions as long as they are reasonable (close to the same form)

2. Allow any alternate solutions as long as the argument is correct and sufficient. 

3. Deduct marks for inappropriate units and precision

\section{EM}

\subsection*{Q1: (8 marks total)}

[1 mark] naming 3 types: paramagnetism, diamagnetism, ferromagnetism. 

[1 mark] mention differing magnetic susceptibility/magnetic dipole moment values which determine response 

[2 marks] paramagnetism, magnetic susceptibility positive, much less than 1, slight attraction to bar magnet (weaker dipole moment relative to ferromagnetic material)

[2 marks] diamagnetism, magnetic susceptibility negative, magnitude much less than 1, slight repulsion to bar magnet, dipole moment induced opposes field (weak dipole moment relative to ferromagnetic, paramagnetic material)

[2 marks] ferromagnetism, magnetic susceptibility much larger than 1, much stronger attraction to bar magnet relative to paramagnet. (large dipole moment relative to para and diamagnetic material)

\subsection*{Q2: (5 marks total)}

[1 mark] free space - vacuum with no matter, free charge or free current.

[1 mark] The equation given defines a wave equation for electric fields. Solutions are plane waves propagating in space

[3 marks] Some mention that light is an EM wave or partly formed by electric field waves. Speed of wave propagation in the equation is equal to speed of light. The speed of light, c,  is equal to \(1/\sqrt{\epsilon_{0}\mu_{0}}\). \(c = 3 \times 10^8 \, ms^{-1}\).

\subsection*{Q3: (14 marks total)}

[1 mark] Use of coulombs law to integrate and find electric field

[3 marks] correct integral setup to evaluate x-component \( E_x \). \( dq = \lambda dx \)

\[ E_x = - \frac{1}{4\pi \epsilon_0} \lambda \int_{0}^{L} \frac{x}{(x^2+d^2)^{3/2}} \, dx \]

[3 marks] correct integral setup to evaluate z-component \( E_z \).

\[ E_z = \frac{1}{4\pi \epsilon_0} \lambda d  \int_{0}^{L} \frac{1}{(x^2+d^2)^{3/2}} \, dx \]

[2 marks] correct evaluation of integral to get \( E_x \)

\[ E_x = -\frac{1}{4 \pi \epsilon_0} \lambda \left [ \frac{1}{d} - \frac{1}{\sqrt{d^2+L^2}} \right ] \]

[2 marks] correct evaluation of integral to get \( E_z \)

\[ \frac{1}{4 \pi \epsilon_0} \frac{\lambda}{d} \frac{L}{\sqrt{d^2+L^2}} \]

[1 mark] total expression of \underline{E}

\[
\frac{1}{4 \pi \epsilon_0} \frac{\lambda}{d} \left [ \left ( -1+\frac{d}{\sqrt{d^2+L^2}} \right ) \hat{x} +\left ( \frac{L}{\sqrt{d^2+L^2}} \right ) \hat{z} \right ]
\]

[2 marks] Correct simplification in \underline{E} for \( d \gg L \)

\[ \frac{1}{4 \pi \epsilon_0} \frac{\lambda L}{d^2} \hat{z} \]

and mention line charge at this distance looks like a point charge

\subsection*{Q4: (11 marks total)}

a) (5 marks)

[1 mark] Use of definition of flux as surface integral

\[ \int \underline{B} \cdot \underline{dA} \]

[3 marks] Correct setup of integral for flux, using \underline{B} for an infinite long wire 
 
\[ \underline{B} = \frac{\mu_0 I}{2\pi r} \hat{\phi} \]
\[ \Phi = \frac{\mu_0 I}{2\pi} \int_{d}^{d+L} \frac{1}{r} \, (L \, dr) \]

[1 mark] correction evaluation of integral to get flux
\[ \Phi = \frac{\mu_0 I L}{2\pi}\ln\left ( \frac{d+L}{d} \right ) \]

b) (5 marks)

[1 mark] Use of emf definition.  \( \mathcal{E} = - \frac{d\Phi}{dt} \)

[2 marks] Correct evaluation of the derivative to get the emf

\[ \mathcal{E} = \frac{\mu_0 I L^2 v}{2\pi r (r+L)} \]

here \( r \) is the perpendicular distance from the wire to the closest side of the square. More explicitly \( r = d + vt \).

[2 marks] Current moves counterclockwise. Justify by Lenz Law or explicitly showing direction of force.

c) (1 mark)

[1 mark] No emf since flux does not change.

\subsection*{Q5: (11 marks total)}

a) (8 marks)

[1 mark] Mention use of Kirchhoff Circuit Laws 

[2 marks] split current of circuit into 3 branches, \( I_1 \), \( I_2 \), \( I_3 \). Current through the ammeter denoted by one of these currents. Use of Kirchoff’s current law eg. \( I_1 + I_2 = I_3 \) (exact description is dependent on labeling)

[2 marks] Use of Kirchoff’s voltage law to obtain 2 more equations of  \( I_1 \), \( I_2 \), and \( I_3 \).

[3 marks] Solve simultaneous equations and obtain current through the ammeter which is 4/3 A. 

b)  (3 marks)

[2 marks] Power generated by the 12V battery is the sum of power delivered to each component. Using \( P = IV \) or \( P = I^2 R \),

\[ P_{tot} = \sum_{i} P_i = \left ( \frac{11}{3} \right )^2 \times 2 + \left ( \frac{7}{3} \right )^2 \times 2 + \left ( \frac{4}{3} \right ) \times 2 + \left ( \frac{4}{3} \right )^2 \times 2 = 44 W \]

[1 mark] Energy delivered in 4 seconds: 
\[ E = P_{tot} t = 44 \times 4 = 176 J \]

\subsection*{Q6: (5 marks total)}

a) (3 marks)

[2 marks] Infinite planes have constant electric fields between. Magnitude of potential difference is \( V = Ed \). Magnitude of electric field is calculated 

\[ E = \frac{V}{d} = \frac{120}{0.15} = 800 \, \text{NC}^{-1} \quad (\text{or units } \mathrm{Vm}^{-1}) \]

[1 mark] The electric field is \(E = 800 \hat{n} NC^{-1} \) where \( \hat{n} \) is the unit vector of perpendicular direction from \( +\sigma \) plane to the \( -\sigma \) plane. (Only mention of direction of field required)

b) (2 marks)

[1 mark] Use Newton’s 2nd Law equate to electric force, \( ma = qE \)

[1 mark] \[ a = \frac{qE}{m} = \frac{0.001 \times 800}{0.023} = 34.7826 \, ms^{-2} \]

\subsection*{Q7: (6 marks total)}

a) (1 mark) 

[1 mark]  resonant frequency is 
\[ \omega_0 = \frac{1}{\sqrt{LC}} = 182574.18 \, \mathrm{rads}^{-1} \]

b) (3 marks)

[1 mark] Impedance Z given by 
\[ Z = \sqrt{R^2 + (X_L - X_C)^2} \]

[1 mark] Current is

\[ I =  \frac{\mathcal{E}_0}{Z} \sin(\omega t - \phi) \]

(allow different trigonometric functions as long as there is a phase factor \( \phi \) ) 

[1 mark] Voltage drop across inductor is 

\[ \Delta{V_L} = I X_L = \frac{\mathcal{E}_0 X_L}{Z} \sin(\omega t - \phi) \]

c) (2 marks)

[2 marks] For \( \omega \gg \omega_{0} \), in phase. For \( \omega \ll \omega_{0} \), antiphase.  

\subsection*{Q8: (6 marks total)}

[3 marks] Apply correct boundary conditions on the interface to parallel and perpendicular components of H and B:

\[ H_1 \sin(\alpha_1) = H_2 \sin(\alpha_2) \]
\[ B_1 \cos(\alpha_1) = B_2 \cos(\alpha_2) \]

[1 mark] use \( B = \mu_0 \mu_r H \)

[2 marks] manipulate equations to obtain 

\[ \frac{1}{\mu_{r1}} \tan(\alpha_1) = \frac{1}{\mu_{r2}} \tan(\alpha_2) \]
\[ \alpha_1 = \tan^{-1}\left ( \frac{\mu_{r1}}{\mu_{r2}} \right ) = 26.366^{\circ} \]

\subsection*{Q9: (15 marks total)}

a) (7 marks)

[1 mark] assume \underline{H}(\underline{r}, t) has solution

\[ \underline{H}(\underline{r}, t) = \underline{H}_0 \exp (i(\underline{k} \cdot \underline{r} - \omega t)) \]

[2 marks] apply Faraday's law, noting relation between \underline{B} and \underline{H} in vacuum, \( \underline{B} = \mu_0 \underline{H} \)

\[ \nabla \times \underline{E} = - \frac{\partial \underline{B}}{\partial t} = - \mu_0 \frac{\partial \underline{H}}{\partial t}  \]
[2 marks] noting \( \nabla \times \underline{E} = i \underline{k} \times \underline{E} \) or by direct calculation. \( - \mu_0 \frac{\partial \underline{H}}{\partial t} = - i \mu_0 \omega \underline{H} \)

\[ i \underline{k} \times \underline{E} = i \mu_0 \omega \underline{H} \]

[2 marks] Taking magnitude of equation, noting \( \omega / k = c = 1 / \sqrt{\epsilon_0 \mu_0} \), simplify to desired result 

\[ H(\underline{r}, t) = \sqrt{\frac{\epsilon_{0}}{\mu_{0}}} E(\underline{r}, t) \]

b) (6 marks)

[1 mark] Use the expression of the Poynting vector, \underline{N}, in a vacuum. 

\[ \underline{N} = \frac{1}{\mu_0} ( \underline{E} \times \underline{B} ) = \underline{E} \times \underline{H} \]

[3 marks] Take the real part of \( \underline{E} \) and \( \underline{H} \), using expression found in a), simplify and give 

\[ \underline{N} = \sqrt{\frac{\mu_0}{\epsilon_0}} (H_0)^2 \cos^2(\underline{k} \cdot \underline{r} - \omega t) \hat{k} \]

[2 marks] 

\[ \langle \underline{N} \rangle = \sqrt{\frac{\mu_0}{\epsilon_0}} (H_0)^2 \langle \cos^2(\underline{k} \cdot \underline{r} - \omega t) \rangle \hat{k} \]

\[ \langle \underline{N} \rangle = \frac{1}{2}\sqrt{\frac{\mu_0}{\epsilon_0}} (H_0)^2 \hat{k} \]

Must justify the time average of the cosine term as 1/2.

c) (2 marks) 

[1 mark] One factor of cosine from component of momentum perpendicular to surface

[1 mark] Another factor of cosine due to the surface area incident increasing. 



\subsection*{Q10: (7 marks total)}

a) (4 marks)

[1 mark] Substitute \( (\underline{E'} \cdot \underline{B'}) \) with transformation terms

\begin{align*}
    \underline{E'} \cdot \underline{B'} & = E'_x B'_x + E'_y B'_y + E'_z B'_z \\ & = E_{x} B_{x} + \gamma(E_{y} - vB_{z}) \gamma(B_{y} + \frac{v}{c^{2}} E_{z}) + \gamma(E_{z} + vB_{y}) \gamma(B_{z} - \frac{v}{c^{2}} E_{y})
\end{align*}

[3 marks] Explicitly expand and simplify algebra, final simplified term should be \( E_x B_x + E_y B_y + E_z B_z = \underline{E} \cdot \underline{B} \) (Converting to dot product form not required)

b) (2 marks)

[1 mark] Use of inverse lorentz transformations to replace \( x \) and \( t \)

[1 mark] Rearrange and simplification to give required relations.

c) (1 mark)

[1 mark] Represents Doppler effect. Gives some reasoning involving frequency shift.

\section{QM}

\subsection*{Q1: (5 marks total)}

a) (3 marks)

[1 mark]The commutator is given as  \( \left[\hat{P}, \hat{Q} \right] = \hat{P}\hat{Q} - \hat{Q}\hat{P} \)

[1 mark] Compatible operators are operators which share a common, complete set of eigenfunctions

[1 mark] Compatible operators imply that the commutator is equal to 0.

b) (2 marks)

[1 mark] The initial \( p \) value is not necessarily the same as the final \( p \) value measured

[1 mark] Since the operators are incompatible, the wavefunction will collapse to different eigenstates after each operator has been applied to it and the measured values may change

\subsection*{Q2: (4 marks total)}

a) (2 marks)

[1 mark] The eigenvalues are real 

[1 mark] Eigenvalues are possible values obtained by measurement of the given quantity associated with the operator. 

b) (2 marks)

[1 mark] The set of eigenfunctions is orthogonal - the inner products are non-vanishing if and only if the eigenfunctions chosen in the inner product are the same. 

[1 mark] Each eigenstate is normalized such that their inner product with themselves gives 1. 

\[ \left< \phi_i | \phi_j \right> = \delta_{ij} \]

\subsection*{Q3: (9 marks total)}

a) (4 marks)

[2 marks] Substitution of \( \hat{p} \) into the LHS or RHS of definition, Integrate by parts, removing vanishings terms since functions vanish at infinity 

[2 marks] for simplifying the remaining integral, arranging into correct form showing LHS equals RHS, stating  \( \hat{p} \) is Hermitian e.g. 

\[ \int g (\hat{Q}f)^{*} \, dx \]

b) (5 marks)

[2 marks] for substituting the operator into the integral and applying integration by parts, removing vanishing terms due to limits and well behaved functions

[1 mark] for applying integration by parts again, again removing vanishing terms after evaluating limits

[2 marks] for arranging into the correct form, stating \(\hat{Q}\) is Hermitian. 

\subsection*{Q4: (20 marks total) }

a) (3 marks)

[1 mark] for expanding the commutation relation and correctly substituting \( \hat{L_{x}} \) in, 

\[ 
\left[ \hat{L_{x}}, \hat{X} \right] = (\hat{Y}\hat{P_{z}} - \hat{Z}\hat{P_{y}})\hat{X} - \hat{X}(\hat{Y}\hat{P_{z}} - \hat{Z}\hat{P_{y}})
\]

[1 mark] For stating that since \( \hat{X} \) commutes with \( \hat{Z} \), \( \hat{P_{z}} \), \( \hat{P_{y}} \), and \( Y \), the second term can be rearranged to \( (\hat{Y}\hat{P_{z}} - \hat{Z}\hat{P_{y}})\hat{X} \).

[1 mark] \( (\hat{Y}\hat{P_{z}} - \hat{Z}\hat{P_{y}})\hat{X} - (\hat{Y}\hat{P_{z}} - \hat{Z}\hat{P_{y}})\hat{X} = 0 \) as required

b) (3 marks)

[1 mark] for expanding the commutation relation and correctly substituting \( \hat{L_{x}} \) in, \([\hat{L_{x}}, \hat{P_{x}}] = (\hat{Y}\hat{P_{z}} - \hat{Z}\hat{P_{y}})\hat{P_{x}} - \hat{P_{x}}(\hat{Y}\hat{P_{z}} - \hat{Z}\hat{P_{y}})\)

[1 mark] for stating that since \( \hat{P_{x}} \) commutes with \( \hat{Z} \), \( \hat{P_{z}} \), \( \hat{P_{y}} \), and \( Y \), the second term can be rearranged to \( (\hat{Y}\hat{P_{z}} - \hat{Z}\hat{P_{y}})\hat{P_{x}} \)

[1 mark] \((\hat{Y}\hat{P_{z}} - \hat{Z}\hat{P_{y}})\hat{P_{x}}  - (\hat{Y}\hat{P_{z}} - \hat{Z}\hat{P_{y}})\hat{P_{x}}\)  = 0 as required

c)  (4 marks)

[1 mark] for expanding the commutation relation and correctly substituting \( \hat{L_{x}} \) in, \([\hat{L_{x}}, \hat{Y}] = (\hat{Y}\hat{P_{z}} - \hat{Z}\hat{P_{y}})\hat{Y} - \hat{Y}(\hat{Y}\hat{P_{z}} - \hat{Z}\hat{P_{y}})\)

[1 mark] For changing the order of commuting terms to get \( \hat{Y}\hat{Y}\hat{P_{z}} - \hat{Z}\hat{P_{y}}\hat{Y} - \hat{Y}\hat{Y}\hat{P_{z}} + \hat{Z}\hat{Y}\hat{P_{y}} \)

[2 marks] for correctly factorizing and simplifying: \( \hat{Z}[\hat{Y}, \hat{P_{y}}] \), using the fact that \( [\hat{Y}, \hat{P_{y}}] = -i\hbar \) and substituting into the answer to yield \( [\hat{L}_x, \hat{Y}] = i\hbar \hat{Z} \)

d) (4 marks)

[1 mark] for expanding the commutation relation and correctly substituting \( \hat{L_{x}} \) in, \([\hat{L_{x}}, \hat{P_{y}}] = (\hat{Y}\hat{P_{z}} - \hat{Z}\hat{P_{y}})\hat{P_{y}} - \hat{P_{y}}(\hat{Y}\hat{P_{z}} - \hat{Z}\hat{P_{y}})\)

[1 mark] For changing the order of commuting terms to get \( \hat{P_{z}}\hat{Y}\hat{P_{y}} - \hat{P_{z}}\hat{P_{y}}\hat{Y} - \hat{Z}\hat{P_{y}}\hat{P_{y}} + \hat{Z}\hat{P_{y}}\hat{P_{y}} \)

[2 marks] for correctly factorizing and simplifying: \( \hat{P_{z}}[\hat{Y}, \hat{P_{y}}] \), using the fact \( [\hat{Y}, \hat{P_{y}}] = -i\hbar \hat{P_{z}} \) to yield \( [\hat{L}_x, \hat{P_{y}}] = i\hbar\hat{P_{z}} \)

e) (6 marks)

[1 mark] for rewriting the commutation as \( [\hat{L_{x}}, \hat{P^{2}}] = [\hat{L_{x}}, \hat{P}_{x}^{2} ] + [\hat{L_{x}}, \hat{P}_{y}^{2}] + [\hat{L_{x}}, \hat{P}_{z}^{2}] \)

[1 mark] Applying the identity \( [\hat{A}, \hat{B^{2}}] = [\hat{A}, \hat{B}]\hat{B} + \hat{B}[\hat{A},\hat{B}] \) to yield 

\[ [\hat{L_{x}}, \hat{P^{2}}] = [\hat{L_{x}}, \hat{P_{x}}]\hat{P_{x}} + \hat{P_{x}}[\hat{L_x}, \hat{P_{x}}] + [\hat{L_{x}}, \hat{P_{y}}]\hat{P_{y}} + \hat{P_{y}}[\hat{L_{x}}, \hat{P_{y}}] + [\hat{L_{x}}, \hat{P_{z}}]\hat{P_{z}} + \hat{P_{z}}[\hat{L_{x}}, \hat{P_{z}}] \]

Note that the identity \( [\hat{A}, \hat{B^{2}}] = [\hat{A}, \hat{B}]\hat{B} + \hat{B}[\hat{A},\hat{B}] \) does NOT need to be explicitly stated.

[3 marks] For \( [\hat{L_{x}}, \hat{P_{z}}] = -i\hbar\hat{P_{y}} \)

[1 mark] For simplifying \( [\hat{L_{x}}, \hat{P^{2}}] = (i\hbar\hat{P_{z}}\hat{P_{y}}) + (i\hbar\hat{P_{y}}\hat{P_{z}}) - (i\hbar\hat{P_{y}}\hat{P_{z}}) - (i\hbar\hat{P_{z}}\hat{P_{y}}) = 0 \)

\subsection*{Q5: (7 marks total)}

a) (4 marks)

[1 mark] for stating the equation for energy and making it relevant to the ground state energy, 

\[ E_{1}^{(1)} = \int_{-\infty}^{\infty} \phi_{1}^{(0)*}\hat{H'}\phi_{1}^{(0)} \, dx \]

[1 mark] Substituting \( \hat{H}' = V_{0} \sin\left(\frac{2 \pi x}{L} \right) \) into the integral for energy with the limits [0,L] and simplifying to get \[ \frac{2V_{0}}{L} \int_{0}^{L} \sin^{2}\left(\frac{ \pi x}{L} \right)\sin\left(\frac{ 2\pi x}{L} \right) \, dx \]

[2 marks] Solving the integral, showing the perturbation equals 0.

b) (3 marks)

[1 mark] Applying perturbation theory, substituting the correct terms in to get the integral: 

\[ E_{n}^{(1)} = 2 \alpha \int_{0}^{L} \sin^{2}\left(\frac{n \pi x}{L} \right) \delta\left(x - \frac{L}{2} \right) \, dx \]

[1 mark] for correctly solving the integral and simplifying to 
\[ 2 \alpha \sin^{2}\left(\frac{n \pi}{2} \right) \]

[1 mark] for stating \(2 \alpha \) for all odd values of n, and 0 for all even values of n 

\subsection*{Q6: (17 marks total) }

a) (4 marks)

[1 mark] \[ \hat{J^{2}} = \left(\hat{\underline{L}} + \hat{\underline{S}} \right) \cdot \left(\hat{\underline{L}} + \hat{\underline{S}} \right) =  \hat{L^{2}} + \hat{S^{2}} + 2\hat{L}_{x}\hat{S}_{x} + 2\hat{L}_{y}\hat{S}_{y} + 2\hat{L}_{z}\hat{S}_{z} \]

[2 marks]  \[ \hat{S_{x}} = \frac{1}{2} (\hat{S_{+}} + \hat{S_{-}}) \quad \hat{S_{y}} = \frac{1}{2i} (\hat{S_{+}} - \hat{S_{-}}) \] 

\[ \hat{L_{x}} = \frac{1}{2} (\hat{L_{+}} + \hat{L_{-}}) \quad \hat{L_{y}} = \frac{1}{2i} (\hat{L_{+}} - \hat{L_{-}}) \]

[1 mark] Final answer \[ \hat{J^{2}} =  \hat{L^{2}} + \hat{S^{2}} + \hat{L}_{+}\hat{S}_{-} +  \hat{L}_{-}\hat{S}_{+} + 2\hat{L}_{z}\hat{S}_{z} \]

b) (4 marks)

[2 marks] Applying operator expression for \( \hat{J^{2}} \) to \( \left|l, -l; s, -s \right> \) and noting \( \hat{L}_{+}\hat{S}_{-} \), \( \hat{L}_{-}\hat{S}_{+} \), since states composing the tensor product are the bottom rung (or show explicitly)

\[ 
\hat{J^{2}}\left|l, -l; s, -s \right> = \left( \hat{L^{2}} + \hat{S^{2}} + \hat{L}_{+}\hat{S}_{-} +  \hat{L}_{-}\hat{S}_{+} + 2\hat{L}_{z}\hat{S}_{z} \right) \left|l, -l \right> \left|s, -s \right>  
\]

\[
\hat{J^{2}}\left|l, -l; s, -s \right> = \left( \hat{L^{2}} + \hat{S^{2}} + 2\hat{L}_{z}\hat{S}_{z} \right)  \left|l, -l \right> \left|s, -s \right>
\]

[2 marks] calculating \( \hat{L^{2}} \), \( \hat{S^{2}} \),\( \hat{L}_{z}\hat{S}_{z} \) acting on \( \left|l, -l \right> \left|s, -s \right> \). Showing it is an eigenstate of  \( \hat{J^{2}} \) with eigenvalue \( \hbar^2 \left ( l(l+1) + s(s+1) + 2ls \right ) \) or equivalent. 

c) (9 marks)

[1 mark] noting there are 6 states which naturally form the basis kets of \( \left|l=1, m; s=\frac{1}{2}, m_{s} \right> \).

\[ \left|1, 1 \right> |\frac{1}{2},\frac{1}{2} \rangle, \quad |1, 0 \rangle |\frac{1}{2}, \frac{1}{2} \rangle, \quad \left|1, -1 \right>|\frac{1}{2}, \frac{1}{2} \rangle \]

\[ \left|1, 1 \right> |\frac{1}{2},-\frac{1}{2} \rangle, \quad |1, 0 \rangle |\frac{1}{2}, -\frac{1}{2} \rangle, \quad \left|1, -1 \right>|\frac{1}{2}, -\frac{1}{2} \rangle \]

[1 mark] Appropriate choice of vector representation for each basis ket. Eg. 

\[
 \left|1, 1 \right> |\frac{1}{2},\frac{1}{2} \rangle \rightarrow \begin{pmatrix} 1\\ 0\\ 0\\ 0\\ 0\\ 0 \end{pmatrix} = e_1, \quad |1, 0 \rangle |\frac{1}{2}, \frac{1}{2} \rangle \rightarrow \begin{pmatrix} 0\\ 0\\ 1\\ 0\\ 0\\ 0 \end{pmatrix} = e_3, \quad \left|1, -1 \right>|\frac{1}{2}, \frac{1}{2} \rangle \rightarrow \begin{pmatrix} 0\\ 0\\ 0\\ 0\\ 1\\ 0 \end{pmatrix} = e_5
\]

\[
 \left|1, 1 \right> |\frac{1}{2},-\frac{1}{2} \rangle \rightarrow \begin{pmatrix} 0\\ 1\\ 0\\ 0\\ 0\\ 0 \end{pmatrix} = e_2, \quad |1, 0 \rangle |\frac{1}{2}, -\frac{1}{2} \rangle \rightarrow \begin{pmatrix} 0\\ 0\\ 0\\ 1\\ 0\\ 0 \end{pmatrix} = e_4, \quad \left|1, -1 \right>|\frac{1}{2}, -\frac{1}{2} \rangle \rightarrow \begin{pmatrix} 0\\ 0\\ 0\\ 0\\ 0\\ 1 \end{pmatrix} = e_6
\]

[2 marks] Calculation of  \( \hat{L^{2}} \) acting on basis kets, giving matrix representation 

\[ 
2\hbar^2 \begin{pmatrix} 1 & 0 & 0 & 0 & 0 & 0\\  0 & 1 & 0 & 0 & 0 & 0\\ 0 & 0 & 1 & 0 & 0 & 0\\ 0 & 0 & 0 & 1 & 0 & 0\\ 0 & 0 & 0 & 0 & 1 & 0\\ 0 & 0 & 0 & 0 & 0 & 1 \end{pmatrix}
\]

[4 marks] Calculation of \( \hat{S}_{+}\hat{L}_{z} \) acting on basis kets giving the matrix representation. Eg. 

\[
\hat{S}_{+}\hat{L}_{z}e_1 = 0, \quad \hat{S}_{+}\hat{L}_{z}e_2 = \hbar^2e_1, \quad \hat{S}_{+}\hat{L}_{z}e_3 = 0, \quad \hat{S}_{+}\hat{L}_{z}e_4 = 0, \quad \hat{S}_{+}\hat{L}_{z}e_5 = 0, \quad \hat{S}_{+}\hat{L}_{z}e_6 = -\hbar^2e_5 
\]

\[ 
S_{+}L_{z} = \hbar^2 \begin{pmatrix} 0 & 1 & 0 & 0 & 0 & 0\\ 0 & 0 & 0 & 0 & 0 & 0\\ 0 & 0 & 0 & 0 & 0 & 0\\ 0 & 0 & 0 & 0 & 0 & 0\\ 0 & 0 & 0 & 0 & 0 & -1\\ 0 & 0 & 0 & 0 & 0 & 0 \end{pmatrix}
\]

[1 mark] Construction of the matrix representation of \( \hat{O} = a\hat{L^{2}} + b\hat{S}_{+}\hat{L}_{z} \)

\[ O = a L^2 + b S_{+}L_z \]

\[
O = 2\hbar^2 a \begin{pmatrix} 1 & 0 & 0 & 0 & 0 & 0\\  0 & 1 & 0 & 0 & 0 & 0\\ 0 & 0 & 1 & 0 & 0 & 0\\ 0 & 0 & 0 & 1 & 0 & 0\\ 0 & 0 & 0 & 0 & 1 & 0\\ 0 & 0 & 0 & 0 & 0 & 1 \end{pmatrix} + \hbar^2 b \begin{pmatrix} 0 & 1 & 0 & 0 & 0 & 0\\ 0 & 0 & 0 & 0 & 0 & 0\\ 0 & 0 & 0 & 0 & 0 & 0\\ 0 & 0 & 0 & 0 & 0 & 0\\ 0 & 0 & 0 & 0 & 0 & -1\\ 0 & 0 & 0 & 0 & 0 & 0 \end{pmatrix}
\]

\[ 
O = \hbar^2 \begin{pmatrix} 2a & b & 0 & 0 & 0 & 0\\ 0 & 2a & 0 & 0 & 0 & 0\\ 0 & 0 & 2a & 0 & 0 & 0\\ 0 & 0 & 0 & 2a & 0 & 0\\ 0 & 0 & 0 & 0 & 2a & -b\\ 0 & 0 & 0 & 0 & 0 & 2a \end{pmatrix}
\]

Note matrix representation will look different depending on the representation chosen for the basis kets. 

\subsection*{Q7: (8 marks total)}

a) (3 marks)

[1 mark] Time-dependent solution is given by 

\[ \Psi(x,t) = \Psi(x,0) \exp(- \frac{iEt}{\hbar}) =  Ae^{ikx} \exp(- \frac{iEt}{\hbar})  \]

[1 mark] For free particle, \( V(x) = 0 \), \( E = \frac{\hbar^2 k^2}{2m} \) where k is a constant, 

\[ \Psi(x,t) = \Psi(x,0) \exp(- \frac{iEt}{\hbar}) =  A \exp\left[ i \left( kx - \frac{\hbar k^2 t}{2m} \right) \right] \]

[1 mark] Probability per unit length is the modulus of the wavefunction squared,

\[ |\Psi(x,t)|^2 = |A|^2 \]

which is independent of time

b) (5 marks)

[1 mark] 

\[ \frac{\partial \Psi}{\partial x} = Aik e^{ikx - \frac{iEt}{\hbar}} \] 

\[ \frac{\partial \Psi^{*}}{\partial x} = -Aik e^{-ikx + \frac{iEt}{\hbar}} \] 

[2 marks] Calculation of the probability flux

\[\Psi^{*}\frac{\partial \Psi}{\partial x} - \Psi\frac{\partial \Psi^{*}}{\partial x} = 2|A|^{2}ik\]
\[\Gamma = \frac{\left|A \right|^{2}\hbar k}{m} \]  

[2 marks] \( \Gamma \) is the number of particles which pass through a given point per unit time. \( \hbar k \) is the momentum hence \( \frac{\hbar k }{m} \) is the velocity.  \( |A|^2 \) is the no. particles per unit length

 \[\Gamma = v \left|A \right|^{2} \]  

where \(v\) is the velocity. 

\subsection*{Q8: (9 marks total) }

a) (5 marks)

[2 marks] \[ \frac{d}{dt} \left< \hat{Q} \right> = \int_{-\infty}^{\infty} \frac{\partial}{\partial t} \left( \Psi^{*}(x) \hat{Q} \Psi(x) \right) \, dx \]

\[ = \int_{-\infty}^{\infty} \frac{\partial \Psi^{*}}{\partial t}\hat{Q}\Psi + \Psi^{*}\hat{Q}\frac{\partial \Psi}{\partial t} \, dx \]

[1 mark] substitute partial derivatives of \(\Psi\), \(\Psi^{*}\) with respect to t using Hamiltonian expression.

[2 marks] Algebraic simplification to 

\[ \frac{1}{i\hbar} \int_{-\infty}^{\infty} \Psi^{*} \left [ \hat{Q}, \hat{H} \right ] \Psi \, dx \]

\[ = \frac{1}{i\hbar} \left<  \left [ \hat{Q}, \hat{H} \right ] \right> \]

b) (4 marks)

[1 mark]\[ \frac{d}{dt}  \left<\hat{X}\right> = \frac{1}{i\hbar}\left< \left[ \hat{X}, \hat{H} \right] \right> \] 

[2 marks] \[\left[ \hat{X}, \hat{H}\right] = \left[ \hat{X}, \frac{\hat{P}^{2}}{2m} + V(x)\right] = \frac{i\hbar \hat{P}}{m} \]

[1 mark] Substitute second expression into first expression to give 

\[m \frac{d}{dt}\left<\hat{X}\right> =  \left<\hat{P}\right> \]

\subsection*{Q9: (17 marks total) }

a) (5 marks)

[2 marks] The general solution of free particle solution is \( Ae^{ik_{1}x} + Be^{-ik_{1}x} \). We can choose \( |A|^{2} = 1 \), to represent a flux of one particle, thus choose \( A = 1 \). 

[1 mark] \[ k_1 = \frac{\sqrt{2mE}}{\hbar} \]

[1 mark] In the \( x > 0 \) region, the solution has the same form as before, \( \psi_{2}(x) = Ce^{ik_{2}x} + De^{-ik_{2}x} \). No potential for beam to scatter off and reflect traveling left, hence set \( D = 0 \). 

[1 mark] \[k_{2} = \frac{\sqrt{2m(E+V_0)}}{\hbar} \]

b) (5 marks)

[1 mark] Calculating derivatives:
\[ \frac{d\psi_1}{dx} = ik_{1}e^{ik_{1}x} - ik_{1}Be^{-ik_{1}x} \]

\[ \frac{d\psi_2}{dx} = ik_{2}Ce^{ik_{2}x} \]

[2 marks] Equate \( \psi_{1} = \psi_{2} \), \( \frac{d\psi_1}{dx} = \frac{d\psi_2}{dx} \) at \(x = 0\) to obtain

\[1 + B = C\]
\[ik_{1} - ik_{1}B = ik_{2}C\]

[2 marks] The addition of the previous 2 equations results in \[ C = \frac{2k_{1}}{k_{1}+k_{2}} \] and the subtraction leads to  \[ B = \frac{k_{1}-k_{2}}{k_{1}+k_{2}} \]

c) (7 marks)

[1 mark] substituting the incident beam \( e^{ik_{1}x} \) into the formula for flux to get \[\Gamma_{incident} = \frac{\hbar k_{1}}{m} \]

[2 marks] substituting the reflected beam \( Be^{-ik_{1}x} \) into the formula for flux to get 

\[ 
\Gamma_{reflected} = \frac{-\hbar k_{1}}{m}\frac{\left(k_{1} - k_{2} \right)^{2}}{ \left(k_{1} + k_{2} \right)^{2}} 
\]

[2 marks] substituting the transmitted beam \( Ce^{ik_{2}x} \) into the formula for flux to get \[\Gamma_{transmitted} = \frac{\hbar k_{2}}{m} \frac{4k_{1}^{2}}{ \left(k_{1} + k_{2} \right)^{2}} \]

[2 marks]

\[ T = \left|\frac{\Gamma_{transmitted}}{\Gamma_{incident}} \right| = \frac{4k_{1}k_{2}}{(k_{1}+k_{2})^{2}} \]

\[R = \left|\frac{\Gamma_{reflected}}{\Gamma_{incident}} \right| = \frac{(k_{1}-k_{2})^{2}}{(k_{1}+k_{2})^{2}} \]

\subsection*{Q10: (16 marks total) }

a) (5 marks total)

[2 marks] To normalize, attach factor to \( \left| \psi \right> \) and set the inner product to 1. Choose real a and assume b is a real constant. 

\[ a \left| \psi \right> = a\left| 0 \right> + ab \left| 1 \right> \]
\[ \left< a\psi | a\psi \right> = |a|^2 \left< \psi | \psi \right> = 1 \]

[1 mark] Expand inner product, simplify algebra to obtain value of \( a \), normalized \( \left| \psi \right> \).

\[ a = \frac{1}{\sqrt{1+b^2}} \]

\[ \left| \psi \right>_{norm} = \frac{1}{\sqrt{1+b^2}} \left| 0 \right> + \frac{b}{\sqrt{1+b^2}} \left| 1 \right> \]

[2 marks] Compute expectation value of Hamiltonian, \( \langle H \rangle \) 

\[ \langle H \rangle = \frac{1}{1+b^2} \frac{\hbar \omega}{2} + \frac{b^2}{1+b^2} \frac{3\hbar \omega}{2} \]

\[ \langle H \rangle = \left ( \frac{1+3b^2}{1+b^2} \right) \frac{\hbar \omega}{2} \]

b) (2 marks)

[1 mark] rewrite \( \hat{x} \) in terms of ladder operators,

\[ \hat{x} = \frac{1}{\sqrt{2} \alpha} (\hat{a}_{+} + \hat{a}_{-}) \]

[1 mark] substitute into \( \hat{H}' \), 

\[ \hat{H}' = \frac{\sqrt{2} q \mathcal{E}}{2 \alpha} (\hat{a}_{+} + \hat{a}_{-}) = \frac{g}{2} (\hat{a}_{+} + \hat{a}_{-}) \]

Expression in terms of g not required. 

c) (9 marks)

[1 mark] Writing expectation value of original Hamiltonian, 

\[ \langle H \rangle = \frac{1}{1+b^2} E_0 + \frac{b^2}{1+b^2} E_1 \]

[3 marks] Calculation of the expectation value of the perturbation, \( \hat{H}' \), using normalized ket

\[ \hat{H}' \left| \psi \right> = \frac{g}{2} \frac{1}{\sqrt{1+b^2}} (b \left| 0 \right> + \left| 1 \right> ) \]

\[ \langle H' \rangle = \langle \psi | \hat{H}' | \psi \rangle = \frac{gb}{1+b^2}\]

[1 mark] Obtain total expectation value of the Hamiltonian. 

\[ \langle H_{tot} \rangle = \langle H \rangle + \langle H' \rangle = \frac{E_0+b^2E_1}{1+b^2} + \frac{gb}{1+b^2} = \frac{E_1 b^2 + gb + E_0}{1+b^2} \]

[1 mark] Find \( b \) to minimize \( \langle H_{tot} \rangle \) by differentiating w.r.t \( b \), setting it to 0. 

\[ \frac{\partial \langle H_{tot} \rangle}{\partial b} = \frac{(2E_1 b + g)(1+b^2) - 2b(E_1 b^2 + gb + E_0)}{(1+b^2)^2} = 0 \]

[2 marks] Solve equation to give b, 

\[ gb^2 - 2(E_1 - E_0)b - g = 0 \]

\[ b = \frac{E_1 - E_0}{g} \pm \sqrt{\frac{(E_1-E_0)^2}{g^2}-1} \]

[1 mark] Choose and justify taking the negative square root. Take the negative square root since smaller b reduces expectation value as \(  \left| \psi \right> \sim \left| 0 \right> + b \left| 1 \right> \), then the ground \( \left| 0 \right> \) state is weighted more heavily. 


\section{CM}

\subsection*{Q1: (4 marks total)}

[1 mark] Conservative force: a force where its work done on an object is independent of the object path.

[2 marks] The potential energy for a conservative force \( \underline{F} \) is the function defined by 

\[ \int_{O}^{r} - \underline{F} \cdot d\underline{r} \] 

where O is a fixed origin point and r is any point in space. Must say the chosen point of origin is arbitrary.  

[1 mark] Not possible to define potential energy for non-conservative force since if the line integral as defined above is path dependent, then each potential energy at each point will be non-unique, have multiple values and is not a function. 

\subsection*{Q2: (6 marks total)}

[2 marks] Underdamped,  \( b^2 < 4k \). Solution is sinusoidal motion which has an exponentially decaying amplitude 

[2 marks] Critically damped, \( b^2 = 4k \). Exponential decay, position decays the fastest to 0 in this case. 

[2 marks] Overdamped, \( b^2 > 4k \). The solution for position is just an exponential decay. \\

Note \( b^2 > 4k \) is equivalent to \( b^2 > 4km \).

\subsection*{Q3: (6 marks total)}

a) (2 marks) 

[1 mark] Use definition of center of mass 

\[ \underline{R} = \frac{m_1 \underline{r}_1 + m_2 \underline{r}_2}{m_1 + m_2} \]

[1 mark] Manipulate to the form 

\[ \underline{R} = \left ( 1 - \frac{m_2}{m_1 + m_2} \right ) \underline{r}_1 + \left ( \frac{m_2}{m_1 + m_2} \right ) \underline{r}_2 \]

\[ k = \frac{m_2}{m_1 + m_2} \]

b) [4 marks] 

[1 mark] calculate the vector \( \underline{R} - \underline{r}_1 \) using results of a), 

\[ \underline{R} - \underline{r}_1 = \left( \frac{m_2}{m_1+m_2} \right )(\underline{r}_2 - \underline{r}_1) \]

[1 mark] Distance from center of mass point to position \( \underline{r}_1 \) given by 

\[ d_{r1} = | \underline{R} - \underline{r}_1 | = \left( \frac{m_2}{m_1+m_2} \right )|(\underline{r}_2 - \underline{r}_1) | = \frac{m_2 d}{m_1+m_2} \]

[1 mark] calculate the vector \( \underline{R} - \underline{r}_2 \) using results of a), 

\[ \underline{R} - \underline{r}_2 = \left( \frac{m_1}{m_1+m_2} \right )(\underline{r}_1 - \underline{r}_2) \]

[1 mark] Distance from center of mass point to position \( \underline{r}_2 \) given by 

\[ d_{r2} = | \underline{R} - \underline{r}_2 | = \left( \frac{m_1}{m_1+m_2} \right )|(\underline{r}_1 - \underline{r}_2) | = \frac{m_1 d}{m_1+m_2} \]

\subsection*{Q4: (18 marks total)}

a) (6 marks)

[1 mark] 

\[ \dot{r} = \frac{dr}{dt} = - \gamma \]
\[ r = - \gamma t + C \]

where C is a constant.

[1 mark] use initial conditions, \( r = r_0 \), \( t = 0 \) to give \( C = r_0 \), 

\[ r = - \gamma t + r_0 \]

[2 marks] express and simplify angular momentum using given formulas, 

\[ \underline{L} = \underline{r} \times \underline{p} = m \underline{r} \times \underline{p} \]

\[ \underline{L} = m (r_0 - \gamma t) \hat{\underline{r}} \times \left ( - \gamma \hat{\underline{r}} + (r_0 - \gamma t ) \dot{\theta} \hat{\underline{\theta}} \right ) = m (r_0 - \gamma t)^2 \dot{\theta} \hat{\underline{k}} \]

[2 marks] Use initial values, \( \dot{\theta} = \delta \), \(t = 0 \) and conservation of angular momentum, 

\[ \underline{L}  = m {r_0}^2 \delta \hat{\underline{k}} = m (r_0 - \gamma t)^2 \dot{\theta} \hat{\underline{k}} \]

\[ \dot{\theta} = \frac{{r_0}^2 \delta}{(r_0 - \gamma t)^2} \]

b) (5 marks)

[1 mark] angular component of acceleration given by  \( r \ddot{\theta} + 2\dot{r}\dot{\theta} \).

[3 marks] obtain \( \ddot{\theta} \) by differentiation with respect to time and substituting values of \( r \), \( \dot_{r} \) and \( \dot{\theta} \) found in a). 

\[ \ddot{\theta} = \frac{2r_0^2 \gamma \delta}{(r_0 - \gamma t)^3} \]

\[ r \ddot{\theta} + 2\dot{r}\dot{\theta} = (r_0 - \gamma t) \frac{2{r_0}^2 \gamma \delta}{(r_0 - \gamma t)^3} + 2 (- \gamma) \frac{{r_0}^2 \delta}{(r_0 - \gamma t)^2}  \]

[1 mark] simplify to give 0. 

c) (7 marks)

[3 marks] expression for kinetic energy in polar coordinates 
\[ K = \frac{1}{2}m (\underline{v} \cdot \underline{v} ) \]

\[ \underline{v} = \dot{r} \underline{\hat{r}} + r \dot{\theta} \underline{\hat{\theta}} \]

\[ K = \frac{m}{2} \left ( \dot{r} \underline{\hat{r}} + r \dot{\theta} \underline{\hat{\theta}} \right ) \cdot \left ( \dot{r} \underline{\hat{r}} + r \dot{\theta} \underline{\hat{\theta}} \right ) = \frac{m}{2} \left ( \dot{r}^2 + (r \dot{\theta})^2 \right ) \]

[3 marks] substitution of \( r \), \( \dot{r} \) and \( \dot{\theta} \) to give K 

\[ K = \frac{m}{2} \left [ \gamma^2 + \frac{{r_0}^4 \delta^2}{(r_0 - \gamma t)^2} \right ] \]

[1 mark] Differentiate to obtain 

\[ \frac{dK}{dt} = -2 \frac{m}{2}\frac{-\gamma {r_0}^4 \delta^2}{(r_0 - \gamma t)^3} = \frac{m {r_0}^4 \gamma \delta^2}{(r_0 - \gamma t)^3} \]

\subsection*{Q5: (9 marks total)}

a) (2 marks)

[2 marks] Using Newton’s 2nd Law, Hooke’s Law, we have the 2 equations of motion: 

\[ m \ddot{x}_1 = - k x_1 - k(x_1 - x_2) = -2k x_1 + k x_2 \]
\[ m \ddot{x}_2 = - k x_2 - k(x_2 - x_1 = -2k x_2 + k x_1 \]

b) (4 marks) 

[2 marks] Adding and subtracting the 2 equations of motion from a), 

\[ m( \ddot{x}_1 + \ddot{x}_2 ) = -k(x_1 + x_2) \]

\[ m( \ddot{x}_1 - \ddot{x}_2 ) = -3k(x_1 - x_2) \]

[2 marks] Using \( y_{1} = x_{1} + x_{2} \), \( y_{2} = x_{1} - x_{2} \)

\[ m \ddot{y}_1 = -k y_1 \]
\[ m \ddot{y}_2 = -3k y_2 \]

Mention both equations represent simple harmonic motion.

c) (3 marks)

 [1 mark] General solution for \( y_1 \) is 

\[ y_1 = A \sin(\omega_1 t) + B \cos(\omega_1 t) \]

where \(\omega_1 = \sqrt{\frac{k}{m}}\)

[2 marks] Use of initial conditions. \( y_1(0) = x_1(0) + x_2(0) = 0 \), giving B = 0. \( \dot{y}_1(0) = \dot{x}_1(0) + \dot{x}_2(0) = v \), giving \( A = \frac{v}{\omega} = v\sqrt{\frac{m}{k}} \). Then 

\[ y_1 = v\sqrt{\frac{m}{k}} \sin \left ( t \sqrt{\frac{k}{m}} \right ) \]

\subsection*{Q6: (16 marks total)}

a) (2 marks)

[1 mark] identification of frictional force \( F_f \) and centripetal acceleration \( F \)

\[ F_f = \mu_s N = \mu_s mg \]
\[ F = \frac{m v^2}{R} \]

[1 mark] frictional force provides centripetal acceleration, \(F_f = F \):

\[ \mu_s mg = \frac{m v^2}{R} \]
\[ v = \sqrt{\mu_s g R} \]
b) [3 marks]

[1 mark] Normal force provides centripetal force,

\[ N \sin(\phi) = \frac{m v^2}{R} \]

[1 mark] Normal force also balances weight =, 

\[ N \cos(\phi) = mg \]

[1 mark] substitute 2nd equation into first to give 

\[ \frac{m v^2}{R} = \frac{mg}{\cos(\phi)} \sin(\phi) \]
\[ v = \sqrt{R g \tan(\phi)} \]

c) (11 marks)

[2 marks] For largest \( v \), frictional force \( F_f \) points inwards. Balancing forces, 

\[ \frac{m v^2}{R} = F_f \cos(\phi) + N \sin(\phi) \]

[2 marks] Balance vertical forces, with \( F_f = \mu_s N \),

\[ mg + \mu_s N \sin(\phi) = N \cos(\phi) \]

\[ N = \frac{mg}{\cos(\phi) - \mu_s \sin(\phi)} \]

[3 marks] substitute expressions into centripetal equation, simplify to obtain \( v_{max} \)

\[ \frac{m v^2}{R} = \mu_s N \cos(\phi) + N \sin(\phi) = N (\mu_s \cos(\phi) + \sin(\phi) ) = mg \left ( \frac{\mu_s \cos(\phi) + \sin(\phi)}{ \cos(\phi) - \mu_s \sin(\phi)} \right ) \]

\[ v_{max} = \sqrt{gR \left ( \frac{\sin(\phi) + \mu_s \cos(\phi)}{\cos(\phi) - \mu_s \sin(\phi)} \right )}  \]

[2 marks] For \( v_{min} \), \(F_f\) is reversed. Terms with \( \mu_s \) switch sign, 

\[ v_{min} = \sqrt{gR \left ( \frac{\sin(\phi) - \mu_s \cos(\phi)}{\cos(\phi) + \mu_s \sin(\phi)} \right )} \]

[2 marks] Calculation of \( v_{max}^{2} - v_{min}^{2} \), simplifying algebra:

\[ v_{max}^{2} - v_{min}^{2} = gR \left ( \frac{\sin(\phi) + \mu_s \cos(\phi)}{\cos(\phi) - \mu_s \sin(\phi)} \right ) - gR \left ( \frac{\sin(\phi) - \mu_s \cos(\phi)}{\cos(\phi) + \mu_s \sin(\phi)} \right ) = \frac{2Rg\mu_{s}}{\cos^{2}(\phi) - \mu_{s}^{2}\sin^{2}(\phi)} \]

\subsection*{Q7: (13 marks total)}

a) (4 marks)

[1 mark] \( \underline{v} \) is arbitrary, choose it to be along one of the axis. (ie. x-axis)

[1 mark] Energy momentum 4-vector of single particle, \( P \) is given by 

\[ P = ( E / c, p, 0, 0 ) \]

where \( E = \gamma m c^2 \), \( p = \gamma m | \underline{v} | = \gamma mv \), \( \gamma = \frac{1}{\sqrt{1-\frac{v^2}{c^2}}} \). Allow equivalent expressions.

[2 marks] Compute scalar product of the 4-vector, substituting definitions to show invariance.

\[ P^2 = E^2 / c^2 - \underline{p}^2 \]
\[ P^2 = \gamma^2 m^2 c^4 / c^2 - \gamma ^2 m^2 v^2 \]
\[ P^2 = m^2 c^2 \left [ \frac{1-v^2/c^2}{1-v^2/c^2} \right] = m^2c^2 \]

b) (5 marks)

[2 marks] 4-momenta of decaying particle, equal to the sum of 4-momenta of the massless particles 

\[ (E_1 / c, p_1, 0, 0) + (E_2 / c, p_2, 0, 0) = (mc, 0, 0, 0) \]

[1 mark] \( p_2 =  - p_1 \), since particles are massless, \( E_1 = E_2 = mc^2 / 2 = |p_1|c \). 4-momenta of both particles in the rest frame is

\[ \left( \frac{mc}{2}, \frac{mc}{2}, 0, 0 \right), \left( \frac{mc}{2}, -\frac{mc}{2}, 0, 0 \right) \]


[2 marks] Inverse Lorentz transforms the 4-momenta back to the lab frame . \( \beta = v / c \)

\[
\begin{pmatrix} \gamma & \beta \gamma & 0 & 0 \\ \beta \gamma & \gamma & 0 & 0 \\ 0 & 0 & 1 & 0 \\ 0 & 0 & 0 & 1 \end{pmatrix} \begin{pmatrix} mc/2\\ mc/2\\ 0\\ 0 \end{pmatrix} = \begin{pmatrix} mc \gamma (1+\beta) /2\\ mc \gamma (1+\beta) /2\\ 0\\ 0 \end{pmatrix}
\]

\[
\begin{pmatrix} \gamma & \beta \gamma & 0 & 0 \\ \beta \gamma & \gamma & 0 & 0 \\ 0 & 0 & 1 & 0 \\ 0 & 0 & 0 & 1 \end{pmatrix} \begin{pmatrix} mc/2\\ -mc/2\\ 0\\ 0 \end{pmatrix} = \begin{pmatrix} mc \gamma (1-\beta) /2\\ -mc \gamma (1-\beta) /2\\ 0\\ 0 \end{pmatrix}
\]


c) (4 marks)

[1 mark] Using \( v = 0.8 c ) \), \( \gamma = 5 / 3 \), \( \beta = 4 / 5 \), the 4-momenta are simplified to 

\[ 
\left ( \frac{3}{2}mc, \frac{3}{2}mc, 0,0 \right), \left ( \frac{1}{6}mc, -\frac{1}{6}mc,0,0 \right)
\]

[1 mark] Expression of wavelength from energy or momenta, 

\[ \lambda  = \frac{h}{p} \]
\[ \lambda = \frac{2h}{3mc} \quad \text{or} \quad \frac{6h}{mc} \]

h is planck’s constant. Use of energy is acceptable. 

[2 marks] Correct calculation using \( m = 10 GeV / c^2 \), 

\[
\lambda = 8.27 \times 10^{-17} m \quad \text{or} \quad 7.44 \times 10^{-16} m
\].

\subsection*{Q8: (10 marks total)}

a) (9 marks)

[1 mark] position of mass given by \( \underline{r} = (x + L \sin(\phi), -L \cos(\phi)) \). 

[3 marks] Calculation of kinetic and potential energy to give Lagrangian. 

\[ \dot{\underline{r}} = (\dot{x} + L\dot{\phi}\cos(\phi), L\dot{\phi}\sin(\phi) ) \]
\[ K = \frac{1}{2}m |\dot{\underline{r}}|^2 = \frac{1}{2}m (\dot{x}^2 + L^2 \dot{\phi}^2 + 2\dot{x}L\dot{\phi}\cos(\phi) ) \]
\[ U = -mgL \cos(\phi) \]
\[ L = K - U = \frac{1}{2}m \left(\dot{x}^2 + L^2 \dot{\phi}^2 + 2\dot{x}L\dot{\phi}\cos(\phi) \right) + mgL \cos(\phi) \]

[3 marks] Obtain and simplify Euler-Lagrange equations, 

\[ \frac{d}{dt}\left ( \frac{\partial L}{\partial \dot{\phi}} \right ) = \frac{\partial L}{\partial \phi} \]

\[ \frac{d}{dt} \left ( mL^2\dot{\phi} + m\dot{x}L\cos{\phi} \right ) = -m\dot{x}L\dot{\phi}\sin(\phi) - mgL\sin(\phi) \]

\[ mL^2 \ddot{\phi} + m\ddot{x}L\cos{\phi} - m\dot{x}L\dot{\phi}\sin{\phi} = -m\dot{x}L\dot{\phi}\sin(\phi) - mgL\sin(\phi) \]

\[ L \ddot{\phi} + \ddot{x}\cos{\phi} = - g\sin{\phi} \]

[2 marks] Compute \( \ddot{x} = 6vt - A \omega^2 \sin(\omega t) \), substitute to obtain equation of motion

\[ L \ddot{\phi} + \left (6vt - A\omega^2\sin(\omega t) \right )\cos{\phi} + g\sin{\phi} = 0 \]

b) 

[1 mark] 2nd term in the equation of motion vanishes, we are left with 

\[  L \ddot{\phi} + g\sin{\phi} = 0 \]
\[ \ddot{\phi} + \frac{g}{L}\sin{\phi} = 0 \]

Allow direct substitution for the expression of x into all steps and explicit calculation.

\subsection*{Q9: (7 marks total)}

a) (6 marks)

[2 marks] apply laws of motion for acceleration of block of mass m, torque on circular disk. Let \( \alpha \) be the angular acceleration of the disk, \( T \) the tension in the rope and \( a \) the acceleration of the block. 

\[ T - mg = -ma \]
\[ RT = I \alpha = \frac{1}{2} M R^2 \alpha \]

[1 mark] Constraint that acceleration of block is equal to acceleration of disk since rope does not slip 

\[ a = R \alpha \]

[3 marks] Solve simultaneous equations to give expressions for T, a and \( \alpha \).

\[ a = \frac{mg}{m + \frac{1}{2}M} \]

\[ T = \frac{1}{2}Ma = \frac{\frac{1}{2}Mmg}{m + \frac{1}{2}M} \]

\[ \alpha = a / R = \frac{mg}{(m + \frac{1}{2}M)R} \]

b) (1 mark) 

[1 mark] substitute values \(m = 3kg \), \( M = 12kg \) and \( R = 0.2m \). \(g = 9.81 ms^{-2} \).

\[ a = 3.27 \, ms^{-2} \]
\[ T = 19.62 \, N \]
\[ \alpha = 16.35 \, rads^{-2} \]

\subsection*{Q10: (8 marks total)}

a) (6 marks)

[1 mark] Net force is the sum of weight and force of air resistance. Equating with Newton’s 2nd Law, 

\[
m\frac{d\underline{v}}{dt} = - mg \hat{\underline{k}} - b\underline{v} 
\]

[3 marks] Solve differential equation, noting a general solution of the form 

\[
\underline{v} = \underline{A} e^{-bt/m} - \underline{B}
\]

[2 marks] Use of the boundary condition that at \( t = \infty \),  \( v \) approaches terminal velocity and weight is balanced by air resistance \( b\underline{v} = - mg \hat{\underline{k}} \), giving \( \underline{B} = - \frac{mg}{b} \hat{\underline{k}} \).

b) (2 marks)

[1 mark] Use of initial boundary conditions, \( \underline{v}(0) = \underline{u} \)

[1 mark] Substitute into solution to give

\[ \underline{A} = \underline{u} + \frac{mg}{b} \hat{\underline{k}} \]


\end{document}