\documentclass{article}
\usepackage{graphicx} % Required for inserting images
\usepackage{amsmath, amssymb} % For math symbols and formatting
\usepackage{amssymb}
\usepackage{hyperref} % For hyperlinks in the Table of Contents
\usepackage{titlesec} % For customizing section titles
\setlength{\parindent}{0pt} % Reads better


\title{Gemini 2.0 response collection}
\author{Jonah Donaldson}
\date{Febuary 2025}

\begin{document}

\maketitle

% Table of Contents
\tableofcontents
\newpage

\section{Introduction}

% Content for the introduction
This is a record of Gemini 2.0's solutions across a number of university-level physics problems. The problems were solved using a zero-shot approach, and no additional changes were made to the model to determine out-of-the-box utility.

\section{Classical Mechanics}

\subsection{Question 1}
Define what is a conservative force in the context of classical mechanics. For a conservative force
$\underline{F}$, mathematically define the potential energy. Why is it not possible to define a potential energy
for a force which is not conservative?

\paragraph{Prompt: \\} 
Answer the given question as a top physics student, giving step-by-step reasoning [4 marks]. Keep your answer under 300 words and format as LaTeX code without using sub-sections. 

\subsubsection{Solution 1}
A conservative force is one for which the work done in moving a particle between two points is independent of the path taken. Equivalently, the work done by the force around any closed loop is zero. This can be mathematically represented as:

$\oint \underline{F} \cdot d\underline{r} = 0$

where the integral is taken around any closed path.
\\

For a conservative force $\underline{F}$, the potential energy $U(\underline{r})$ at a point $\underline{r}$ relative to a reference point $\underline{r_0}$ can be defined as the negative of the work done by the force in moving the particle from $\underline{r_0}$ to $\underline{r}$:

$U(\underline{r}) = - \int_{\underline{r_0}}^{\underline{r}} \underline{F} \cdot d\underline{r}$
\\
This definition is valid because the path independence of the work done ensures that the integral depends only on the endpoints $\underline{r_0}$ and $\underline{r}$, not the specific path taken. Furthermore, a conservative force can be expressed as the negative gradient of a scalar potential:
\\

$\underline{F} = - \nabla U$.
\\

For a non-conservative force, the work done is path-dependent. Therefore, the integral from $\underline{r_0}$ to $\underline{r}$ would yield different values for different paths. This means we cannot define a unique scalar value for the potential energy at a given point, as it would depend on how we got there. The concept of potential energy relies on the existence of a scalar potential, which is only guaranteed for conservative forces. Friction is a classic example of a non-conservative force, as the work done against it depends on the total distance traveled, not just the initial and final positions.


\subsubsection{Solution 2}
A conservative force is one for which the work done in moving a particle between two points is independent of the path taken. Equivalently, the work done by a conservative force around any closed loop is zero.

Mathematically, this can be expressed as:

\begin{equation}
\oint \underline{F} \cdot d\underline{r} = 0
\end{equation}

This implies that the force can be expressed as the negative gradient of a scalar potential energy function, U:

\begin{equation}
\underline{F} = -\nabla U
\end{equation}

Thus, we can define the potential energy difference between two points A and B as:

\begin{equation}
U_B - U_A = -\int_A^B \underline{F} \cdot d\underline{r}
\end{equation}

**Reasoning:**
1. **Path Independence:** For a conservative force, the path independence ensures that the integral above only depends on the endpoints A and B, not the path taken. This means the potential energy difference is uniquely defined.

2. **Gradient of a Scalar Function:** If the work done is path-independent, then the force field is conservative and can be derived from a scalar potential. The negative sign is a convention, indicating that the force acts in the direction of decreasing potential energy.

3. **Non-Conservative Forces:** For a non-conservative force (e.g., friction), the work done *does* depend on the path taken. Therefore, the integral would yield different values for different paths between the same two points, making it impossible to define a unique potential energy difference. The work done by a non-conservative force is path-dependent and hence we cannot construct a scalar function which just depends on the position.

In essence, potential energy is a well-defined concept only when the work done is solely dependent on the initial and final positions and not on the details of the journey.

\subsubsection{Solution 3}

A conservative force is one for which the work done in moving a particle between two points is independent of the path taken. Equivalently, the work done by the force around any closed loop is zero. This can be mathematically represented as:

$\oint \underline{F} \cdot d\underline{r} = 0$.

**Step 1:**  A force is conservative if and only if it can be expressed as the negative gradient of a scalar potential energy function, U.
This can be written as:

$\underline{F} = -\nabla U$.

**Step 2:** For a conservative force, the potential energy difference between two points A and B is defined as the negative of the work done by the force in moving a particle from A to B:

$U(\underline{r}_B) - U(\underline{r}_A) = -\int_{\underline{r}_A}^{\underline{r}_B} \underline{F} \cdot d\underline{r}$.

**Step 3:** It is not possible to define a potential energy for a non-conservative force because the work done by such a force depends on the path taken.
If the work done depended on the path, then the integral above would have different values for different paths between A and B, leading to an ambiguous definition of potential energy.

**Step 4:** A unique value for the potential energy difference can only be assigned if the work done is path-independent, which is only true for conservative forces.
Friction and drag are prime examples of non-conservative forces where energy is not conserved, as it is dissipated as heat.

\subsection{Question 2}

The damped harmonic oscillator is governed by an equation of motion

\begin{equation*}
\frac{d^2x}{dt^2} + b\frac{dx}{dt} + kx = 0
\end{equation*}

State the 3 cases of damping and qualitatively describe the behavior of each case.

\paragraph{Prompt: \\} 
Answer the given question as a top physics student, giving step-by-step reasoning [6 marks].
Format your answer as LaTeX code without using sub-sections. 

\subsubsection{Solution 1}
\noindent \textbf{Answer:}

The given equation of motion for the damped harmonic oscillator is:
\begin{equation*}
\frac{d^2x}{dt^2} + b\frac{dx}{dt} + kx = 0
\end{equation*}
where $b$ is the damping coefficient and $k$ is the spring constant. The behavior of the system depends on the roots of the characteristic equation, which is obtained by assuming a solution of the form $x(t) = e^{rt}$. Substituting this into the equation of motion, we get:

\begin{equation*}
r^2 + br + k = 0
\end{equation*}
The roots of this quadratic equation are given by:

\begin{equation*}
r = \frac{-b \pm \sqrt{b^2 - 4k}}{2}
\end{equation*}
The nature of the roots, and thus the behavior of the system, is determined by the discriminant, $\Delta = b^2 - 4k$.

\noindent \textbf{Case 1: Overdamping ($\Delta > 0$ or $b^2 > 4k$)}

\textit{Reasoning}: In this case, the discriminant is positive, leading to two distinct real roots. Both roots are negative because $b$ is positive.

\textit{Behavior}: The system returns to equilibrium without oscillating. The displacement decays exponentially to zero. The larger the damping coefficient $b$, the slower the system returns to equilibrium. This is akin to a mass-spring system immersed in a very viscous fluid.

\noindent \textbf{Case 2: Critical Damping ($\Delta = 0$ or $b^2 = 4k$)}

\textit{Reasoning}: Here, the discriminant is zero, resulting in a single real root, $r = -\frac{b}{2}$.

\textit{Behavior}: The system returns to equilibrium as quickly as possible without oscillating. This is the fastest non-oscillatory return to equilibrium. The displacement decays exponentially to zero, and it represents the boundary between oscillatory and non-oscillatory behavior. Critical damping is often desired in systems where a quick return to equilibrium without oscillations is necessary, such as in door-closing mechanisms and instrument needles.

\noindent \textbf{Case 3: Underdamping ($\Delta < 0$ or $b^2 < 4k$)}

\textit{Reasoning}: In this scenario, the discriminant is negative, yielding two complex conjugate roots. The roots can be written as $r = -\alpha \pm i\omega_d$, where $\alpha = \frac{b}{2}$ and $\omega_d = \frac{\sqrt{4k - b^2}}{2}$ is the damped angular frequency.

\textit{Behavior}: The system oscillates with a decreasing amplitude over time. The displacement is given by $x(t) = Ae^{-\alpha t}\cos(\omega_d t + \phi)$, where $A$ and $\phi$ are constants determined by the initial conditions. The amplitude of oscillation decays exponentially at a rate determined by $\alpha$, while the frequency of oscillation is $\omega_d$, which is lower than the natural frequency $\omega_0 = \sqrt{k}$ of the undamped oscillator. This behavior is observed in systems with relatively weak damping, like a pendulum swinging in air.

\noindent \textbf{Summary}
Overdamped - Exponential decay to zero with no oscillation
Critically damped - Fastest decay to zero with no oscillation
Underdamped - Oscillations with decaying amplitude


\subsubsection{Solution 2}
\noindent \textbf{Answer:}

The given equation represents a damped harmonic oscillator, where the term  $b\frac{dx}{dt}$ signifies the damping force, proportional to the velocity. The behavior of the system depends on the discriminant of the characteristic equation associated with this differential equation. The characteristic equation is given by:
\begin{equation*}
r^2 + br + k = 0
\end{equation*}

The roots of this quadratic equation are:
\begin{equation*}
r = \frac{-b \pm \sqrt{b^2 - 4k}}{2}
\end{equation*}

The discriminant, $\Delta$, is given by:
\begin{equation*}
\Delta = b^2 - 4k
\end{equation*}

We can now analyze the three cases of damping based on the value of $\Delta$:

\noindent \textbf{Case 1: Overdamped ($\Delta > 0$ or $b^2 > 4k$)}

\textit{Reasoning}: When the discriminant is positive, we have two distinct real roots. The general solution is a linear combination of two decaying exponentials. No oscillations are present in the motion.

\textit{Behavior}: In this case, the system returns to equilibrium without oscillating. The damping force is strong enough to prevent oscillations. The system returns to equilibrium relatively slowly, as the large damping force resists motion.

\noindent \textbf{Case 2: Critically Damped ($\Delta = 0$ or $b^2 = 4k$)}

\textit{Reasoning}: When the discriminant is zero, we have a repeated real root. The general solution is a linear combination of a decaying exponential and a term that is the product of time and a decaying exponential.

\textit{Behavior}: This represents the boundary between oscillatory and non-oscillatory behavior. The system returns to equilibrium as quickly as possible without oscillating. Critically damped systems are often desired in applications where a system needs to return to equilibrium rapidly without overshooting, like in a mechanical door closer or car suspension.

\noindent \textbf{Case 3: Underdamped ($\Delta < 0$ or $b^2 < 4k$)}

\textit{Reasoning}: When the discriminant is negative, we have two complex conjugate roots. The general solution can be expressed as a decaying exponential multiplied by a sinusoidal function.

\textit{Behavior}: In this case, the system oscillates as it returns to equilibrium. The amplitude of the oscillations decreases exponentially over time. The frequency of oscillation is lower than the natural frequency of the undamped oscillator. This is the most common scenario in real-world oscillatory systems, like a mass on a spring with some air resistance.

\subsubsection{Solution 3}
\noindent The given equation represents a damped harmonic oscillator, where \\
\(\frac{d^2x}{dt^2}\) is the acceleration, \\
\(b\frac{dx}{dt}\) is the damping force proportional to velocity, and \\
\(kx\) is the restoring force proportional to displacement.

\noindent The characteristic equation for this differential equation is given by:
\begin{equation*}
r^2 + br + k = 0
\end{equation*}
\noindent The roots of this equation determine the behavior of the 
oscillator. The roots are given by the quadratic formula:

r = \frac{-b \pm \sqrt{b^2 - 4k}}{2}

\noindent The behavior depends on the discriminant, \(b^2 - 4k\). We can identify three cases based on the discriminant:

\noindent \textbf{Case 1: Overdamping} (\(b^2 - 4k > 0\))

\noindent \textit{Reasoning:} In this case, the discriminant is positive, leading to two distinct real roots. \\
\textit{Behavior:} The system returns to equilibrium without oscillating. The displacement decays exponentially to zero. The larger the value of \(b\) (damping coefficient) relative to \(\sqrt{4k}\), the slower the system returns to equilibrium.

\noindent \textbf{Case 2: Critical Damping} (\(b^2 - 4k = 0\))

\noindent \textit{Reasoning:} Here, the discriminant is zero, resulting in a single real root (or two equal real roots). \\
\textit{Behavior:} The system returns to equilibrium as quickly as possible without oscillating. This represents the fastest non-oscillatory return to equilibrium. The displacement also decays exponentially to zero. Critical damping is often desired in systems where oscillations are undesirable, and a fast return to equilibrium is needed.

\noindent \textbf{Case 3: Underdamping} (\(b^2 - 4k < 0\))

\noindent \textit{Reasoning:} In this scenario, the discriminant is negative, leading to a pair of complex conjugate roots. \\
\textit{Behavior:} The system oscillates with decreasing amplitude as it returns to equilibrium. The solution involves sinusoidal functions multiplied by an exponentially decaying term. The frequency of oscillation is less than the natural frequency of the undamped oscillator. The smaller the value of \(b\) relative to \(\sqrt{4k}\), the longer the oscillations persist before the system comes to rest.

\subsection{Question 3}
Q3.

Two objects of masses $m_1$ and $m_2$ are separated by a distance $d$. The object of mass $m_1$ is at
position $\underline{r}_1$ and the object of mass $m_2$ is at position $\underline{r}_2$.

a) Starting from the definition of center of mass, show that its position vector can be given by
$$\underline{R} = (1-k)\underline{r}_1 + k\underline{r}_2$$
and find the appropriate value of $k$.

b) Show that the center of mass lies on a line connecting the 2 masses and its distance from the 2
masses are $dm_2/(m_1+m_2)$ from $\underline{r}_1$ and $dm_1/(m_1+m_2)$ from $\underline{r}_2$.

\paragraph{Prompt: \\} 
Answer the given question as a top physics student, giving step-by-step reasoning a) [2 marks] b) [4 marks]. Format your answer as LaTeX code without using sub-sections.

\subsubsection{Solution 1}
\textbf{Solution:}

\textbf{a)}

\textit{Step 1: State the definition of the center of mass.}

The position vector of the center of mass, $\underline{R}$, for a system of two particles is defined as:
$$\underline{R} = \frac{m_1 \underline{r}_1 + m_2 \underline{r}_2}{m_1 + m_2}$$

\textit{Step 2: Rearrange the equation to match the desired form.}

We can rewrite the equation as:
$$\underline{R} = \frac{m_1}{m_1 + m_2}\underline{r}_1 + \frac{m_2}{m_1 + m_2}\underline{r}_2$$

$$\underline{R} = \left(1 - \frac{m_2}{m_1 + m_2}\right)\underline{r}_1 + \frac{m_2}{m_1 + m_2}\underline{r}_2$$

\textit{Step 3: Identify the value of k.}

Comparing this with the given equation $\underline{R} = (1-k)\underline{r}_1 + k\underline{r}_2$, we can see that:
$$k = \frac{m_2}{m_1 + m_2}$$

\textit{Final Answer:}

Therefore, the position vector of the center of mass can be given by $\underline{R} = (1-k)\underline{r}_1 + k\underline{r}_2$, where $k = \frac{m_2}{m_1+m_2}$.

\textbf{b)}

\textit{Step 1: Express the vector connecting the two masses.}

The vector connecting the two masses is given by $\underline{r}_2 - \underline{r}_1$. Since the center of mass $\underline{R}$ is a weighted average of $\underline{r}_1$ and $\underline{r}_2$, it must lie on the line connecting the two masses.

\textit{Step 2: Find the vector from r1 to R.}

The vector from $\underline{r}_1$ to $\underline{R}$ is given by:
$$\underline{R} - \underline{r}_1 = \left( \frac{m_1\underline{r}_1 + m_2\underline{r}_2}{m_1 + m_2} \right) - \underline{r}_1 = \frac{m_2(\underline{r}_2 - \underline{r}_1)}{m_1 + m_2}$$

\textit{Step 3: Calculate the magnitude of the vector from r1 to R.}

The distance between $m_1$ and the center of mass is the magnitude of this vector:
$$|\underline{R} - \underline{r}_1| = \left| \frac{m_2(\underline{r}_2 - \underline{r}_1)}{m_1 + m_2} \right| = \frac{m_2}{m_1 + m_2} |\underline{r}_2 - \underline{r}_1|$$

Since $d = |\underline{r}_2 - \underline{r}_1|$, the distance is:
$$|\underline{R} - \underline{r}_1| = \frac{m_2}{m_1 + m_2} d$$

\textit{Step 4: Find the vector from r2 to R.}

The vector from $\underline{r}_2$ to $\underline{R}$ is given by:
$$\underline{R} - \underline{r}_2 = \left( \frac{m_1\underline{r}_1 + m_2\underline{r}_2}{m_1 + m_2} \right) - \underline{r}_2 = \frac{m_1(\underline{r}_1 - \underline{r}_2)}{m_1 + m_2} = -\frac{m_1(\underline{r}_2 - \underline{r}_1)}{m_1 + m_2}$$

\textit{Step 5: Calculate the magnitude of the vector from r2 to R.}
The distance between $m_2$ and the center of mass is the magnitude of this vector:
$$|\underline{R} - \underline{r}_2| = \left| -\frac{m_1(\underline{r}_2 - \underline{r}_1)}{m_1 + m_2} \right| = \frac{m_1}{m_1 + m_2} |\underline{r}_2 - \underline{r}_1|$$

Since $d = |\underline{r}_2 - \underline{r}_1|$, the distance is:
$$|\underline{R} - \underline{r}_2| = \frac{m_1}{m_1 + m_2} d$$

\textit{Final Answer:}
The center of mass lies on the line connecting the two masses, and its distances from the two masses are $\frac{dm_2}{m_1 + m_2}$ from $\underline{r}_1$ and $\frac{dm_1}{m_1 + m_2}$ from $\underline{r}_2$.


\subsubsection{Solution 2}
a)

The definition of the center of mass $\underline{R}$ for a system of two particles is given by:

$$ \underline{R} = \frac{m_1\underline{r}_1 + m_2\underline{r}_2}{m_1 + m_2} $$

We can rewrite this expression as:

$$ \underline{R} = \frac{m_1}{m_1 + m_2}\underline{r}_1 + \frac{m_2}{m_1 + m_2}\underline{r}_2 $$

Now, let's factor out $\underline{r}_1$ from the first term:

$$ \underline{R} = \left(1 - \frac{m_2}{m_1 + m_2}\right)\underline{r}_1 + \frac{m_2}{m_1 + m_2}\underline{r}_2 $$
Comparing this with the given equation $\underline{R} = (1-k)\underline{r}_1 + k\underline{r}_2$, we can identify that:

$$ k = \frac{m_2}{m_1 + m_2} $$

Thus, we have shown that the position vector of the center of mass can be given by $\underline{R} = (1-k)\underline{r}_1 + k\underline{r}_2$ and found the appropriate value of $k$ to be $k = \frac{m_2}{m_1 + m_2}$.

b)

To show that the center of mass lies on the line connecting the two masses, we can express $\underline{R}$ as a linear combination of $\underline{r}_1$ and $\underline{r}_2$. Since we have already found that:
$$ \underline{R} = \left(1 - \frac{m_2}{m_1 + m_2}\right)\underline{r}_1 + \frac{m_2}{m_1 + m_2}\underline{r}_2 $$
And both coefficients are positive and sum to 1, we can conclude that $\underline{R}$ lies on the line segment connecting $\underline{r}_1$ and $\underline{r}_2$.

Now let's find the distance of the center of mass from $\underline{r}_1$ and $\underline{r}_2$. The vector from $\underline{r}_1$ to $\underline{R}$ is given by:
$$ \underline{R} - \underline{r}_1 = \left( \frac{m_2}{m_1 + m_2} \right) (\underline{r}_2 - \underline{r}_1) $$
The magnitude of this vector is the distance between the center of mass and $\underline{r}_1$:
$$ |\underline{R} - \underline{r}_1| = \left| \frac{m_2}{m_1 + m_2} \right| |\underline{r}_2 - \underline{r}_1| $$
Since the distance between the two masses is given as $d = |\underline{r}_2 - \underline{r}_1|$, the distance from $\underline{r}_1$ to the center of mass is:
$$ |\underline{R} - \underline{r}_1| = \frac{m_2}{m_1 + m_2} d $$
Similarly, the vector from $\underline{r}_2$ to $\underline{R}$ is:
$$ \underline{R} - \underline{r}_2 = \left( \frac{m_1}{m_1 + m_2} - 1 \right) \underline{r}_2 + \left( \frac{m_1}{m_1 + m_2} \right) \underline{r}_1 $$
$$ \underline{R} - \underline{r}_2 = -\left( \frac{m_1}{m_1 + m_2} \right) (\underline{r}_2 - \underline{r}_1) $$

The magnitude of this vector is the distance between the center of mass and $\underline{r}_2$:
$$ |\underline{R} - \underline{r}_2| = \left| -\frac{m_1}{m_1 + m_2} \right| |\underline{r}_2 - \underline{r}_1| $$
The distance from $\underline{r}_2$ to the center of mass is:
$$ |\underline{R} - \underline{r}_2| = \frac{m_1}{m_1 + m_2} d $$
Thus, we have shown that the center of mass lies on the line connecting the two masses, and its distances from $\underline{r}_1$ and $\underline{r}_2$ are $\frac{dm_2}{m_1 + m_2}$ and $\frac{dm_1}{m_1 + m_2}$, respectively.


\subsubsection{Solution 3}
\textbf{Solution:}

\textbf{a)}

The definition of the center of mass is given by:

$$\underline{R} = \frac{\sum m_i \underline{r}_i}{\sum m_i}$$

For two objects, this becomes:

$$\underline{R} = \frac{m_1 \underline{r}_1 + m_2 \underline{r}_2}{m_1 + m_2}$$

We want to express this in the form $\underline{R} = (1-k)\underline{r}_1 + k\underline{r}_2$.  Let's manipulate the definition:

$$\underline{R} = \frac{m_1}{m_1 + m_2} \underline{r}_1 + \frac{m_2}{m_1 + m_2} \underline{r}_2$$

We can rewrite the first term as:

$$\frac{m_1}{m_1 + m_2} \underline{r}_1 = \left(1 - \frac{m_2}{m_1 + m_2}\right) \underline{r}_1$$

Substituting this back into the equation for $\underline{R}$:

$$\underline{R} = \left(1 - \frac{m_2}{m_1 + m_2}\right) \underline{r}_1 + \frac{m_2}{m_1 + m_2} \underline{r}_2$$

Comparing this to the desired form $\underline{R} = (1-k)\underline{r}_1 + k\underline{r}_2$, we see that:

$$k = \frac{m_2}{m_1 + m_2}$$

Thus, the position vector of the center of mass can be given by $\underline{R} = (1-k)\underline{r}_1 + k\underline{r}_2$ with $k = \frac{m_2}{m_1 + m_2}$. [2 marks]

\textbf{b)}

The vector pointing from $\underline{r}_1$ to $\underline{r}_2$ is given by $\underline{r}_2 - \underline{r}_1$. Since $\underline{R}$ is a weighted average of $\underline{r}_1$ and $\underline{r}_2$ with coefficients that sum to 1, it must lie on the line segment connecting the two masses. To demonstrate this explicitly, we can rewrite $\underline{R}$ as follows:

$$\underline{R} = \underline{r}_1 + \frac{m_2}{m_1+m_2} (\underline{r}_2 - \underline{r}_1)$$

This equation expresses $\underline{R}$ as the starting point $\underline{r}_1$ plus a fraction of the vector connecting $\underline{r}_1$ to $\underline{r}_2$.  Since $k = \frac{m_2}{m_1+m_2}$ is between 0 and 1, $\underline{R}$ lies on the line segment between the two masses.

The distance between $m_1$ and the center of mass is given by:

$$|\underline{R} - \underline{r}_1| = \left| \left(1 - \frac{m_2}{m_1 + m_2}\right)\underline{r}_1 + \frac{m_2}{m_1 + m_2}\underline{r}_2 - \underline{r}_1 \right|$$

$$|\underline{R} - \underline{r}_1| = \left| \frac{m_2}{m_1 + m_2}(\underline{r}_2 - \underline{r}_1) \right|$$

Since $d = |\underline{r}_2 - \underline{r}_1|$, we have:

$$|\underline{R} - \underline{r}_1| = \frac{m_2}{m_1 + m_2} d = \frac{dm_2}{m_1+m_2}$$

Similarly, the distance between $m_2$ and the center of mass is given by:

$$|\underline{R} - \underline{r}_2| = \left| \left(1 - \frac{m_2}{m_1 + m_2}\right)\underline{r}_1 + \frac{m_2}{m_1 + m_2}\underline{r}_2 - \underline{r}_2 \right|$$

$$|\underline{R} - \underline{r}_2| = \left| \frac{m_1}{m_1 + m_2}(\underline{r}_1 - \underline{r}_2) \right|$$

$$|\underline{R} - \underline{r}_2| = \frac{m_1}{m_1 + m_2} d = \frac{dm_1}{m_1+m_2}$$

Thus, the center of mass lies on the line connecting the two masses, and its distances from the two masses are $\frac{dm_2}{m_1+m_2}$ from $\underline{r}_1$ and $\frac{dm_1}{m_1+m_2}$ from $\underline{r}_2$. [4 marks]

\subsection{Question 4}
Q4.

    In 2D polar coordinates, the velocity, $\underline{v}$, and acceleration, $\underline{a}$, are given by 
    
    10

    \[\underline{v}=\dot{r}\underline{\hat{r}}+r\dot{\theta}\underline{\hat{\theta}}\]

    \[\underline{a}=(\ddot{r}-r\dot{\theta}^{2})\underline{\hat{r}}+(r\ddot{\theta}+2\dot{r}\dot{\theta})\underline{\hat{\theta}}\]

A particle of mass, $m$, rotates with angular frequency, $\dot{\theta}$, dependent on time and has radial velocity
given by $\dot{r}=-\gamma,$ where $\gamma$ is a constant. At time $t=0,$ the particle has radial position $r_{0}$ and
\[\dot{\theta}=\delta.\]

a) Assuming angular momentum is conserved, derive an expression for the angular frequency $\dot{\theta}$.
Write your answer in terms of $r_{0}$, $\gamma$, and $\delta$.

b) What is the angular component of the acceleration?

c) Derive an expression for the kinetic energy of the particle and show that
\[\frac{dK}{dt}=\frac{mr_{0}^{4}\gamma\delta^{2}}{(r_{0}-\gamma t)^{3}}\]

\paragraph{Prompt: \\} 
Answer the given question as a top physics student, giving step-by-step reasoning a) [6 marks] b) [5 marks] c) [7 marks]. Format your answer as LaTeX code without using sub-sections. 

\subsubsection{Solution 1}
\textbf{a)}

**Step 1: Define Angular Momentum**

The angular momentum, $L$, of the particle about the origin is given by:

\[L = \underline{r} \times m\underline{v}\]

In polar coordinates, $\underline{r} = r\underline{\hat{r}}$ and $\underline{v} = \dot{r}\underline{\hat{r}} + r\dot{\theta}\underline{\hat{\theta}}$. Since $\underline{\hat{r}} \times \underline{\hat{r}} = 0$ and $\underline{\hat{r}} \times \underline{\hat{\theta}} = \underline{\hat{z}}$, we have

\[L = mr^2\dot{\theta}\underline{\hat{z}}\]

**Step 2: Apply Conservation of Angular Momentum**

Since angular momentum is conserved, the initial angular momentum is equal to the angular momentum at any time t:

\[L_0 = L\]

\[mr_0^2\delta = mr^2\dot{\theta}\]

**Step 3: Express r in terms of t**

We are given $\dot{r} = -\gamma$. Integrating with respect to time, and using the initial condition $r(0) = r_0$, we get:

\[r = r_0 - \gamma t\]

**Step 4: Substitute and Solve for $\dot{\theta}$**

Substituting the expression for $r$ into the conservation of angular momentum equation:

\[r_0^2\delta = (r_0 - \gamma t)^2\dot{\theta}\]

\[\dot{\theta} = \frac{r_0^2\delta}{(r_0 - \gamma t)^2}\]

Therefore, the angular frequency $\dot{\theta}$ is:

\[\boxed{\dot{\theta} = \frac{r_0^2\delta}{(r_0 - \gamma t)^2}}\]

\textbf{b)}

**Step 1: Identify the Angular Component of Acceleration**

The angular component of the acceleration, $a_{\theta}$, is given by:

\[a_{\theta} = r\ddot{\theta} + 2\dot{r}\dot{\theta}\]

**Step 2: Calculate $\ddot{\theta}$**

Differentiating the expression for $\dot{\theta}$ with respect to time:

\[\ddot{\theta} = \frac{d}{dt}\left(\frac{r_0^2\delta}{(r_0 - \gamma t)^2}\right)\]

\[\ddot{\theta} = \frac{2r_0^2\delta\gamma}{(r_0 - \gamma t)^3}\]

**Step 3: Substitute into the Expression for $a_{\theta}$**

Substituting $\dot{r} = -\gamma$, $\dot{\theta} = \frac{r_0^2\delta}{(r_0 - \gamma t)^2}$, $r = r_0 - \gamma t$, and $\ddot{\theta} = \frac{2r_0^2\delta\gamma}{(r_0 - \gamma t)^3}$ into the expression for $a_{\theta}$:

\[a_{\theta} = (r_0 - \gamma t)\left(\frac{2r_0^2\delta\gamma}{(r_0 - \gamma t)^3}\right) + 2(-\gamma)\left(\frac{r_0^2\delta}{(r_0 - \gamma t)^2}\right)\]

\[a_{\theta} = \frac{2r_0^2\delta\gamma}{(r_0 - \gamma t)^2} - \frac{2r_0^2\delta\gamma}{(r_0 - \gamma t)^2}\]

\[a_{\theta} = 0\]

Therefore, the angular component of the acceleration is:

\[\boxed{a_{\theta} = 0}\]

\textbf{c)}

**Step 1: Express Kinetic Energy in Polar Coordinates**

The kinetic energy, $K$, of the particle is given by:

\[K = \frac{1}{2}mv^2 = \frac{1}{2}m(\dot{r}^2 + r^2\dot{\theta}^2)\]

**Step 2: Substitute for $\dot{r}$, $r$, and $\dot{\theta}$**

Substituting $\dot{r} = -\gamma$, $r = r_0 - \gamma t$, and $\dot{\theta} = \frac{r_0^2\delta}{(r_0 - \gamma t)^2}$:

\[K = \frac{1}{2}m\left(\gamma^2 + (r_0 - \gamma t)^2\left(\frac{r_0^2\delta}{(r_0 - \gamma t)^2}\right)^2\right)\]

\[K = \frac{1}{2}m\left(\gamma^2 + \frac{r_0^4\delta^2}{(r_0 - \gamma t)^2}\right)\]

**Step 3: Calculate $\frac{dK}{dt}$**

Differentiating $K$ with respect to time:

\[\frac{dK}{dt} = \frac{1}{2}m\left(0 + \frac{d}{dt}\left(\frac{r_0^4\delta^2}{(r_0 - \gamma t)^2}\right)\right)\]

\[\frac{dK}{dt} = \frac{1}{2}m\left(2\frac{r_0^4\delta^2}{(r_0 - \gamma t)^3}\gamma\right)\]

\[\frac{dK}{dt} = \frac{mr_0^4\gamma\delta^2}{(r_0 - \gamma t)^3}\]

Therefore, the rate of change of kinetic energy is:

\[\boxed{\frac{dK}{dt} = \frac{mr_0^4\gamma\delta^2}{(r_0 - \gamma t)^3}}\]

\subsubsection{Solution 2}
\textbf{a)}

**Step 1: Define angular momentum.**

Angular momentum, $L$, is given by the cross product of the position vector and the linear momentum. In polar coordinates:

$L = \underline{r} \times m\underline{v} = r\underline{\hat{r}} \times m(\dot{r}\underline{\hat{r}} + r\dot{\theta}\underline{\hat{\theta}}) = mr^2\dot{\theta}\underline{\hat{z}}$.

**Step 2: Apply conservation of angular momentum.**

Since angular momentum is conserved, $L$ is constant. Therefore, $L$ at any time $t$ is equal to $L$ at $t=0$.

At $t=0$, $r=r_0$ and $\dot{\theta}=\delta$, so $L = mr_0^2\delta$.

At time $t$, $\dot{r} = -\gamma$, integrating with respect to time gives $r = r_0 - \gamma t$.

Equating the angular momentum at time $t$ to the initial angular momentum:

$mr^2\dot{\theta} = mr_0^2\delta$

$m(r_0 - \gamma t)^2\dot{\theta} = mr_0^2\delta$

**Step 3: Solve for $\dot{\theta}$.**

$\dot{\theta} = \frac{r_0^2\delta}{(r_0 - \gamma t)^2}$

**b)**

**Step 1: Identify the angular component of acceleration.**

The angular component of acceleration is given by $a_{\theta} = r\ddot{\theta} + 2\dot{r}\dot{\theta}$.

**Step 2: Calculate $\ddot{\theta}$.**

We have $\dot{\theta} = \frac{r_0^2\delta}{(r_0 - \gamma t)^2}$. Differentiating with respect to time:

$\ddot{\theta} = \frac{d}{dt} \left( \frac{r_0^2\delta}{(r_0 - \gamma t)^2} \right) = \frac{2r_0^2\delta\gamma}{(r_0 - \gamma t)^3}$

**Step 3: Substitute into the angular acceleration equation.**

We have $\dot{r} = -\gamma$, $r = r_0 - \gamma t$, $\dot{\theta} = \frac{r_0^2\delta}{(r_0 - \gamma t)^2}$, and $\ddot{\theta} = \frac{2r_0^2\delta\gamma}{(r_0 - \gamma t)^3}$. Substituting these into $a_{\theta} = r\ddot{\theta} + 2\dot{r}\dot{\theta}$:

$a_{\theta} = (r_0 - \gamma t)\left(\frac{2r_0^2\delta\gamma}{(r_0 - \gamma t)^3}\right) + 2(-\gamma)\left(\frac{r_0^2\delta}{(r_0 - \gamma t)^2}\right)$

$a_{\theta} = \frac{2r_0^2\delta\gamma}{(r_0 - \gamma t)^2} - \frac{2r_0^2\delta\gamma}{(r_0 - \gamma t)^2} = 0$

**c)**

**Step 1: Write the expression for kinetic energy.**

The kinetic energy, $K$, in polar coordinates is given by:

$K = \frac{1}{2}m(\dot{r}^2 + r^2\dot{\theta}^2)$

**Step 2: Substitute the given values.**

We have $\dot{r} = -\gamma$, $r = r_0 - \gamma t$, and $\dot{\theta} = \frac{r_0^2\delta}{(r_0 - \gamma t)^2}$. Substituting:

$K = \frac{1}{2}m\left(\gamma^2 + (r_0 - \gamma t)^2\left(\frac{r_0^2\delta}{(r_0 - \gamma t)^2}\right)^2\right)$

$K = \frac{1}{2}m\left(\gamma^2 + \frac{r_0^4\delta^2}{(r_0 - \gamma t)^2}\right)$

**Step 3: Differentiate $K$ with respect to time.**

$\frac{dK}{dt} = \frac{1}{2}m\left(0 + \frac{d}{dt}\left(\frac{r_0^4\delta^2}{(r_0 - \gamma t)^2}\right)\right)$

$\frac{dK}{dt} = \frac{1}{2}m\left(r_0^4\delta^2 \cdot \frac{2\gamma}{(r_0 - \gamma t)^3}\right)$

$\frac{dK}{dt} = \frac{mr_0^4\gamma\delta^2}{(r_0 - \gamma t)^3}$


\subsubsection{Solution 3}
\textbf{a)}

**Step 1: Define angular momentum.**

Angular momentum, $L$, is given by the cross product of the position vector and the linear momentum:

\[L = \underline{r} \times m\underline{v}\]

**Step 2: Express angular momentum in polar coordinates.**

In polar coordinates, $\underline{r} = r\underline{\hat{r}}$ and $\underline{v} = \dot{r}\underline{\hat{r}} + r\dot{\theta}\underline{\hat{\theta}}$. Thus,

\[L = r\underline{\hat{r}} \times m(\dot{r}\underline{\hat{r}} + r\dot{\theta}\underline{\hat{\theta}}) = mr^2\dot{\theta}(\underline{\hat{r}} \times \underline{\hat{\theta}}) = mr^2\dot{\theta}\underline{\hat{z}}\]

**Step 3: Apply conservation of angular momentum.**

Since angular momentum is conserved, $L$ is constant. Therefore,

\[L = mr^2\dot{\theta} = \text{constant}\]

At $t=0$, $r=r_0$ and $\dot{\theta}=\delta$, so the constant is $mr_0^2\delta$.

**Step 4: Express $r$ in terms of $t$.**

We are given $\dot{r} = -\gamma$. Integrating with respect to time, we get:

\[r = r_0 - \gamma t\]

**Step 5: Derive the expression for $\dot{\theta}$.**

Using the conservation of angular momentum and the expression for $r$:

\[mr^2\dot{\theta} = mr_0^2\delta\]
\[\dot{\theta} = \frac{r_0^2\delta}{r^2} = \frac{r_0^2\delta}{(r_0 - \gamma t)^2}\]

Thus, the angular frequency is:
\[\dot{\theta} = \frac{r_0^2\delta}{(r_0 - \gamma t)^2}\]

\textbf{b)}

**Step 1: Identify the angular component of acceleration.**

The angular component of acceleration is given by:

\[a_\theta = r\ddot{\theta} + 2\dot{r}\dot{\theta}\]

**Step 2: Calculate $\ddot{\theta}$.**

Differentiating $\dot{\theta}$ with respect to time:

\[\ddot{\theta} = \frac{d}{dt}\left(\frac{r_0^2\delta}{(r_0 - \gamma t)^2}\right) = \frac{2r_0^2\delta\gamma}{(r_0 - \gamma t)^3}\]

**Step 3: Substitute $\dot{r}$, $\dot{\theta}$, and $\ddot{\theta}$ into the expression for $a_\theta$.**

\[a_\theta = (r_0 - \gamma t)\left(\frac{2r_0^2\delta\gamma}{(r_0 - \gamma t)^3}\right) + 2(-\gamma)\left(\frac{r_0^2\delta}{(r_0 - \gamma t)^2}\right)\]

**Step 4: Simplify the expression.**

\[a_\theta = \frac{2r_0^2\delta\gamma}{(r_0 - \gamma t)^2} - \frac{2r_0^2\delta\gamma}{(r_0 - \gamma t)^2} = 0\]

Therefore, the angular component of the acceleration is:

\[a_\theta = 0\]

\textbf{c)}

**Step 1: Express kinetic energy in polar coordinates.**

The kinetic energy, $K$, is given by:

\[K = \frac{1}{2}m|\underline{v}|^2 = \frac{1}{2}m(\dot{r}^2 + r^2\dot{\theta}^2)\]

**Step 2: Substitute the given values for $\dot{r}$, $r$, and $\dot{\theta}$.**

\[K = \frac{1}{2}m\left((-\gamma)^2 + (r_0 - \gamma t)^2\left(\frac{r_0^2\delta}{(r_0 - \gamma t)^2}\right)^2\right)\]
\[K = \frac{1}{2}m\left(\gamma^2 + \frac{r_0^4\delta^2}{(r_0 - \gamma t)^2}\right)\]

**Step 3: Differentiate $K$ with respect to time.**

\[\frac{dK}{dt} = \frac{1}{2}m\left(0 + \frac{d}{dt}\left(\frac{r_0^4\delta^2}{(r_0 - \gamma t)^2}\right)\right)\]
\[\frac{dK}{dt} = \frac{1}{2}m\left(r_0^4\delta^2 \cdot \frac{2\gamma}{(r_0 - \gamma t)^3}\right)\]

**Step 4: Simplify the expression.**

\[\frac{dK}{dt} = \frac{mr_0^4\gamma\delta^2}{(r_0 - \gamma t)^3}\]

Thus, we have shown that:
\[\frac{dK}{dt} = \frac{mr_0^4\gamma\delta^2}{(r_0 - \gamma t)^3}\]

\subsection{Question 5}

\begin{center}
\includegraphics[width=0.8\textwidth]{image.png} % Replace image.png with the actual image file
\end{center}

Two objects X and Y of the same mass, $m$, are connected by a spring and each object is connected
to a fixed wall by a spring (as seen in the figure). All springs have the same spring constant,
$k$. Initially object X is at rest whilst object Y moves with an initial velocity $v\hat{i}$. The horizontal
displacement from equilibrium of X and Y is defined by $x_1$ and $x_2$ respectively.

a) Assuming no other forces besides those caused by the springs, derive the equations of motion
for object X and object Y.

b) By defining new coordinates $y_1 = x_1 + x_2$, $y_2 = x_1 - x_2$, show that this uncouples the coordinates
and $y_1$ and $y_2$ obey equations of undamped, undriven simple harmonic motion.

c) With the initial conditions of the system, show the solution of $y_1$ is given by \[y_{1}=v\sqrt{\frac{m}{k}}sin(t\sqrt{\frac{k}{m}})\]

\paragraph{Prompt: \\} 
Answer the given question as a top physics student, giving step-by-step reasoning a) [2 marks] b) [4 marks] c) [3 marks]. Format your answer as LaTeX code without using sub-sections.

\subsubsection{Solution 1}

\textbf{Answer:}

\textbf{a)}

\textit{Reasoning:}
We apply Newton's second law, $F=ma$, to each object. The forces acting on each object are due to the extension or compression of the springs. Hooke's law states that the force exerted by a spring is proportional to its extension/compression, $F=-kx$.

For object X, the force from the left spring is $-kx_1$ and the force from the middle spring is $k(x_2 - x_1)$.

For object Y, the force from the middle spring is $-k(x_2 - x_1)$ and the force from the right spring is $-kx_2$.

\textit{Equations of motion:}

For object X:
\[ m\ddot{x}_1 = -kx_1 + k(x_2 - x_1) \]
\[ m\ddot{x}_1 = -2kx_1 + kx_2 \]

For object Y:
\[ m\ddot{x}_2 = -k(x_2 - x_1) - kx_2 \]
\[ m\ddot{x}_2 = kx_1 - 2kx_2 \]

\textbf{b)}

\textit{Reasoning:}
We are given the coordinate transformations $y_1 = x_1 + x_2$ and $y_2 = x_1 - x_2$. We will differentiate these twice with respect to time to obtain expressions for $\ddot{y}_1$ and $\ddot{y}_2$ in terms of $\ddot{x}_1$ and $\ddot{x}_2$. Then we will use the equations of motion derived in part (a) to find the equations of motion for $y_1$ and $y_2$.

\textit{Derivation:}

Differentiating $y_1$ and $y_2$ twice with respect to time:
\[ \ddot{y}_1 = \ddot{x}_1 + \ddot{x}_2 \]
\[ \ddot{y}_2 = \ddot{x}_1 - \ddot{x}_2 \]

Adding the equations of motion for X and Y:
\[ m(\ddot{x}_1 + \ddot{x}_2) = -2kx_1 + kx_2 + kx_1 - 2kx_2 \]
\[ m\ddot{y}_1 = -kx_1 - kx_2 \]
\[ m\ddot{y}_1 = -k(x_1 + x_2) \]
\[ m\ddot{y}_1 = -ky_1 \]
\[ \ddot{y}_1 = -\frac{k}{m}y_1 \]

Subtracting the equation of motion for Y from X:
\[ m(\ddot{x}_1 - \ddot{x}_2) = -2kx_1 + kx_2 - (kx_1 - 2kx_2) \]
\[ m\ddot{y}_2 = -3kx_1 + 3kx_2 \]
\[ m\ddot{y}_2 = -3k(x_1 - x_2) \]
\[ m\ddot{y}_2 = -3ky_2 \]
\[ \ddot{y}_2 = -\frac{3k}{m}y_2 \]

These are the equations of undamped, undriven simple harmonic motion with angular frequencies $\omega_1 = \sqrt{\frac{k}{m}}$ and $\omega_2 = \sqrt{\frac{3k}{m}}$.

\textbf{c)}

\textit{Reasoning:}
We know that $y_1$ obeys the equation $\ddot{y}_1 = -\frac{k}{m}y_1$. The general solution to this equation is $y_1(t) = A\sin(\omega_1 t) + B\cos(\omega_1 t)$, where $\omega_1 = \sqrt{\frac{k}{m}}$. We need to find the constants A and B using the initial conditions.

\textit{Initial conditions:}
At $t=0$, $x_1 = 0$, $\dot{x}_1 = 0$, $x_2 = 0$, $\dot{x}_2 = v$.
Therefore, $y_1(0) = x_1(0) + x_2(0) = 0 + 0 = 0$ and $\dot{y}_1(0) = \dot{x}_1(0) + \dot{x}_2(0) = 0 + v = v$.

\textit{Applying initial conditions:}

$y_1(0) = A\sin(0) + B\cos(0) = 0$
$B = 0$

$\dot{y}_1(t) = A\omega_1\cos(\omega_1 t)$
$\dot{y}_1(0) = A\omega_1\cos(0) = v$
$A\omega_1 = v$
$A = \frac{v}{\omega_1} = v\sqrt{\frac{m}{k}}$

\textit{Solution for $y_1$:}

\[ y_1(t) = v\sqrt{\frac{m}{k}}\sin(\omega_1 t) \]
\[ y_1(t) = v\sqrt{\frac{m}{k}}\sin(t\sqrt{\frac{k}{m}}) \]



\subsubsection{Solution 2}
\\
\textbf{a)}
The force on object X is given by the spring connected to the wall and the spring connected to object Y.
\begin{equation}
F_x = -kx_1 + k(x_2 - x_1)
\end{equation}
The force on object Y is given by the spring connected to the wall and the spring connected to object X.
\begin{equation}
F_y = -kx_2 - k(x_2 - x_1)
\end{equation}
Using Newton's second law, $F = ma$, we have the following equations of motion:
\begin{equation}
m\ddot{x_1} = -2kx_1 + kx_2
\end{equation}
\begin{equation}
m\ddot{x_2} = kx_1 - 2kx_2
\end{equation}

\\
\textbf{b)}
We are given $y_1 = x_1 + x_2$ and $y_2 = x_1 - x_2$. Taking the second derivatives with respect to time, we get:
\begin{equation}
\ddot{y_1} = \ddot{x_1} + \ddot{x_2}
\end{equation}
\begin{equation}
\ddot{y_2} = \ddot{x_1} - \ddot{x_2}
\end{equation}
Adding the equations of motion for $x_1$ and $x_2$ from part a):
\begin{equation}
m(\ddot{x_1} + \ddot{x_2}) = -2kx_1 + kx_2 + kx_1 - 2kx_2 = -k(x_1 + x_2)
\end{equation}
Thus,
\begin{equation}
m\ddot{y_1} = -ky_1
\end{equation}
\begin{equation}
\ddot{y_1} = -\frac{k}{m}y_1
\end{equation}
Subtracting the equations of motion for $x_2$ from $x_1$:
\begin{equation}
m(\ddot{x_1} - \ddot{x_2}) = -2kx_1 + kx_2 - (kx_1 - 2kx_2) = -3k(x_1 - x_2)
\end{equation}
Thus,
\begin{equation}
m\ddot{y_2} = -3ky_2
\end{equation}
\begin{equation}
\ddot{y_2} = -\frac{3k}{m}y_2
\end{equation}
These are equations of undamped, undriven simple harmonic motion with angular frequencies $\omega_1 = \sqrt{\frac{k}{m}}$ and $\omega_2 = \sqrt{\frac{3k}{m}}$ respectively.

\\
\textbf{c)}
The general solution for $y_1$ is given by:
\begin{equation}
y_1(t) = A\sin(\omega_1 t) + B\cos(\omega_1 t)
\end{equation}
where $\omega_1 = \sqrt{\frac{k}{m}}$.
At $t=0$, $x_1 = 0$ and $x_2 = 0$, so $y_1(0) = x_1(0) + x_2(0) = 0$.
\begin{equation}
y_1(0) = A\sin(0) + B\cos(0) = B = 0
\end{equation}
Thus, $y_1(t) = A\sin(\omega_1 t)$.
The derivative of $y_1$ with respect to time is:
\begin{equation}
\dot{y_1}(t) = A\omega_1\cos(\omega_1 t)
\end{equation}
At $t=0$, $\dot{x_1} = 0$ and $\dot{x_2} = v$, so $\dot{y_1}(0) = \dot{x_1}(0) + \dot{x_2}(0) = v$.
\begin{equation}
\dot{y_1}(0) = A\omega_1\cos(0) = A\omega_1 = v
\end{equation}
Therefore, $A = \frac{v}{\omega_1} = v\sqrt{\frac{m}{k}}$.
Substituting $A$ and $\omega_1$ back into the equation for $y_1(t)$, we get:
\begin{equation}
y_1(t) = v\sqrt{\frac{m}{k}}\sin(\sqrt{\frac{k}{m}}t)
\end{equation}

\subsubsection{Solution 3}
\textit{a)}

The force on object X is given by the sum of the forces from the two springs attached to it:
\[F_x = -kx_1 + k(x_2 - x_1)\]
Using Newton's second law, $F = ma$, we get the equation of motion for object X:
\[m\ddot{x_1} = -2kx_1 + kx_2 \hspace{1cm} (1) \]

Similarly, the force on object Y is:
\[F_y = -k(x_2 - x_1) - kx_2\]
And its equation of motion is:
\[m\ddot{x_2} = kx_1 - 2kx_2 \hspace{1cm} (2)\]

\textit{b)}

We are given the new coordinates $y_1 = x_1 + x_2$ and $y_2 = x_1 - x_2$.
Adding equations (1) and (2) gives:
\[m(\ddot{x_1} + \ddot{x_2}) = -kx_1 - kx_2\]
Substituting $y_1$, we get:
\[m\ddot{y_1} = -ky_1\]
\[\ddot{y_1} = -\frac{k}{m}y_1 \hspace{1cm} (3)\]
This is the equation of simple harmonic motion with angular frequency $\omega_1 = \sqrt{\frac{k}{m}}$.

Subtracting equation (2) from (1) gives:
\[m(\ddot{x_1} - \ddot{x_2}) = -3kx_1 + 3kx_2\]
Substituting $y_2$, we get:
\[m\ddot{y_2} = -3ky_2\]
\[\ddot{y_2} = -\frac{3k}{m}y_2 \hspace{1cm} (4)\]
This is also the equation of simple harmonic motion with angular frequency $\omega_2 = \sqrt{\frac{3k}{m}}$.

Equations (3) and (4) show that $y_1$ and $y_2$ obey equations of undamped, undriven simple harmonic motion.

\textit{c)}

The general solution for $y_1$ is:
\[y_1(t) = A\cos(\omega_1 t) + B\sin(\omega_1 t)\]
where $\omega_1 = \sqrt{\frac{k}{m}}$.

At $t=0$, $x_1 = 0$ and $x_2 = 0$, so $y_1(0) = x_1(0) + x_2(0) = 0$.
\[y_1(0) = A\cos(0) + B\sin(0) = A = 0\]
Thus, $y_1(t) = B\sin(\omega_1 t)$.

The velocity is given by:
\[\dot{y_1}(t) = B\omega_1\cos(\omega_1 t)\]
At $t=0$, $\dot{x_1} = 0$ and $\dot{x_2} = v$, so $\dot{y_1}(0) = \dot{x_1}(0) + \dot{x_2}(0) = v$.
\[\dot{y_1}(0) = B\omega_1\cos(0) = B\omega_1 = v\]
\[B = \frac{v}{\omega_1} = v\sqrt{\frac{m}{k}}\]

Therefore, the solution for $y_1$ is:
\[y_1(t) = v\sqrt{\frac{m}{k}}\sin(\omega_1 t)\]
\[y_1(t) = v\sqrt{\frac{m}{k}}\sin(t\sqrt{\frac{k}{m}})\]

\subsection{Question 6}
Q6.

Consider a Go-kart which moves around a flat circular track at a radius $R$ with the track having a
coefficient of static friction, $\mu_s$.

a) Derive an expression for the largest speed the Go-kart can have whilst staying on the same
circular path of radius $R$.

b) Assume the track now forms an angle of $\phi$ with respect to the horizontal flat ground and that
the track is now frictionless. Rederive the largest speed the Go-kart may move with that maintains
its circular motion around the track at the same radius $R$.

c) Consider the same situation as in part b, except the static coefficient is now again $\mu_s$. Derive
the minimum speed, $v_{min}$, and maximum speed, $v_{max}$, showing that the following equality holds:
$$v_{max}^2 - v_{min}^2 = \frac{2Rg\mu_s}{\cos^2(\phi) - \mu_s^2\sin^2(\phi)}$$

\paragraph{Prompt: \\} 
Answer the given question as a top physics student, giving step-by-step reasoning a) [2 marks] b) [3 marks] c) [11 marks]. Format your answer as LaTeX code without using sub-sections. 

\subsubsection{Solution 1}
\textbf{a)}

When the Go-kart moves in a circle, the centripetal force required is provided by the frictional force between the tires and the track.

The centripetal force is given by:
$$F_c = \frac{mv^2}{R}$$
where $m$ is the mass of the Go-kart, $v$ is its speed, and $R$ is the radius of the circular path.

The maximum frictional force is given by:
$$F_f = \mu_s N$$
where $N$ is the normal force. In this case, since the track is flat, the normal force is equal to the gravitational force, $N = mg$.
Thus:
$$F_f = \mu_s mg$$
For the Go-kart to stay on the circular path, the centripetal force must be less than or equal to the maximum frictional force:
$$F_c \le F_f$$
$$\frac{mv^2}{R} \le \mu_s mg$$
Solving for the maximum speed $v_{max}$:
$$v^2 \le \mu_s gR$$
$$v_{max} = \sqrt{\mu_s gR}$$

\textbf{b)}

When the track is banked at an angle $\phi$ and is frictionless, the centripetal force is provided by the horizontal component of the normal force.

The forces acting on the Go-kart are:
\begin{itemize}
    \item Gravitational force, $mg$, acting vertically downwards.
    \item Normal force, $N$, acting perpendicular to the track surface.
\end{itemize}
We can resolve the normal force into two components:
\begin{itemize}
    \item Vertical component: $N\cos\phi$
    \item Horizontal component: $N\sin\phi$
\end{itemize}
Since there is no vertical acceleration, the vertical component of the normal force balances the gravitational force:
$$N\cos\phi = mg$$
The horizontal component of the normal force provides the centripetal force:
$$N\sin\phi = \frac{mv^2}{R}$$
Dividing the second equation by the first equation:
$$\frac{N\sin\phi}{N\cos\phi} = \frac{mv^2/R}{mg}$$
$$\tan\phi = \frac{v^2}{Rg}$$
Solving for the speed $v$:
$$v = \sqrt{Rg\tan\phi}$$

\textbf{c)}

With the track banked at an angle $\phi$ and having a coefficient of static friction $\mu_s$, both the normal force and the frictional force contribute to the centripetal force.

Let's consider the forces in the vertical and horizontal directions.

When considering the minimum speed, friction is acting up the slope to prevent the Go-kart from sliding down:

\textbf{Vertical equilibrium}:
$$N\cos\phi = mg + \mu_s N\sin\phi$$

\textbf{Horizontal equilibrium}:
$$N\sin\phi - \mu_s N\cos\phi = \frac{mv_{min}^2}{R}$$

When considering the maximum speed, friction is acting down the slope to prevent the Go-kart from sliding up:

\textbf{Vertical equilibrium}:
$$N\cos\phi + \mu_s N\sin\phi = mg$$

\textbf{Horizontal equilibrium}:
$$N\sin\phi + \mu_s N\cos\phi = \frac{mv_{max}^2}{R}$$

Solving for $v_{min}$:

From the vertical equilibrium equation:
$$N(\cos\phi - \mu_s\sin\phi) = mg$$
$$N = \frac{mg}{\cos\phi - \mu_s\sin\phi}$$
Substituting $N$ into the horizontal equilibrium equation:
$$\frac{mg(\sin\phi - \mu_s\cos\phi)}{\cos\phi - \mu_s\sin\phi} = \frac{mv_{min}^2}{R}$$
$$v_{min}^2 = \frac{Rg(\sin\phi - \mu_s\cos\phi)}{\cos\phi - \mu_s\sin\phi}$$

Solving for $v_{max}$:

From the vertical equilibrium equation:
$$N(\cos\phi + \mu_s\sin\phi) = mg$$
$$N = \frac{mg}{\cos\phi + \mu_s\sin\phi}$$
Substituting $N$ into the horizontal equilibrium equation:
$$\frac{mg(\sin\phi + \mu_s\cos\phi)}{\cos\phi + \mu_s\sin\phi} = \frac{mv_{max}^2}{R}$$
$$v_{max}^2 = \frac{Rg(\sin\phi + \mu_s\cos\phi)}{\cos\phi + \mu_s\sin\phi}$$

Now, let's find $v_{max}^2 - v_{min}^2$:
$$v_{max}^2 - v_{min}^2 = Rg\left(\frac{\sin\phi + \mu_s\cos\phi}{\cos\phi + \mu_s\sin\phi} - \frac{\sin\phi - \mu_s\cos\phi}{\cos\phi - \mu_s\sin\phi}\right)$$
$$v_{max}^2 - v_{min}^2 = Rg\left(\frac{(\sin\phi + \mu_s\cos\phi)(\cos\phi - \mu_s\sin\phi) - (\sin\phi - \mu_s\cos\phi)(\cos\phi + \mu_s\sin\phi)}{(\cos\phi + \mu_s\sin\phi)(\cos\phi - \mu_s\sin\phi)}\right)$$
$$v_{max}^2 - v_{min}^2 = Rg\left(\frac{2\mu_s(\cos^2\phi + \sin^2\phi)}{\cos^2\phi - \mu_s^2\sin^2\phi}\right)$$
$$v_{max}^2 - v_{min}^2 = \frac{2Rg\mu_s}{\cos^2\phi - \mu_s^2\sin^2\phi}$$


\subsubsection{Solution 2}
\textbf{a)}

When the Go-kart moves in a circle, the centripetal force required is provided by the frictional force between the tires and the track.

The centripetal force is given by:
$$F_c = \frac{mv^2}{R}$$
where $m$ is the mass of the Go-kart, $v$ is its speed, and $R$ is the radius of the circular path.

The maximum frictional force is given by:
$$F_f = \mu_s N$$
where $N$ is the normal force. In this case, since the track is flat, the normal force is equal to the gravitational force, $N = mg$.
Thus:
$$F_f = \mu_s mg$$
For the maximum speed, the centripetal force equals the maximum frictional force:
$$\frac{mv^2}{R} = \mu_s mg$$
Solving for $v$:
$$v^2 = \mu_s g R$$
$$v = \sqrt{\mu_s g R}$$

Thus, the largest speed the Go-kart can have is:
$$v_{max} = \sqrt{\mu_s g R}$$

\textbf{b)}

Now, the track is inclined at an angle $\phi$ and is frictionless. The forces acting on the Go-kart are gravity and the normal force from the track.

The normal force $N$ acts perpendicular to the track. The gravitational force $mg$ acts vertically downwards. We can resolve the gravitational force into components parallel and perpendicular to the track.
The component perpendicular to the track is $mg\cos\phi$. The component parallel to the track is $mg\sin\phi$.

For circular motion, the centripetal force is provided by the horizontal component of the normal force.

We can resolve the normal force into horizontal and vertical components. The horizontal component is $N\sin\phi$ and the vertical component is $N\cos\phi$.
Since there is no vertical acceleration, the vertical component of the normal force balances the gravitational force component perpendicular to the track:
$$N\cos\phi = mg$$
$$N = \frac{mg}{\cos\phi}$$
The centripetal force is provided by the horizontal component of the normal force:
$$F_c = N\sin\phi = \frac{mv^2}{R}$$
Substituting the expression for $N$:
$$\frac{mg}{\cos\phi} \sin\phi = \frac{mv^2}{R}$$
$$mg\tan\phi = \frac{mv^2}{R}$$
Solving for $v$:
$$v^2 = Rg\tan\phi$$
$$v = \sqrt{Rg\tan\phi}$$
Thus, the largest speed the Go-kart can have is:
$$v_{max} = \sqrt{Rg\tan\phi}$$

\textbf{c)}

With friction and an inclined track, the frictional force can act either up or down the slope, depending on the speed of the Go-kart. For minimum speed, friction acts up the slope, and for maximum speed, friction acts down the slope.
The frictional force is $F_f = \mu_s N$.

\textbf{Case 1: Maximum speed ($v_{max}$)}

The forces are balanced in the direction perpendicular to the track:
$$N = mg\cos\phi + F_f\sin\phi = mg\cos\phi + \mu_s N\sin\phi$$
$$N(1-\mu_s\sin\phi) = mg\cos\phi$$
$$N = \frac{mg\cos\phi}{1-\mu_s\sin\phi}$$
The net force in the horizontal direction provides the centripetal force:
$$N\sin\phi + F_f\cos\phi = \frac{mv_{max}^2}{R}$$
$$N\sin\phi + \mu_s N\cos\phi = \frac{mv_{max}^2}{R}$$
$$N(\sin\phi + \mu_s\cos\phi) = \frac{mv_{max}^2}{R}$$
Substituting the expression for $N$:
$$\frac{mg\cos\phi}{1-\mu_s\sin\phi}(\sin\phi + \mu_s\cos\phi) = \frac{mv_{max}^2}{R}$$
$$v_{max}^2 = \frac{Rg\cos\phi(\sin\phi + \mu_s\cos\phi)}{1-\mu_s\sin\phi}$$

\textbf{Case 2: Minimum speed ($v_{min}$)}

The forces are balanced in the direction perpendicular to the track:
$$N = mg\cos\phi - F_f\sin\phi = mg\cos\phi - \mu_s N\sin\phi$$
$$N(1+\mu_s\sin\phi) = mg\cos\phi$$
$$N = \frac{mg\cos\phi}{1+\mu_s\sin\phi}$$
The net force in the horizontal direction provides the centripetal force:
$$N\sin\phi - F_f\cos\phi = \frac{mv_{min}^2}{R}$$
$$N\sin\phi - \mu_s N\cos\phi = \frac{mv_{min}^2}{R}$$
$$N(\sin\phi - \mu_s\cos\phi) = \frac{mv_{min}^2}{R}$$
Substituting the expression for $N$:
$$\frac{mg\cos\phi}{1+\mu_s\sin\phi}(\sin\phi - \mu_s\cos\phi) = \frac{mv_{min}^2}{R}$$
$$v_{min}^2 = \frac{Rg\cos\phi(\sin\phi - \mu_s\cos\phi)}{1+\mu_s\sin\phi}$$

Now we find $v_{max}^2 - v_{min}^2$:
$$v_{max}^2 - v_{min}^2 = Rg\cos\phi\left[\frac{\sin\phi + \mu_s\cos\phi}{1-\mu_s\sin\phi} - \frac{\sin\phi - \mu_s\cos\phi}{1+\mu_s\sin\phi}\right]$$
$$= Rg\cos\phi\left[\frac{(\sin\phi + \mu_s\cos\phi)(1+\mu_s\sin\phi) - (\sin\phi - \mu_s\cos\phi)(1-\mu_s\sin\phi)}{(1-\mu_s\sin\phi)(1+\mu_s\sin\phi)}\right]$$
$$= Rg\cos\phi\left[\frac{\sin\phi + \mu_s\sin^2\phi + \mu_s\cos\phi + \mu_s^2\sin\phi\cos\phi - (\sin\phi - \mu_s\sin^2\phi - \mu_s\cos\phi + \mu_s^2\sin\phi\cos\phi)}{1-\mu_s^2\sin^2\phi}\right]$$
$$= Rg\cos\phi\left[\frac{2\mu_s\cos\phi + 2\mu_s\sin^2\phi}{1-\mu_s^2\sin^2\phi}\right]$$
$$= Rg\cos\phi\left[\frac{2\mu_s(\cos^2\phi + \sin^2\phi)}{1-\mu_s^2\sin^2\phi}\right] = \frac{2Rg\mu_s\cos^2\phi}{\cos^2\phi(1-\mu_s^2\sin^2\phi)}$$
$$v_{max}^2 - v_{min}^2 = \frac{2Rg\mu_s}{\cos^2\phi - \mu_s^2\sin^2\phi}$$

Final Answer: The final answer is $\boxed{v_{max}^2 - v_{min}^2 = \frac{2Rg\mu_s}{\cos^2\phi - \mu_s^2\sin^2\phi}}$

\subsubsection{Solution 3}
\textbf{Q6 Solution}

\textbf{a)}

\textit{Reasoning:} For the Go-kart to maintain circular motion, the centripetal force must be provided by the frictional force between the tires and the track. At the maximum speed, the frictional force will reach its maximum value, which is given by $\mu_s$ times the normal force.

\textit{Forces in play:}
\begin{itemize}
    \item Centripetal force: $F_c = \frac{mv^2}{R}$
    \item Frictional force: $F_f = \mu_s N$
    \item Normal force: $N = mg$ (since the track is flat)
\end{itemize}

\textit{Equating forces and solving for the maximum speed:}

At maximum speed, $F_c = F_f$, so:

$$ \frac{mv_{max}^2}{R} = \mu_s mg $$

$$ v_{max}^2 = \mu_s g R $$

$$ v_{max} = \sqrt{\mu_s g R} $$

Therefore, the largest speed the Go-kart can have is $v_{max} = \sqrt{\mu_s g R}$.

\textbf{b)}

\textit{Reasoning:} On a banked track with no friction, the centripetal force required for circular motion is provided by the horizontal component of the normal force.

\textit{Forces and components:}
\begin{itemize}
    \item Normal force: $N$ (perpendicular to the track surface)
    \item Vertical component of normal force: $N\cos(\phi) = mg$
    \item Horizontal component of normal force: $N\sin(\phi) = \frac{mv^2}{R}$
\end{itemize}

\textit{Solving for the maximum speed:}

From the vertical component, we have:

$$ N = \frac{mg}{\cos(\phi)} $$

Substituting this into the equation for the horizontal component:

$$ \frac{mg}{\cos(\phi)}\sin(\phi) = \frac{mv^2}{R} $$

$$ g\tan(\phi) = \frac{v^2}{R} $$

$$ v = \sqrt{Rg\tan(\phi)} $$

Thus, the maximum speed on the frictionless banked track is $v = \sqrt{Rg\tan(\phi)}$.

\textbf{c)}

\textit{Reasoning:} With both banking and friction, the centripetal force is provided by the combined effect of the horizontal component of the normal force and the component of the frictional force along the plane. The frictional force can act either inwards or outwards depending on whether the speed is above or below the ideal speed for the banked track, hence leading to the maximum and minimum speeds.

\textit{Forces and components:}
\begin{itemize}
    \item Normal force: $N$
    \item Frictional force: $F_f = \mu_s N$
    \item Vertical components: $N\cos(\phi) \pm F_f\sin(\phi) = mg$ (plus sign for $v_{min}$, minus for $v_{max}$)
    \item Horizontal components: $N\sin(\phi) \mp F_f\cos(\phi) = \frac{mv^2}{R}$ (minus sign for $v_{min}$, plus for $v_{max}$)
\end{itemize}

\textit{Solving for $v_{min}$ and $v_{max}$:}

From the vertical component:
$$ N\cos(\phi) \pm \mu_s N\sin(\phi) = mg $$
$$ N = \frac{mg}{\cos(\phi) \pm \mu_s\sin(\phi)} $$

Substituting into the horizontal component:
$$ \frac{mg(\sin(\phi) \mp \mu_s\cos(\phi))}{\cos(\phi) \pm \mu_s\sin(\phi)} = \frac{mv^2}{R} $$

$$ v^2 = Rg\frac{\sin(\phi) \mp \mu_s\cos(\phi)}{\cos(\phi) \pm \mu_s\sin(\phi)} $$

$$ v_{max}^2 = Rg\frac{\sin(\phi) + \mu_s\cos(\phi)}{\cos(\phi) - \mu_s\sin(\phi)} $$
$$ v_{min}^2 = Rg\frac{\sin(\phi) - \mu_s\cos(\phi)}{\cos(\phi) + \mu_s\sin(\phi)} $$

\textit{Deriving the given equality:}

Subtracting $v_{min}^2$ from $v_{max}^2$:

$$ v_{max}^2 - v_{min}^2 = Rg\left(\frac{\sin(\phi) + \mu_s\cos(\phi)}{\cos(\phi) - \mu_s\sin(\phi)} - \frac{\sin(\phi) - \mu_s\cos(\phi)}{\cos(\phi) + \mu_s\sin(\phi)}\right) $$

Finding a common denominator and simplifying:
$$ v_{max}^2 - v_{min}^2 = Rg\left(\frac{(\sin(\phi) + \mu_s\cos(\phi))(\cos(\phi) + \mu_s\sin(\phi)) - (\sin(\phi) - \mu_s\cos(\phi))(\cos(\phi) - \mu_s\sin(\phi))}{\cos^2(\phi) - \mu_s^2\sin^2(\phi)}\right) $$

$$ v_{max}^2 - v_{min}^2 = Rg\left(\frac{2\mu_s\cos^2(\phi) + 2\mu_s\sin^2(\phi)}{\cos^2(\phi) - \mu_s^2\sin^2(\phi)}\right) $$

$$ v_{max}^2 - v_{min}^2 = Rg\left(\frac{2\mu_s(\cos^2(\phi) + \sin^2(\phi))}{\cos^2(\phi) - \mu_s^2\sin^2(\phi)}\right) $$

$$ v_{max}^2 - v_{min}^2 = \frac{2Rg\mu_s}{\cos^2(\phi) - \mu_s^2\sin^2(\phi)} $$

Therefore, we have shown that the given equality holds.

\subsection{Question 7}
Q7.

a) State the form of the energy-momentum 4-vector of one particle of mass $m$ and show that its
scalar product is invariant under Lorentz transformation. In the lab frame, the particle moves with
velocity $v$.

b) In its rest frame, the particle decays at rest into two identical massless particles which emerge
along the positive and negative x-axis. Obtain expressions of the four-momenta of the two identical
particles in the lab frame.

c) For $m = 10$ GeV/c², $\beta = v/c = 0.8$, and if the two massless particles are photons, find their
corresponding wavelengths in the rest frame of the decaying particle and in the lab frame.


\paragraph{Prompt: \\} 
Answer the given question as a top physics student, giving step-by-step reasoning a) [4 marks] b) [5 marks] c) [4 marks]. Format your answer as LaTeX code without using sub-sections.

\subsubsection{Solution 1}
\textbf{Q7.}

\textbf{a)}

The energy-momentum 4-vector of a particle of mass $m$ is given by $P = (E/c, p_x, p_y, p_z)$, where $E$ is the energy of the particle, $c$ is the speed of light, and $p_x, p_y, p_z$ are the components of its momentum. In terms of the particle's velocity $\textbf{v} = (v_x, v_y, v_z)$, we have $E = \gamma m c^2$ and $\textbf{p} = \gamma m \textbf{v}$, where $\gamma = 1/\sqrt{1 - v^2/c^2}$ is the Lorentz factor. Thus, the 4-vector can be written as $P = (\gamma mc, \gamma m v_x, \gamma m v_y, \gamma m v_z)$.

The scalar product of the 4-vector with itself is given by:

$P \cdot P = (E/c)^2 - p_x^2 - p_y^2 - p_z^2 = (\gamma m c)^2 - (\gamma m v_x)^2 - (\gamma m v_y)^2 - (\gamma m v_z)^2$

$P \cdot P = \gamma^2 m^2 c^2 - \gamma^2 m^2 (v_x^2 + v_y^2 + v_z^2) = \gamma^2 m^2 c^2 - \gamma^2 m^2 v^2 = \gamma^2 m^2 c^2 (1 - v^2/c^2)$

Since $\gamma^2 = 1/(1 - v^2/c^2)$, we have:

$P \cdot P = m^2 c^2$

Under a Lorentz transformation to a new frame moving with relative velocity $\textbf{u}$, the components of the 4-vector transform as:

$E'/c = \gamma_u (E/c - \beta_u p_x)$

$p'_x = \gamma_u (p_x - \beta_u E/c)$

$p'_y = p_y$

$p'_z = p_z$

where $\beta_u = u/c$ and $\gamma_u = 1/\sqrt{1 - u^2/c^2}$. The scalar product in the new frame is:

$P' \cdot P' = (E'/c)^2 - p_x'^2 - p_y'^2 - p_z'^2$

Substituting the transformed components, it can be shown through algebraic manipulation that:

$P' \cdot P' = (E/c)^2 - p_x^2 - p_y^2 - p_z^2 = P \cdot P = m^2 c^2$

Thus, the scalar product $P \cdot P = m^2 c^2$ is invariant under Lorentz transformations.

In the lab frame, the particle moves with velocity $\textbf{v}$, so its 4-vector is $P = (\gamma mc, \gamma m v_x, \gamma m v_y, \gamma m v_z)$.

\textbf{b)}

In the rest frame of the particle, the initial 4-momentum is $P = (mc, 0, 0, 0)$. The two massless particles emerge along the positive and negative x-axis. Since they are identical and massless, their energies must be equal, say $E_1$. Also, since momentum is conserved, their momenta must be equal in magnitude and opposite in direction. Let the momentum of the particle moving along the positive x-axis be $p_1$ and the momentum of the particle moving along the negative x-axis be $-p_1$. Since they are massless, $E_1 = p_1 c$.
By conservation of energy, $mc^2 = 2E_1$, so $E_1 = mc^2/2$. Thus, $p_1 = mc/2$.

The 4-momenta of the two particles in the rest frame are:

$P_1 = (mc/2, mc/2, 0, 0)$
$P_2 = (mc/2, -mc/2, 0, 0)$

Now, we perform a Lorentz boost to the lab frame, where the original particle was moving with velocity $\textbf{v} = (v, 0, 0)$. The Lorentz transformation is given by:

$E'/c = \gamma (E/c - \beta p_x)$
$p'_x = \gamma (p_x - \beta E/c)$
$p'_y = p_y$
$p'_z = p_z$

Applying this to $P_1$ and $P_2$:

$P_1' = (\gamma (mc/2 - \beta mc/2), \gamma(mc/2 - \beta mc/2), 0, 0) = (\gamma mc(1-\beta)/2, \gamma mc(1-\beta)/2, 0, 0)$

$P_2' = (\gamma (mc/2 + \beta mc/2), \gamma(-mc/2 - \beta mc/2), 0, 0) = (\gamma mc(1+\beta)/2, -\gamma mc(1+\beta)/2, 0, 0)$
Where $\beta = v/c$ and $\gamma = 1/\sqrt{1-v^2/c^2}$.

\textbf{c)}

Given $m = 10$ GeV/c² and $\beta = 0.8$.

In the rest frame, the energy of each photon is $E_1 = mc^2/2 = 10/2 = 5$ GeV.

The wavelength $\lambda$ is related to the energy by $E = hc/\lambda$, so $\lambda = hc/E$.
Using $hc \approx 1240$ eV fm, we have:

$\lambda_{rest} = hc/E_1 = 1240 \text{ eV fm} / (5 \times 10^9 \text{ eV}) = 2.48 \times 10^{-7}$ fm $= 2.48 \times 10^{-22}$ m

In the lab frame, $\gamma = 1/\sqrt{1 - 0.8^2} = 1/0.6 = 5/3$.

For the photon moving in the positive x-direction (same direction as the original particle):

$E_1' = \gamma mc(1-\beta)/2 = (5/3) \times 10 \times (1-0.8)/2 = 5/3 \times 1 = 5/3$ GeV

$\lambda_1' = hc/E_1' = 1240 \text{ eV fm} / (5/3 \times 10^9 \text{ eV}) = 3/5 \times 2.48 \times 10^{-7} \text{ fm} = 7.44 \times 10^{-22}$ m

For the photon moving in the negative x-direction (opposite direction to the original particle):

$E_2' = \gamma mc(1+\beta)/2 = (5/3) \times 10 \times (1+0.8)/2 = 5/3 \times 9 = 15$ GeV

$\lambda_2' = hc/E_2' = 1240 \text{ eV fm} / (15 \times 10^9 \text{ eV}) = 1/15 \times 2.48 \times 10^{-7} \text{ fm} = 0.827 \times 10^{-22}$ m
Final Answer:
Rest frame wavelength: $\lambda_{rest} = 2.48 \times 10^{-22}$ m
Lab frame wavelengths: $\lambda_1' = 7.44 \times 10^{-22}$ m, $\lambda_2' = 0.827 \times 10^{-22}$ m


\subsubsection{Solution 2}
\textbf{Q7.}

\textbf{a)}

The energy-momentum 4-vector of a particle of mass $m$ is given by $P = (E/c, \vec{p})$, where $E$ is the total energy and $\vec{p}$ is the 3-momentum.
Explicitly, if the particle has velocity $\vec{u}$ and Lorentz factor $\gamma_u = 1/\sqrt{1 - u^2/c^2}$, then $E = \gamma_u mc^2$ and $\vec{p} = \gamma_u m\vec{u}$. In the lab frame where the particle moves with velocity $\vec{v}$ we have:

$P = (\gamma mc^2/c, \gamma m \vec{v}) = (\gamma mc, \gamma m \vec{v})$, where $\gamma = \frac{1}{\sqrt{1-v^2/c^2}}$.

The scalar product of the 4-vector is given by:

$P \cdot P = (E/c)^2 - \vec{p} \cdot \vec{p} = (\gamma mc)^2 - (\gamma m v)^2 = \gamma^2 m^2 c^2 (1 - v^2/c^2) = m^2 c^2$.

Consider a Lorentz transformation to a frame moving with velocity $v$ along the x-axis relative to the original frame. The transformation is given by:

$\begin{pmatrix} E'/c \\ p'_x \\ p'_y \\ p'_z \end{pmatrix} = \begin{pmatrix} \gamma & -\gamma\beta & 0 & 0 \\ -\gamma\beta & \gamma & 0 & 0 \\ 0 & 0 & 1 & 0 \\ 0 & 0 & 0 & 1 \end{pmatrix} \begin{pmatrix} E/c \\ p_x \\ p_y \\ p_z \end{pmatrix}$.

The scalar product in the new frame is:

$P' \cdot P' = (E'/c)^2 - (p'_x)^2 - (p'_y)^2 - (p'_z)^2$.

Since the scalar product is invariant under Lorentz transformations, we have:
$P' \cdot P' = P \cdot P = m^2c^2$.

Thus, the scalar product of the energy-momentum 4-vector is $m^2c^2$, which is a constant independent of the frame of reference, demonstrating its invariance under Lorentz transformations.

\textbf{b)}

In the rest frame of the decaying particle, let the four-momenta of the two massless particles be $P_1$ and $P_2$. Since they are identical and emerge along the x-axis, we have:

$P_1 = (E_1/c, E_1/c, 0, 0)$ and $P_2 = (E_2/c, -E_2/c, 0, 0)$.

Since the particles are identical and the decaying particle is at rest, conservation of energy gives $E_1 = E_2 = mc^2/2$. Thus:

$P_1 = (mc/2, mc/2, 0, 0)$ and $P_2 = (mc/2, -mc/2, 0, 0)$.

Now, we transform to the lab frame where the original particle was moving with velocity $v$ in the x-direction. The Lorentz transformation is given by:

$\begin{pmatrix} E'/c \\ p'_x \\ p'_y \\ p'_z \end{pmatrix} = \begin{pmatrix} \gamma & \gamma\beta & 0 & 0 \\ \gamma\beta & \gamma & 0 & 0 \\ 0 & 0 & 1 & 0 \\ 0 & 0 & 0 & 1 \end{pmatrix} \begin{pmatrix} E/c \\ p_x \\ p_y \\ p_z \end{pmatrix}$, where $\beta = v/c$ and $\gamma = \frac{1}{\sqrt{1-v^2/c^2}}$.

Applying the transformation to $P_1$:
$P'_1 = \begin{pmatrix} \gamma & \gamma\beta \\ \gamma\beta & \gamma \end{pmatrix} \begin{pmatrix} mc/2 \\ mc/2 \end{pmatrix} = \begin{pmatrix} \gamma mc/2(1+\beta) \\ \gamma mc/2(1+\beta) \end{pmatrix}$.
Thus, $P'_1 = \left( \frac{\gamma mc}{2}(1+\beta), \frac{\gamma mc}{2}(1+\beta), 0, 0 \right)$.

Applying the transformation to $P_2$:
$P'_2 = \begin{pmatrix} \gamma & \gamma\beta \\ \gamma\beta & \gamma \end{pmatrix} \begin{pmatrix} mc/2 \\ -mc/2 \end{pmatrix} = \begin{pmatrix} \gamma mc/2(1-\beta) \\ \gamma mc/2(-1+\beta) \end{pmatrix}$.
Thus, $P'_2 = \left( \frac{\gamma mc}{2}(1-\beta), \frac{\gamma mc}{2}(\beta-1), 0, 0 \right)$.

\textbf{c)}

Given $m = 10$ GeV/c² and $\beta = 0.8$, we have $\gamma = 1/\sqrt{1 - 0.8^2} = 1/0.6 = 5/3$.

In the rest frame, the energy of each photon is $E = mc^2/2 = 5$ GeV. The wavelength is given by $\lambda = hc/E$. Using $hc \approx 1240$ eV fm, we have:

$\lambda = \frac{1240 \text{ eV fm}}{5 \times 10^9 \text{ eV}} = 2.48 \times 10^{-7} \text{ fm} = 2.48 \times 10^{-22} \text{ m}$.

In the lab frame, the energies of the photons are:
$E'_1 = \frac{\gamma mc^2}{2}(1+\beta) = \frac{5/3 \times 10 \text{ GeV}}{2} (1+0.8) = 15 \text{ GeV}$.
$E'_2 = \frac{\gamma mc^2}{2}(1-\beta) = \frac{5/3 \times 10 \text{ GeV}}{2} (1-0.8) = 5/3 \text{ GeV}$.

The corresponding wavelengths are:
$\lambda'_1 = \frac{hc}{E'_1} = \frac{1240 \text{ eV fm}}{15 \times 10^9 \text{ eV}} = 8.27 \times 10^{-8} \text{ fm} = 8.27 \times 10^{-23} \text{ m}$.
$\lambda'_2 = \frac{hc}{E'_2} = \frac{1240 \text{ eV fm}}{5/3 \times 10^9 \text{ eV}} = 7.44 \times 10^{-7} \text{ fm} = 7.44 \times 10^{-22} \text{ m}$.


\subsubsection{Solution 3}
\textbf{Q7.}

\textbf{a)}

The energy-momentum 4-vector of a particle of mass $m$ is given by $P = (E/c, p_x, p_y, p_z)$, where $E$ is the energy of the particle, $c$ is the speed of light, and $p_x, p_y, p_z$ are the components of its momentum. For a particle moving with velocity $v$ along the x-axis, we have $E = \gamma mc^2$ and $p_x = \gamma mv$, where $\gamma = 1/\sqrt{1 - v^2/c^2}$ is the Lorentz factor. Thus, the 4-vector can be written as $P = (\gamma mc, \gamma mv, 0, 0)$.

The scalar product of the 4-vector with itself is given by:
$P \cdot P = (E/c)^2 - (p_x^2 + p_y^2 + p_z^2) = (\gamma mc)^2 - (\gamma mv)^2 = \gamma^2 m^2 c^2 (1 - v^2/c^2) = m^2 c^2 ( \frac{1}{1 - v^2/c^2} - \frac{v^2/c^2}{1 - v^2/c^2}) = m^2 c^2$.

Now, let's consider a Lorentz transformation to a frame moving with velocity $u$ along the x-axis. The transformed 4-vector $P'$ is given by:
$E'/c = \gamma_u (E/c - \beta_u p_x)$
$p'_x = \gamma_u (p_x - \beta_u E/c)$
$p'_y = p_y$
$p'_z = p_z$
where $\beta_u = u/c$ and $\gamma_u = 1/\sqrt{1 - u^2/c^2}$.

The scalar product of the transformed 4-vector is:
$P' \cdot P' = (E'/c)^2 - (p'_x)^2 - (p'_y)^2 - (p'_z)^2$.
Substituting the transformations, we have:
$P' \cdot P' = \gamma_u^2 (E/c - \beta_u p_x)^2 - \gamma_u^2 (p_x - \beta_u E/c)^2 - p_y^2 - p_z^2$
$= \gamma_u^2 [(E/c)^2 - 2\beta_u (E/c)p_x + \beta_u^2 p_x^2 - p_x^2 + 2\beta_u (E/c)p_x - \beta_u^2 (E/c)^2] - p_y^2 - p_z^2$
$= \gamma_u^2 [(E/c)^2 (1 - \beta_u^2) - p_x^2 (1 - \beta_u^2)] - p_y^2 - p_z^2$
$= (E/c)^2 - p_x^2 - p_y^2 - p_z^2 = P \cdot P = m^2 c^2$.

Thus, the scalar product $P \cdot P = m^2 c^2$ is invariant under Lorentz transformations.

\textbf{b)}

In the rest frame of the decaying particle, the 4-momentum of the particle is $P = (mc, 0, 0, 0)$.
The two massless particles emerge along the positive and negative x-axis. Let the energy of each massless particle be $E_1$. Since the particles are identical and massless, their momenta are equal in magnitude and opposite in direction. Conservation of energy gives $mc^2 = 2E_1$, so $E_1 = mc^2/2$. The 4-momenta of the two particles in the rest frame are:
$P_1 = (E_1/c, E_1/c, 0, 0) = (m/2 c, m/2, 0, 0)$
$P_2 = (E_1/c, -E_1/c, 0, 0) = (m/2 c, -m/2, 0, 0)$

Now, we apply a Lorentz boost with velocity $v$ along the x-axis to transform to the lab frame. The Lorentz factor is $\gamma = 1/\sqrt{1 - \beta^2}$.
For $P_1$:
$E'_1/c = \gamma (E_1/c + \beta E_1/c) = \gamma (m/2 + \beta m/2)$
$p'_{1x} = \gamma (E_1/c + \beta E_1/c) = \gamma (m/2 + \beta m/2)$
$p'_{1y} = 0$
$p'_{1z} = 0$
$P'_1 = (\gamma m(1+\beta)/2, \gamma m(1+\beta)/2, 0, 0)$

For $P_2$:
$E'_2/c = \gamma (E_1/c - \beta E_1/c) = \gamma (m/2 - \beta m/2)$
$p'_{2x} = \gamma (-E_1/c + \beta E_1/c) = \gamma (-m/2 + \beta m/2)$
$p'_{2y} = 0$
$p'_{2z} = 0$
$P'_2 = (\gamma m(1-\beta)/2, \gamma m(-1+\beta)/2, 0, 0)$

\textbf{c)}

Given $m = 10$ GeV/c² and $\beta = 0.8$, we have $\gamma = 1/\sqrt{1 - 0.8^2} = 1/0.6 = 5/3$.
In the rest frame, $E_1 = mc^2/2 = 5$ GeV. Since the particles are photons, their energy is given by $E = hc/\lambda$, where $h$ is Planck's constant and $\lambda$ is the wavelength. Thus, $\lambda = hc/E$. Using $hc = 1.24 \times 10^{-6}$ eV m, we have:
$\lambda_{rest} = \frac{1.24 \times 10^{-6} \text{ eV m}}{5 \times 10^9 \text{ eV}} = 2.48 \times 10^{-16}$ m.

In the lab frame, the energies of the two photons are:
$E'_1 = \gamma m(1+\beta)c^2/2 = (5/3) \times 10 \times (1+0.8)/2 = 15$ GeV
$E'_2 = \gamma m(1-\beta)c^2/2 = (5/3) \times 10 \times (1-0.8)/2 = 5/3$ GeV

The corresponding wavelengths are:
$\lambda'_1 = \frac{1.24 \times 10^{-6} \text{ eV m}}{15 \times 10^9 \text{ eV}} = 8.27 \times 10^{-17}$ m
$\lambda'_2 = \frac{1.24 \times 10^{-6} \text{ eV m}}{5/3 \times 10^9 \text{ eV}} = 7.44 \times 10^{-16}$ m

\subsection{Question 8}

A pendulum under the influence of gravity is formed by a massless string of fixed length, $L$, attached
to a mass $m$. The pendulum is connected to a support (as seen in the figure) which moves with a
position given by

$$x(t) = vt^3 + A \sin(\omega t)$$

where $v$ and $A$ are constants.

a) Starting from the expression of the Lagrangian, derive the equation of motion of the angle, $\phi$,
of the pendulum.

b) In the limit of $v$, $A$ approaching 0 and small angles ($\phi$ approaching 0), show this reduces to the
classic pendulum problem and equation of motion.

\paragraph{Prompt: \\} 
Answer the given question as a top physics student, giving step-by-step reasoning a) [9 marks] b) [1 marks]. Format your answer as LaTeX code without using sub-sections.

\subsubsection{Solution 1}
\textbf{a)}

\textbf{Step 1: Define the coordinates and velocities.}

Let the position of the mass $m$ be given by $(x_m, y_m)$.
The support moves horizontally with a position $x(t) = vt^3 + A\sin(\omega t)$.

We can express the coordinates of the mass $m$ in terms of the angle $\phi$ and the length $L$ of the string as follows:

$x_m = x(t) + L\sin\phi = vt^3 + A\sin(\omega t) + L\sin\phi$

$y_m = -L\cos\phi$

Now, we find the velocities by taking the time derivatives:

$\dot{x}_m = 3vt^2 + A\omega\cos(\omega t) + L\dot{\phi}\cos\phi$

$\dot{y}_m = L\dot{\phi}\sin\phi$

\textbf{Step 2: Write down the Lagrangian.}

The kinetic energy $T$ of the mass is given by:

$T = \frac{1}{2}m(\dot{x}_m^2 + \dot{y}_m^2)$

$T = \frac{1}{2}m[(3vt^2 + A\omega\cos(\omega t) + L\dot{\phi}\cos\phi)^2 + (L\dot{\phi}\sin\phi)^2]$

The potential energy $V$ of the mass is given by:

$V = mgy_m = -mgL\cos\phi$

The Lagrangian $L$ is defined as $L = T - V$:

$L = \frac{1}{2}m[(3vt^2 + A\omega\cos(\omega t) + L\dot{\phi}\cos\phi)^2 + (L\dot{\phi}\sin\phi)^2] + mgL\cos\phi$

\textbf{Step 3: Apply the Euler-Lagrange equation.}

The Euler-Lagrange equation for the angle $\phi$ is:

$\frac{d}{dt}\left(\frac{\partial L}{\partial \dot{\phi}}\right) - \frac{\partial L}{\partial \phi} = 0$

First, we find the partial derivatives:

$\frac{\partial L}{\partial \dot{\phi}} = m(3vt^2 + A\omega\cos(\omega t) + L\dot{\phi}\cos\phi)L\cos\phi + mL^2\dot{\phi}\sin^2\phi = m[3vt^2L\cos\phi + A\omega L \cos(\omega t)\cos\phi + L^2\dot{\phi}]$
$\frac{\partial L}{\partial \phi} = m(3vt^2 + A\omega\cos(\omega t) + L\dot{\phi}\cos\phi)(-L\dot{\phi}\sin\phi) + mL^2\dot{\phi}^2\sin\phi\cos\phi - mgL\sin\phi$
$\frac{\partial L}{\partial \phi} = -m(3vt^2L\dot{\phi}\sin\phi + A\omega L\dot{\phi}\cos(\omega t)\sin\phi) -mgL\sin\phi$

Now, we take the time derivative of $\frac{\partial L}{\partial \dot{\phi}}$:

$\frac{d}{dt}\left(\frac{\partial L}{\partial \dot{\phi}}\right) = m[6vtL\cos\phi - 3vt^2L\dot{\phi}\sin\phi - A\omega^2 L \sin(\omega t)\cos\phi - A\omega L \cos(\omega t)\dot{\phi}\sin\phi + L^2\ddot{\phi}]$

\textbf{Step 4: Obtain the equation of motion.}

Substituting into the Euler-Lagrange equation:

$m[6vtL\cos\phi - 3vt^2L\dot{\phi}\sin\phi - A\omega^2 L \sin(\omega t)\cos\phi - A\omega L \cos(\omega t)\dot{\phi}\sin\phi + L^2\ddot{\phi}] - [-m(3vt^2L\dot{\phi}\sin\phi + A\omega L\dot{\phi}\cos(\omega t)\sin\phi) -mgL\sin\phi] = 0$

Simplifying, we get:

$L^2\ddot{\phi} + 6vtL\cos\phi - A\omega^2 L \sin(\omega t)\cos\phi + gL\sin\phi = 0$

Dividing by $mL^2$:

$\ddot{\phi} + \frac{6vt}{L}\cos\phi - \frac{A\omega^2}{L} \sin(\omega t)\cos\phi + \frac{g}{L}\sin\phi = 0$

\textbf{b)}

When $v \to 0$, $A \to 0$, and for small angles $\phi \approx 0$, we can use the approximations $\sin\phi \approx \phi$ and $\cos\phi \approx 1$.
The equation of motion becomes:
$\ddot{\phi} + \frac{g}{L}\phi = 0$

This is the equation of motion for a simple pendulum, which is a harmonic oscillator with angular frequency $\omega_0 = \sqrt{\frac{g}{L}}$.


\subsubsection{Solution 2}
\textbf{a)}

\textbf{Step 1: Define the coordinates and velocities.}

Let the position of the mass $m$ be given by $(x_m, y_m)$.
We can express these coordinates in terms of the angle $\phi$ and the support's position $x(t)$.
\begin{align*}
x_m &= x(t) + L \sin \phi = vt^3 + A \sin(\omega t) + L \sin \phi \\
y_m &= -L \cos \phi
\end{align*}

Now, we differentiate with respect to time to find the velocities:
\begin{align*}
\dot{x}_m &= 3vt^2 + A\omega \cos(\omega t) + L \dot{\phi} \cos \phi \\
\dot{y}_m &= L \dot{\phi} \sin \phi
\end{align*}

\textbf{Step 2: Write down the Lagrangian.}

The Lagrangian is given by $L = T - V$, where $T$ is the kinetic energy and $V$ is the potential energy.

The kinetic energy is:
$$T = \frac{1}{2} m (\dot{x}_m^2 + \dot{y}_m^2)$$

The potential energy is:
$$V = mgy_m = -mgL \cos \phi$$

Substituting the velocities, we get:
\begin{align*}
T &= \frac{1}{2} m \left[ (3vt^2 + A\omega \cos(\omega t) + L \dot{\phi} \cos \phi)^2 + (L \dot{\phi} \sin \phi)^2 \right] \\
&= \frac{1}{2} m \left[ (3vt^2 + A\omega \cos(\omega t))^2 + 2(3vt^2 + A\omega \cos(\omega t))L \dot{\phi} \cos \phi + L^2 \dot{\phi}^2 \cos^2 \phi + L^2 \dot{\phi}^2 \sin^2 \phi \right] \\
&= \frac{1}{2} m \left[ (3vt^2 + A\omega \cos(\omega t))^2 + 2(3vt^2 + A\omega \cos(\omega t))L \dot{\phi} \cos \phi + L^2 \dot{\phi}^2 \right]
\end{align*}

Thus, the Lagrangian is:
$$L = \frac{1}{2} m \left[ (3vt^2 + A\omega \cos(\omega t))^2 + 2(3vt^2 + A\omega \cos(\omega t))L \dot{\phi} \cos \phi + L^2 \dot{\phi}^2 \right] + mgL \cos \phi$$

\textbf{Step 3: Apply the Euler-Lagrange equation.}

The Euler-Lagrange equation for the variable $\phi$ is:
$$\frac{d}{dt} \left( \frac{\partial L}{\partial \dot{\phi}} \right) - \frac{\partial L}{\partial \phi} = 0$$

First, we compute the partial derivatives:
\begin{align*}
\frac{\partial L}{\partial \dot{\phi}} &= m \left[ (3vt^2 + A\omega \cos(\omega t))L \cos \phi + L^2 \dot{\phi} \right] \\
\frac{\partial L}{\partial \phi} &= -m(3vt^2 + A\omega \cos(\omega t))L \dot{\phi} \sin \phi - mgL \sin \phi
\end{align*}

Next, we differentiate $\frac{\partial L}{\partial \dot{\phi}}$ with respect to time:
\begin{align*}
\frac{d}{dt} \left( \frac{\partial L}{\partial \dot{\phi}} \right) &= m \left[ (6vt - A\omega^2 \sin(\omega t))L \cos \phi - (3vt^2 + A\omega \cos(\omega t))L \dot{\phi} \sin \phi + L^2 \ddot{\phi} \right]
\end{align*}

\textbf{Step 4: Derive the equation of motion.}

Substituting into the Euler-Lagrange equation:
\begin{align*}
&m \left[ (6vt - A\omega^2 \sin(\omega t))L \cos \phi - (3vt^2 + A\omega \cos(\omega t))L \dot{\phi} \sin \phi + L^2 \ddot{\phi} \right] \\
&+ m(3vt^2 + A\omega \cos(\omega t))L \dot{\phi} \sin \phi + mgL \sin \phi = 0
\end{align*}
Dividing by $mL$:
$$(6vt - A\omega^2 \sin(\omega t)) \cos \phi + L \ddot{\phi} + g \sin \phi = 0$$

Thus, the equation of motion is:
$$L \ddot{\phi} + (6vt - A\omega^2 \sin(\omega t)) \cos \phi + g \sin \phi = 0$$

\textbf{b)}

\textbf{Step 1: Apply the given limits.}

In the limit of $v \to 0$ and $A \to 0$, the equation of motion becomes:
$$L \ddot{\phi} + g \sin \phi = 0$$

\textbf{Step 2: Apply the small angle approximation.}

For small angles, $\sin \phi \approx \phi$. The equation further simplifies to:
$$L \ddot{\phi} + g \phi = 0$$

\textbf{Step 3: Recognize the classic pendulum equation.}

Dividing by $L$, we get:
$$\ddot{\phi} + \frac{g}{L} \phi = 0$$
This is the equation of motion for a simple harmonic oscillator with angular frequency $\omega = \sqrt{\frac{g}{L}}$, which is the classic pendulum equation of motion.


\subsubsection{Solution 3}
\textbf{a)}

\textbf{Step 1: Define the coordinates and velocities.}

Let the position of the mass $m$ be given by $(x_m, y_m)$.
We can express these coordinates in terms of the angle $\phi$ and the support's position $x(t)$.
\begin{align*}
x_m &= x(t) + L \sin \phi = vt^3 + A \sin(\omega t) + L \sin \phi \\
y_m &= -L \cos \phi
\end{align*}

Now, we differentiate with respect to time to find the velocities:
\begin{align*}
\dot{x}_m &= 3vt^2 + A\omega \cos(\omega t) + L \dot{\phi} \cos \phi \\
\dot{y}_m &= L \dot{\phi} \sin \phi
\end{align*}

\textbf{Step 2: Write down the Lagrangian.}

The Lagrangian is given by $L = T - V$, where $T$ is the kinetic energy and $V$ is the potential energy.

The kinetic energy is:
$$T = \frac{1}{2} m (\dot{x}_m^2 + \dot{y}_m^2)$$

The potential energy is:
$$V = mgy_m = -mgL \cos \phi$$

Substituting the velocities, we get:
\begin{align*}
T &= \frac{1}{2} m \left[ (3vt^2 + A\omega \cos(\omega t) + L \dot{\phi} \cos \phi)^2 + (L \dot{\phi} \sin \phi)^2 \right] \\
&= \frac{1}{2} m \left[ (3vt^2 + A\omega \cos(\omega t))^2 + 2(3vt^2 + A\omega \cos(\omega t))L \dot{\phi} \cos \phi + L^2 \dot{\phi}^2 \cos^2 \phi + L^2 \dot{\phi}^2 \sin^2 \phi \right] \\
&= \frac{1}{2} m \left[ (3vt^2 + A\omega \cos(\omega t))^2 + 2(3vt^2 + A\omega \cos(\omega t))L \dot{\phi} \cos \phi + L^2 \dot{\phi}^2 \right]
\end{align*}

Thus, the Lagrangian is:
$$L = \frac{1}{2} m \left[ (3vt^2 + A\omega \cos(\omega t))^2 + 2(3vt^2 + A\omega \cos(\omega t))L \dot{\phi} \cos \phi + L^2 \dot{\phi}^2 \right] + mgL \cos \phi$$

\textbf{Step 3: Apply the Euler-Lagrange equation.}

The Euler-Lagrange equation for the variable $\phi$ is:
$$\frac{d}{dt} \left( \frac{\partial L}{\partial \dot{\phi}} \right) - \frac{\partial L}{\partial \phi} = 0$$

First, we compute the partial derivatives:
\begin{align*}
\frac{\partial L}{\partial \dot{\phi}} &= m \left[ (3vt^2 + A\omega \cos(\omega t))L \cos \phi + L^2 \dot{\phi} \right] \\
\frac{\partial L}{\partial \phi} &= -m(3vt^2 + A\omega \cos(\omega t))L \dot{\phi} \sin \phi - mgL \sin \phi
\end{align*}

Next, we differentiate $\frac{\partial L}{\partial \dot{\phi}}$ with respect to time:
\begin{align*}
\frac{d}{dt} \left( \frac{\partial L}{\partial \dot{\phi}} \right) &= m \left[ (6vt - A\omega^2 \sin(\omega t))L \cos \phi - (3vt^2 + A\omega \cos(\omega t))L \dot{\phi} \sin \phi + L^2 \ddot{\phi} \right]
\end{align*}

\textbf{Step 4: Derive the equation of motion.}

Substituting into the Euler-Lagrange equation:
\begin{align*}
&m \left[ (6vt - A\omega^2 \sin(\omega t))L \cos \phi - (3vt^2 + A\omega \cos(\omega t))L \dot{\phi} \sin \phi + L^2 \ddot{\phi} \right] \\
&+ m(3vt^2 + A\omega \cos(\omega t))L \dot{\phi} \sin \phi + mgL \sin \phi = 0
\end{align*}
Dividing by $mL$:
$$(6vt - A\omega^2 \sin(\omega t)) \cos \phi + L \ddot{\phi} + g \sin \phi = 0$$

Thus, the equation of motion is:
$$L \ddot{\phi} + (6vt - A\omega^2 \sin(\omega t)) \cos \phi + g \sin \phi = 0$$

\textbf{b)}

\textbf{Step 1: Apply the given limits.}

In the limit of $v \to 0$ and $A \to 0$, the equation of motion becomes:
$$L \ddot{\phi} + g \sin \phi = 0$$

\textbf{Step 2: Apply the small angle approximation.}

For small angles, $\sin \phi \approx \phi$. The equation further simplifies to:
$$L \ddot{\phi} + g \phi = 0$$

\textbf{Step 3: Recognize the classic pendulum equation.}

Dividing by $L$, we get:
$$\ddot{\phi} + \frac{g}{L} \phi = 0$$
This is the equation of motion for a simple harmonic oscillator with angular frequency $\omega = \sqrt{\frac{g}{L}}$, which is the classic pendulum equation of motion.


\subsection{Question 9}
A rope connects to a block of mass $m$ and wraps around a circular disk of mass $M$ and radius $R$.
Due to gravity, the rope unwinds and the block falls down.

a) Derive expressions for the angular acceleration of the disk, the tension in the rope, and the
acceleration of the block. Work under the assumption that the rope does not slip and that the
moment of inertia of the disk is given by \(I=\frac{1}{2}MR^{2}\) .

b) Calculate the angular acceleration of the disk, tension in the rope, and acceleration of the block
for the case \(m=3\) kg, \(M=12\) kg, \(R=0.2\) m.

\paragraph{Prompt: \\} 
Answer the given question as a top physics student, giving step-by-step reasoning a) [6 marks] b) [1 marks]. Format your answer as LaTeX code without using sub-sections.

\subsubsection{Solution 1}

\textit{\textbf{a) Deriving expressions for angular acceleration, tension, and acceleration:}}

\textit{\textbf{Step 1: Draw a free body diagram and identify forces.}}

Let's denote the tension in the rope as $T$, the acceleration of the block as $a$, and the angular acceleration of the disk as $\alpha$. For the block, the forces are gravity ($mg$) acting downward and tension ($T$) acting upward. For the disk, the torque is due to the tension at the edge of the disk.

\textit{\textbf{Step 2: Apply Newton's second law for the block.}}

For the block of mass $m$, applying Newton's second law in the vertical direction, we get:
$$mg - T = ma \quad (1)$$

\textit{\textbf{Step 3: Apply Newton's second law for rotation for the disk.}}

For the disk, the torque $\tau$ is given by $TR$. Applying Newton's second law for rotation, $\tau = I\alpha$, where $I$ is the moment of inertia. Given that \(I=\frac{1}{2}MR^{2}\), we have:
$$TR = \frac{1}{2}MR^{2}\alpha \quad (2)$$

\textit{\textbf{Step 4: Relate linear acceleration and angular acceleration.}}

Since the rope does not slip, the linear acceleration of the block is related to the angular acceleration of the disk by:
$$a = R\alpha \quad (3)$$

\textit{\textbf{Step 5: Solve the equations.}}

From equation (2), we can express the tension $T$ as:
$$T = \frac{1}{2}MR\alpha \quad (4)$$
Substitute equation (3) into equation (4):
$$T = \frac{1}{2}Ma \quad (5)$$
Now, substitute equation (5) into equation (1):
$$mg - \frac{1}{2}Ma = ma$$
$$mg = a(m + \frac{1}{2}M)$$
Solving for $a$:
$$a = \frac{mg}{m + \frac{1}{2}M} = \frac{2mg}{2m+M} \quad (6)$$
Using equation (3), we find $\alpha$:
$$\alpha = \frac{a}{R} = \frac{2mg}{R(2m+M)} \quad (7)$$
Finally, using equation (5), we find $T$:
$$T = \frac{1}{2}Ma = \frac{M}{2}\left(\frac{2mg}{2m+M}\right) = \frac{Mmg}{2m+M} \quad (8)$$

\textit{\textbf{Expressions:}}

Angular acceleration of the disk:
$$\alpha = \frac{2mg}{R(2m+M)}$$

Tension in the rope:
$$T = \frac{Mmg}{2m+M}$$

Acceleration of the block:
$$a = \frac{2mg}{2m+M}$$

\textit{\textbf{b) Calculating values for given parameters:}}

\textit{\textbf{Step 6: Substitute given values into the derived expressions.}}

Given \(m=3\) kg, \(M=12\) kg, \(R=0.2\) m.

Acceleration of the block:
$$a = \frac{2(3)(9.8)}{2(3)+12} = \frac{58.8}{18} = 3.2667 \, \text{m/s}^2$$
Angular acceleration of the disk:
$$\alpha = \frac{a}{R} = \frac{3.2667}{0.2} = 16.3335 \, \text{rad/s}^2$$
Tension in the rope:
$$T = \frac{(12)(3)(9.8)}{2(3)+12} = \frac{352.8}{18} = 19.6 \, \text{N}$$

\textit{\textbf{Values:}}

Angular acceleration: $\alpha = 16.33 \, \text{rad/s}^2$

Tension: $T = 19.6 \, \text{N}$

Acceleration: $a = 3.27 \, \text{m/s}^2$

\subsubsection{Solution 2}
\textbf{a)}

\textbf{Step 1: Draw a free-body diagram and define variables.}

Let $a$ be the downward acceleration of the block, $T$ be the tension in the rope, and $\alpha$ be the angular acceleration of the disk.

\textbf{Step 2: Apply Newton's second law to the block.}

The forces acting on the block are gravity ($mg$) downwards and tension ($T$) upwards. Applying Newton's second law in the vertical direction:
$$mg - T = ma$$

\textbf{Step 3: Apply Newton's second law for rotation to the disk.}

The torque acting on the disk is due to the tension in the rope, which is $\tau = TR$. The moment of inertia of the disk is given as $I = \frac{1}{2}MR^2$. Applying Newton's second law for rotation:
$$\tau = I\alpha$$
$$TR = \frac{1}{2}MR^2\alpha$$

\textbf{Step 4: Relate linear acceleration to angular acceleration.}

Since the rope does not slip, the linear acceleration of the block is related to the angular acceleration of the disk by:
$$a = R\alpha$$

\textbf{Step 5: Solve the system of equations.}

From the equation in Step 3, we have:
$$T = \frac{1}{2}MR\alpha$$
Substitute $a = R\alpha$, or $\alpha = \frac{a}{R}$, into the above equation:
$$T = \frac{1}{2}Ma$$
Now, substitute this expression for $T$ into the equation from Step 2:
$$mg - \frac{1}{2}Ma = ma$$
$$mg = ma + \frac{1}{2}Ma$$
$$mg = a(m + \frac{1}{2}M)$$
Solve for $a$:
$$a = \frac{mg}{m + \frac{1}{2}M}$$
$$a = \frac{2mg}{2m + M}$$
Now, substitute this back into the expression for $T$:
$$T = \frac{1}{2}M \left(\frac{2mg}{2m + M}\right)$$
$$T = \frac{Mmg}{2m + M}$$
Finally, substitute $a$ into the expression for $\alpha$:
$$\alpha = \frac{a}{R} = \frac{2mg}{R(2m+M)}$$

\textbf{Expressions derived:}
Angular acceleration of the disk: $$\alpha = \frac{2mg}{R(2m+M)}$$
Tension in the rope: $$T = \frac{Mmg}{2m+M}$$
Acceleration of the block: $$a = \frac{2mg}{2m+M}$$

\textbf{b)}

\textbf{Step 1: Substitute the given values into the expressions.}
Given $m = 3$ kg, $M = 12$ kg, and $R = 0.2$ m. Also, $g = 9.81$ m/s$^2$.

Acceleration of the block:
$$a = \frac{2(3)(9.81)}{2(3) + 12} = \frac{58.86}{18} = 3.27 \text{ m/s}^2$$
Tension in the rope:
$$T = \frac{(12)(3)(9.81)}{2(3) + 12} = \frac{353.16}{18} = 19.62 \text{ N}$$
Angular acceleration of the disk:
$$\alpha = \frac{2(3)(9.81)}{0.2(2(3) + 12)} = \frac{58.86}{3.6} = 16.35 \text{ rad/s}^2$$

\textbf{Numerical answers:}
Angular acceleration of the disk: $$\alpha = 16.35 \text{ rad/s}^2$$
Tension in the rope: $$T = 19.62 \text{ N}$$
Acceleration of the block: $$a = 3.27 \text{ m/s}^2$$


\subsubsection{Solution 3}
\textbf{a)}

\textbf{Step 1: Draw a free-body diagram and define variables.}

Let $a$ be the downward acceleration of the block, $T$ be the tension in the rope, and $\alpha$ be the angular acceleration of the disk.

\textbf{Step 2: Apply Newton's second law to the block.}

The forces acting on the block are gravity ($mg$) and tension ($T$). Applying Newton's second law in the vertical direction:
\[mg - T = ma\]

\textbf{Step 3: Apply Newton's second law for rotation to the disk.}

The torque acting on the disk is due to the tension in the rope: $\tau = TR$.
The moment of inertia of the disk is given as $I = \frac{1}{2}MR^2$.
Applying Newton's second law for rotation:
\[TR = I\alpha = \frac{1}{2}MR^2 \alpha\]

\textbf{Step 4: Relate the linear acceleration of the block to the angular acceleration of the disk.}

Since the rope does not slip, the linear acceleration of the block is related to the angular acceleration of the disk by:
\[a = R\alpha\]

\textbf{Step 5: Solve for the angular acceleration ($\alpha$), tension ($T$), and linear acceleration ($a$).}

From Step 3, we have:
\[T = \frac{1}{2}MR\alpha\]
From Step 4, we have:
\[\alpha = \frac{a}{R}\]
Substituting this into the equation for $T$:
\[T = \frac{1}{2}Ma\]
Now substitute this expression for $T$ into the equation from Step 2:
\[mg - \frac{1}{2}Ma = ma\]
\[mg = a(m + \frac{1}{2}M)\]
Solving for $a$:
\[a = \frac{mg}{m + \frac{1}{2}M} = \frac{2mg}{2m + M}\]
Now we can find $\alpha$:
\[\alpha = \frac{a}{R} = \frac{2mg}{R(2m + M)}\]
And finally, we can find $T$:
\[T = \frac{1}{2}Ma = \frac{1}{2}M \cdot \frac{2mg}{2m + M} = \frac{Mmg}{2m + M}\]

\textbf{Final expressions:}

Angular acceleration of the disk: \[\alpha = \frac{2mg}{R(2m+M)}\]
Tension in the rope: \[T = \frac{Mmg}{2m + M}\]
Acceleration of the block: \[a = \frac{2mg}{2m + M}\]

\textbf{b)}

\textbf{Step 1: Substitute the given values into the expressions derived in part a).}

Given $m = 3$ kg, $M = 12$ kg, $R = 0.2$ m, and assuming $g = 9.81 \, \text{m/s}^2$:

\[a = \frac{2(3)(9.81)}{2(3) + 12} = \frac{58.86}{18} = 3.27 \, \text{m/s}^2\]
\[\alpha = \frac{a}{R} = \frac{3.27}{0.2} = 16.35 \, \text{rad/s}^2\]
\[T = \frac{(12)(3)(9.81)}{2(3) + 12} = \frac{353.16}{18} = 19.62 \, \text{N}\]

\textbf{Final answers:}

Angular acceleration of the disk: \[\alpha = 16.35 \, \text{rad/s}^2\]
Tension in the rope: \[T = 19.62 \, \text{N}\]
Acceleration of the block: \[a = 3.27 \, \text{m/s}^2\]

\subsection{Question 10}
Q10.

On Earth, a ball of mass $m$ is dropped from an airplane moving with horizontal velocity $u$. The
air resistance on the ball causes a force opposite to the ball's velocity $v$, namely

$$F = -bv$$

where $b$ is a positive constant.

a) Using Newton's laws of motion, show that

$$m\frac{dv}{dt} = -mg\hat{k} - bv$$

and show the solution to this equation is of the form

$$v = Ae^{-bt/m} - \frac{mg}{b}\hat{k}$$

where $A$ is a constant vector.

b) Given that the ball initially starts with velocity $u$, find an expression for $A$.

\paragraph{Prompt: \\} 
Answer the given question as a top physics student, giving step-by-step reasoning a) [6 marks] b) [2 marks]. Format your answer as LaTeX code without using sub-sections. 

\subsubsection{Solution 1}
**a)**

**Step 1: Apply Newton's Second Law**

Newton's second law states that the net force acting on an object is equal to the product of its mass and acceleration:
$$F_{net} = ma$$
where $F_{net}$ is the net force, $m$ is the mass, and $a$ is the acceleration.
Since acceleration is the time derivative of velocity, we can write $a = \frac{dv}{dt}$.

**Step 2: Identify the Forces**

In this case, there are two forces acting on the ball:
1.  Gravitational force: $F_g = -mg\hat{k}$, where $g$ is the acceleration due to gravity and $\hat{k}$ is the unit vector in the vertical direction (downward is negative).
2.  Air resistance force: $F_{air} = -bv$, where $b$ is a positive constant and $v$ is the velocity of the ball.

**Step 3: Formulate the Equation of Motion**

The net force is the sum of the gravitational and air resistance forces:
$$F_{net} = F_g + F_{air} = -mg\hat{k} - bv$$
Applying Newton's second law:
$$m\frac{dv}{dt} = -mg\hat{k} - bv$$

**Step 4: Solve the Differential Equation**

We are given that the solution is of the form:
$$v = Ae^{-bt/m} - \frac{mg}{b}\hat{k}$$
Let's differentiate this with respect to time:
$$\frac{dv}{dt} = -\frac{b}{m}Ae^{-bt/m}$$

**Step 5: Substitute into the Equation of Motion**
Substitute $v$ and $\frac{dv}{dt}$ into the equation of motion:
$$m\left(-\frac{b}{m}Ae^{-bt/m}\right) = -mg\hat{k} - b\left(Ae^{-bt/m} - \frac{mg}{b}\hat{k}\right)$$
$$-bAe^{-bt/m} = -mg\hat{k} - bAe^{-bt/m} + mg\hat{k}$$
$$-bAe^{-bt/m} = -bAe^{-bt/m}$$

This equation holds true, confirming that the given solution is correct.

**b)**

**Step 7: Apply Initial Conditions**

At $t=0$, the initial velocity is given as $v(0) = u$. We can assume that $u$ is in the horizontal direction, so we can write $u = u\hat{i}$, where $\hat{i}$ is the unit vector in the horizontal direction.

**Step 8: Substitute Initial Conditions into the Solution**

Substitute $t=0$ and $v(0) = u\hat{i}$ into the solution:
$$u\hat{i} = Ae^{-b(0)/m} - \frac{mg}{b}\hat{k}$$
$$u\hat{i} = A - \frac{mg}{b}\hat{k}$$

**Step 9: Solve for A**

From the above equation, we can find the constant vector $A$:
$$A = u\hat{i} + \frac{mg}{b}\hat{k}$$

\subsubsection{Solution 2}
**a)**

**Step 1: Apply Newton's Second Law**

Newton's Second Law states that the net force acting on an object is equal to the product of its mass and acceleration:

$$F_{net} = ma$$

In this case, the ball experiences two forces: gravity and air resistance. We can express the acceleration as the time derivative of velocity:

$$F_{net} = m\frac{dv}{dt}$$

**Step 2: Identify the Forces**

The gravitational force acts downward and is given by:

$$F_g = -mg\hat{k}$$

where $\hat{k}$ is the unit vector in the vertical direction.

The air resistance force is given as:

$$F_{air} = -bv$$

**Step 3: Combine the Forces**

The net force is the sum of the gravitational and air resistance forces:

$$F_{net} = F_g + F_{air} = -mg\hat{k} - bv$$

**Step 4: Form the Differential Equation**

Equating the net force to $m\frac{dv}{dt}$, we get the differential equation:

$$m\frac{dv}{dt} = -mg\hat{k} - bv$$

**Step 5: Solve the Differential Equation**

To solve this first-order linear vector differential equation, we can first consider the homogeneous part:

$$m\frac{dv}{dt} = -bv$$

$$\frac{dv}{v} = -\frac{b}{m}dt$$

Integrating both sides:

$$\int \frac{dv}{v} = \int -\frac{b}{m}dt$$

$$\ln|v| = -\frac{b}{m}t + C$$

$$v = e^{-\frac{b}{m}t + C} = e^C e^{-\frac{b}{m}t}$$

Let $A = e^C$, then the homogeneous solution is:

$$v_h = Ae^{-\frac{b}{m}t}$$

where $A$ is a constant vector.

Now we need to find a particular solution. We can guess a constant solution of the form:

$$v_p = C\hat{k}$$

Substituting into the original differential equation:

$$0 = -mg\hat{k} - bC\hat{k}$$

$$C = -\frac{mg}{b}$$

So the particular solution is:

$$v_p = -\frac{mg}{b}\hat{k}$$

**Step 6: Combine the Solutions**

The general solution is the sum of the homogeneous and particular solutions:

$$v = v_h + v_p = Ae^{-bt/m} - \frac{mg}{b}\hat{k}$$

**b)**

**Step 7: Apply the Initial Condition**

At $t=0$, the ball has an initial velocity $u$. Let's assume $u$ is in the horizontal direction, so we can write $u = u\hat{i}$ where $\hat{i}$ is the unit vector in the horizontal direction.
Thus,

$$v(0) = u\hat{i}$$

Substituting $t=0$ into the general solution:

$$v(0) = Ae^{-b(0)/m} - \frac{mg}{b}\hat{k} = A - \frac{mg}{b}\hat{k}$$

**Step 8: Solve for A**

Equating the initial condition to the solution at $t=0$:

$$u\hat{i} = A - \frac{mg}{b}\hat{k}$$

Therefore,

$$A = u\hat{i} + \frac{mg}{b}\hat{k}$$


\subsubsection{Solution 3}
**a)**

**Step 1: Apply Newton's Second Law**

Newton's Second Law states that the net force acting on an object is equal to the product of its mass and acceleration:

$$F_{net} = ma$$

In this case, the ball experiences two forces: gravity and air resistance. We can express the acceleration as the time derivative of velocity:

$$F_{net} = m\frac{dv}{dt}$$

**Step 2: Identify the Forces**

The gravitational force acts downward and is given by:

$$F_g = -mg\hat{k}$$

where $\hat{k}$ is the unit vector in the vertical direction. The air resistance force is given as:

$$F_{air} = -bv$$

**Step 3: Combine the Forces**

The net force is the sum of the gravitational and air resistance forces:

$$F_{net} = F_g + F_{air} = -mg\hat{k} - bv$$

**Step 4: Form the Differential Equation**

Substituting this into Newton's Second Law, we get:

$$m\frac{dv}{dt} = -mg\hat{k} - bv$$

This is the differential equation governing the motion of the ball.

**Step 5: Solve the Differential Equation**

The given solution is of the form:

$$v = Ae^{-bt/m} - \frac{mg}{b}\hat{k}$$

Let's differentiate this with respect to time:

$$\frac{dv}{dt} = -\frac{b}{m}Ae^{-bt/m}$$

**Step 6: Substitute into the Differential Equation**

Substituting the proposed solution and its derivative into the differential equation:

$$m\left(-\frac{b}{m}Ae^{-bt/m}\right) = -mg\hat{k} - b\left(Ae^{-bt/m} - \frac{mg}{b}\hat{k}\right)$$

$$-bAe^{-bt/m} = -mg\hat{k} - bAe^{-bt/m} + mg\hat{k}$$

$$-bAe^{-bt/m} = -bAe^{-bt/m}$$

**Step 7: Verify the Solution**

The equation holds true, confirming that the proposed solution is indeed a solution to the differential equation.

**b)**

**Step 1: Apply the Initial Condition**

At time $t=0$, the ball has an initial velocity $u$. We can substitute these into the solution:

$$v(0) = u$$

$$u = Ae^{-b(0)/m} - \frac{mg}{b}\hat{k}$$

$$u = A - \frac{mg}{b}\hat{k}$$

**Step 2: Solve for A**

Rearranging the equation to solve for $A$, we get:

$$A = u + \frac{mg}{b}\hat{k}$$

This is the expression for the constant vector $A$ in terms of the initial velocity $u$ and other given parameters.

\section{Quantum Mechanics}

\subsection{Question 1}
\begin{enumerate}
    \item[(a)] Define the commutator of 2 operators, $\hat{P}$ and $\hat{Q}$. Define what it means if $\hat{P}$ and $\hat{Q}$ are compatible operators. Explain what compatibility implies about their commutator?
    \item[(b)] Suppose that $\hat{P}$ and $\hat{Q}$ are not compatible. For a given system, $\hat{P}$ is first measured giving a value $p$, then $\hat{Q}$ is measured giving the value $q$. If $\hat{P}$ is then measured again, what can be said about the possible results of the measurement and why?
\end{enumerate}

\paragraph{Prompt: \\} 
Answer the given question as a top physics student, giving step-by-step reasoning keeping your answer under 300 words. Format your answer as LaTeX code without using sub-sections. Available marks: a) [3 marks] b) [2 marks].

\subsubsection{Solution 1}
\noindent \textbf{Question 1}

\noindent (a) [3 marks] The commutator of two operators, $\hat{P}$ and $\hat{Q}$, is defined as:

\[ [\hat{P}, \hat{Q}] = \hat{P}\hat{Q} - \hat{Q}\hat{P} \]

\noindent Compatible operators $\hat{P}$ and $\hat{Q}$ are operators that can be measured simultaneously with infinite precision. This means that the measurement of one does not affect the outcome of a subsequent measurement of the other. Compatibility implies that their commutator is zero:

\[ [\hat{P}, \hat{Q}] = 0 \]

\noindent This can also be expressed as $\hat{P}\hat{Q} = \hat{Q}\hat{P}$ for compatible operators.

\noindent (b) [2 marks] If $\hat{P}$ and $\hat{Q}$ are not compatible, then $[\hat{P}, \hat{Q}] \neq 0$. After the initial measurement of $\hat{P}$, the system is in an eigenstate of $\hat{P}$ with eigenvalue $p$. However, the subsequent measurement of $\hat{Q}$ collapses the wavefunction into an eigenstate of $\hat{Q}$ with eigenvalue $q$. This state is, in general, not an eigenstate of $\hat{P}$ as the operators are not compatible. Therefore, when $\hat{P}$ is measured again, the result is no longer guaranteed to be $p$. The outcome will be probabilistic, governed by the coefficients of the expansion of the new state in terms of the eigenstates of $\hat{P}$.


\subsubsection{Solution 2}
\noindent \textbf{Question 1}

\noindent (a) [3 marks]

\noindent The commutator of two operators, $\hat{P}$ and $\hat{Q}$, is defined as:
\[
[\hat{P}, \hat{Q}] = \hat{P}\hat{Q} - \hat{Q}\hat{P}
\]
If $\hat{P}$ and $\hat{Q}$ are compatible operators, it means that the measurement of one operator does not affect the measurement outcome of the other. In other words, the order of measurements does not matter. If the operators are compatible then they can be simultaneously measured to arbitrary precision.

\noindent Compatibility implies that the commutator of the two operators is zero:
\[
[\hat{P}, \hat{Q}] = 0
\]
\noindent This is because if the order of measurements does not affect the outcome, then $\hat{P}\hat{Q} = \hat{Q}\hat{P}$.

\noindent (b) [2 marks]

\noindent If $\hat{P}$ and $\hat{Q}$ are not compatible, and $\hat{P}$ is first measured giving a value $p$, followed by a measurement of $\hat{Q}$ giving $q$, then a subsequent measurement of $\hat{P}$ will not necessarily yield the same result $p$.
\noindent The act of measuring $\hat{Q}$ collapses the system's wavefunction into an eigenstate of $\hat{Q}$, which is generally not an eigenstate of $\hat{P}$ since they are incompatible.
\noindent Therefore, the second measurement of $\hat{P}$ will yield one of the eigenvalues of $\hat{P}$, but not necessarily the original value $p$, as the state of the system was altered by the measurement of $\hat{Q}$. The precise value obtained will depend on the wavefunction after the measurement of $\hat{Q}$.


\subsubsection{Solution 3}
\textit{Question 1}

\textit{(a) [3 marks]}

The commutator of two operators $\hat{P}$ and $\hat{Q}$, denoted as $[\hat{P}, \hat{Q}]$, is defined as:

\[ [\hat{P}, \hat{Q}] = \hat{P}\hat{Q} - \hat{Q}\hat{P} \]

Two operators $\hat{P}$ and $\hat{Q}$ are said to be compatible if they share a complete set of simultaneous eigenstates. In simpler terms, if we can measure one operator without affecting the outcome of a subsequent measurement of the other operator, they are compatible.

If $\hat{P}$ and $\hat{Q}$ are compatible, their commutator is zero:

\[ [\hat{P}, \hat{Q}] = 0 \]

This means that the order of operation does not matter, i.e., $\hat{P}\hat{Q} = \hat{Q}\hat{P}$.

\textit{(b) [2 marks]}

If $\hat{P}$ and $\hat{Q}$ are not compatible, their commutator is non-zero, $[\hat{P}, \hat{Q}] \neq 0$. In this scenario, the initial measurement of $\hat{P}$ puts the system into an eigenstate of $\hat{P}$ with eigenvalue $p$.  However, since $\hat{P}$ and $\hat{Q}$ are not compatible, this eigenstate of $\hat{P}$ is not an eigenstate of $\hat{Q}$. The subsequent measurement of $\hat{Q}$ collapses the system into an eigenstate of $\hat{Q}$ with eigenvalue $q$, which is not an eigenstate of $\hat{P}$.

Therefore, if $\hat{P}$ is measured again, the system is no longer in the original eigenstate of $\hat{P}$. The measurement of $\hat{Q}$ disturbed the state such that $\hat{P}$ could take on any of its eigenvalues, with specific probability determined by the new wavefunction. It is not guaranteed that the measurement of $\hat{P}$ will be $p$ again.


\subsection{Question 2}

\begin{enumerate}
    \item[(a)] Quantum mechanical operators are Hermitian. What mathematical property do the eigenvalues of a Hermitian operator have? What is the physical interpretation of its eigenvalues?
    \item[(b)] Explain what is meant if it is stated that the set of eigenfunctions $\{\phi_n\}$ of a Hermitian operator is orthonormal.
\end{enumerate}


\paragraph{Prompt: \\} 
Answer the given question as a top physics student, giving step-by-step reasoning keeping your answer under 300 words. Format your answer as LaTeX code without using sub-sections. Available marks: a) [2 marks] b) [2 marks].

\subsubsection{Solution 1}
Okay, here is the answer to Question 2, formatted as requested:

\textbf{Question 2}

\textbf{(a)}

\textit{Step 1}: Recall the defining property of a Hermitian operator $\hat{A}$. It satisfies $\hat{A} = \hat{A}^\dagger$, where $\hat{A}^\dagger$ is the Hermitian conjugate.

\textit{Step 2}: Consider the eigenvalue equation $\hat{A}\phi_n = a_n\phi_n$. The eigenvalues $a_n$ of a Hermitian operator are always \textbf{real}.

\textit{Step 3}: In quantum mechanics, physical observables are represented by Hermitian operators. The eigenvalues correspond to the possible \textbf{measured values} of that observable.

\textbf{(b)}

\textit{Step 1}: Orthonormality combines two properties: orthogonality and normalization.

\textit{Step 2}: Orthogonality means that the inner product of two different eigenfunctions is zero. Mathematically, $\langle \phi_m | \phi_n \rangle = \int \phi_m^* \phi_n d\tau = 0$ if $m \neq n$.

\textit{Step 3}: Normalization means that the inner product of an eigenfunction with itself is one. Mathematically, $\langle \phi_n | \phi_n \rangle = \int \phi_n^* \phi_n d\tau = 1$.

\textit{Step 4}: Combining these, the set $\{\phi_n\}$ is orthonormal if $\langle \phi_m | \phi_n \rangle = \delta_{mn}$, where $\delta_{mn}$ is the Kronecker delta, which is 1 if $m=n$ and 0 otherwise.


\subsubsection{Solution 2}
Okay, here is the answer to Question 2, formatted as requested:

\textbf{Question 2}

\textbf{(a) [2 marks]}

\textit{Step 1:} The eigenvalues of a Hermitian operator are real. Mathematically, if $\hat{A}$ is a Hermitian operator and $\lambda$ is an eigenvalue, then $\lambda = \lambda^*$, where $\lambda^*$ denotes the complex conjugate of $\lambda$.

\textit{Step 2:} Physically, the eigenvalues of a Hermitian operator represent the possible results of a measurement of the observable corresponding to that operator. Since physical measurements always yield real numbers, it is crucial that the eigenvalues are real.

\textbf{(b) [2 marks]}

\textit{Step 1:} The statement that a set of eigenfunctions $\{\phi_n\}$ of a Hermitian operator is orthonormal means two things: they are orthogonal and they are normalized.

\textit{Step 2:} Orthogonality means that the inner product of any two different eigenfunctions is zero. Mathematically, this is expressed as $\int \phi_m^* \phi_n d\tau = 0$ if $m \neq n$.

\textit{Step 3:} Normalization means that the inner product of each eigenfunction with itself is one. Mathematically, this is expressed as $\int \phi_n^* \phi_n d\tau = 1$ for all $n$.

\textit{Step 4:} Combining these, orthonormality is expressed using the Kronecker delta: $\int \phi_m^* \phi_n d\tau = \delta_{mn}$, where $\delta_{mn} = 1$ if $m = n$ and $\delta_{mn} = 0$ if $m \neq n$.


\subsubsection{Solution 3}
Okay, here is the answer to Question 2, formatted as requested:

\textbf{Question 2}

\textbf{(a) [2 marks]}

\textit{Step 1:} The eigenvalues of a Hermitian operator $\hat{A}$ are \textit{real}. This can be represented mathematically as $\lambda = \lambda^*$ where $\lambda$ is an eigenvalue.

\textit{Step 2:} Physically, the eigenvalues represent the possible results of a measurement of the observable associated with the operator. Since measured values in experiments are real quantities, the eigenvalues must be real.

\textbf{(b) [2 marks]}

\textit{Step 1:} A set of eigenfunctions $\{\phi_n\}$ is orthonormal if each eigenfunction is orthogonal to every other eigenfunction in the set. Mathematically, this means $\int \phi_m^* \phi_n d\tau = 0$ if $m \neq n$, where the integral is over all space.

\textit{Step 2:} Additionally, each eigenfunction is normalized, meaning the integral of its probability density over all space is 1. Mathematically, this is represented as $\int \phi_n^* \phi_n d\tau = 1$.

\textit{Step 3:} Combining these conditions, the orthonormality of the set can be concisely expressed using the Kronecker delta: $\int \phi_m^* \phi_n d\tau = \delta_{mn}$, where $\delta_{mn} = 1$ if $m = n$ and $\delta_{mn} = 0$ if $m \neq n$.


\subsection{Question 3} 

For one dimension, an operator $\hat{Q}$ is Hermitian if and only if
\[
\int_{-\infty}^{\infty} f^* \hat{Q} g \, dx = \int_{-\infty}^{\infty} g (\hat{Q} f)^* \, dx
\]
where $f(x), g(x)$ are well-behaved functions which vanish at infinity, $*$ denotes the complex conjugate.

a) Use the definition above to determine whether $\hat{p} = -i\hbar \frac{d}{dx}$ is a Hermitian operator.

b) Determine whether $\hat{Q} = \frac{d^2}{dx^2}$ is a Hermitian operator.

\paragraph{Prompt: \\} 
Answer the given question as a top physics student, giving step-by-step reasoning. Format your answer as LaTeX code without using sub-sections. Available marks: a) [4 marks], b) [5 marks].

\subsubsection{Solution 1}
\textbf{Question 3}

\textbf{a)}

We are given the definition of a Hermitian operator and asked to determine if  $\hat{p} = -i\hbar \frac{d}{dx}$ is Hermitian.  We need to show that $\int_{-\infty}^{\infty} f^* \hat{p} g \, dx = \int_{-\infty}^{\infty} g (\hat{p} f)^* \, dx$.

Let's start with the left-hand side (LHS) of the equation:

\[
\text{LHS} = \int_{-\infty}^{\infty} f^* \left(-i\hbar \frac{d}{dx}\right) g \, dx = -i\hbar \int_{-\infty}^{\infty} f^* \frac{dg}{dx} \, dx
\]

We can solve this integral using integration by parts. Let $u = f^*$ and $dv = \frac{dg}{dx} dx$. Then $du = \frac{df^*}{dx} dx$ and $v = g$. Integration by parts states that $\int u \, dv = uv - \int v \, du$. Applying this, we get:

\[
\text{LHS} = -i\hbar \left[ f^* g \Big|_{-\infty}^{\infty} - \int_{-\infty}^{\infty} g \frac{df^*}{dx} \, dx \right]
\]

Since $f$ and $g$ vanish at infinity, the term $f^* g \Big|_{-\infty}^{\infty}$ is zero. Thus:

\[
\text{LHS} = i\hbar \int_{-\infty}^{\infty} g \frac{df^*}{dx} \, dx
\]

Now let's consider the right-hand side (RHS) of the equation:

\[
\text{RHS} = \int_{-\infty}^{\infty} g \left(-i\hbar \frac{d}{dx} f\right)^* \, dx = \int_{-\infty}^{\infty} g \left(i\hbar \frac{df}{dx}\right)^* \, dx = i\hbar \int_{-\infty}^{\infty} g \frac{df^*}{dx} \, dx
\]

Comparing the LHS and RHS, we see that:

\[
\text{LHS} = i\hbar \int_{-\infty}^{\infty} g \frac{df^*}{dx} \, dx = \text{RHS}
\]

Therefore, $\hat{p}$ is a Hermitian operator.

\textbf{b)}

We need to determine if $\hat{Q} = \frac{d^2}{dx^2}$ is Hermitian. We need to show that $\int_{-\infty}^{\infty} f^* \hat{Q} g \, dx = \int_{-\infty}^{\infty} g (\hat{Q} f)^* \, dx$.

Let's start with the left-hand side (LHS) of the equation:

\[
\text{LHS} = \int_{-\infty}^{\infty} f^* \frac{d^2 g}{dx^2} \, dx
\]

We can use integration by parts. Let $u = f^*$ and $dv = \frac{d^2 g}{dx^2} dx$. Then $du = \frac{df^*}{dx} dx$ and $v = \frac{dg}{dx}$. Applying integration by parts:

\[
\text{LHS} = f^* \frac{dg}{dx} \Big|_{-\infty}^{\infty} - \int_{-\infty}^{\infty} \frac{df^*}{dx} \frac{dg}{dx} \, dx
\]

Since $f$ and $g$ vanish at infinity, the first term is zero. Thus:

\[
\text{LHS} = - \int_{-\infty}^{\infty} \frac{df^*}{dx} \frac{dg}{dx} \, dx
\]

We apply integration by parts again. Let $u = \frac{dg}{dx}$ and $dv = \frac{df^*}{dx} dx$. Then $du = \frac{d^2 g}{dx^2} dx$ and $v = f^*$. Applying integration by parts:

\[
\text{LHS} = - \left[ \frac{dg}{dx} f^* \Big|_{-\infty}^{\infty} - \int_{-\infty}^{\infty} f^* \frac{d^2g}{dx^2} \, dx \right]
\]

This result isn't helpful, let's return to our previous result and let $u = \frac{df^*}{dx}$ and $dv = \frac{dg}{dx}dx$, so that $du = \frac{d^2f^*}{dx^2}dx$ and $v=g$. Applying integration by parts we obtain:

\[
\text{LHS} = -\left( \frac{df^*}{dx} g \Big|_{-\infty}^{\infty} - \int_{-\infty}^{\infty} g \frac{d^2f^*}{dx^2} \, dx \right)
\]

Since the functions vanish at infinity:

\[
\text{LHS} = \int_{-\infty}^{\infty} g \frac{d^2 f^*}{dx^2} \, dx
\]
Now for the right-hand side (RHS):

\[
\text{RHS} = \int_{-\infty}^{\infty} g \left(\frac{d^2 f}{dx^2}\right)^* \, dx = \int_{-\infty}^{\infty} g \frac{d^2 f^*}{dx^2} \, dx
\]

Comparing the LHS and RHS, we see that:

\[
\text{LHS} = \int_{-\infty}^{\infty} g \frac{d^2 f^*}{dx^2} \, dx = \text{RHS}
\]

Therefore, $\hat{Q} = \frac{d^2}{dx^2}$ is a Hermitian operator.

\subsubsection{Solution 2}

\textbf{Question 3}

\textbf{a)}

We are given the definition of a Hermitian operator and asked to determine if  $\hat{p} = -i\hbar \frac{d}{dx}$ is Hermitian.  We need to show that $\int_{-\infty}^{\infty} f^* \hat{p} g \, dx = \int_{-\infty}^{\infty} g (\hat{p} f)^* \, dx$.

Let's start with the left-hand side (LHS) of the equation:

\[
\text{LHS} = \int_{-\infty}^{\infty} f^* \left(-i\hbar \frac{d}{dx}\right) g \, dx = -i\hbar \int_{-\infty}^{\infty} f^* \frac{dg}{dx} \, dx
\]

We can solve this integral using integration by parts. Let $u = f^*$ and $dv = \frac{dg}{dx}dx$. Then $du = \frac{df^*}{dx} dx$ and $v = g$. Integration by parts states that $\int u \, dv = uv - \int v \, du$. Applying this, we get:

\[
\text{LHS} = -i\hbar \left[ f^* g \Big|_{-\infty}^{\infty} - \int_{-\infty}^{\infty} g \frac{df^*}{dx} \, dx \right]
\]

Since $f$ and $g$ vanish at infinity, the first term $f^* g \Big|_{-\infty}^{\infty}$ is zero. Therefore,

\[
\text{LHS} = i\hbar \int_{-\infty}^{\infty} g \frac{df^*}{dx} \, dx
\]

Now let's consider the right-hand side (RHS) of the equation:

\[
\text{RHS} = \int_{-\infty}^{\infty} g \left(-i\hbar \frac{d}{dx} f\right)^* \, dx = \int_{-\infty}^{\infty} g \left(i\hbar \frac{df}{dx}\right)^* \, dx = i\hbar \int_{-\infty}^{\infty} g \frac{df^*}{dx} \, dx
\]

Comparing the LHS and RHS, we see that

\[
\text{LHS} = i\hbar \int_{-\infty}^{\infty} g \frac{df^*}{dx} \, dx = \text{RHS}
\]

Thus, $\hat{p} = -i\hbar \frac{d}{dx}$ is a Hermitian operator.

\textbf{b)}

We are asked to determine if $\hat{Q} = \frac{d^2}{dx^2}$ is a Hermitian operator. We need to show that $\int_{-\infty}^{\infty} f^* \hat{Q} g \, dx = \int_{-\infty}^{\infty} g (\hat{Q} f)^* \, dx$.

Let's start with the left-hand side (LHS) of the equation:

\[
\text{LHS} = \int_{-\infty}^{\infty} f^* \frac{d^2g}{dx^2} \, dx
\]

We will use integration by parts again. Let $u = f^*$ and $dv = \frac{d^2g}{dx^2}dx$. Then $du = \frac{df^*}{dx}dx$ and $v = \frac{dg}{dx}$. Applying integration by parts:

\[
\text{LHS} = f^* \frac{dg}{dx} \Big|_{-\infty}^{\infty} - \int_{-\infty}^{\infty} \frac{df^*}{dx} \frac{dg}{dx} \, dx
\]

Since $f$ and $g$ vanish at infinity, their derivatives also vanish at infinity. Thus, the first term is zero.

\[
\text{LHS} = - \int_{-\infty}^{\infty} \frac{df^*}{dx} \frac{dg}{dx} \, dx
\]
We apply integration by parts a second time. Let $u = \frac{dg}{dx}$ and $dv = \frac{df^*}{dx}dx$. Then $du = \frac{d^2g}{dx^2}dx$ and $v = f^*$.
\[
\text{LHS} = - \left[ \frac{dg}{dx} f^* \Big|_{-\infty}^{\infty} - \int_{-\infty}^{\infty} f^* \frac{d^2g}{dx^2} dx \right]
\]

Again, the boundary term vanishes, leaving:

\[
\text{LHS} = \frac{dg}{dx} f^* \Big|_{-\infty}^{\infty} - \left[ g \frac{df^*}{dx} \Big|_{-\infty}^{\infty} - \int_{-\infty}^{\infty} g \frac{d^2 f^*}{dx^2} \, dx \right]
\]
The boundary terms vanish:

\[
\text{LHS} =  \int_{-\infty}^{\infty} g \frac{d^2 f^*}{dx^2} \, dx
\]

Now let's consider the right-hand side (RHS) of the equation:

\[
\text{RHS} = \int_{-\infty}^{\infty} g \left(\frac{d^2 f}{dx^2}\right)^* \, dx = \int_{-\infty}^{\infty} g \frac{d^2 f^*}{dx^2} \, dx
\]

Comparing the LHS and RHS, we see that

\[
\text{LHS} = \int_{-\infty}^{\infty} g \frac{d^2 f^*}{dx^2} \, dx = \text{RHS}
\]

Thus, $\hat{Q} = \frac{d^2}{dx^2}$ is a Hermitian operator.

\subsubsection{Solution 3}
Okay, here is the solution to the question, formatted as requested:

\textbf{a)}

\textit{Step 1: Substitute the operator into the left-hand side of the Hermitian definition.}

\[
\int_{-\infty}^{\infty} f^* \hat{p} g \, dx = \int_{-\infty}^{\infty} f^* \left( -i\hbar \frac{d g}{dx} \right) dx = -i\hbar \int_{-\infty}^{\infty} f^* \frac{dg}{dx} dx
\]

\textit{Step 2: Integrate by parts.}

Let $u = f^*$ and $dv = \frac{dg}{dx} dx$. Then $du = \frac{df^*}{dx} dx$ and $v = g$.

Using integration by parts formula $\int u \, dv = uv - \int v \, du$, we have:

\[
-i\hbar \int_{-\infty}^{\infty} f^* \frac{dg}{dx} dx = -i\hbar \left[ f^* g \right]_{-\infty}^{\infty} + i\hbar \int_{-\infty}^{\infty} \frac{df^*}{dx} g \, dx
\]

\textit{Step 3: Apply the boundary conditions.}

Since $f$ and $g$ vanish at infinity, $f^* g \Big|_{-\infty}^{\infty} = 0$. Thus,

\[
-i\hbar \left[ f^* g \right]_{-\infty}^{\infty} + i\hbar \int_{-\infty}^{\infty} \frac{df^*}{dx} g \, dx = i\hbar \int_{-\infty}^{\infty} \frac{df^*}{dx} g \, dx
\]

\textit{Step 4: Manipulate the result into the required form.}

We can rewrite the integral as:

\[
i\hbar \int_{-\infty}^{\infty} \frac{df^*}{dx} g \, dx = \int_{-\infty}^{\infty} g \left( i\hbar \frac{df^*}{dx} \right) dx = \int_{-\infty}^{\infty} g \left( -i\hbar \frac{df}{dx} \right)^* dx = \int_{-\infty}^{\infty} g (\hat{p} f)^* dx
\]

\textit{Step 5: Conclude.}

We have shown that

\[
\int_{-\infty}^{\infty} f^* \hat{p} g \, dx = \int_{-\infty}^{\infty} g (\hat{p} f)^* dx
\]

Therefore, $\hat{p}$ is a Hermitian operator.

\textbf{b)}

\textit{Step 1: Substitute the operator into the left-hand side of the Hermitian definition.}

\[
\int_{-\infty}^{\infty} f^* \hat{Q} g \, dx = \int_{-\infty}^{\infty} f^* \frac{d^2 g}{dx^2} dx
\]

\textit{Step 2: Integrate by parts.}

Let $u = f^*$ and $dv = \frac{d^2 g}{dx^2} dx$. Then $du = \frac{df^*}{dx} dx$ and $v = \frac{dg}{dx}$.

Using integration by parts formula:

\[
\int_{-\infty}^{\infty} f^* \frac{d^2 g}{dx^2} dx = \left[ f^* \frac{dg}{dx} \right]_{-\infty}^{\infty} - \int_{-\infty}^{\infty} \frac{df^*}{dx} \frac{dg}{dx} dx
\]

\textit{Step 3: Apply the boundary conditions.}

Since $f$ and $g$ vanish at infinity, their derivatives must also vanish (if we assume the functions go to zero faster than any linear function).  Thus, $f^* \frac{dg}{dx} \Big|_{-\infty}^{\infty} = 0$. So,

\[
\left[ f^* \frac{dg}{dx} \right]_{-\infty}^{\infty} - \int_{-\infty}^{\infty} \frac{df^*}{dx} \frac{dg}{dx} dx = - \int_{-\infty}^{\infty} \frac{df^*}{dx} \frac{dg}{dx} dx
\]

\textit{Step 4: Integrate by parts again.}

Let $u = \frac{dg}{dx}$ and $dv = \frac{df^*}{dx} dx$. Then $du = \frac{d^2 g}{dx^2} dx$ and $v = f^*$.

\[
- \int_{-\infty}^{\infty} \frac{df^*}{dx} \frac{dg}{dx} dx = - \left[ f^* \frac{dg}{dx} \right]_{-\infty}^{\infty} + \int_{-\infty}^{\infty} f^* \frac{d^2 g}{dx^2} dx
\]
This is the negative of what we started with and is not helpful.
Let $u = \frac{df^*}{dx}$ and $dv = \frac{dg}{dx} dx$. Then $du = \frac{d^2 f^*}{dx^2} dx$ and $v = g$.
\[
-\int_{-\infty}^{\infty} \frac{df^*}{dx} \frac{dg}{dx} dx = - \left[ \frac{df^*}{dx} g \right]_{-\infty}^{\infty} + \int_{-\infty}^{\infty} \frac{d^2 f^*}{dx^2} g \, dx
\]

\textit{Step 5: Apply the boundary conditions.}

Since $f$ and $g$ vanish at infinity, $\frac{df^*}{dx} g \Big|_{-\infty}^{\infty} = 0$. Thus,

\[
- \left[ \frac{df^*}{dx} g \right]_{-\infty}^{\infty} + \int_{-\infty}^{\infty} \frac{d^2 f^*}{dx^2} g \, dx = \int_{-\infty}^{\infty} \frac{d^2 f^*}{dx^2} g \, dx
\]

\textit{Step 6: Manipulate the result into the required form.}

\[
\int_{-\infty}^{\infty} \frac{d^2 f^*}{dx^2} g \, dx = \int_{-\infty}^{\infty} g \frac{d^2 f^*}{dx^2} dx = \int_{-\infty}^{\infty} g \left( \frac{d^2 f}{dx^2} \right)^* dx = \int_{-\infty}^{\infty} g (\hat{Q} f)^* dx
\]

\textit{Step 7: Conclude.}

We have shown that

\[
\int_{-\infty}^{\infty} f^* \hat{Q} g \, dx = \int_{-\infty}^{\infty} g (\hat{Q} f)^* dx
\]

Therefore, $\hat{Q} = \frac{d^2}{dx^2}$ is a Hermitian operator.


\subsection{Question 4}

Let $\hat{L}_x$ be the x-component of the angular momentum operator. Let $\hat{X}, \hat{Y}, \hat{Z}$ be the x, y, z components of the position operator respectively and $\hat{P}_x, \hat{P}_y, \hat{P}_z$ be the x, y, z components of the momentum operator respectively. Derive the following commutator relations:   \\

a) $\left[\hat{L}_x, \hat{X}\right] = 0$  

b) $\left[\hat{L}_x, \hat{P}_x\right] = 0$  

c) $\left[\hat{L}_x, \hat{Y}\right] = i\hbar \hat{Z}$  

d) $\left[\hat{L}_x, \hat{P}_y\right] = i\hbar \hat{P}_z$  

e) $\left[\hat{L}_x, \hat{P}^2\right] = 0$

\paragraph{Prompt: \\} 
Answer the given question as a top physics student, giving step-by-step reasoning. Format your answer as LaTeX code without using sub-sections. Available marks: a) [3 marks], b) [3 marks], c) [4 marks], d) [4 marks], e) [6 marks].

\subsubsection{Solution 1}
Okay, here are the derivations of the commutator relations, formatted as requested, demonstrating the reasoning of a top physics student.

\begin{center}
\textbf{Question 4}
\end{center}

We start by recalling the definition of the x-component of the angular momentum operator:
\[ \hat{L}_x = \hat{Y}\hat{P}_z - \hat{Z}\hat{P}_y \]
We also know the canonical commutation relations:
\[ [\hat{X},\hat{P}_x] = \hat{X}\hat{P}_x - \hat{P}_x\hat{X} = i\hbar \]
\[ [\hat{Y},\hat{P}_y] = \hat{Y}\hat{P}_y - \hat{P}_y\hat{Y} = i\hbar \]
\[ [\hat{Z},\hat{P}_z] = \hat{Z}\hat{P}_z - \hat{P}_z\hat{Z} = i\hbar \]
All other commutators between position and momentum operators are zero (e.g., $[\hat{X},\hat{Y}] = 0$, $[\hat{X}, \hat{P}_y] = 0$, etc.).

\textbf{a)}
\[ [\hat{L}_x, \hat{X}] = [\hat{Y}\hat{P}_z - \hat{Z}\hat{P}_y, \hat{X}] \]
Using the commutator identity $[\hat{A}\hat{B}, \hat{C}] = \hat{A}[\hat{B},\hat{C}] + [\hat{A},\hat{C}]\hat{B}$, we can expand:
\[ [\hat{L}_x, \hat{X}] = [\hat{Y}\hat{P}_z, \hat{X}] - [\hat{Z}\hat{P}_y, \hat{X}] \]
\[ [\hat{L}_x, \hat{X}] = \hat{Y}[\hat{P}_z, \hat{X}] + [\hat{Y}, \hat{X}]\hat{P}_z - \hat{Z}[\hat{P}_y, \hat{X}] - [\hat{Z}, \hat{X}]\hat{P}_y \]
Since all commutators on the right-hand side are zero, we have:
\[ [\hat{L}_x, \hat{X}] = 0 \]

\textbf{b)}
\[ [\hat{L}_x, \hat{P}_x] = [\hat{Y}\hat{P}_z - \hat{Z}\hat{P}_y, \hat{P}_x] \]
Expanding the commutator:
\[ [\hat{L}_x, \hat{P}_x] = [\hat{Y}\hat{P}_z, \hat{P}_x] - [\hat{Z}\hat{P}_y, \hat{P}_x] \]
\[ [\hat{L}_x, \hat{P}_x] = \hat{Y}[\hat{P}_z, \hat{P}_x] + [\hat{Y}, \hat{P}_x]\hat{P}_z - \hat{Z}[\hat{P}_y, \hat{P}_x] - [\hat{Z}, \hat{P}_x]\hat{P}_y \]
Again, all commutators on the right-hand side are zero, so:
\[ [\hat{L}_x, \hat{P}_x] = 0 \]

\textbf{c)}
\[ [\hat{L}_x, \hat{Y}] = [\hat{Y}\hat{P}_z - \hat{Z}\hat{P}_y, \hat{Y}] \]
Expanding:
\[ [\hat{L}_x, \hat{Y}] = [\hat{Y}\hat{P}_z, \hat{Y}] - [\hat{Z}\hat{P}_y, \hat{Y}] \]
\[ [\hat{L}_x, \hat{Y}] = \hat{Y}[\hat{P}_z, \hat{Y}] + [\hat{Y}, \hat{Y}]\hat{P}_z - \hat{Z}[\hat{P}_y, \hat{Y}] - [\hat{Z}, \hat{Y}]\hat{P}_y \]
\[ [\hat{L}_x, \hat{Y}] = 0 - 0 - \hat{Z}(-i\hbar) - 0 \]
\[ [\hat{L}_x, \hat{Y}] = i\hbar \hat{Z} \]

\textbf{d)}
\[ [\hat{L}_x, \hat{P}_y] = [\hat{Y}\hat{P}_z - \hat{Z}\hat{P}_y, \hat{P}_y] \]
Expanding:
\[ [\hat{L}_x, \hat{P}_y] = [\hat{Y}\hat{P}_z, \hat{P}_y] - [\hat{Z}\hat{P}_y, \hat{P}_y] \]
\[ [\hat{L}_x, \hat{P}_y] = \hat{Y}[\hat{P}_z, \hat{P}_y] + [\hat{Y}, \hat{P}_y]\hat{P}_z - \hat{Z}[\hat{P}_y, \hat{P}_y] - [\hat{Z}, \hat{P}_y]\hat{P}_y \]
\[ [\hat{L}_x, \hat{P}_y] = 0 + i\hbar \hat{P}_z - 0 - 0 \]
\[ [\hat{L}_x, \hat{P}_y] = i\hbar \hat{P}_z \]

\textbf{e)}
First, we define $\hat{P}^2 = \hat{P}_x^2 + \hat{P}_y^2 + \hat{P}_z^2$.
Then:
\[ [\hat{L}_x, \hat{P}^2] = [\hat{L}_x, \hat{P}_x^2 + \hat{P}_y^2 + \hat{P}_z^2] \]
Using the linearity of the commutator, we get:
\[ [\hat{L}_x, \hat{P}^2] = [\hat{L}_x, \hat{P}_x^2] + [\hat{L}_x, \hat{P}_y^2] + [\hat{L}_x, \hat{P}_z^2] \]
We use the identity $[\hat{A},\hat{B}^2] = \hat{B}[\hat{A},\hat{B}] + [\hat{A},\hat{B}]\hat{B}$ and the result of part b) to find that the first term is zero. Then, using the result of part d):
\[ [\hat{L}_x, \hat{P}_y^2] = \hat{P}_y[\hat{L}_x,\hat{P}_y] + [\hat{L}_x, \hat{P}_y]\hat{P}_y = \hat{P}_y(i\hbar \hat{P}_z) + (i\hbar \hat{P}_z)\hat{P}_y = i\hbar (\hat{P}_y\hat{P}_z + \hat{P}_z\hat{P}_y) \]
Similarly, we can obtain an expression for $[\hat{L}_x, \hat{P}_z^2]$:
\[ [\hat{L}_x, \hat{P}_z^2] = \hat{P}_z[\hat{L}_x,\hat{P}_z] + [\hat{L}_x, \hat{P}_z]\hat{P}_z = \hat{P}_z(-i\hbar \hat{P}_y) + (-i\hbar \hat{P}_y)\hat{P}_z = -i\hbar (\hat{P}_y\hat{P}_z + \hat{P}_z\hat{P}_y) \]
Putting these results together:
\[ [\hat{L}_x, \hat{P}^2] = 0 + i\hbar (\hat{P}_y\hat{P}_z + \hat{P}_z\hat{P}_y) - i\hbar (\hat{P}_y\hat{P}_z + \hat{P}_z\hat{P}_y) = 0 \]

Thus, we have derived all the required commutator relations.


\subsubsection{Solution 2}
Okay, here are the derivations of the commutator relations, formatted as requested, with step-by-step reasoning, just as a top physics student would present them:

\textbf{Question 4}

\textbf{a) $\left[\hat{L}_x, \hat{X}\right] = 0$ [3 marks]}

\textit{Step 1: Recall the definition of $\hat{L}_x$ and the commutator.}

$\hat{L}_x = \hat{Y}\hat{P}_z - \hat{Z}\hat{P}_y$

$\left[\hat{A}, \hat{B}\right] = \hat{A}\hat{B} - \hat{B}\hat{A}$

\textit{Step 2: Substitute the definition of $\hat{L}_x$ into the commutator.}

$\left[\hat{L}_x, \hat{X}\right] = \left[\hat{Y}\hat{P}_z - \hat{Z}\hat{P}_y, \hat{X}\right]$

\textit{Step 3: Apply the commutator linearity property: $\left[\hat{A} - \hat{B}, \hat{C}\right] = \left[\hat{A}, \hat{C}\right] - \left[\hat{B}, \hat{C}\right]$.}

$\left[\hat{L}_x, \hat{X}\right] = \left[\hat{Y}\hat{P}_z, \hat{X}\right] - \left[\hat{Z}\hat{P}_y, \hat{X}\right]$

\textit{Step 4: Apply the commutator product rule: $\left[\hat{A}\hat{B}, \hat{C}\right] = \hat{A}\left[\hat{B}, \hat{C}\right] + \left[\hat{A}, \hat{C}\right]\hat{B}$.}

$\left[\hat{L}_x, \hat{X}\right] = \hat{Y}\left[\hat{P}_z, \hat{X}\right] + \left[\hat{Y}, \hat{X}\right]\hat{P}_z - \hat{Z}\left[\hat{P}_y, \hat{X}\right] - \left[\hat{Z}, \hat{X}\right]\hat{P}_y$

\textit{Step 5: Recall that position and momentum operators for different components commute: $\left[\hat{X}, \hat{Y}\right] = \left[\hat{X}, \hat{Z}\right] = \left[\hat{X}, \hat{P}_y\right] = \left[\hat{X}, \hat{P}_z\right] = 0$.}

$\left[\hat{L}_x, \hat{X}\right] = \hat{Y}(0) + (0)\hat{P}_z - \hat{Z}(0) - (0)\hat{P}_y = 0$

\textbf{b) $\left[\hat{L}_x, \hat{P}_x\right] = 0$ [3 marks]}

\textit{Step 1: Substitute the definition of $\hat{L}_x$ into the commutator.}

$\left[\hat{L}_x, \hat{P}_x\right] = \left[\hat{Y}\hat{P}_z - \hat{Z}\hat{P}_y, \hat{P}_x\right]$

\textit{Step 2: Apply the commutator linearity property.}

$\left[\hat{L}_x, \hat{P}_x\right] = \left[\hat{Y}\hat{P}_z, \hat{P}_x\right] - \left[\hat{Z}\hat{P}_y, \hat{P}_x\right]$

\textit{Step 3: Apply the commutator product rule.}

$\left[\hat{L}_x, \hat{P}_x\right] = \hat{Y}\left[\hat{P}_z, \hat{P}_x\right] + \left[\hat{Y}, \hat{P}_x\right]\hat{P}_z - \hat{Z}\left[\hat{P}_y, \hat{P}_x\right] - \left[\hat{Z}, \hat{P}_x\right]\hat{P}_y$

\textit{Step 4: Recall that momentum operators commute with each other, and position and momentum operators for different components commute: $\left[\hat{P}_x, \hat{P}_y\right] = \left[\hat{P}_x, \hat{P}_z\right] = \left[\hat{Y}, \hat{P}_x\right] = \left[\hat{Z}, \hat{P}_x\right] = 0$.}

$\left[\hat{L}_x, \hat{P}_x\right] = \hat{Y}(0) + (0)\hat{P}_z - \hat{Z}(0) - (0)\hat{P}_y = 0$

\textbf{c) $\left[\hat{L}_x, \hat{Y}\right] = i\hbar \hat{Z}$ [4 marks]}

\textit{Step 1: Substitute the definition of $\hat{L}_x$ into the commutator.}

$\left[\hat{L}_x, \hat{Y}\right] = \left[\hat{Y}\hat{P}_z - \hat{Z}\hat{P}_y, \hat{Y}\right]$

\textit{Step 2: Apply the commutator linearity property.}

$\left[\hat{L}_x, \hat{Y}\right] = \left[\hat{Y}\hat{P}_z, \hat{Y}\right] - \left[\hat{Z}\hat{P}_y, \hat{Y}\right]$

\textit{Step 3: Apply the commutator product rule.}

$\left[\hat{L}_x, \hat{Y}\right] = \hat{Y}\left[\hat{P}_z, \hat{Y}\right] + \left[\hat{Y}, \hat{Y}\right]\hat{P}_z - \hat{Z}\left[\hat{P}_y, \hat{Y}\right] - \left[\hat{Z}, \hat{Y}\right]\hat{P}_y$

\textit{Step 4: Recall that an operator commutes with itself, and position and momentum operators for different components commute. Also, recall the canonical commutation relation: $\left[\hat{Y}, \hat{P}_y\right] = i\hbar$.}

$\left[\hat{L}_x, \hat{Y}\right] = \hat{Y}(0) + (0)\hat{P}_z - \hat{Z}(i\hbar) - (0)\hat{P}_y = -i\hbar\hat{Z}$
The correct answer should be $i\hbar\hat{Z}$, therefore we have made an error somewhere. Let's backtrack:

$\left[\hat{L}_x, \hat{Y}\right] = \left[\hat{Y}\hat{P}_z - \hat{Z}\hat{P}_y, \hat{Y}\right] = \hat{Y}\left[\hat{P}_z, \hat{Y}\right] + \left[\hat{Y}, \hat{Y}\right]\hat{P}_z - \hat{Z}\left[\hat{P}_y, \hat{Y}\right] - \left[\hat{Z}, \hat{Y}\right]\hat{P}_y$.

$\left[\hat{P}_z, \hat{Y}\right] = 0$
$\left[\hat{Y}, \hat{Y}\right] = 0$
$\left[\hat{P}_y, \hat{Y}\right] = -i\hbar$
$\left[\hat{Z}, \hat{Y}\right] = 0$

$\left[\hat{L}_x, \hat{Y}\right] = \hat{Y}(0) + (0)\hat{P}_z - \hat{Z}(-i\hbar) - (0)\hat{P}_y = i\hbar\hat{Z}$

\textbf{d) $\left[\hat{L}_x, \hat{P}_y\right] = i\hbar \hat{P}_z$ [4 marks]}

\textit{Step 1: Substitute the definition of $\hat{L}_x$ into the commutator.}

$\left[\hat{L}_x, \hat{P}_y\right] = \left[\hat{Y}\hat{P}_z - \hat{Z}\hat{P}_y, \hat{P}_y\right]$

\textit{Step 2: Apply the commutator linearity property.}

$\left[\hat{L}_x, \hat{P}_y\right] = \left[\hat{Y}\hat{P}_z, \hat{P}_y\right] - \left[\hat{Z}\hat{P}_y, \hat{P}_y\right]$

\textit{Step 3: Apply the commutator product rule.}

$\left[\hat{L}_x, \hat{P}_y\right] = \hat{Y}\left[\hat{P}_z, \hat{P}_y\right] + \left[\hat{Y}, \hat{P}_y\right]\hat{P}_z - \hat{Z}\left[\hat{P}_y, \hat{P}_y\right] - \left[\hat{Z}, \hat{P}_y\right]\hat{P}_y$

\textit{Step 4: Recall that momentum operators commute with each other, and position and momentum operators for different components commute. Also, recall the canonical commutation relation.}

$\left[\hat{L}_x, \hat{P}_y\right] = \hat{Y}(0) + (i\hbar)\hat{P}_z - \hat{Z}(0) - (0)\hat{P}_y = i\hbar\hat{P}_z$

\textbf{e) $\left[\hat{L}_x, \hat{P}^2\right] = 0$ [6 marks]}

\textit{Step 1: Expand $\hat{P}^2$.}

$\hat{P}^2 = \hat{P}_x^2 + \hat{P}_y^2 + \hat{P}_z^2$

\textit{Step 2: Substitute into the commutator.}

$\left[\hat{L}_x, \hat{P}^2\right] = \left[\hat{L}_x, \hat{P}_x^2 + \hat{P}_y^2 + \hat{P}_z^2\right]$

\textit{Step 3: Apply the commutator linearity property.}

$\left[\hat{L}_x, \hat{P}^2\right] = \left[\hat{L}_x, \hat{P}_x^2\right] + \left[\hat{L}_x, \hat{P}_y^2\right] + \left[\hat{L}_x, \hat{P}_z^2\right]$

\textit{Step 4: Apply the commutator product rule: $\left[\hat{A}, \hat{B}^2\right] = \hat{B}\left[\hat{A}, \hat{B}\right] + \left[\hat{A}, \hat{B}\right]\hat{B}$.}

$\left[\hat{L}_x, \hat{P}^2\right] = \hat{P}_x\left[\hat{L}_x, \hat{P}_x\right] + \left[\hat{L}_x, \hat{P}_x\right]\hat{P}_x + \hat{P}_y\left[\hat{L}_x, \hat{P}_y\right] + \left[\hat{L}_x, \hat{P}_y\right]\hat{P}_y + \hat{P}_z\left[\hat{L}_x, \hat{P}_z\right] + \left[\hat{L}_x, \hat{P}_z\right]\hat{P}_z$

\textit{Step 5: Use the results from parts (b) and (d), and a similar derivation for $\left[\hat{L}_x, \hat{P}_z\right] = -i\hbar \hat{P}_y$.}

$\left[\hat{L}_x, \hat{P}^2\right] = \hat{P}_x(0) + (0)\hat{P}_x + \hat{P}_y(i\hbar\hat{P}_z) + (i\hbar\hat{P}_z)\hat{P}_y + \hat{P}_z(-i\hbar\hat{P}_y) + (-i\hbar\hat{P}_y)\hat{P}_z$

\textit{Step 6: Simplify.}

$\left[\hat{L}_x, \hat{P}^2\right] = i\hbar(\hat{P}_y\hat{P}_z + \hat{P}_z\hat{P}_y - \hat{P}_z\hat{P}_y - \hat{P}_y\hat{P}_z) = 0$


\subsubsection{Solution 3}

\textit{We are given the x-component of the angular momentum operator, $\hat{L}_x$, and we are asked to evaluate several commutators involving the position operators ($\hat{X}, \hat{Y}, \hat{Z}$) and the momentum operators ($\hat{P}_x, \hat{P}_y, \hat{P}_z$).}

\textit{First, recall the definition of $\hat{L}_x$ in terms of the position and momentum operators:}

\begin{equation}
\hat{L}_x = \hat{Y}\hat{P}_z - \hat{Z}\hat{P}_y
\end{equation}

\textit{We will also need the fundamental commutation relations:}

\begin{equation}
[\hat{X}, \hat{P}_x] = [\hat{Y}, \hat{P}_y] = [\hat{Z}, \hat{P}_z] = i\hbar
\end{equation}

\begin{equation}
[\hat{X}, \hat{Y}] = [\hat{Y}, \hat{Z}] = [\hat{Z}, \hat{X}] = 0
\end{equation}

\begin{equation}
[\hat{P}_x, \hat{P}_y] = [\hat{P}_y, \hat{P}_z] = [\hat{P}_z, \hat{P}_x] = 0
\end{equation}

\textit{Finally, we can apply the commutator identity, valid for any operators $\hat{A}, \hat{B}, \hat{C}$:}

\begin{equation}
[\hat{A}, \hat{B}\hat{C}] = [\hat{A}, \hat{B}]\hat{C} + \hat{B}[\hat{A}, \hat{C}]
\end{equation}
\\

a) \textit{Calculate $[\hat{L}_x, \hat{X}]$:}

\textit{Step 1: Substitute the definition of $\hat{L}_x$.}

\begin{equation}
[\hat{L}_x, \hat{X}] = [\hat{Y}\hat{P}_z - \hat{Z}\hat{P}_y, \hat{X}]
\end{equation}

\textit{Step 2: Use the linearity of the commutator.}

\begin{equation}
[\hat{L}_x, \hat{X}] = [\hat{Y}\hat{P}_z, \hat{X}] - [\hat{Z}\hat{P}_y, \hat{X}]
\end{equation}

\textit{Step 3: Apply the commutator identity.}

\begin{equation}
[\hat{L}_x, \hat{X}] = [\hat{Y}, \hat{X}]\hat{P}_z + \hat{Y}[\hat{P}_z, \hat{X}] - [\hat{Z}, \hat{X}]\hat{P}_y - \hat{Z}[\hat{P}_y, \hat{X}]
\end{equation}

\textit{Step 4: Use the fundamental commutation relations.}

\begin{equation}
[\hat{L}_x, \hat{X}] = (0)\hat{P}_z + \hat{Y}(0) - (0)\hat{P}_y - \hat{Z}(0) = 0
\end{equation}

\textit{Therefore, $[\hat{L}_x, \hat{X}] = 0$ [3 marks]}

\\

b) \textit{Calculate $[\hat{L}_x, \hat{P}_x]$:}

\textit{Step 1: Substitute the definition of $\hat{L}_x$.}

\begin{equation}
[\hat{L}_x, \hat{P}_x] = [\hat{Y}\hat{P}_z - \hat{Z}\hat{P}_y, \hat{P}_x]
\end{equation}

\textit{Step 2: Use the linearity of the commutator.}

\begin{equation}
[\hat{L}_x, \hat{P}_x] = [\hat{Y}\hat{P}_z, \hat{P}_x] - [\hat{Z}\hat{P}_y, \hat{P}_x]
\end{equation}

\textit{Step 3: Apply the commutator identity.}

\begin{equation}
[\hat{L}_x, \hat{P}_x] = [\hat{Y}, \hat{P}_x]\hat{P}_z + \hat{Y}[\hat{P}_z, \hat{P}_x] - [\hat{Z}, \hat{P}_x]\hat{P}_y - \hat{Z}[\hat{P}_y, \hat{P}_x]
\end{equation}

\textit{Step 4: Use the fundamental commutation relations.}

\begin{equation}
[\hat{L}_x, \hat{P}_x] = (0)\hat{P}_z + \hat{Y}(0) - (0)\hat{P}_y - \hat{Z}(0) = 0
\end{equation}

\textit{Therefore, $[\hat{L}_x, \hat{P}_x] = 0$ [3 marks]}

\\

c) \textit{Calculate $[\hat{L}_x, \hat{Y}]$:}

\textit{Step 1: Substitute the definition of $\hat{L}_x$.}

\begin{equation}
[\hat{L}_x, \hat{Y}] = [\hat{Y}\hat{P}_z - \hat{Z}\hat{P}_y, \hat{Y}]
\end{equation}

\textit{Step 2: Use the linearity of the commutator.}

\begin{equation}
[\hat{L}_x, \hat{Y}] = [\hat{Y}\hat{P}_z, \hat{Y}] - [\hat{Z}\hat{P}_y, \hat{Y}]
\end{equation}

\textit{Step 3: Apply the commutator identity.}

\begin{equation}
[\hat{L}_x, \hat{Y}] = [\hat{Y}, \hat{Y}]\hat{P}_z + \hat{Y}[\hat{P}_z, \hat{Y}] - [\hat{Z}, \hat{Y}]\hat{P}_y - \hat{Z}[\hat{P}_y, \hat{Y}]
\end{equation}

\textit{Step 4: Use the fundamental commutation relations.}

\begin{equation}
[\hat{L}_x, \hat{Y}] = (0)\hat{P}_z + \hat{Y}(0) - (0)\hat{P}_y - \hat{Z}(-i\hbar) = i\hbar\hat{Z}
\end{equation}

\textit{Therefore, $[\hat{L}_x, \hat{Y}] = i\hbar \hat{Z}$ [4 marks]}

\\

d) \textit{Calculate $[\hat{L}_x, \hat{P}_y]$:}

\textit{Step 1: Substitute the definition of $\hat{L}_x$.}

\begin{equation}
[\hat{L}_x, \hat{P}_y] = [\hat{Y}\hat{P}_z - \hat{Z}\hat{P}_y, \hat{P}_y]
\end{equation}

\textit{Step 2: Use the linearity of the commutator.}

\begin{equation}
[\hat{L}_x, \hat{P}_y] = [\hat{Y}\hat{P}_z, \hat{P}_y] - [\hat{Z}\hat{P}_y, \hat{P}_y]
\end{equation}

\textit{Step 3: Apply the commutator identity.}

\begin{equation}
[\hat{L}_x, \hat{P}_y] = [\hat{Y}, \hat{P}_y]\hat{P}_z + \hat{Y}[\hat{P}_z, \hat{P}_y] - [\hat{Z}, \hat{P}_y]\hat{P}_y - \hat{Z}[\hat{P}_y, \hat{P}_y]
\end{equation}

\textit{Step 4: Use the fundamental commutation relations.}

\begin{equation}
[\hat{L}_x, \hat{P}_y] = (i\hbar)\hat{P}_z + \hat{Y}(0) - (0)\hat{P}_y - \hat{Z}(0) = i\hbar\hat{P}_z
\end{equation}

\textit{Therefore, $[\hat{L}_x, \hat{P}_y] = i\hbar \hat{P}_z$ [4 marks]}

\\

e) \textit{Calculate $[\hat{L}_x, \hat{P}^2]$:}

\textit{Step 1: Expand $\hat{P}^2$.}

\begin{equation}
[\hat{L}_x, \hat{P}^2] = [\hat{L}_x, \hat{P}_x^2 + \hat{P}_y^2 + \hat{P}_z^2]
\end{equation}

\textit{Step 2: Use the linearity of the commutator.}

\begin{equation}
[\hat{L}_x, \hat{P}^2] = [\hat{L}_x, \hat{P}_x^2] + [\hat{L}_x, \hat{P}_y^2] + [\hat{L}_x, \hat{P}_z^2]
\end{equation}

\textit{Step 3: Apply the commutator identity, $[\hat{A}, \hat{B}^2] = [\hat{A}, \hat{B}]\hat{B} + \hat{B}[\hat{A}, \hat{B}]$.}

\begin{equation}
[\hat{L}_x, \hat{P}^2] = ([\hat{L}_x, \hat{P}_x]\hat{P}_x + \hat{P}_x[\hat{L}_x, \hat{P}_x]) + ([\hat{L}_x, \hat{P}_y]\hat{P}_y + \hat{P}_y[\hat{L}_x, \hat{P}_y]) + ([\hat{L}_x, \hat{P}_z]\hat{P}_z + \hat{P}_z[\hat{L}_x, \hat{P}_z])
\end{equation}

\textit{Step 4: Substitute the results from parts b) and d) and the result for $[\hat{L}_x, \hat{P}_z]$ (derived similarly to part d)).}

\begin{equation}
[\hat{L}_x, \hat{P}^2] = (0\hat{P}_x + \hat{P}_x0) + (i\hbar \hat{P}_z \hat{P}_y + \hat{P}_y i\hbar \hat{P}_z) + (-i\hbar \hat{P}_y \hat{P}_z + \hat{P}_z (-i\hbar \hat{P}_y))
\end{equation}

\textit{Step 5: Simplify.}

\begin{equation}
[\hat{L}_x, \hat{P}^2] = 0 + i\hbar(\hat{P}_z\hat{P}_y - \hat{P}_y\hat{P}_z) + i\hbar(-\hat{P}_y\hat{P}_z + \hat{P}_z\hat{P}_y) = 0
\end{equation}

\textit{Therefore, $[\hat{L}_x, \hat{P}^2] = 0$ [6 marks]}

\subsection{Question 5}
An infinite square well of length L can be defined mathematically by the potential,
\[
V(x) = 
\begin{cases} 
0 & \text{for } 0 \leq x \leq L, \\
+\infty & \text{otherwise},
\end{cases}
\]

A particle of mass \(m\) in the well has eigenfunctions
\[
\phi_n(x) = \sqrt{\frac{2}{L}} \sin\left(\frac{n \pi x}{L}\right)
\]
with corresponding energy eigenvalues
\[
E_n = \frac{n^2 \hbar^2 \pi^2}{2mL^2}.
\]

(a) Calculate the first-order correction to the ground state energy if the system is perturbed by
\[
\hat{H}' = V_0 \sin\left(\frac{2\pi x}{L}\right).
\]

(b) Calculate and derive an expression for the first-order correction to all energy eigenvalues given that the system is perturbed by
\[
\hat{H}' = L \alpha \delta\left(x - \frac{L}{2}\right),
\]
where \(\alpha\) is a constant and \(\delta\) is the Dirac delta function.

\paragraph{Prompt: \\} 
Answer the given question as a top physics student, giving step-by-step reasoning. Format your answer as LaTeX code without using sub-sections. Available marks: a) [4 marks], b) [3 marks].

\subsubsection{Solution 1}

\textbf{(a)}

The first-order correction to the ground state energy is given by,

\[
E_1^{(1)} = \langle \phi_1 | \hat{H}' | \phi_1 \rangle
\]

We are given that \(n=1\) for the ground state, thus,

\[
\phi_1(x) = \sqrt{\frac{2}{L}} \sin\left(\frac{\pi x}{L}\right)
\]

and
\[
\hat{H}' = V_0 \sin\left(\frac{2\pi x}{L}\right).
\]
Thus,
\[
E_1^{(1)} = \int_0^L \left(\sqrt{\frac{2}{L}} \sin\left(\frac{\pi x}{L}\right)\right) V_0 \sin\left(\frac{2\pi x}{L}\right) \left(\sqrt{\frac{2}{L}} \sin\left(\frac{\pi x}{L}\right)\right) dx
\]

\[
E_1^{(1)} = \frac{2V_0}{L} \int_0^L \sin^2\left(\frac{\pi x}{L}\right) \sin\left(\frac{2\pi x}{L}\right) dx
\]

Using the trigonometric identity \(\sin(2\theta) = 2\sin(\theta)\cos(\theta)\), we get

\[
E_1^{(1)} = \frac{4V_0}{L} \int_0^L \sin^3\left(\frac{\pi x}{L}\right) \cos\left(\frac{\pi x}{L}\right) dx
\]

Let \(u = \sin\left(\frac{\pi x}{L}\right)\), so \(du = \frac{\pi}{L} \cos\left(\frac{\pi x}{L}\right) dx\).
The limits of integration remain 0 and 0, since \(\sin(0) = \sin(\pi) = 0\).
Therefore,

\[
E_1^{(1)} = \frac{4V_0}{\pi} \int_0^0 u^3 du = 0
\]

Thus, the first-order correction to the ground state energy is 0.

\textbf{[4 marks]}

\textbf{(b)}

The first-order correction to the energy eigenvalues is given by

\[
E_n^{(1)} = \langle \phi_n | \hat{H}' | \phi_n \rangle
\]

We are given
\[
\hat{H}' = L \alpha \delta\left(x - \frac{L}{2}\right).
\]

Thus,

\[
E_n^{(1)} = \int_0^L \left(\sqrt{\frac{2}{L}} \sin\left(\frac{n\pi x}{L}\right)\right) L \alpha \delta\left(x - \frac{L}{2}\right) \left(\sqrt{\frac{2}{L}} \sin\left(\frac{n\pi x}{L}\right)\right) dx
\]

\[
E_n^{(1)} = 2\alpha \int_0^L \sin^2\left(\frac{n\pi x}{L}\right) \delta\left(x - \frac{L}{2}\right) dx
\]

Using the sifting property of the Dirac delta function, we get

\[
E_n^{(1)} = 2\alpha \sin^2\left(\frac{n\pi}{L} \cdot \frac{L}{2}\right) = 2\alpha \sin^2\left(\frac{n\pi}{2}\right)
\]

The expression can be simplified as

\[
E_n^{(1)} =
\begin{cases}
0 & \text{for } n \text{ even} \\
2\alpha & \text{for } n \text{ odd}
\end{cases}
\]
Thus, the first-order correction to the energy eigenvalue \(E_n\) is \(2\alpha\) for odd \(n\) and \(0\) for even \(n\).

\textbf{[3 marks]}

\subsubsection{Solution 2}
Okay, here are the solutions to the provided questions, formatted as requested:

\textbf{Question 5}

\textbf{(a)}

\textit{Step 1: Identify the unperturbed ground state energy and wave function.}

The ground state corresponds to \(n=1\), so the unperturbed ground state energy is
\[E_1^{(0)} = \frac{1^2 \hbar^2 \pi^2}{2mL^2} = \frac{\hbar^2 \pi^2}{2mL^2}.\]
The corresponding wave function is
\[\phi_1(x) = \sqrt{\frac{2}{L}} \sin\left(\frac{\pi x}{L}\right).\]

\textit{Step 2: Apply the formula for the first-order energy correction.}

The first-order correction to the energy is given by
\[E_1^{(1)} = \langle \phi_1 | \hat{H}' | \phi_1 \rangle = \int_0^L \phi_1^*(x) \hat{H}' \phi_1(x) \, dx.\]

\textit{Step 3: Substitute the given perturbation and wave function.}

Substituting the given perturbation and the ground state wave function, we get
\[E_1^{(1)} = \int_0^L \left(\sqrt{\frac{2}{L}} \sin\left(\frac{\pi x}{L}\right)\right) V_0 \sin\left(\frac{2\pi x}{L}\right) \left(\sqrt{\frac{2}{L}} \sin\left(\frac{\pi x}{L}\right)\right) \, dx.\]

\textit{Step 4: Simplify the integral.}

This simplifies to
\[E_1^{(1)} = \frac{2V_0}{L} \int_0^L \sin^2\left(\frac{\pi x}{L}\right) \sin\left(\frac{2\pi x}{L}\right) \, dx.\]

\textit{Step 5: Evaluate the integral.}
We know that
\[ \int_0^L \sin^2\left(\frac{\pi x}{L}\right) \sin\left(\frac{2\pi x}{L}\right) \, dx = 0, \]
from orthogonality.

\textit{Step 6: State the first-order correction.}
Thus,
\[E_1^{(1)} = \frac{2V_0}{L} \cdot 0 = 0.\]
The first-order correction to the ground state energy is zero.

\textbf{(b)}

\textit{Step 1: Apply the formula for the first-order energy correction.}
For a general state \(n\), the first-order energy correction is given by
\[E_n^{(1)} = \langle \phi_n | \hat{H}' | \phi_n \rangle = \int_0^L \phi_n^*(x) \hat{H}' \phi_n(x) \, dx.\]

\textit{Step 2: Substitute the given perturbation and wave function.}
Substituting the given perturbation and the \(n\)th state wave function, we get
\[E_n^{(1)} = \int_0^L \left(\sqrt{\frac{2}{L}} \sin\left(\frac{n\pi x}{L}\right)\right) L \alpha \delta\left(x - \frac{L}{2}\right) \left(\sqrt{\frac{2}{L}} \sin\left(\frac{n\pi x}{L}\right)\right) \, dx.\]

\textit{Step 3: Simplify the integral using the properties of the Dirac delta function.}
The Dirac delta function has the property that
\[\int_a^b f(x) \delta(x-c) \, dx = f(c)\]
if \(a < c < b\), and 0 otherwise.  Therefore,
\[E_n^{(1)} = \frac{2}{L} \cdot L \alpha \sin^2\left(\frac{n\pi}{L} \cdot \frac{L}{2}\right) = 2\alpha \sin^2\left(\frac{n\pi}{2}\right).\]

\textit{Step 4: Analyze the result.}
We see that \(\sin^2\left(\frac{n\pi}{2}\right)\) is 1 when \(n\) is odd and 0 when \(n\) is even.

\textit{Step 5: State the first-order correction.}
Thus, the first-order correction to the energy eigenvalues is
\[ E_n^{(1)} = 
\begin{cases}
2\alpha & \text{if } n \text{ is odd}, \\
0 & \text{if } n \text{ is even}.
\end{cases}
\]
This can also be expressed as:
\[ E_n^{(1)} = 2\alpha \left(\frac{1-(-1)^n}{2}\right) \]

\subsubsection{Solution 3}
Okay, here's my solution to Question 5, formatted as requested:

\textbf{Question 5}

\textbf{(a)}

\textit{Step 1: Identify the relevant equation for first-order perturbation theory.}

The first-order correction to the energy of the \(n\)-th state is given by

\[
E_n^{(1)} = \langle \phi_n | \hat{H}' | \phi_n \rangle,
\]

where \(\hat{H}'\) is the perturbation Hamiltonian and \(\phi_n\) is the unperturbed wavefunction.

\textit{Step 2: Apply the equation to the ground state.}

We are asked to find the first-order correction to the ground state energy, so we need to calculate \(E_1^{(1)}\). The ground state corresponds to \(n=1\), thus \(\phi_1(x) = \sqrt{\frac{2}{L}} \sin\left(\frac{\pi x}{L}\right)\).

\textit{Step 3: Substitute the given perturbation and ground state wavefunction into the equation.}

Substituting the given perturbation \(\hat{H}' = V_0 \sin\left(\frac{2\pi x}{L}\right)\) and the ground state wavefunction into the equation, we get:

\[
E_1^{(1)} = \int_0^L \left(\sqrt{\frac{2}{L}} \sin\left(\frac{\pi x}{L}\right)\right) V_0 \sin\left(\frac{2\pi x}{L}\right) \left(\sqrt{\frac{2}{L}} \sin\left(\frac{\pi x}{L}\right)\right) dx
\]

\textit{Step 4: Simplify and evaluate the integral.}

Simplifying the expression, we have:

\[
E_1^{(1)} = \frac{2V_0}{L} \int_0^L \sin^2\left(\frac{\pi x}{L}\right) \sin\left(\frac{2\pi x}{L}\right) dx
\]

Using the trigonometric identity \(\sin(2\theta) = 2\sin(\theta)\cos(\theta)\), we can rewrite the integral as:

\[
E_1^{(1)} = \frac{4V_0}{L} \int_0^L \sin^3\left(\frac{\pi x}{L}\right) \cos\left(\frac{\pi x}{L}\right) dx
\]

Now, let \(u = \sin\left(\frac{\pi x}{L}\right)\), then \(du = \frac{\pi}{L}\cos\left(\frac{\pi x}{L}\right)dx\). The limits of integration change from \(x=0\) to \(u=0\) and from \(x=L\) to \(u=0\). Therefore, the integral becomes:

\[
E_1^{(1)} = \frac{4V_0}{\pi} \int_0^0 u^3 du = 0
\]

\textit{Step 5: State the final answer.}

Therefore, the first-order correction to the ground state energy is:

\[
E_1^{(1)} = 0
\]

\textbf{(b)}

\textit{Step 1: Apply the equation for first-order perturbation theory to the general case.}

We need to calculate \(E_n^{(1)}\) for a general \(n\). The unperturbed wavefunction is \(\phi_n(x) = \sqrt{\frac{2}{L}} \sin\left(\frac{n \pi x}{L}\right)\).

\textit{Step 2: Substitute the given perturbation and general wavefunction into the equation.}

Substituting the given perturbation \(\hat{H}' = L \alpha \delta\left(x - \frac{L}{2}\right)\) and the general wavefunction into the equation, we have:

\[
E_n^{(1)} = \int_0^L \left(\sqrt{\frac{2}{L}} \sin\left(\frac{n \pi x}{L}\right)\right) L \alpha \delta\left(x - \frac{L}{2}\right) \left(\sqrt{\frac{2}{L}} \sin\left(\frac{n \pi x}{L}\right)\right) dx
\]

\textit{Step 3: Simplify the expression using the properties of the Dirac delta function.}

Simplifying, we get:

\[
E_n^{(1)} = 2 \alpha \int_0^L \sin^2\left(\frac{n \pi x}{L}\right) \delta\left(x - \frac{L}{2}\right) dx
\]

The sifting property of the Dirac delta function states that \(\int_a^b f(x) \delta(x-c) dx = f(c)\) if \(a < c < b\), and \(0\) otherwise. In our case, \(a=0\), \(b=L\), and \(c=\frac{L}{2}\), which satisfies \(0 < \frac{L}{2} < L\).

\textit{Step 4: Evaluate the integral using the sifting property.}

Applying the sifting property, we obtain:

\[
E_n^{(1)} = 2 \alpha \sin^2\left(\frac{n \pi}{L} \cdot \frac{L}{2}\right) = 2 \alpha \sin^2\left(\frac{n \pi}{2}\right)
\]

\textit{Step 5: State the final answer.}

Therefore, the first-order correction to the energy eigenvalue \(E_n\) is:

\[
E_n^{(1)} = 2 \alpha \sin^2\left(\frac{n \pi}{2}\right)
\]

This can be further simplified. For even \(n\), \(\sin\left(\frac{n\pi}{2}\right) = 0\). For odd \(n\), \(\sin\left(\frac{n\pi}{2}\right) = \pm 1\), so \(\sin^2\left(\frac{n\pi}{2}\right) = 1\). Thus, we can write:

\[
E_n^{(1)} = 
\begin{cases}
2\alpha & \text{for } n \text{ odd} \\
0 & \text{for } n \text{ even}
\end{cases}
\]

\subsection{Question 6}

Let \(\hat{L}\) and \(\hat{S}\) be the angular momentum and spin angular momentum quantum operators.

\textbf{(a)} Given that \(\hat{J} = \hat{L} + \hat{S}\), show that
\[
\hat{J}^2 = \hat{L}^2 + \hat{S}^2 + \hat{L}_+ \hat{S}_- + \hat{L}_- \hat{S}_+ + 2 \hat{L}_z \hat{S}_z
\]
Where you may use \(\hat{L}_\pm = \hat{L}_x \pm i \hat{L}_y\) and \(\hat{S}_\pm = \hat{S}_x \pm i \hat{S}_y\).

\textbf{(b)} Consider the state \(|l, m; s, m_s\rangle = |l, m\rangle |s, m_s\rangle\). Here \(l\) is a quantum number of \(\hat{L}^2\), \(m\) is a quantum number of \(\hat{L}_z\), \(s\) is a quantum number of \(\hat{S}^2\), and \(m_s\) is a quantum number of \(\hat{S}_z\). Show that the state \(|l, -l; s, -s\rangle\) is an eigenvector of \(\hat{J}^2\) and find the corresponding eigenvalue.

\textbf{(c)} Now consider an operator \(\hat{O} = a \hat{L}^2 + b \hat{S}_+ \hat{L}_z\), where \(a, b\) are constants. You are given the following result:
\[
\hat{S}_+ |s, m_s\rangle = \hbar \sqrt{s(s+1) - m_s(m_s + 1)} |s, m_s + 1\rangle
\]
Find the matrix representation of \(\hat{O}\) for a chosen basis of kets \(|l = 1, m; s = \frac{1}{2}, m_s\rangle\).

\paragraph{Prompt: \\} 
Answer the given question as a top physics student, giving step-by-step reasoning. Format your answer as LaTeX code without using sub-sections. Available marks: a) [4 marks], b) [4 marks], c) [9 marks].

\subsubsection{Solution 1}
Okay, here are the solutions to Question 6, formatted as requested:

\textbf{Question 6}

\textbf{(a)} We are given that $\hat{J} = \hat{L} + \hat{S}$. Thus,
\[
\hat{J}^2 = (\hat{L} + \hat{S}) \cdot (\hat{L} + \hat{S}) = \hat{L}^2 + \hat{S}^2 + \hat{L} \cdot \hat{S} + \hat{S} \cdot \hat{L}
\]
Since components of \(\hat{L}\) and \(\hat{S}\) commute, \(\hat{L} \cdot \hat{S} = \hat{S} \cdot \hat{L}\). Therefore,
\[
\hat{J}^2 = \hat{L}^2 + \hat{S}^2 + 2 \hat{L} \cdot \hat{S} = \hat{L}^2 + \hat{S}^2 + 2(\hat{L}_x \hat{S}_x + \hat{L}_y \hat{S}_y + \hat{L}_z \hat{S}_z)
\]
We are given \(\hat{L}_\pm = \hat{L}_x \pm i \hat{L}_y\) and \(\hat{S}_\pm = \hat{S}_x \pm i \hat{S}_y\). From these, we can write
\[
\hat{L}_x = \frac{1}{2} (\hat{L}_+ + \hat{L}_-) \quad \text{and} \quad \hat{L}_y = \frac{1}{2i} (\hat{L}_+ - \hat{L}_-)
\]
\[
\hat{S}_x = \frac{1}{2} (\hat{S}_+ + \hat{S}_-) \quad \text{and} \quad \hat{S}_y = \frac{1}{2i} (\hat{S}_+ - \hat{S}_-)
\]
Substituting these into the expression for $\hat{J}^2$,
\[
\hat{J}^2 = \hat{L}^2 + \hat{S}^2 + 2\left[\frac{1}{4} (\hat{L}_+ + \hat{L}_-)(\hat{S}_+ + \hat{S}_-) - \frac{1}{4} (\hat{L}_+ - \hat{L}_-)(\hat{S}_+ - \hat{S}_-) + \hat{L}_z \hat{S}_z\right]
\]
Expanding and simplifying,
\[
\hat{J}^2 = \hat{L}^2 + \hat{S}^2 + \frac{1}{2} (\hat{L}_+ \hat{S}_- + \hat{L}_- \hat{S}_+) + \frac{1}{2} (\hat{L}_- \hat{S}_+ + \hat{L}_+ \hat{S}_-) + 2 \hat{L}_z \hat{S}_z
\]
Finally,
\[
\hat{J}^2 = \hat{L}^2 + \hat{S}^2 + \hat{L}_+ \hat{S}_- + \hat{L}_- \hat{S}_+ + 2 \hat{L}_z \hat{S}_z
\]

\textbf{(b)} We want to show that \(|l, -l; s, -s\rangle\) is an eigenvector of \(\hat{J}^2\). We use the expression for \(\hat{J}^2\) derived in part (a).
\[
\hat{J}^2 |l, -l; s, -s\rangle = (\hat{L}^2 + \hat{S}^2 + \hat{L}_+ \hat{S}_- + \hat{L}_- \hat{S}_+ + 2 \hat{L}_z \hat{S}_z) |l, -l; s, -s\rangle
\]
Now, we know that \(\hat{L}_- |l, -l\rangle = 0\) and \(\hat{S}_- |s, -s\rangle = 0\). Thus,
\[
\hat{J}^2 |l, -l; s, -s\rangle = (\hat{L}^2 + \hat{S}^2 + 2 \hat{L}_z \hat{S}_z) |l, -l; s, -s\rangle
\]
Applying the operators,
\[
\hat{J}^2 |l, -l; s, -s\rangle = (\hbar^2 l(l+1) + \hbar^2 s(s+1) + 2(-l)(-s)\hbar^2) |l, -l; s, -s\rangle
\]
\[
\hat{J}^2 |l, -l; s, -s\rangle = \hbar^2 (l(l+1) + s(s+1) + 2ls) |l, -l; s, -s\rangle
\]
\[
\hat{J}^2 |l, -l; s, -s\rangle = \hbar^2 (l+s)(l+s+1) |l, -l; s, -s\rangle
\]
Thus, \(|l, -l; s, -s\rangle\) is an eigenvector of \(\hat{J}^2\) with eigenvalue \(\hbar^2 (l+s)(l+s+1)\).

\textbf{(c)} We are given \(\hat{O} = a \hat{L}^2 + b \hat{S}_+ \hat{L}_z\), and the basis kets are \(|l=1, m; s=\frac{1}{2}, m_s\rangle\). The possible values of \(m\) are \(-1, 0, 1\) and \(m_s\) are \(-\frac{1}{2}, \frac{1}{2}\). The basis states are:
\[
|1, -1; \frac{1}{2}, -\frac{1}{2}\rangle, \quad |1, -1; \frac{1}{2}, \frac{1}{2}\rangle, \quad |1, 0; \frac{1}{2}, -\frac{1}{2}\rangle, \quad |1, 0; \frac{1}{2}, \frac{1}{2}\rangle, \quad |1, 1; \frac{1}{2}, -\frac{1}{2}\rangle, \quad |1, 1; \frac{1}{2}, \frac{1}{2}\rangle
\]
We are given that \(\hat{S}_+ |s, m_s\rangle = \hbar \sqrt{s(s+1) - m_s(m_s+1)} |s, m_s+1\rangle\). We need to evaluate the matrix elements \(\langle l', m'; s', m_s' | \hat{O} | l, m; s, m_s \rangle\).

For the given operator \(\hat{O}\), the only non-zero matrix elements will occur when \(m'_s = m_s + 1\), due to the \(\hat{S}_+\) operator.
The only non-zero matrix elements are:
\[
\langle 1, -1; \frac{1}{2}, \frac{1}{2} | \hat{O} | 1, -1; \frac{1}{2}, -\frac{1}{2} \rangle = \langle 1, -1; \frac{1}{2}, \frac{1}{2} | a \hat{L}^2 + b \hat{S}_+ \hat{L}_z | 1, -1; \frac{1}{2}, -\frac{1}{2} \rangle
\]
\[
\langle 1, 0; \frac{1}{2}, \frac{1}{2} | \hat{O} | 1, 0; \frac{1}{2}, -\frac{1}{2} \rangle = \langle 1, 0; \frac{1}{2}, \frac{1}{2} | a \hat{L}^2 + b \hat{S}_+ \hat{L}_z | 1, 0; \frac{1}{2}, -\frac{1}{2} \rangle
\]
\[
\langle 1, 1; \frac{1}{2}, \frac{1}{2} | \hat{O} | 1, 1; \frac{1}{2}, -\frac{1}{2} \rangle = \langle 1, 1; \frac{1}{2}, \frac{1}{2} | a \hat{L}^2 + b \hat{S}_+ \hat{L}_z | 1, 1; \frac{1}{2}, -\frac{1}{2} \rangle
\]

\[
\langle 1, -1; \frac{1}{2}, \frac{1}{2} | \hat{O} | 1, -1; \frac{1}{2}, -\frac{1}{2} \rangle = 2a\hbar^2 + b\hbar\sqrt{\frac{1}{2}(\frac{3}{2}) - (-\frac{1}{2})(\frac{1}{2})} (-\hbar) = 2a\hbar^2 - b\hbar^2
\]
\[
\langle 1, 0; \frac{1}{2}, \frac{1}{2} | \hat{O} | 1, 0; \frac{1}{2}, -\frac{1}{2} \rangle = 2a\hbar^2 + b\hbar\sqrt{\frac{1}{2}(\frac{3}{2}) - (-\frac{1}{2})(\frac{1}{2})} (0) = 2a\hbar^2
\]
\[
\langle 1, 1; \frac{1}{2}, \frac{1}{2} | \hat{O} | 1, 1; \frac{1}{2}, -\frac{1}{2} \rangle = 2a\hbar^2 + b\hbar\sqrt{\frac{1}{2}(\frac{3}{2}) - (-\frac{1}{2})(\frac{1}{2})} (\hbar) = 2a\hbar^2 + b\hbar^2
\]

Thus, we order the basis states as \(|1, -1; \frac{1}{2}, -\frac{1}{2}\rangle, |1, -1; \frac{1}{2}, \frac{1}{2}\rangle, |1, 0; \frac{1}{2}, -\frac{1}{2}\rangle, |1, 0; \frac{1}{2}, \frac{1}{2}\rangle, |1, 1; \frac{1}{2}, -\frac{1}{2}\rangle, |1, 1; \frac{1}{2}, \frac{1}{2}\rangle\).
The matrix representation of \(\hat{O}\) is:
\[
\begin{pmatrix}
2a\hbar^2 & 0 & 0 & 0 & 0 & 0 \\
-b\hbar^2 & 2a\hbar^2 & 0 & 0 & 0 & 0 \\
0 & 0 & 2a\hbar^2 & 0 & 0 & 0 \\
0 & 0 & 0 & 2a\hbar^2 & 0 & 0 \\
0 & 0 & 0 & 0 & 2a\hbar^2 & 0 \\
0 & 0 & 0 & 0 & b\hbar^2 & 2a\hbar^2
\end{pmatrix}
\]

\subsubsection{Solution 2}
Okay, here are the solutions to Question 6, formatted as requested:

\textbf{Question 6}

\textbf{(a)} We are given that $\hat{J} = \hat{L} + \hat{S}$. Thus,
\[
\hat{J}^2 = (\hat{L} + \hat{S}) \cdot (\hat{L} + \hat{S}) = \hat{L}^2 + \hat{S}^2 + \hat{L} \cdot \hat{S} + \hat{S} \cdot \hat{L}
\]
Since $\hat{L}$ and $\hat{S}$ operate on different vector spaces, they commute. Thus, $[\hat{L}_i, \hat{S}_j] = 0$ for $i, j = x, y, z$. Then $\hat{L} \cdot \hat{S} = \hat{S} \cdot \hat{L}$.

We have
\[
\hat{J}^2 = \hat{L}^2 + \hat{S}^2 + 2\hat{L} \cdot \hat{S} = \hat{L}^2 + \hat{S}^2 + 2(\hat{L}_x \hat{S}_x + \hat{L}_y \hat{S}_y + \hat{L}_z \hat{S}_z)
\]
We can rewrite $\hat{L}_x$ and $\hat{L}_y$ in terms of $\hat{L}_+$ and $\hat{L}_-$ as follows:
\[
\hat{L}_x = \frac{1}{2} (\hat{L}_+ + \hat{L}_-), \quad \hat{L}_y = \frac{1}{2i} (\hat{L}_+ - \hat{L}_-)
\]
Similarly,
\[
\hat{S}_x = \frac{1}{2} (\hat{S}_+ + \hat{S}_-), \quad \hat{S}_y = \frac{1}{2i} (\hat{S}_+ - \hat{S}_-)
\]
Substituting into the expression for $\hat{J}^2$,
\[
\begin{aligned}
\hat{J}^2 &= \hat{L}^2 + \hat{S}^2 + 2 \left[ \frac{1}{4} (\hat{L}_+ + \hat{L}_-) (\hat{S}_+ + \hat{S}_-) - \frac{1}{4} (\hat{L}_+ - \hat{L}_-) (\hat{S}_+ - \hat{S}_-) + \hat{L}_z \hat{S}_z \right] \\
&= \hat{L}^2 + \hat{S}^2 + 2 \left[ \frac{1}{4} (2\hat{L}_+ \hat{S}_- + 2\hat{L}_- \hat{S}_+) + \hat{L}_z \hat{S}_z \right] \\
&= \hat{L}^2 + \hat{S}^2 + \hat{L}_+ \hat{S}_- + \hat{L}_- \hat{S}_+ + 2 \hat{L}_z \hat{S}_z
\end{aligned}
\]

\textbf{(b)} We are given the state \(|l, -l; s, -s\rangle\). We want to apply $\hat{J}^2$ to this state.
\[
\begin{aligned}
\hat{J}^2 |l, -l; s, -s\rangle &= (\hat{L}^2 + \hat{S}^2 + \hat{L}_+ \hat{S}_- + \hat{L}_- \hat{S}_+ + 2 \hat{L}_z \hat{S}_z) |l, -l; s, -s\rangle \\
&= \hat{L}^2 |l, -l; s, -s\rangle + \hat{S}^2 |l, -l; s, -s\rangle + \hat{L}_+ \hat{S}_- |l, -l; s, -s\rangle \\
&+ \hat{L}_- \hat{S}_+ |l, -l; s, -s\rangle + 2 \hat{L}_z \hat{S}_z |l, -l; s, -s\rangle
\end{aligned}
\]
Using the properties of the operators:
\[
\begin{aligned}
\hat{L}^2 |l, -l; s, -s\rangle &= \hbar^2 l(l+1) |l, -l; s, -s\rangle \\
\hat{S}^2 |l, -l; s, -s\rangle &= \hbar^2 s(s+1) |l, -l; s, -s\rangle \\
\hat{L}_- |l, -l; s, -s\rangle &= 0, \quad \hat{S}_- |l, -l; s, -s\rangle = 0 \\
\hat{L}_+ \hat{S}_- |l, -l; s, -s\rangle &= 0 \\
\hat{L}_- \hat{S}_+ |l, -l; s, -s\rangle &= 0 \\
\hat{L}_z |l, -l; s, -s\rangle &= -l \hbar |l, -l; s, -s\rangle \\
\hat{S}_z |l, -l; s, -s\rangle &= -s \hbar |l, -l; s, -s\rangle \\
2 \hat{L}_z \hat{S}_z |l, -l; s, -s\rangle &= 2 (-l \hbar) (-s \hbar) |l, -l; s, -s\rangle = 2ls \hbar^2 |l, -l; s, -s\rangle
\end{aligned}
\]
Therefore,
\[
\begin{aligned}
\hat{J}^2 |l, -l; s, -s\rangle &= \hbar^2 [l(l+1) + s(s+1) + 2ls] |l, -l; s, -s\rangle \\
&= \hbar^2 (l^2 + l + s^2 + s + 2ls) |l, -l; s, -s\rangle \\
&= \hbar^2 (l+s)(l+s+1) |l, -l; s, -s\rangle
\end{aligned}
\]
Thus, \(|l, -l; s, -s\rangle\) is an eigenvector of \(\hat{J}^2\) with eigenvalue \(\hbar^2 (l+s)(l+s+1)\).

\textbf{(c)} We are given \(\hat{O} = a \hat{L}^2 + b \hat{S}_+ \hat{L}_z\), and we want to find the matrix representation of \(\hat{O}\) for the basis \(|l=1, m; s=\frac{1}{2}, m_s\rangle\). The possible values for \(m\) are \(-1, 0, 1\) and for \(m_s\) are \(-\frac{1}{2}, \frac{1}{2}\). Thus, our basis states are:
\[
|1, -1\rangle |\frac{1}{2}, -\frac{1}{2}\rangle, \quad |1, -1\rangle |\frac{1}{2}, \frac{1}{2}\rangle, \quad |1, 0\rangle |\frac{1}{2}, -\frac{1}{2}\rangle, \quad |1, 0\rangle |\frac{1}{2}, \frac{1}{2}\rangle, \quad |1, 1\rangle |\frac{1}{2}, -\frac{1}{2}\rangle, \quad |1, 1\rangle |\frac{1}{2}, \frac{1}{2}\rangle
\]
We order the basis as
\[
|1, -1; \frac{1}{2}, -\frac{1}{2}\rangle, |1, -1; \frac{1}{2}, \frac{1}{2}\rangle, |1, 0; \frac{1}{2}, -\frac{1}{2}\rangle, |1, 0; \frac{1}{2}, \frac{1}{2}\rangle, |1, 1; \frac{1}{2}, -\frac{1}{2}\rangle, |1, 1; \frac{1}{2}, \frac{1}{2}\rangle
\]
We have \(\hat{L}^2 |1, m\rangle = 2\hbar^2 |1, m\rangle\), \(\hat{L}_z |1, m\rangle = m \hbar |1, m\rangle\), and \(\hat{S}_+ |s, m_s\rangle = \hbar \sqrt{s(s+1) - m_s(m_s+1)} |s, m_s+1\rangle\).
The matrix elements are given by \(\langle l', m'; s', m_s' | \hat{O} | l, m; s, m_s \rangle\).
For the diagonal elements,
\[
\begin{aligned}
\langle 1, m; \frac{1}{2}, m_s | a \hat{L}^2 + b \hat{S}_+ \hat{L}_z | 1, m; \frac{1}{2}, m_s \rangle &= 2 a \hbar^2 + b \langle 1, m; \frac{1}{2}, m_s | \hat{S}_+ \hat{L}_z | 1, m; \frac{1}{2}, m_s \rangle \\
&= 2 a \hbar^2 + b m \hbar \langle 1, m; \frac{1}{2}, m_s | \hat{S}_+ | 1, m; \frac{1}{2}, m_s \rangle \\
&= 2 a \hbar^2
\end{aligned}
\]
For the off-diagonal elements, we have non-zero contributions from \(b \hat{S}_+ \hat{L}_z\) when \(m_s = -\frac{1}{2}\):
\[
\begin{aligned}
\langle 1, m; \frac{1}{2}, \frac{1}{2} | \hat{O} | 1, m; \frac{1}{2}, -\frac{1}{2} \rangle &= \langle 1, m; \frac{1}{2}, \frac{1}{2} | b \hat{S}_+ \hat{L}_z | 1, m; \frac{1}{2}, -\frac{1}{2} \rangle \\
&= b m \hbar \langle 1, m; \frac{1}{2}, \frac{1}{2} | \hat{S}_+ | 1, m; \frac{1}{2}, -\frac{1}{2} \rangle \\
&= b m \hbar^2 \sqrt{\frac{1}{2}(\frac{3}{2}) - (-\frac{1}{2})(\frac{1}{2})} \langle 1, m; \frac{1}{2}, \frac{1}{2} | 1, m; \frac{1}{2}, \frac{1}{2} \rangle \\
&= b m \hbar^2
\end{aligned}
\]
Therefore, the matrix representation of \(\hat{O}\) is:
\[
\begin{pmatrix}
2a\hbar^2 & -b\hbar^2 & 0 & 0 & 0 & 0 \\
0 & 2a\hbar^2 & 0 & 0 & 0 & 0 \\
0 & 0 & 2a\hbar^2 & 0 & 0 & 0 \\
0 & 0 & 0 & 2a\hbar^2 & b\hbar^2 & 0 \\
0 & 0 & 0 & 0 & 2a\hbar^2 & 0 \\
0 & 0 & 0 & 0 & 0 & 2a\hbar^2
\end{pmatrix}
\]

I hope these solutions are clear and well-explained. Let me know if you have any more questions!

\subsubsection{Solution 3}
Okay, here are the solutions to Question 6, formatted as requested:

\textbf{Question 6}

\textbf{(a)} We are given that $\hat{J} = \hat{L} + \hat{S}$. Thus,
\[
\hat{J}^2 = (\hat{L} + \hat{S}) \cdot (\hat{L} + \hat{S}) = \hat{L}^2 + \hat{S}^2 + \hat{L} \cdot \hat{S} + \hat{S} \cdot \hat{L}
\]
Since $\hat{L}$ and $\hat{S}$ operate on different vector spaces, they commute. Thus,
\[
\hat{J}^2 = \hat{L}^2 + \hat{S}^2 + 2\hat{L} \cdot \hat{S} = \hat{L}^2 + \hat{S}^2 + 2(\hat{L}_x \hat{S}_x + \hat{L}_y \hat{S}_y + \hat{L}_z \hat{S}_z)
\]
Now we use $\hat{L}_\pm = \hat{L}_x \pm i \hat{L}_y$ and $\hat{S}_\pm = \hat{S}_x \pm i \hat{S}_y$. This gives us $\hat{L}_x = \frac{1}{2}(\hat{L}_+ + \hat{L}_-)$ and $\hat{L}_y = \frac{1}{2i}(\hat{L}_+ - \hat{L}_-) = \frac{-i}{2}(\hat{L}_+ - \hat{L}_-)$. Similarly, $\hat{S}_x = \frac{1}{2}(\hat{S}_+ + \hat{S}_-)$ and $\hat{S}_y = \frac{-i}{2}(\hat{S}_+ - \hat{S}_-)$. Substituting these expressions into the equation for $\hat{J}^2$:
\[
\hat{J}^2 = \hat{L}^2 + \hat{S}^2 + 2\left[ \frac{1}{4}(\hat{L}_+ + \hat{L}_-)(\hat{S}_+ + \hat{S}_-) - \frac{1}{4}(\hat{L}_+ - \hat{L}_-)(\hat{S}_+ - \hat{S}_-) + \hat{L}_z \hat{S}_z \right]
\]
\[
\hat{J}^2 = \hat{L}^2 + \hat{S}^2 + 2\left[ \frac{1}{2} \hat{L}_+ \hat{S}_- + \frac{1}{2} \hat{L}_- \hat{S}_+ + \hat{L}_z \hat{S}_z \right]
\]
\[
\hat{J}^2 = \hat{L}^2 + \hat{S}^2 + \hat{L}_+ \hat{S}_- + \hat{L}_- \hat{S}_+ + 2 \hat{L}_z \hat{S}_z
\]

\textbf{(b)} We want to apply $\hat{J}^2$ to the state $|l, -l; s, -s\rangle$. Using the result from part (a):
\[
\hat{J}^2 |l, -l; s, -s\rangle = (\hat{L}^2 + \hat{S}^2 + \hat{L}_+ \hat{S}_- + \hat{L}_- \hat{S}_+ + 2 \hat{L}_z \hat{S}_z) |l, -l; s, -s\rangle
\]
We know that $\hat{L}^2 |l, -l\rangle = \hbar^2 l(l+1) |l, -l\rangle$ and $\hat{S}^2 |s, -s\rangle = \hbar^2 s(s+1) |s, -s\rangle$. Also, $\hat{L}_z |l, -l\rangle = -l\hbar |l, -l\rangle$ and $\hat{S}_z |s, -s\rangle = -s\hbar |s, -s\rangle$.
Furthermore, since $-l$ and $-s$ are the minimum possible values of $m$ and $m_s$, we have $\hat{L}_- |l, -l\rangle = 0$ and $\hat{S}_- |s, -s\rangle = 0$. Thus:
\[
\hat{J}^2 |l, -l; s, -s\rangle = (\hbar^2 l(l+1) + \hbar^2 s(s+1) + 0 + 0 + 2(-l\hbar)(-s\hbar)) |l, -l; s, -s\rangle
\]
\[
\hat{J}^2 |l, -l; s, -s\rangle = (\hbar^2 l(l+1) + \hbar^2 s(s+1) + 2\hbar^2 ls) |l, -l; s, -s\rangle
\]
\[
\hat{J}^2 |l, -l; s, -s\rangle = \hbar^2 (l^2 + l + s^2 + s + 2ls) |l, -l; s, -s\rangle
\]
\[
\hat{J}^2 |l, -l; s, -s\rangle = \hbar^2 (l+s)(l+s+1) |l, -l; s, -s\rangle
\]
Therefore, $|l, -l; s, -s\rangle$ is an eigenvector of $\hat{J}^2$ with eigenvalue $\hbar^2 (l+s)(l+s+1)$.

\textbf{(c)} We have $\hat{O} = a \hat{L}^2 + b \hat{S}_+ \hat{L}_z$ and we want to find its matrix representation for $l=1$ and $s=\frac{1}{2}$. The basis states are $|1, 1; \frac{1}{2}, \frac{1}{2}\rangle$, $|1, 1; \frac{1}{2}, -\frac{1}{2}\rangle$, $|1, 0; \frac{1}{2}, \frac{1}{2}\rangle$, $|1, 0; \frac{1}{2}, -\frac{1}{2}\rangle$, $|1, -1; \frac{1}{2}, \frac{1}{2}\rangle$, $|1, -1; \frac{1}{2}, -\frac{1}{2}\rangle$.
We are given $\hat{S}_+ |s, m_s\rangle = \hbar \sqrt{s(s+1) - m_s(m_s + 1)} |s, m_s + 1\rangle$.
Let's calculate the matrix elements. We denote the basis states as $|m, m_s\rangle$ for brevity.
$\langle m', m_s' | \hat{O} | m, m_s \rangle = \langle m', m_s' | a \hat{L}^2 | m, m_s \rangle + \langle m', m_s' | b \hat{S}_+ \hat{L}_z | m, m_s \rangle$
$\langle m', m_s' | \hat{O} | m, m_s \rangle = a \hbar^2 l(l+1) \delta_{m,m'} \delta_{m_s, m_s'} + b \langle m', m_s' | \hat{S}_+ (m \hbar) | m, m_s \rangle$
$\langle m', m_s' | \hat{O} | m, m_s \rangle = a \hbar^2 (1)(1+1) \delta_{m,m'} \delta_{m_s, m_s'} + b m \hbar \langle m', m_s' | \hat{S}_+ | m, m_s \rangle$
$\langle m', m_s' | \hat{O} | m, m_s \rangle = 2a \hbar^2 \delta_{m,m'} \delta_{m_s, m_s'} + b m \hbar \left( \hbar \sqrt{\frac{1}{2}(\frac{1}{2}+1) - m_s(m_s + 1)} \delta_{m, m'} \delta_{m_s'+1, m_s} \right)$
$\langle m', m_s' | \hat{O} | m, m_s \rangle = 2a \hbar^2 \delta_{m,m'} \delta_{m_s, m_s'} + b m \hbar^2 \sqrt{\frac{3}{4} - m_s(m_s + 1)} \delta_{m, m'} \delta_{m_s'+1, m_s}$
The only non-zero matrix elements from the second term will be when $m_s = -\frac{1}{2}$, thus $m_s' = \frac{1}{2}$.
The matrix representation in the basis $\{ |1,\frac{1}{2}\rangle, |1, -\frac{1}{2}\rangle, |0, \frac{1}{2}\rangle, |0, -\frac{1}{2}\rangle, |-1, \frac{1}{2}\rangle, |-1, -\frac{1}{2}\rangle \}$ is:
\[
\hat{O} = \hbar^2 \begin{pmatrix}
2a & 0 & 0 & 0 & 0 & 0 \\
b & 2a & 0 & 0 & 0 & 0 \\
0 & 0 & 2a & 0 & 0 & 0 \\
0 & 0 & 0 & 2a & 0 & 0 \\
0 & 0 & 0 & 0 & 2a & 0 \\
0 & 0 & 0 & 0 & -b & 2a
\end{pmatrix}
\]



\subsection{Question 7}
A beam of particles each of mass $m$ moves in a space with potential energy $V(x) = 0$, described by the wavefunction $\psi(x) = A e^{ikx}$.  

a) What is the corresponding time-dependent solution $\Psi(x,t)$? Show that the probability per unit length of finding a particle is independent of both space and time.  

b) Evaluate the particle flux  
\[
\Gamma = -\frac{i \hbar}{2m} \left[ \Psi^* \frac{\partial \Psi}{\partial x} - \Psi \frac{\partial \Psi^*}{\partial x} \right]
\]
for the state $\Psi$, giving a physical interpretation of the result in terms of the velocity of the particles.


\paragraph{Prompt: \\} 
Answer the given question as a top physics student, giving step-by-step reasoning. Format your answer as LaTeX code without using sub-sections. Available marks: a) [3 marks], b) [5 marks].


\subsubsection{Solution 1}
Okay, here is the solution to Question 7, formatted as requested:

\textbf{Question 7}

\textbf{a)}

The time-independent Schrödinger equation is given by

\[
-\frac{\hbar^2}{2m} \frac{d^2 \psi(x)}{dx^2} + V(x) \psi(x) = E \psi(x)
\]

Since $V(x) = 0$, the equation becomes

\[
-\frac{\hbar^2}{2m} \frac{d^2 \psi(x)}{dx^2} = E \psi(x)
\]

Given $\psi(x) = A e^{ikx}$, we have

\[
\frac{d \psi(x)}{dx} = ik A e^{ikx}
\]

\[
\frac{d^2 \psi(x)}{dx^2} = -k^2 A e^{ikx} = -k^2 \psi(x)
\]

Substituting this into the Schrödinger equation, we get

\[
-\frac{\hbar^2}{2m} (-k^2 \psi(x)) = E \psi(x)
\]

Thus, the energy is

\[
E = \frac{\hbar^2 k^2}{2m}
\]

The time-dependent solution $\Psi(x,t)$ can be written as

\[
\Psi(x,t) = \psi(x) e^{-iEt/\hbar} = A e^{ikx} e^{-i(\hbar^2 k^2 / 2m)t/\hbar} = A e^{i(kx - \omega t)}
\]

where $\omega = \frac{\hbar k^2}{2m}$.

The probability density is given by $|\Psi(x,t)|^2$:

\[
|\Psi(x,t)|^2 = \Psi^*(x,t) \Psi(x,t) = (A^* e^{-i(kx - \omega t)}) (A e^{i(kx - \omega t)}) = |A|^2
\]

This is independent of both space ($x$) and time ($t$). Hence the probability per unit length of finding a particle is independent of both space and time.

\textbf{b)}

The particle flux $\Gamma$ is given by

\[
\Gamma = -\frac{i \hbar}{2m} \left[ \Psi^* \frac{\partial \Psi}{\partial x} - \Psi \frac{\partial \Psi^*}{\partial x} \right]
\]

We have $\Psi(x,t) = A e^{i(kx - \omega t)}$ and $\Psi^*(x,t) = A^* e^{-i(kx - \omega t)}$.

Then

\[
\frac{\partial \Psi}{\partial x} = ik A e^{i(kx - \omega t)} = ik \Psi
\]

\[
\frac{\partial \Psi^*}{\partial x} = -ik A^* e^{-i(kx - \omega t)} = -ik \Psi^*
\]

Substituting these into the expression for $\Gamma$, we get

\[
\Gamma = -\frac{i \hbar}{2m} \left[ \Psi^* (ik \Psi) - \Psi (-ik \Psi^*) \right] = -\frac{i \hbar}{2m} \left[ ik |\Psi|^2 + ik |\Psi|^2 \right]
\]

\[
\Gamma = -\frac{i \hbar}{2m} (2ik |\Psi|^2) = \frac{\hbar k}{m} |\Psi|^2 = \frac{\hbar k}{m} |A|^2
\]

The physical interpretation of this result is related to the velocity of the particles. The velocity $v$ of a particle is given by $p = mv = \hbar k$, so $v = \frac{\hbar k}{m}$.

Thus, $\Gamma = v |A|^2$. This represents the flux of particles, which is the product of their velocity and their probability density. It tells us the number of particles per unit time passing through a given point.  The higher the probability of finding a particle, or the faster the particle is moving, the larger the flux.


\subsubsection{Solution 2}
Okay, here is the solution to Question 7, formatted as requested:

\textbf{Question 7}

\textbf{a)}

To find the time-dependent solution $\Psi(x,t)$, we use the time-dependent Schrödinger equation (TDSE):

\[
i\hbar \frac{\partial \Psi}{\partial t} = \hat{H} \Psi
\]

Since $V(x) = 0$, the Hamiltonian is simply the kinetic energy operator:

\[
\hat{H} = -\frac{\hbar^2}{2m} \frac{\partial^2}{\partial x^2}
\]

We assume a separable solution of the form:

\[
\Psi(x,t) = \psi(x) T(t)
\]

Plugging this into the TDSE and dividing by $\psi(x) T(t)$, we get:

\[
\frac{i\hbar}{T(t)} \frac{dT(t)}{dt} = \frac{1}{\psi(x)} \left( -\frac{\hbar^2}{2m} \frac{d^2 \psi(x)}{dx^2} \right) = E
\]

Where $E$ is a separation constant, representing the energy of the particle.

The time-dependent part is then:

\[
\frac{dT(t)}{dt} = -\frac{iE}{\hbar} T(t)
\]

Which has the solution:

\[
T(t) = e^{-iEt/\hbar}
\]

Therefore, the time-dependent solution is:

\[
\Psi(x,t) = \psi(x) T(t) = A e^{ikx} e^{-iEt/\hbar}
\]

We can find $E$ by plugging $\psi(x)$ back into the spatial part of the separated TDSE:

\[
-\frac{\hbar^2}{2m} \frac{d^2}{dx^2} (Ae^{ikx}) = E(Ae^{ikx})
\]

\[
-\frac{\hbar^2}{2m} (ik)^2 Ae^{ikx} = E Ae^{ikx}
\]
\[
\frac{\hbar^2 k^2}{2m} = E
\]

Thus,

\[
\Psi(x,t) = A e^{i(kx - \omega t)}, \quad \text{where} \quad \omega = \frac{E}{\hbar} = \frac{\hbar k^2}{2m}
\]

The probability density is given by:

\[
P(x,t) = |\Psi(x,t)|^2 = \Psi^*(x,t) \Psi(x,t) = (A^* e^{-i(kx - \omega t)}) (A e^{i(kx - \omega t)}) = |A|^2
\]

Since the probability density is constant and equal to $|A|^2$, it is independent of both space $x$ and time $t$.

\textbf{b)}

We are given the particle flux formula:

\[
\Gamma = -\frac{i \hbar}{2m} \left[ \Psi^* \frac{\partial \Psi}{\partial x} - \Psi \frac{\partial \Psi^*}{\partial x} \right]
\]

And we have $\Psi(x,t) = A e^{i(kx - \omega t)}$.

First, we find the complex conjugate:

\[
\Psi^*(x,t) = A^* e^{-i(kx - \omega t)}
\]

Next, we calculate the derivatives:

\[
\frac{\partial \Psi}{\partial x} = ik A e^{i(kx - \omega t)}
\]

\[
\frac{\partial \Psi^*}{\partial x} = -ik A^* e^{-i(kx - \omega t)}
\]

Now we substitute these into the flux formula:

\[
\Gamma = -\frac{i \hbar}{2m} \left[ (A^* e^{-i(kx - \omega t)})(ik A e^{i(kx - \omega t)}) - (A e^{i(kx - \omega t)})(-ik A^* e^{-i(kx - \omega t)}) \right]
\]

\[
\Gamma = -\frac{i \hbar}{2m} \left[ ik |A|^2 + ik |A|^2 \right]
\]

\[
\Gamma = -\frac{i \hbar}{2m} (2ik |A|^2) = \frac{\hbar k}{m} |A|^2
\]

We know that the classical momentum is $p = mv$ and in quantum mechanics, $p = \hbar k$. Thus, the velocity is $v = \frac{\hbar k}{m}$. The particle flux can be expressed as:

\[
\Gamma = v |A|^2
\]

\textit{Physical Interpretation}:

The particle flux $\Gamma$ represents the number of particles passing a given point per unit time. The result $\Gamma = v |A|^2$ tells us that the flux is equal to the velocity of the particles $v$ multiplied by the probability density $|A|^2$.  This makes intuitive sense, as a higher velocity will result in more particles passing a point per unit time, and a higher probability density means there are more particles available to move. Thus the number of particles flowing past a point per unit time is proportional to their probability density multiplied by their velocity.


\subsubsection{Solution 3}
Okay, here is the solution to Question 7, formatted as requested:

\textbf{Question 7}

\textbf{a)}

We are given the spatial part of the wavefunction, $\psi(x) = A e^{ikx}$, and we want to find the time-dependent solution $\Psi(x,t)$ for particles of mass $m$ in a region with potential energy $V(x) = 0$.

We know that the time evolution of a wavefunction is generally given by:

\[ \Psi(x,t) = \psi(x) \phi(t) \]

where $\phi(t)$ is the time-dependent part.  For a stationary state with energy $E$, $\phi(t)$ takes the form:

\[ \phi(t) = e^{-iEt/\hbar} \]

Since $V(x) = 0$, the energy $E$ of the particles is purely kinetic, and is given by $E = \frac{p^2}{2m}$, where $p$ is the momentum. The de Broglie relation gives us $p = \hbar k$, so we can write the energy in terms of $k$ as:

\[ E = \frac{(\hbar k)^2}{2m} = \frac{\hbar^2 k^2}{2m} \]

Substituting this into the expression for $\phi(t)$, we get:

\[ \phi(t) = e^{-i(\hbar^2 k^2 / 2m)t/\hbar} = e^{-i \hbar k^2 t / 2m} \]

Therefore, the full time-dependent wavefunction is:

\[ \Psi(x,t) = \psi(x) \phi(t) = A e^{ikx} e^{-i \hbar k^2 t / 2m} = A e^{i(kx - \hbar k^2 t / 2m)} \]

Now, we want to show that the probability per unit length of finding a particle is independent of both space and time. This probability per unit length is given by the probability density, $|\Psi(x,t)|^2$:

\[ |\Psi(x,t)|^2 = \Psi^*(x,t) \Psi(x,t) \]

The complex conjugate of $\Psi(x,t)$ is:

\[ \Psi^*(x,t) = A^* e^{-i(kx - \hbar k^2 t / 2m)} \]

Thus, the probability density is:

\[ |\Psi(x,t)|^2 = A^* e^{-i(kx - \hbar k^2 t / 2m)} \cdot A e^{i(kx - \hbar k^2 t / 2m)} = |A|^2 e^{-i(kx - \hbar k^2 t / 2m) + i(kx - \hbar k^2 t / 2m)} = |A|^2 e^0 = |A|^2 \]

Since $|A|^2$ is a constant, the probability density is independent of both $x$ and $t$.

\textbf{b)}

We are asked to evaluate the particle flux $\Gamma$ for the state $\Psi(x,t)$, where:

\[ \Gamma = -\frac{i \hbar}{2m} \left[ \Psi^* \frac{\partial \Psi}{\partial x} - \Psi \frac{\partial \Psi^*}{\partial x} \right] \]

We already have $\Psi(x,t) = A e^{i(kx - \hbar k^2 t / 2m)}$ and $\Psi^*(x,t) = A^* e^{-i(kx - \hbar k^2 t / 2m)}$. We need to calculate the partial derivatives with respect to $x$:

\[ \frac{\partial \Psi}{\partial x} = A (ik) e^{i(kx - \hbar k^2 t / 2m)} = ik \Psi \]

\[ \frac{\partial \Psi^*}{\partial x} = A^* (-ik) e^{-i(kx - \hbar k^2 t / 2m)} = -ik \Psi^* \]

Now, substitute these into the expression for $\Gamma$:

\[ \Gamma = -\frac{i \hbar}{2m} \left[ \Psi^* (ik \Psi) - \Psi (-ik \Psi^*) \right] = -\frac{i \hbar}{2m} \left[ ik \Psi^* \Psi + ik \Psi \Psi^* \right] \]

\[ \Gamma = -\frac{i \hbar}{2m} (2ik |\Psi|^2) = \frac{\hbar k}{m} |\Psi|^2 \]

We found in part (a) that $|\Psi|^2 = |A|^2$, so:

\[ \Gamma = \frac{\hbar k}{m} |A|^2 \]

To interpret this result, recall that the momentum $p = \hbar k$, and thus the velocity $v$ of the particles is $v = \frac{p}{m} = \frac{\hbar k}{m}$. Therefore, we can write the particle flux as:

\[ \Gamma = v |A|^2 \]

The flux $\Gamma$ represents the number of particles crossing a unit area per unit time.  Since $|A|^2$ represents the probability density (number of particles per unit length), the flux is equal to the velocity of the particles multiplied by their density. This makes physical sense: the faster the particles are moving and the greater their density, the greater the flux will be.


\subsection{Question 8}
Q8.  
The expectation of an operator $\hat{Q}$ in one dimension can be written  
\[
\langle \hat{Q} \rangle = \int_{-\infty}^{\infty} \Psi^*(x) \hat{Q} \Psi(x) \, dx
\]

a) For an operator $\hat{Q}$ which does not vary with time, show the rate of change with time of the expectation value of $\hat{Q}$ can be written  
\[
\frac{d}{dt} \langle \hat{Q} \rangle = \frac{1}{i\hbar} \langle [\hat{Q}, \hat{H}] \rangle
\]  
where $\hat{H}$ is the Hamiltonian operator.

b) A particle of mass $m$ is subject to a time-independent potential $V(x)$. By evaluating $[\hat{X}, \hat{H}]$, where $\hat{X}$ is the position operator, show that  
\[
m \frac{d}{dt} \langle \hat{X} \rangle = \langle \hat{P} \rangle
\]

You may use the additional information:  
Time-dependent Schrödinger equation:  
\[
\frac{\partial \Psi}{\partial t} = \frac{1}{i\hbar} \hat{H} \Psi
\]


\paragraph{Prompt: \\} 
Answer the given question as a top physics student, giving step-by-step reasoning. Format your answer as LaTeX code without using sub-sections. Available marks: a) [5 marks], b) [4 marks].


\subsubsection{Solution 1}

\textbf{Question 8}

\textbf{a)}

We are given the expectation value of an operator $\hat{Q}$:

\[
\langle \hat{Q} \rangle = \int_{-\infty}^{\infty} \Psi^*(x) \hat{Q} \Psi(x) \, dx
\]

We want to find the time derivative of this expectation value. Since $\hat{Q}$ does not vary with time, only $\Psi$ and $\Psi^*$ are time-dependent. Applying the product rule for differentiation:

\[
\frac{d}{dt} \langle \hat{Q} \rangle = \int_{-\infty}^{\infty} \left( \frac{\partial \Psi^*}{\partial t} \hat{Q} \Psi + \Psi^* \hat{Q} \frac{\partial \Psi}{\partial t} \right) dx
\]

We are given the time-dependent Schrödinger equation:

\[
\frac{\partial \Psi}{\partial t} = \frac{1}{i\hbar} \hat{H} \Psi
\]

Taking the complex conjugate of this equation:

\[
\frac{\partial \Psi^*}{\partial t} = -\frac{1}{i\hbar} (\hat{H} \Psi)^* = -\frac{1}{i\hbar} \hat{H}^* \Psi^*
\]

Assuming the Hamiltonian is Hermitian, $\hat{H}^* = \hat{H}$. Therefore,

\[
\frac{\partial \Psi^*}{\partial t} = -\frac{1}{i\hbar} \hat{H} \Psi^*
\]

Substituting these expressions into our time derivative equation:

\[
\frac{d}{dt} \langle \hat{Q} \rangle = \int_{-\infty}^{\infty} \left( -\frac{1}{i\hbar} (\hat{H} \Psi^*) \hat{Q} \Psi + \Psi^* \hat{Q} \frac{1}{i\hbar} \hat{H} \Psi \right) dx
\]

\[
\frac{d}{dt} \langle \hat{Q} \rangle = \frac{1}{i\hbar} \int_{-\infty}^{\infty} \left( -\Psi^* \hat{H} \hat{Q} \Psi + \Psi^* \hat{Q} \hat{H} \Psi \right) dx
\]

We can rewrite this using the fact that $\int_{-\infty}^{\infty} (\hat{H} \Psi^*) \hat{Q} \Psi \, dx = \int_{-\infty}^{\infty} \Psi^* \hat{H} \hat{Q} \Psi \, dx$ because $\hat{H}$ is Hermitian:

\[
\frac{d}{dt} \langle \hat{Q} \rangle = \frac{1}{i\hbar} \int_{-\infty}^{\infty} \Psi^* (\hat{Q} \hat{H} - \hat{H} \hat{Q}) \Psi \, dx
\]

The term inside the parentheses is the commutator $[\hat{Q}, \hat{H}] = \hat{Q} \hat{H} - \hat{H} \hat{Q}$. Thus:

\[
\frac{d}{dt} \langle \hat{Q} \rangle = \frac{1}{i\hbar} \int_{-\infty}^{\infty} \Psi^* [\hat{Q}, \hat{H}] \Psi \, dx
\]

\[
\frac{d}{dt} \langle \hat{Q} \rangle = \frac{1}{i\hbar} \langle [\hat{Q}, \hat{H}] \rangle
\]

\textbf{b)}

We are asked to evaluate $[\hat{X}, \hat{H}]$ where $\hat{H}$ describes a particle of mass $m$ subject to a time-independent potential $V(x)$. The Hamiltonian is given by:

\[
\hat{H} = \frac{\hat{P}^2}{2m} + V(\hat{X})
\]

The position operator is $\hat{X} = x$ and the momentum operator is $\hat{P} = -i\hbar \frac{\partial}{\partial x}$.

We need to calculate the commutator:

\[
[\hat{X}, \hat{H}] = \hat{X}\hat{H} - \hat{H}\hat{X}
\]

\[
[\hat{X}, \hat{H}] = \hat{X} \left( \frac{\hat{P}^2}{2m} + V(\hat{X}) \right) - \left( \frac{\hat{P}^2}{2m} + V(\hat{X}) \right) \hat{X}
\]

Since $[\hat{X}, V(\hat{X})] = 0$ (they commute), we have:

\[
[\hat{X}, \hat{H}] = \frac{1}{2m} (\hat{X}\hat{P}^2 - \hat{P}^2\hat{X})
\]

\[
[\hat{X}, \hat{H}] = \frac{1}{2m} [\hat{X}, \hat{P}^2]
\]

Using the commutator identity $[\hat{A}, \hat{B}\hat{C}] = [\hat{A}, \hat{B}]\hat{C} + \hat{B}[\hat{A}, \hat{C}]$, and given $[\hat{X}, \hat{P}] = i\hbar$:

\[
[\hat{X}, \hat{P}^2] = [\hat{X}, \hat{P}]\hat{P} + \hat{P}[\hat{X}, \hat{P}] = i\hbar \hat{P} + \hat{P} i\hbar = 2i\hbar \hat{P}
\]

Therefore,

\[
[\hat{X}, \hat{H}] = \frac{1}{2m} (2i\hbar \hat{P}) = \frac{i\hbar \hat{P}}{m}
\]

Now, using the result from part (a) with $\hat{Q} = \hat{X}$:

\[
\frac{d}{dt} \langle \hat{X} \rangle = \frac{1}{i\hbar} \langle [\hat{X}, \hat{H}] \rangle
\]

\[
\frac{d}{dt} \langle \hat{X} \rangle = \frac{1}{i\hbar} \left\langle \frac{i\hbar \hat{P}}{m} \right\rangle
\]

\[
\frac{d}{dt} \langle \hat{X} \rangle = \frac{1}{m} \langle \hat{P} \rangle
\]

\[
m \frac{d}{dt} \langle \hat{X} \rangle = \langle \hat{P} \rangle
\]

\subsubsection{Solution 2}
Okay, here is the solution to Question 8, formatted in LaTeX as requested:

```latex
\noindent Q8.

\noindent a) We are given the expectation value of an operator $\hat{Q}$ as

\[
\langle \hat{Q} \rangle = \int_{-\infty}^{\infty} \Psi^*(x) \hat{Q} \Psi(x) \, dx
\]

\noindent We want to find the rate of change of this expectation value with time, $\frac{d}{dt} \langle \hat{Q} \rangle$. We can do this by differentiating the above expression with respect to time:

\[
\frac{d}{dt} \langle \hat{Q} \rangle = \frac{d}{dt} \int_{-\infty}^{\infty} \Psi^*(x) \hat{Q} \Psi(x) \, dx
\]

\noindent Since the limits of integration are time-independent, we can move the time derivative inside the integral. Also, $\hat{Q}$ does not vary with time:

\[
\frac{d}{dt} \langle \hat{Q} \rangle = \int_{-\infty}^{\infty} \left( \frac{\partial \Psi^*}{\partial t} \hat{Q} \Psi + \Psi^* \hat{Q} \frac{\partial \Psi}{\partial t} \right) \, dx
\]

\noindent Using the time-dependent Schrödinger equation, we have:

\[
\frac{\partial \Psi}{\partial t} = \frac{1}{i\hbar} \hat{H} \Psi
\]

\noindent Taking the complex conjugate, we get:

\[
\frac{\partial \Psi^*}{\partial t} = -\frac{1}{i\hbar} (\hat{H} \Psi)^* = -\frac{1}{i\hbar} \hat{H}^* \Psi^* = -\frac{1}{i\hbar} \hat{H} \Psi^*
\]
\noindent (Since $\hat{H}$ is Hermitian, $\hat{H}^* = \hat{H}$)

\noindent Substituting these into our expression for $\frac{d}{dt} \langle \hat{Q} \rangle$:

\[
\frac{d}{dt} \langle \hat{Q} \rangle = \int_{-\infty}^{\infty} \left( -\frac{1}{i\hbar} \hat{H} \Psi^* \hat{Q} \Psi + \Psi^* \hat{Q} \frac{1}{i\hbar} \hat{H} \Psi \right) \, dx
\]

\[
\frac{d}{dt} \langle \hat{Q} \rangle = \frac{1}{i\hbar} \int_{-\infty}^{\infty} \left( \Psi^* \hat{Q} \hat{H} \Psi -  \Psi^* \hat{H} \hat{Q} \Psi \right) \, dx
\]
\noindent (Using $\int \Psi_1^* \hat{H} \Psi_2 = \int (\hat{H} \Psi_1)^* \Psi_2$ because $\hat{H}$ is Hermitian)

\[
\frac{d}{dt} \langle \hat{Q} \rangle = \frac{1}{i\hbar} \int_{-\infty}^{\infty} \Psi^* (\hat{Q} \hat{H} - \hat{H} \hat{Q}) \Psi \, dx
\]

\noindent We recognize the term in the parentheses as the commutator $[\hat{Q}, \hat{H}]$:

\[
\frac{d}{dt} \langle \hat{Q} \rangle = \frac{1}{i\hbar} \int_{-\infty}^{\infty} \Psi^* [\hat{Q}, \hat{H}] \Psi \, dx
\]

\noindent Finally, we can write this as:

\[
\frac{d}{dt} \langle \hat{Q} \rangle = \frac{1}{i\hbar} \langle [\hat{Q}, \hat{H}] \rangle
\]

\noindent This completes the proof for part a).

\noindent b) We are asked to evaluate $[\hat{X}, \hat{H}]$ where $\hat{H}$ is the Hamiltonian for a particle of mass $m$ in a time-independent potential $V(x)$. The Hamiltonian is given by:

\[
\hat{H} = \frac{\hat{P}^2}{2m} + V(x)
\]

\noindent The position and momentum operators are $\hat{X} = x$ and $\hat{P} = -i\hbar \frac{\partial}{\partial x}$ respectively.

\noindent Thus the commutator is:

\[
[\hat{X}, \hat{H}] = \left[ \hat{X}, \frac{\hat{P}^2}{2m} + V(x) \right]
\]

\noindent Since $[\hat{X}, V(x)] = 0$ (position operator commutes with any function of position):

\[
[\hat{X}, \hat{H}] = \left[ \hat{X}, \frac{\hat{P}^2}{2m} \right] = \frac{1}{2m} [\hat{X}, \hat{P}^2]
\]

\noindent Using the commutator identity $[\hat{A}, \hat{B}\hat{C}] = [\hat{A}, \hat{B}] \hat{C} + \hat{B} [\hat{A}, \hat{C}]$:

\[
[\hat{X}, \hat{P}^2] = [\hat{X}, \hat{P}] \hat{P} + \hat{P} [\hat{X}, \hat{P}]
\]

\noindent We know that $[\hat{X}, \hat{P}] = i\hbar$, so:

\[
[\hat{X}, \hat{P}^2] = i\hbar \hat{P} + \hat{P} i\hbar = 2i\hbar \hat{P}
\]

\noindent Therefore:

\[
[\hat{X}, \hat{H}] = \frac{1}{2m} (2i\hbar \hat{P}) = \frac{i\hbar \hat{P}}{m}
\]

\noindent Now, from part a), we have:

\[
\frac{d}{dt} \langle \hat{X} \rangle = \frac{1}{i\hbar} \langle [\hat{X}, \hat{H}] \rangle
\]

\noindent Substituting our result for $[\hat{X}, \hat{H}]$:

\[
\frac{d}{dt} \langle \hat{X} \rangle = \frac{1}{i\hbar} \left\langle \frac{i\hbar \hat{P}}{m} \right\rangle = \frac{1}{m} \langle \hat{P} \rangle
\]

\noindent Multiplying both sides by $m$:

\[
m \frac{d}{dt} \langle \hat{X} \rangle = \langle \hat{P} \rangle
\]

\noindent This completes the proof for part b).
```

\subsubsection{Solution 3}
Okay, here is the solution to Question 8, formatted in LaTeX as requested:

```latex
\noindent Q8.

\noindent (a)

\noindent We begin with the expression for the expectation value of an operator $\hat{Q}$:

\begin{equation}
\langle \hat{Q} \rangle = \int_{-\infty}^{\infty} \Psi^*(x,t) \hat{Q} \Psi(x,t) \, dx
\end{equation}

\noindent Differentiating with respect to time, we get:

\begin{equation}
\frac{d}{dt} \langle \hat{Q} \rangle = \int_{-\infty}^{\infty} \frac{\partial \Psi^*}{\partial t} \hat{Q} \Psi \, dx + \int_{-\infty}^{\infty} \Psi^* \frac{\partial \hat{Q}}{\partial t} \Psi \, dx + \int_{-\infty}^{\infty} \Psi^* \hat{Q} \frac{\partial \Psi}{\partial t} \, dx
\end{equation}

\noindent Since $\hat{Q}$ is time-independent, $\frac{\partial \hat{Q}}{\partial t} = 0$. Thus:

\begin{equation}
\frac{d}{dt} \langle \hat{Q} \rangle = \int_{-\infty}^{\infty} \frac{\partial \Psi^*}{\partial t} \hat{Q} \Psi \, dx + \int_{-\infty}^{\infty} \Psi^* \hat{Q} \frac{\partial \Psi}{\partial t} \, dx
\end{equation}

\noindent From the time-dependent Schrödinger equation, we have:

\begin{equation}
\frac{\partial \Psi}{\partial t} = \frac{1}{i\hbar} \hat{H} \Psi
\end{equation}

\noindent Taking the complex conjugate, we have:

\begin{equation}
\frac{\partial \Psi^*}{\partial t} = -\frac{1}{i\hbar} (\hat{H} \Psi)^* = \frac{1}{i\hbar} \Psi^* \hat{H}^\dagger
\end{equation}

\noindent As the Hamiltonian is Hermitian, $\hat{H}^\dagger = \hat{H}$, so:

\begin{equation}
\frac{\partial \Psi^*}{\partial t} = -\frac{1}{i\hbar} \Psi^* \hat{H}
\end{equation}

\noindent Substituting these into our expression for $\frac{d}{dt} \langle \hat{Q} \rangle$, we get:

\begin{equation}
\frac{d}{dt} \langle \hat{Q} \rangle = \int_{-\infty}^{\infty} \left( -\frac{1}{i\hbar} \Psi^* \hat{H} \right) \hat{Q} \Psi \, dx + \int_{-\infty}^{\infty} \Psi^* \hat{Q} \left( \frac{1}{i\hbar} \hat{H} \Psi \right) \, dx
\end{equation}

\noindent Simplifying, we have:

\begin{equation}
\frac{d}{dt} \langle \hat{Q} \rangle = \frac{1}{i\hbar} \int_{-\infty}^{\infty} \Psi^* (\hat{Q}\hat{H} - \hat{H}\hat{Q}) \Psi \, dx
\end{equation}

\noindent Recognizing that $\hat{Q}\hat{H} - \hat{H}\hat{Q} = [\hat{Q}, \hat{H}]$, we arrive at the desired result:

\begin{equation}
\frac{d}{dt} \langle \hat{Q} \rangle = \frac{1}{i\hbar} \langle [\hat{Q}, \hat{H}] \rangle
\end{equation}

\noindent (b)

\noindent We are asked to evaluate $[\hat{X}, \hat{H}]$. The Hamiltonian for a particle of mass $m$ in a potential $V(x)$ is given by:

\begin{equation}
\hat{H} = \frac{\hat{P}^2}{2m} + V(\hat{X})
\end{equation}

\noindent Thus:

\begin{equation}
[\hat{X}, \hat{H}] = \left[\hat{X}, \frac{\hat{P}^2}{2m} + V(\hat{X})\right] = \left[\hat{X}, \frac{\hat{P}^2}{2m}\right] + [\hat{X}, V(\hat{X})]
\end{equation}

\noindent Since an operator commutes with any function of itself, $[\hat{X}, V(\hat{X})] = 0$. Therefore:

\begin{equation}
[\hat{X}, \hat{H}] = \left[\hat{X}, \frac{\hat{P}^2}{2m}\right] = \frac{1}{2m} [\hat{X}, \hat{P}^2]
\end{equation}

\noindent Using the commutator identity $[\hat{A}, \hat{B}\hat{C}] = [\hat{A}, \hat{B}]\hat{C} + \hat{B}[\hat{A}, \hat{C}]$, we have:

\begin{equation}
[\hat{X}, \hat{P}^2] = [\hat{X}, \hat{P}]\hat{P} + \hat{P}[\hat{X}, \hat{P}] = i\hbar \hat{P} + \hat{P}i\hbar = 2i\hbar \hat{P}
\end{equation}

\noindent Therefore:

\begin{equation}
[\hat{X}, \hat{H}] = \frac{1}{2m} (2i\hbar \hat{P}) = \frac{i\hbar \hat{P}}{m}
\end{equation}

\noindent From part (a), we know:

\begin{equation}
\frac{d}{dt} \langle \hat{X} \rangle = \frac{1}{i\hbar} \langle [\hat{X}, \hat{H}] \rangle
\end{equation}

\noindent Substituting our result for $[\hat{X}, \hat{H}]$, we get:

\begin{equation}
\frac{d}{dt} \langle \hat{X} \rangle = \frac{1}{i\hbar} \left\langle \frac{i\hbar \hat{P}}{m} \right\rangle = \frac{1}{m} \langle \hat{P} \rangle
\end{equation}

\noindent Multiplying both sides by $m$, we arrive at the desired result:

\begin{equation}
m \frac{d}{dt} \langle \hat{X} \rangle = \langle \hat{P} \rangle
\end{equation}
```

Hopefully, this is a clear and thorough explanation that demonstrates the reasoning of a top physics student. Please let me know if you have any more questions!


\subsection{Question 9}
Consider a beam of particles each of mass $m$ with energy $E > 0$ incident from the left, subject to a one-dimensional potential step defined by
\[
V(x) =
\begin{cases}
0 & \text{for } x \leq 0, \\
-V_0 & \text{for } x > 0,
\end{cases}
\]
a) Show that $\psi_1(x) = e^{ik_1x} + Be^{-ik_1x}$ is the general solution in the region $x \leq 0$ and $\psi_2(x) = Ce^{ik_2x}$ is the general solution in the region $x > 0$, where $B$ and $C$ are constants. Define $k_1$, $k_2$ as part of your answer.

b) By applying appropriate boundary conditions, show that
\[
C = \frac{2k_1}{k_1 + k_2}
\]
and
\[
B = \frac{k_1 - k_2}{k_1 + k_2}.
\]

c) Calculate the incident, reflected, and transmitted flux for this scattering potential and derive that the probability for transmission, $T$, and the probability for reflection, $R$, are given by
\[
T = \frac{4k_1k_2}{(k_1 + k_2)^2}
\]
and
\[
R = \frac{(k_1 - k_2)^2}{(k_1 + k_2)^2}.
\]

You may use that the particle flux for a particle beam is given by
\[
\Gamma(x) = -\frac{i\hbar}{2m} \left( \psi^* \frac{d\psi}{dx} - \psi \frac{d\psi^*}{dx} \right).
\]


\paragraph{Prompt: \\} 
Answer the given question as a top physics student, giving step-by-step reasoning. Format your answer as LaTeX code without using sub-sections. Available marks: a) [5 marks], b) [5 marks], c) [7 marks].

\subsubsection{Solution 1}
Okay, here is the solution to Question 9, presented as a top physics student would, with step-by-step reasoning in LaTeX format, without using sub-sections:

\textbf{Question 9}

\textbf{a)}

We begin by considering the time-independent Schrödinger equation for a particle of mass $m$ and energy $E$ in a potential $V(x)$:
\[
-\frac{\hbar^2}{2m} \frac{d^2\psi(x)}{dx^2} + V(x)\psi(x) = E\psi(x).
\]

For the region $x \leq 0$, we have $V(x) = 0$, so the equation becomes:
\[
-\frac{\hbar^2}{2m} \frac{d^2\psi_1(x)}{dx^2} = E\psi_1(x).
\]

Rearranging, we get:
\[
\frac{d^2\psi_1(x)}{dx^2} = -\frac{2mE}{\hbar^2}\psi_1(x).
\]

We define $k_1^2 = \frac{2mE}{\hbar^2}$. Since $E > 0$, $k_1$ is real. The equation now becomes:
\[
\frac{d^2\psi_1(x)}{dx^2} = -k_1^2\psi_1(x).
\]

This is a second-order linear homogeneous differential equation with constant coefficients. The general solution is of the form:
\[
\psi_1(x) = Ae^{ik_1x} + Be^{-ik_1x}.
\]
Where $A$, $B$ are constants. Since the particles are incident from the left, we set $A=1$.

For the region $x > 0$, we have $V(x) = -V_0$, so the Schrödinger equation becomes:
\[
-\frac{\hbar^2}{2m} \frac{d^2\psi_2(x)}{dx^2} - V_0\psi_2(x) = E\psi_2(x).
\]

Rearranging, we obtain:
\[
\frac{d^2\psi_2(x)}{dx^2} = -\frac{2m(E + V_0)}{\hbar^2}\psi_2(x).
\]

We define $k_2^2 = \frac{2m(E + V_0)}{\hbar^2}$. Since $E > 0$ and $V_0 > 0$, $k_2$ is real. The equation then becomes:
\[
\frac{d^2\psi_2(x)}{dx^2} = -k_2^2\psi_2(x).
\]

The general solution for this region is:
\[
\psi_2(x) = Ce^{ik_2x} + De^{-ik_2x}.
\]
Where $C$, $D$ are constants.
We discard the term with $De^{-ik_2x}$ because it represents a wave traveling in the negative $x$ direction, and there are no particles incident from the right.

Thus, the general solutions are:

\[
\psi_1(x) = e^{ik_1x} + Be^{-ik_1x}, \quad k_1 = \sqrt{\frac{2mE}{\hbar^2}},
\]
\[
\psi_2(x) = Ce^{ik_2x}, \quad k_2 = \sqrt{\frac{2m(E + V_0)}{\hbar^2}}.
\]

\textbf{b)}

To find the constants $B$ and $C$, we apply boundary conditions at $x = 0$. The wave function and its first derivative must be continuous at the boundary.

Continuity of $\psi(x)$ at $x = 0$:
\[
\psi_1(0) = \psi_2(0).
\]

Substituting the expressions for $\psi_1(x)$ and $\psi_2(x)$, we get:
\[
1 + B = C. \qquad (1)
\]

Continuity of $\frac{d\psi(x)}{dx}$ at $x = 0$:
\[
\frac{d\psi_1(0)}{dx} = \frac{d\psi_2(0)}{dx}.
\]

Taking the derivatives, we have:
\[
ik_1e^{ik_1x} - ik_1Be^{-ik_1x} |_{x=0} = ik_2Ce^{ik_2x} |_{x=0}.
\]

This simplifies to:
\[
ik_1 - ik_1B = ik_2C.
\]

\[
k_1(1 - B) = k_2C. \qquad (2)
\]

Substituting equation (1) into equation (2), we get:
\[
k_1(1 - B) = k_2(1 + B).
\]

Solving for $B$:
\[
k_1 - k_1B = k_2 + k_2B,
\]
\[
k_1 - k_2 = (k_1 + k_2)B,
\]
\[
B = \frac{k_1 - k_2}{k_1 + k_2}.
\]

Now, substituting the value of $B$ back into equation (1):
\[
C = 1 + \frac{k_1 - k_2}{k_1 + k_2} = \frac{k_1 + k_2 + k_1 - k_2}{k_1 + k_2} = \frac{2k_1}{k_1 + k_2}.
\]

Thus, we have found:
\[
C = \frac{2k_1}{k_1 + k_2},
\]
\[
B = \frac{k_1 - k_2}{k_1 + k_2}.
\]

\textbf{c)}

The particle flux is given by:
\[
\Gamma(x) = -\frac{i\hbar}{2m} \left( \psi^* \frac{d\psi}{dx} - \psi \frac{d\psi^*}{dx} \right).
\]

For the incident wave, $\psi_{inc}(x) = e^{ik_1x}$, we have:
\[
\frac{d\psi_{inc}}{dx} = ik_1e^{ik_1x}, \quad \psi_{inc}^* = e^{-ik_1x}, \quad \frac{d\psi_{inc}^*}{dx} = -ik_1e^{-ik_1x}.
\]

The incident flux is:
\[
\Gamma_{inc} = -\frac{i\hbar}{2m} \left( e^{-ik_1x}(ik_1e^{ik_1x}) - e^{ik_1x}(-ik_1e^{-ik_1x}) \right) = -\frac{i\hbar}{2m} (ik_1 + ik_1) = \frac{\hbar k_1}{m}.
\]

For the reflected wave, $\psi_{ref}(x) = Be^{-ik_1x}$, we have:
\[
\frac{d\psi_{ref}}{dx} = -ik_1Be^{-ik_1x}, \quad \psi_{ref}^* = B^*e^{ik_1x}, \quad \frac{d\psi_{ref}^*}{dx} = ik_1B^*e^{ik_1x}.
\]

The reflected flux is:
\[
\Gamma_{ref} = -\frac{i\hbar}{2m} \left( B^*e^{ik_1x}(-ik_1Be^{-ik_1x}) - Be^{-ik_1x}(ik_1B^*e^{ik_1x}) \right) = -\frac{\hbar k_1}{m} |B|^2.
\]

For the transmitted wave, $\psi_{trans}(x) = Ce^{ik_2x}$, we have:
\[
\frac{d\psi_{trans}}{dx} = ik_2Ce^{ik_2x}, \quad \psi_{trans}^* = C^*e^{-ik_2x}, \quad \frac{d\psi_{trans}^*}{dx} = -ik_2C^*e^{-ik_2x}.
\]

The transmitted flux is:
\[
\Gamma_{trans} = -\frac{i\hbar}{2m} \left( C^*e^{-ik_2x}(ik_2Ce^{ik_2x}) - Ce^{ik_2x}(-ik_2C^*e^{-ik_2x}) \right) = \frac{\hbar k_2}{m} |C|^2.
\]

The probability of reflection $R$ is given by $|\Gamma_{ref}| / \Gamma_{inc}$ and the probability of transmission $T$ is given by $\Gamma_{trans} / \Gamma_{inc}$:
\[
R = \frac{|\Gamma_{ref}|}{\Gamma_{inc}} = \frac{\frac{\hbar k_1}{m} |B|^2}{\frac{\hbar k_1}{m}} = |B|^2 = \left( \frac{k_1 - k_2}{k_1 + k_2} \right)^2 = \frac{(k_1 - k_2)^2}{(k_1 + k_2)^2}.
\]
\[
T = \frac{\Gamma_{trans}}{\Gamma_{inc}} = \frac{\frac{\hbar k_2}{m} |C|^2}{\frac{\hbar k_1}{m}} = \frac{k_2}{k_1} |C|^2 = \frac{k_2}{k_1} \left( \frac{2k_1}{k_1 + k_2} \right)^2 = \frac{k_2}{k_1} \frac{4k_1^2}{(k_1 + k_2)^2} = \frac{4k_1k_2}{(k_1 + k_2)^2}.
\]

Thus, we have derived:
\[
R = \frac{(k_1 - k_2)^2}{(k_1 + k_2)^2},
\]
\[
T = \frac{4k_1k_2}{(k_1 + k_2)^2}.
\]
Final Answer: The final answer is $\boxed{T = \frac{4k_1k_2}{(k_1 + k_2)^2}, R = \frac{(k_1 - k_2)^2}{(k_1 + k_2)^2}}$

\subsubsection{Solution 2}
Okay, here is the solution to Question 9, presented as a top physics student would, with step-by-step reasoning in LaTeX format, without using sub-sections:

\textbf{Question 9}

\textbf{a)}

We begin by considering the time-independent Schrödinger equation (TISE) for a particle of mass $m$ in a potential $V(x)$:

\[ -\frac{\hbar^2}{2m} \frac{d^2 \psi(x)}{dx^2} + V(x)\psi(x) = E\psi(x). \]

\textbf{Step 1:}  For the region $x \leq 0$, $V(x) = 0$. Substituting this into the TISE, we get:

\[ -\frac{\hbar^2}{2m} \frac{d^2 \psi_1(x)}{dx^2} = E\psi_1(x). \]

Rearranging,

\[ \frac{d^2 \psi_1(x)}{dx^2} = -\frac{2mE}{\hbar^2}\psi_1(x). \]

\textbf{Step 2:} We define $k_1^2 = \frac{2mE}{\hbar^2}$, so the equation becomes:

\[ \frac{d^2 \psi_1(x)}{dx^2} = -k_1^2 \psi_1(x). \]

This is a second-order differential equation with the general solution:

\[ \psi_1(x) = Ae^{ik_1x} + Be^{-ik_1x}. \]

Since the particles are incident from the left, we set $A=1$, obtaining
\[ \psi_1(x) = e^{ik_1x} + Be^{-ik_1x}. \]
Note that $k_1 = \frac{\sqrt{2mE}}{\hbar}$.

\textbf{Step 3:} For the region $x > 0$, $V(x) = -V_0$. The TISE becomes:

\[ -\frac{\hbar^2}{2m} \frac{d^2 \psi_2(x)}{dx^2} - V_0\psi_2(x) = E\psi_2(x). \]

Rearranging,

\[ \frac{d^2 \psi_2(x)}{dx^2} = -\frac{2m(E + V_0)}{\hbar^2}\psi_2(x). \]

\textbf{Step 4:} We define $k_2^2 = \frac{2m(E + V_0)}{\hbar^2}$, leading to:

\[ \frac{d^2 \psi_2(x)}{dx^2} = -k_2^2 \psi_2(x). \]

This differential equation has the general solution:

\[ \psi_2(x) = Ce^{ik_2x} + De^{-ik_2x}. \]

\textbf{Step 5:} Since there is no reflection from $x = \infty$, the term with $e^{-ik_2x}$ must vanish. Thus, $D = 0$. The solution in this region simplifies to:

\[ \psi_2(x) = Ce^{ik_2x}. \]
Note that $k_2 = \frac{\sqrt{2m(E+V_0)}}{\hbar}$.

\textbf{b)}

\textbf{Step 1:} To determine the constants $B$ and $C$, we apply boundary conditions at $x = 0$. The wave function and its first derivative must be continuous at the boundary.

\textbf{Step 2:} Continuity of the wave function, $\psi_1(0) = \psi_2(0)$, gives:

\[ e^{ik_1(0)} + Be^{-ik_1(0)} = Ce^{ik_2(0)}. \]

This simplifies to:

\[ 1 + B = C. \]

\textbf{Step 3:} Continuity of the first derivative, $\frac{d\psi_1}{dx}\bigg|_{x=0} = \frac{d\psi_2}{dx}\bigg|_{x=0}$, gives:

\[ ik_1e^{ik_1(0)} - ik_1Be^{-ik_1(0)} = ik_2Ce^{ik_2(0)}. \]

This simplifies to:

\[ k_1 - k_1B = k_2C. \]

\textbf{Step 4:} Substituting $C = 1 + B$ into the second equation:

\[ k_1 - k_1B = k_2(1 + B). \]

Solving for $B$:

\[ k_1 - k_1B = k_2 + k_2B \]
\[ k_1 - k_2 = k_1B + k_2B \]
\[ k_1 - k_2 = B(k_1 + k_2) \]
\[ B = \frac{k_1 - k_2}{k_1 + k_2}. \]

\textbf{Step 5:} Substituting $B$ back into $C = 1 + B$:

\[ C = 1 + \frac{k_1 - k_2}{k_1 + k_2} = \frac{k_1 + k_2 + k_1 - k_2}{k_1 + k_2} = \frac{2k_1}{k_1 + k_2}. \]

\textbf{c)}

\textbf{Step 1:} We calculate the particle flux using the given formula:

\[ \Gamma(x) = -\frac{i\hbar}{2m} \left( \psi^* \frac{d\psi}{dx} - \psi \frac{d\psi^*}{dx} \right). \]

\textbf{Step 2:} For the incident wave, $\psi_i(x) = e^{ik_1x}$. Then $\frac{d\psi_i}{dx} = ik_1e^{ik_1x}$ and $\psi_i^* = e^{-ik_1x}$. The incident flux $\Gamma_i$ is:

\[ \Gamma_i = -\frac{i\hbar}{2m} \left( e^{-ik_1x}(ik_1e^{ik_1x}) - e^{ik_1x}(-ik_1e^{-ik_1x}) \right) = -\frac{i\hbar}{2m}(ik_1 + ik_1) = \frac{\hbar k_1}{m}. \]

\textbf{Step 3:} For the reflected wave, $\psi_r(x) = Be^{-ik_1x}$. Then $\frac{d\psi_r}{dx} = -ik_1Be^{-ik_1x}$ and $\psi_r^* = B^*e^{ik_1x}$. The reflected flux $\Gamma_r$ is:

\[ \Gamma_r = -\frac{i\hbar}{2m} \left( B^*e^{ik_1x}(-ik_1Be^{-ik_1x}) - Be^{-ik_1x}(ik_1B^*e^{ik_1x}) \right) = -\frac{i\hbar}{2m}(-ik_1|B|^2 - ik_1|B|^2) = -|B|^2\frac{\hbar k_1}{m}. \]

\textbf{Step 4:} For the transmitted wave, $\psi_t(x) = Ce^{ik_2x}$. Then $\frac{d\psi_t}{dx} = ik_2Ce^{ik_2x}$ and $\psi_t^* = C^*e^{-ik_2x}$. The transmitted flux $\Gamma_t$ is:

\[ \Gamma_t = -\frac{i\hbar}{2m} \left( C^*e^{-ik_2x}(ik_2Ce^{ik_2x}) - Ce^{ik_2x}(-ik_2C^*e^{-ik_2x}) \right) = -\frac{i\hbar}{2m}(ik_2|C|^2 + ik_2|C|^2) = |C|^2\frac{\hbar k_2}{m}. \]

\textbf{Step 5:} The reflection probability $R$ is the ratio of the reflected flux to the incident flux:

\[ R = \frac{|\Gamma_r|}{\Gamma_i} = \frac{|B|^2 \frac{\hbar k_1}{m}}{\frac{\hbar k_1}{m}} = |B|^2 = \left|\frac{k_1 - k_2}{k_1 + k_2}\right|^2 = \frac{(k_1 - k_2)^2}{(k_1 + k_2)^2}. \]

\textbf{Step 6:} The transmission probability $T$ is the ratio of the transmitted flux to the incident flux:

\[ T = \frac{\Gamma_t}{\Gamma_i} = \frac{|C|^2 \frac{\hbar k_2}{m}}{\frac{\hbar k_1}{m}} = |C|^2 \frac{k_2}{k_1} = \left|\frac{2k_1}{k_1 + k_2}\right|^2 \frac{k_2}{k_1} = \frac{4k_1^2}{(k_1 + k_2)^2} \frac{k_2}{k_1} = \frac{4k_1k_2}{(k_1 + k_2)^2}. \]

\textbf{Step 7:} It is important to note that $R + T = 1$:
\[ R + T = \frac{(k_1 - k_2)^2}{(k_1 + k_2)^2} + \frac{4k_1k_2}{(k_1 + k_2)^2} = \frac{k_1^2 - 2k_1k_2 + k_2^2 + 4k_1k_2}{(k_1 + k_2)^2} = \frac{k_1^2 + 2k_1k_2 + k_2^2}{(k_1 + k_2)^2} = \frac{(k_1+k_2)^2}{(k_1+k_2)^2} = 1. \]
This confirms the conservation of probability.


\subsubsection{Solution 3}
Okay, here's my solution to Question 9, formatted as requested:

\noindent \textbf{Question 9}

\noindent a) We begin by considering the time-independent Schrödinger equation:
\[
-\frac{\hbar^2}{2m} \frac{d^2 \psi(x)}{dx^2} + V(x) \psi(x) = E \psi(x).
\]
\noindent \textbf{Region 1 ($x \leq 0$):} Here, $V(x) = 0$, so the equation becomes
\[
-\frac{\hbar^2}{2m} \frac{d^2 \psi_1(x)}{dx^2} = E \psi_1(x).
\]
\noindent Rearranging gives
\[
\frac{d^2 \psi_1(x)}{dx^2} = -\frac{2mE}{\hbar^2} \psi_1(x).
\]
\noindent Defining $k_1^2 = \frac{2mE}{\hbar^2}$, we have
\[
\frac{d^2 \psi_1(x)}{dx^2} + k_1^2 \psi_1(x) = 0.
\]
\noindent This is a second-order differential equation with the general solution $\psi_1(x) = Ae^{ik_1x} + Be^{-ik_1x}$. Since the particles are incident from the left, we assume $A=1$. Thus, $\psi_1(x) = e^{ik_1x} + Be^{-ik_1x}$.

\noindent \textbf{Region 2 ($x > 0$):} In this region, $V(x) = -V_0$, so the Schrödinger equation becomes
\[
-\frac{\hbar^2}{2m} \frac{d^2 \psi_2(x)}{dx^2} - V_0 \psi_2(x) = E \psi_2(x).
\]
\noindent Rearranging yields
\[
\frac{d^2 \psi_2(x)}{dx^2} = -\frac{2m(E + V_0)}{\hbar^2} \psi_2(x).
\]
\noindent Defining $k_2^2 = \frac{2m(E + V_0)}{\hbar^2}$, we get
\[
\frac{d^2 \psi_2(x)}{dx^2} + k_2^2 \psi_2(x) = 0.
\]
\noindent The general solution is $\psi_2(x) = Ce^{ik_2x} + De^{-ik_2x}$. Since we are only interested in the transmitted wave (particles moving to the right in Region 2), we set $D=0$. Thus, $\psi_2(x) = Ce^{ik_2x}$.

\noindent b) We apply the boundary conditions at $x = 0$. The wave function and its first derivative must be continuous:

\noindent \textbf{Continuity of $\psi(x)$ at $x=0$:}
\[
\psi_1(0) = \psi_2(0).
\]
\noindent Substituting the expressions for $\psi_1(x)$ and $\psi_2(x)$, we get
\[
e^{ik_1(0)} + Be^{-ik_1(0)} = Ce^{ik_2(0)}.
\]
\noindent This simplifies to
\[
1 + B = C. \hspace{1cm} (1)
\]

\noindent \textbf{Continuity of $\frac{d\psi(x)}{dx}$ at $x=0$:}
\[
\frac{d\psi_1}{dx}\bigg|_{x=0} = \frac{d\psi_2}{dx}\bigg|_{x=0}.
\]
\noindent Calculating the derivatives, we have
\[
ik_1e^{ik_1x} - ik_1Be^{-ik_1x}\bigg|_{x=0} = ik_2Ce^{ik_2x}\bigg|_{x=0}.
\]
\noindent Evaluating at $x=0$, we get
\[
ik_1 - ik_1B = ik_2C.
\]
\noindent Dividing by $i$ yields
\[
k_1 - k_1B = k_2C. \hspace{1cm} (2)
\]
\noindent Substituting $C = 1 + B$ from equation (1) into equation (2), we obtain
\[
k_1 - k_1B = k_2(1 + B).
\]
\noindent Rearranging to solve for $B$, we get
\[
k_1 - k_2 = k_1B + k_2B = (k_1 + k_2)B,
\]
\noindent so
\[
B = \frac{k_1 - k_2}{k_1 + k_2}.
\]
\noindent Substituting this back into equation (1), we find
\[
C = 1 + \frac{k_1 - k_2}{k_1 + k_2} = \frac{k_1 + k_2 + k_1 - k_2}{k_1 + k_2} = \frac{2k_1}{k_1 + k_2}.
\]

\noindent c) The particle flux is given by
\[
\Gamma(x) = -\frac{i\hbar}{2m} \left( \psi^* \frac{d\psi}{dx} - \psi \frac{d\psi^*}{dx} \right).
\]

\noindent \textbf{Incident Flux:} For the incident wave, $\psi_{inc} = e^{ik_1x}$, and $\psi_{inc}^* = e^{-ik_1x}$. Then, $\frac{d\psi_{inc}}{dx} = ik_1e^{ik_1x}$ and $\frac{d\psi_{inc}^*}{dx} = -ik_1e^{-ik_1x}$.
\[
\Gamma_{inc} = -\frac{i\hbar}{2m} \left( e^{-ik_1x}(ik_1e^{ik_1x}) - e^{ik_1x}(-ik_1e^{-ik_1x}) \right) = -\frac{i\hbar}{2m} \left( ik_1 + ik_1 \right) = \frac{\hbar k_1}{m}.
\]

\noindent \textbf{Reflected Flux:} For the reflected wave, $\psi_{ref} = Be^{-ik_1x}$, and $\psi_{ref}^* = B^*e^{ik_1x}$. Then, $\frac{d\psi_{ref}}{dx} = -ik_1Be^{-ik_1x}$ and $\frac{d\psi_{ref}^*}{dx} = ik_1B^*e^{ik_1x}$.
\[
\Gamma_{ref} = -\frac{i\hbar}{2m} \left( B^*e^{ik_1x}(-ik_1Be^{-ik_1x}) - Be^{-ik_1x}(ik_1B^*e^{ik_1x}) \right) = -\frac{i\hbar}{2m}(-ik_1|B|^2 - ik_1|B|^2) = -\frac{\hbar k_1}{m}|B|^2.
\]

\noindent \textbf{Transmitted Flux:} For the transmitted wave, $\psi_{trans} = Ce^{ik_2x}$, and $\psi_{trans}^* = C^*e^{-ik_2x}$. Then, $\frac{d\psi_{trans}}{dx} = ik_2Ce^{ik_2x}$ and $\frac{d\psi_{trans}^*}{dx} = -ik_2C^*e^{-ik_2x}$.
\[
\Gamma_{trans} = -\frac{i\hbar}{2m} \left( C^*e^{-ik_2x}(ik_2Ce^{ik_2x}) - Ce^{ik_2x}(-ik_2C^*e^{-ik_2x}) \right) = -\frac{i\hbar}{2m} \left( ik_2|C|^2 + ik_2|C|^2 \right) = \frac{\hbar k_2}{m}|C|^2.
\]

\noindent The reflection probability is given by $R = \frac{|\Gamma_{ref}|}{\Gamma_{inc}}$, and the transmission probability is given by $T = \frac{\Gamma_{trans}}{\Gamma_{inc}}$. Thus,

\[
R = \frac{\frac{\hbar k_1}{m}|B|^2}{\frac{\hbar k_1}{m}} = |B|^2 = \left(\frac{k_1 - k_2}{k_1 + k_2}\right)^2 = \frac{(k_1 - k_2)^2}{(k_1 + k_2)^2}.
\]

\[
T = \frac{\frac{\hbar k_2}{m}|C|^2}{\frac{\hbar k_1}{m}} = \frac{k_2}{k_1}|C|^2 = \frac{k_2}{k_1}\left(\frac{2k_1}{k_1 + k_2}\right)^2 = \frac{k_2}{k_1}\frac{4k_1^2}{(k_1 + k_2)^2} = \frac{4k_1k_2}{(k_1 + k_2)^2}.
\]

\subsection{Question 10}

Consider the system of a quantum harmonic oscillator with eigenstates written \(|n\rangle\) and corresponding eigenvalues \(E_n = \hbar\omega(n + \frac{1}{2})\). 

You are given the raising and lowering operators \(\hat{a}_\pm\) defined by:
\[
\hat{a}_\pm = \frac{1}{\sqrt{2}} \left(\alpha \hat{x} \mp \frac{i}{\hbar \alpha} \hat{p}\right)
\]
\[
\alpha = \sqrt{\frac{m\omega}{\hbar}}
\]
where \(\hat{x}, \hat{p}\) are the one-dimensional position and momentum operators respectively.

\begin{enumerate}
    \item[(a)] Normalize the state \(|\psi\rangle = |0\rangle + b |1\rangle\) and calculate the expectation value of the Hamiltonian for \(|\psi\rangle\). \(b\) is a constant.
    
    \item[(b)] A perturbation of the form \(\hat{H}' = q \mathcal{E} \hat{x}\) is introduced to the system, where \(\hat{x}\) is the one-dimensional position operator. Rewrite the perturbation in terms of raising and lowering operators \(\hat{a}_\pm\).
    
    \item[(c)] By applying the variational principle, show that using \(|\psi\rangle\) as a trial state, the value of \(b\) which minimizes energy is:
    \[
    b = \frac{E_1 - E_0}{g} - \sqrt{\frac{(E_1 - E_0)^2}{g^2} - 1}
    \]
    where \(g = \sqrt{2}q\mathcal{E}/\alpha\).
\end{enumerate}


\paragraph{Prompt: \\} 
Answer the given question as a top physics student, giving step-by-step reasoning. Format your answer as LaTeX code without using sub-sections. Available marks: a) [5 marks], b) [2 marks], c) [9 marks].

\subsubsection{Solution 1}
Okay, here is the solution to Question 10, formatted as requested:

\noindent\textbf{Question 10}

\textbf{(a)} [5 marks]

\noindent\textit{Step 1: Normalize the state $|\psi\rangle$.}

\noindent To normalize the state $|\psi\rangle = |0\rangle + b|1\rangle$, we require $\langle \psi | \psi \rangle = 1$.

\noindent Therefore:

\noindent $\langle \psi | \psi \rangle = (|0\rangle + b|1\rangle)^\dagger (|0\rangle + b|1\rangle) = (\langle 0| + b^*\langle 1|) (|0\rangle + b|1\rangle)$

\noindent $= \langle 0|0 \rangle + b^*\langle 1|0\rangle + b\langle 0|1\rangle + b^*b\langle 1|1\rangle = 1 + |b|^2$

\noindent For normalization, we need $1 + |b|^2 = 1$, which is incorrect since $b$ can be non-zero. We need to divide by a normalisation constant. Let $|\psi\rangle = A(|0\rangle + b|1\rangle)$

\noindent $A^2 (1 + |b|^2) = 1$

\noindent $A = \frac{1}{\sqrt{1 + |b|^2}}$

\noindent Therefore the normalised state is $|\psi\rangle = \frac{1}{\sqrt{1+|b|^2}}(|0\rangle + b|1\rangle)$.

\noindent\textit{Step 2: Calculate the expectation value of the Hamiltonian.}

\noindent The Hamiltonian for the quantum harmonic oscillator is $\hat{H} = \hbar\omega(\hat{n} + \frac{1}{2})$, where $\hat{n}$ is the number operator. Thus, the eigenvalues are $E_n = \hbar\omega(n+\frac{1}{2})$.

\noindent The expectation value of the Hamiltonian is given by $\langle \psi | \hat{H} | \psi \rangle$.

\noindent $\langle \psi | \hat{H} | \psi \rangle = \frac{1}{1+|b|^2} (\langle 0| + b^*\langle 1|) \hat{H} (|0\rangle + b|1\rangle)$

\noindent $= \frac{1}{1+|b|^2} (\langle 0| + b^*\langle 1|) (E_0|0\rangle + bE_1|1\rangle)$

\noindent $= \frac{1}{1+|b|^2} (E_0\langle 0|0\rangle + bE_1\langle 0|1\rangle + b^*E_0\langle 1|0\rangle + b^*bE_1\langle 1|1\rangle)$

\noindent $= \frac{1}{1+|b|^2} (E_0 + |b|^2 E_1) = \frac{\hbar\omega(0.5 + 1.5|b|^2)}{1+|b|^2}$.

\textbf{(b)} [2 marks]

\noindent\textit{Step 1: Express $\hat{x}$ in terms of $\hat{a}_+$ and $\hat{a}_-$.}

\noindent We are given $\hat{a}_\pm = \frac{1}{\sqrt{2}} \left(\alpha \hat{x} \mp \frac{i}{\hbar \alpha} \hat{p}\right)$.

\noindent Adding the expressions for $\hat{a}_+$ and $\hat{a}_-$, we get:

\noindent $\hat{a}_+ + \hat{a}_- = \frac{1}{\sqrt{2}} (\alpha \hat{x} - \frac{i}{\hbar \alpha} \hat{p} + \alpha \hat{x} + \frac{i}{\hbar \alpha} \hat{p}) = \frac{2\alpha \hat{x}}{\sqrt{2}} = \sqrt{2}\alpha \hat{x}$.

\noindent Therefore, $\hat{x} = \frac{1}{\sqrt{2}\alpha}(\hat{a}_+ + \hat{a}_-)$.

\noindent\textit{Step 2: Rewrite the perturbation in terms of $\hat{a}_+$ and $\hat{a}_-$.}

\noindent The perturbation is given by $\hat{H}' = q\mathcal{E}\hat{x}$.

\noindent Substituting the expression for $\hat{x}$, we get:

\noindent $\hat{H}' = q\mathcal{E} \frac{1}{\sqrt{2}\alpha}(\hat{a}_+ + \hat{a}_-) = \frac{q\mathcal{E}}{\sqrt{2}\alpha}(\hat{a}_+ + \hat{a}_-)$.

\textbf{(c)} [9 marks]

\noindent\textit{Step 1: Apply the variational principle.}

\noindent The variational principle states that the expectation value of the Hamiltonian with respect to any trial wavefunction is greater than or equal to the ground state energy.

\noindent The expectation value of the perturbed Hamiltonian $\hat{H} + \hat{H}'$ using the trial state $|\psi\rangle$ is:

\noindent $E(b) = \langle \psi | \hat{H} + \hat{H}' | \psi \rangle$

\noindent We want to minimize $E(b)$ with respect to $b$.
\noindent Assuming \(b\) is real:

\noindent $E(b) = \frac{1}{1+b^2} (\langle 0| + b\langle 1|) (\hat{H} + \hat{H}') (|0\rangle + b|1\rangle)$

\noindent $= \frac{1}{1+b^2} (E_0 + b^2 E_1 + \frac{q\mathcal{E}}{\sqrt{2}\alpha} (\langle 0| + b\langle 1|) (\hat{a}_+ + \hat{a}_-) (|0\rangle + b|1\rangle))$

\noindent $= \frac{1}{1+b^2} (E_0 + b^2 E_1 + \frac{q\mathcal{E}}{\sqrt{2}\alpha} (\langle 0| + b\langle 1|) (|1\rangle + b\sqrt{2}|2\rangle + b|0\rangle))$

\noindent $= \frac{1}{1+b^2} (E_0 + b^2 E_1 + \frac{q\mathcal{E}}{\sqrt{2}\alpha} (b + b)) = \frac{1}{1+b^2} (E_0 + b^2 E_1 + \frac{2bq\mathcal{E}}{\sqrt{2}\alpha})$

\noindent $ = \frac{1}{1+b^2} (E_0 + b^2 E_1 + gb)$ , where \(g = \frac{\sqrt{2}q\mathcal{E}}{\alpha}\)

\noindent\textit{Step 2: Minimize $E(b)$ with respect to $b$.}

\noindent To minimize $E(b)$, we take the derivative with respect to $b$ and set it to zero:

\noindent $\frac{dE(b)}{db} = \frac{(2bE_1 + g)(1+b^2) - 2b(E_0 + b^2 E_1 + gb)}{(1+b^2)^2} = 0$

\noindent $2bE_1 + g + 2b^3 E_1 + gb^2 - 2bE_0 - 2b^3 E_1 - 2gb^2 = 0$

\noindent $2b(E_1 - E_0) + g - gb^2 = 0$

\noindent $gb^2 - 2b(E_1 - E_0) - g = 0$

\noindent\textit{Step 3: Solve the quadratic equation for $b$.}

\noindent The quadratic equation for $b$ is:

\noindent $gb^2 - 2(E_1 - E_0)b - g = 0$

\noindent Using the quadratic formula:

\noindent $b = \frac{2(E_1 - E_0) \pm \sqrt{4(E_1 - E_0)^2 + 4g^2}}{2g} = \frac{(E_1 - E_0) \pm \sqrt{(E_1 - E_0)^2 + g^2}}{g}$

\noindent $ = \frac{E_1 - E_0}{g} \pm \sqrt{\frac{(E_1 - E_0)^2}{g^2} + 1}$

\noindent Since we want to minimize the energy, and are given the negative answer, we choose the negative sign:

\noindent $b = \frac{E_1 - E_0}{g} - \sqrt{\frac{(E_1 - E_0)^2}{g^2} + 1}$.


\subsubsection{Solution 2}
Okay, here is the solution to Question 10, formatted as requested:

\begin{enumerate}
\item[(a)]

To normalize the state $|\psi\rangle = |0\rangle + b|1\rangle$, we require $\langle\psi|\psi\rangle = 1$.

\textit{Step 1: Calculate $\langle\psi|\psi\rangle$.}

$\langle\psi|\psi\rangle = (|0\rangle + b|1\rangle)^\dagger (|0\rangle + b|1\rangle) = (\langle 0| + b^*\langle 1|)(|0\rangle + b|1\rangle)$

$\langle\psi|\psi\rangle = \langle 0|0\rangle + b^*\langle 1|0\rangle + b\langle 0|1\rangle + |b|^2\langle 1|1\rangle = 1 + |b|^2$

\textit{Step 2: Set $\langle\psi|\psi\rangle = 1$ and solve for $|b|$.}

$1 + |b|^2 = 1$

$|b|^2 = 0$
$b=0$
It seems there was an error in the question text, as the state is already normalised when \(b=0\). We shall continue with a general complex \(b\).
$1 + |b|^2 = 1$
$|b|^2 + 1 = N^2$ where N is the normalisation constant.
$|\psi\rangle = \frac{1}{\sqrt{1+|b|^2}} (|0\rangle + b|1\rangle)$
The expectation value of the Hamiltonian is given by $\langle\psi|\hat{H}|\psi\rangle$.

\textit{Step 3: Calculate $\hat{H}|\psi\rangle$.}

$\hat{H}|\psi\rangle = \hat{H} \left( \frac{1}{\sqrt{1+|b|^2}} (|0\rangle + b|1\rangle) \right) = \frac{1}{\sqrt{1+|b|^2}}(\hat{H}|0\rangle + b\hat{H}|1\rangle)$

$\hat{H}|\psi\rangle = \frac{1}{\sqrt{1+|b|^2}}(E_0|0\rangle + bE_1|1\rangle)$

\textit{Step 4: Calculate $\langle\psi|\hat{H}|\psi\rangle$.}

$\langle\psi|\hat{H}|\psi\rangle = \frac{1}{\sqrt{1+|b|^2}}(\langle 0| + b^*\langle 1|)\frac{1}{\sqrt{1+|b|^2}}(E_0|0\rangle + bE_1|1\rangle)$

$\langle\psi|\hat{H}|\psi\rangle = \frac{1}{1+|b|^2}(E_0\langle 0|0\rangle + bE_1\langle 0|1\rangle + b^*E_0\langle 1|0\rangle + |b|^2E_1\langle 1|1\rangle)$

$\langle\psi|\hat{H}|\psi\rangle = \frac{1}{1+|b|^2}(E_0 + |b|^2E_1)$
Assuming $b$ is real,
$\langle\psi|\hat{H}|\psi\rangle = \frac{E_0 + b^2E_1}{1+b^2}$

\item[(b)]

We are given $\hat{H}' = q\mathcal{E}\hat{x}$. We need to rewrite this in terms of $\hat{a}_\pm$.

\textit{Step 1: Express $\hat{x}$ in terms of $\hat{a}_\pm$.}

From the definition of $\hat{a}_\pm$, we have:

$\hat{a}_+ = \frac{1}{\sqrt{2}} \left(\alpha \hat{x} - \frac{i}{\hbar \alpha} \hat{p}\right)$

$\hat{a}_- = \frac{1}{\sqrt{2}} \left(\alpha \hat{x} + \frac{i}{\hbar \alpha} \hat{p}\right)$

Adding these two equations:

$\hat{a}_+ + \hat{a}_- = \frac{2\alpha \hat{x}}{\sqrt{2}}$

$\hat{x} = \frac{1}{\alpha\sqrt{2}}(\hat{a}_+ + \hat{a}_-)$

\textit{Step 2: Substitute $\hat{x}$ into $\hat{H}'$.}

$\hat{H}' = q\mathcal{E}\hat{x} = q\mathcal{E}\frac{1}{\alpha\sqrt{2}}(\hat{a}_+ + \hat{a}_-) = \frac{q\mathcal{E}}{\alpha\sqrt{2}}(\hat{a}_+ + \hat{a}_-)$

\item[(c)]

We want to minimize the energy using the variational principle with the trial state $|\psi\rangle = \frac{1}{\sqrt{1+b^2}} (|0\rangle + b|1\rangle)$, assuming $b$ is real.

\end{enumerate}

\textit{Step 1: Calculate the expectation value of the total Hamiltonian $\hat{H} + \hat{H}'$.}

$\langle\psi|\hat{H} + \hat{H}'|\psi\rangle = \langle\psi|\hat{H}|\psi\rangle + \langle\psi|\hat{H}'|\psi\rangle$

We already calculated $\langle\psi|\hat{H}|\psi\rangle = \frac{E_0 + b^2E_1}{1+b^2}$. Now we need $\langle\psi|\hat{H}'|\psi\rangle$.

\textit{Step 2: Calculate $\langle\psi|\hat{H}'|\psi\rangle$.}

$\langle\psi|\hat{H}'|\psi\rangle = \frac{1}{1+b^2} (\langle 0| + b\langle 1|)\frac{q\mathcal{E}}{\alpha\sqrt{2}}(\hat{a}_+ + \hat{a}_-)(|0\rangle + b|1\rangle)$

$\langle\psi|\hat{H}'|\psi\rangle = \frac{q\mathcal{E}}{\alpha\sqrt{2}(1+b^2)}(\langle 0| + b\langle 1|)(|1\rangle + b\sqrt{2}|0\rangle + b|2\rangle) = \frac{g}{1+b^2}(b + b)$

$\langle\psi|\hat{H}'|\psi\rangle = \frac{2gb}{1+b^2}$, where $g = \frac{q\mathcal{E}}{\alpha\sqrt{2}}$.

\textit{Step 3: Find the total energy expectation value.}

$E(b) = \langle\psi|\hat{H} + \hat{H}'|\psi\rangle = \frac{E_0 + b^2E_1}{1+b^2} + \frac{2gb}{1+b^2} = \frac{E_0 + b^2E_1 + 2gb}{1+b^2}$

\textit{Step 4: Minimize $E(b)$ with respect to $b$ by setting $\frac{dE}{db} = 0$.}

$\frac{dE}{db} = \frac{(2bE_1 + 2g)(1+b^2) - 2b(E_0 + b^2E_1 + 2gb)}{(1+b^2)^2} = 0$

$2bE_1 + 2g + 2b^3E_1 + 2b^2g - 2bE_0 - 2b^3E_1 - 4b^2g = 0$

$2b(E_1 - E_0) + 2g - 2b^2g = 0$

$b^2g - b(E_1 - E_0) - g = 0$
\textit{Step 5: Solve the quadratic equation for $b$.}

$b = \frac{(E_1 - E_0) \pm \sqrt{(E_1 - E_0)^2 + 4g^2}}{2g}$
$b = \frac{(E_1 - E_0) \pm \sqrt{(E_1 - E_0)^2(1+\frac{4g^2}{(E_1-E_0)^2})}}{2g}$

Since we want the energy to be minimized, we choose the negative sign.
$b = \frac{E_1-E_0}{g} - \sqrt{\frac{(E_1-E_0)^2}{g^2} + 1}$
Note: The provided solution in the question has a typo, it should be \(+1\) under the square root instead of \(-1\).

Final Answer: The value of $b$ which minimizes the energy is given by:
\[
b = \frac{E_1 - E_0}{g} - \sqrt{\frac{(E_1 - E_0)^2}{g^2} + 1}
\]

\subsubsection{Solution 3}
Okay, here is the solution to Question 10, formatted as requested:

\noindent \\
\textbf{Question 10} \\

\noindent \\
\noindent (a) \\

\noindent \\
\textbf{Step 1: Normalization} \\
To normalize the state \(|\psi\rangle = |0\rangle + b |1\rangle\), we require \(\langle\psi|\psi\rangle = 1\).

\[
\langle\psi|\psi\rangle = (|0\rangle + b|1\rangle)^\dagger(|0\rangle + b|1\rangle) = \langle 0|0\rangle + b^*\langle 1|0\rangle + b\langle 0|1\rangle + |b|^2 \langle 1|1\rangle
\]

Since \( |0\rangle \) and \( |1\rangle \) are orthonormal, \( \langle 0|0\rangle = \langle 1|1\rangle = 1 \) and \( \langle 0|1\rangle = \langle 1|0\rangle = 0 \). Thus,

\[
\langle\psi|\psi\rangle = 1 + |b|^2
\]

For normalization, we need \(1 + |b|^2 = 1\). However this would imply that \(b=0\). We should have \(1 + |b|^2 = N^2\), where \(N\) is a normalisation constant.
\[
|\psi\rangle = \frac{1}{\sqrt{1+|b|^2}}(|0\rangle + b|1\rangle)
\]
Thus,
\[
\langle\psi|\psi\rangle = \frac{1}{1+|b|^2}(1 + |b|^2) = 1
\]
Assuming \(b\) is real, we have the normalized state:

\[
|\psi\rangle = \frac{1}{\sqrt{1+b^2}}(|0\rangle + b|1\rangle)
\]
\noindent \\
\textbf{Step 2: Expectation Value of the Hamiltonian} \\
The expectation value of the Hamiltonian is given by \(\langle\psi|\hat{H}|\psi\rangle\).

\[
\langle\psi|\hat{H}|\psi\rangle = \frac{1}{1+b^2} (|0\rangle + b|1\rangle)^\dagger \hat{H} (|0\rangle + b|1\rangle)
\]

Since \(\hat{H}|n\rangle = E_n|n\rangle\),

\[
\hat{H}|\psi\rangle = \frac{1}{\sqrt{1+b^2}} (E_0|0\rangle + b E_1|1\rangle)
\]
Then,

\[
\langle\psi|\hat{H}|\psi\rangle = \frac{1}{1+b^2} (\langle 0| + b\langle 1|) (E_0|0\rangle + b E_1|1\rangle)
\]

\[
= \frac{1}{1+b^2} (E_0 \langle 0|0\rangle + b E_1 \langle 0|1\rangle + b E_0 \langle 1|0\rangle + b^2 E_1 \langle 1|1\rangle)
\]

\[
= \frac{1}{1+b^2} (E_0 + b^2 E_1)
\]

\[
= \frac{\hbar\omega(\frac{1}{2} + b^2\frac{3}{2})}{1+b^2}
\]
\noindent \\
\noindent (b) \\

\noindent \\
\textbf{Step 1: Expressing \(\hat{x}\) in terms of \(\hat{a}_\pm\)} \\
We are given:

\[
\hat{a}_\pm = \frac{1}{\sqrt{2}} \left(\alpha \hat{x} \mp \frac{i}{\hbar \alpha} \hat{p}\right)
\]

From these, we can write:

\[
\hat{a}_+ + \hat{a}_- = \frac{1}{\sqrt{2}}(\alpha \hat{x} - \frac{i}{\hbar \alpha} \hat{p} + \alpha \hat{x} + \frac{i}{\hbar \alpha} \hat{p}) = \sqrt{2} \alpha \hat{x}
\]

Therefore,

\[
\hat{x} = \frac{1}{\sqrt{2} \alpha} (\hat{a}_+ + \hat{a}_-)
\]
\noindent \\
\textbf{Step 2: Rewriting the Perturbation} \\
The perturbation is given by \(\hat{H}' = q \mathcal{E} \hat{x}\). Substituting the expression for \(\hat{x}\):

\[
\hat{H}' = q \mathcal{E} \frac{1}{\sqrt{2} \alpha} (\hat{a}_+ + \hat{a}_-)
\]

Let \(g = \sqrt{2}q\mathcal{E}/\alpha\). Then we can rewrite \(\hat{H}'\) as:
\[
\hat{H}' = \frac{g}{2}(\hat{a}_+ + \hat{a}_-)
\]
\noindent \\
\noindent (c) \\

\noindent \\
\textbf{Step 1: Applying the Variational Principle} \\
The variational principle states that the expectation value of the Hamiltonian with respect to any trial wavefunction is greater than or equal to the ground state energy. We want to minimize \(\langle\psi|\hat{H} + \hat{H}'|\psi\rangle\) with respect to \(b\).

\noindent \\
\textbf{Step 2: Calculating \(\langle\psi|\hat{H} + \hat{H}'|\psi\rangle\)} \\
We have:
\[
\hat{H}'|\psi\rangle = \frac{1}{\sqrt{1+b^2}}\frac{g}{2}(\hat{a}_+ + \hat{a}_-)(|0\rangle + b|1\rangle)
\]
\[
= \frac{1}{\sqrt{1+b^2}}\frac{g}{2}(|1\rangle + b\sqrt{2}|2\rangle + b|0\rangle)
\]
Then,

\[
\langle\psi|\hat{H}'|\psi\rangle = \frac{1}{1+b^2}(\langle 0| + b\langle 1|)\frac{g}{2}(|1\rangle + b\sqrt{2}|2\rangle + b|0\rangle)
\]
\[
= \frac{1}{1+b^2}\frac{g}{2}(b + b) = \frac{gb}{1+b^2}
\]
So,
\[
\langle\psi|\hat{H} + \hat{H}'|\psi\rangle = \frac{E_0 + b^2 E_1}{1+b^2} + \frac{gb}{1+b^2} = \frac{E_0 + b^2 E_1 + gb}{1+b^2}
\]
\noindent \\
\textbf{Step 3: Minimizing the Energy} \\
Let \(E(b) = \langle\psi|\hat{H} + \hat{H}'|\psi\rangle\). To minimize \(E(b)\), we take the derivative with respect to \(b\) and set it to zero:

\[
\frac{dE(b)}{db} = \frac{(2bE_1 + g)(1+b^2) - 2b(E_0 + b^2 E_1 + gb)}{(1+b^2)^2} = 0
\]

This simplifies to:
\[
(2bE_1 + g)(1+b^2) - 2b(E_0 + b^2 E_1 + gb) = 0
\]
\[
2bE_1 + g + 2b^3 E_1 + gb^2 - 2bE_0 - 2b^3 E_1 - 2gb^2 = 0
\]
\[
2bE_1 + g - 2bE_0 - gb^2 = 0
\]
\[
gb^2 - 2b(E_1 - E_0) - g = 0
\]
\noindent \\
\textbf{Step 4: Solving for \(b\)} \\
This is a quadratic equation in \(b\). Using the quadratic formula:

\[
b = \frac{2(E_1 - E_0) \pm \sqrt{4(E_1 - E_0)^2 + 4g^2}}{2g}
\]
\[
b = \frac{E_1 - E_0}{g} \pm \sqrt{\frac{(E_1 - E_0)^2}{g^2} + 1}
\]
We want the negative root for the minimum energy, so
\[
b = \frac{E_1 - E_0}{g} - \sqrt{\frac{(E_1 - E_0)^2}{g^2} + 1}
\]

Final Answer: The final answer is \(b = \frac{E_1 - E_0}{g} - \sqrt{\frac{(E_1 - E_0)^2}{g^2} + 1}\)


\section{Electrodynamics}

\subsection{Question 1}

Name the three types of magnetic materials. Explain what determines their response when brought near a bar magnet. 

\paragraph{Prompt: \\} 
Answer the given question as a top physics student, giving step-by-step reasoning keeping your answer under 300 words. Format your answer as LaTeX code without using sub-sections. Available marks: [8 marks]

\subsubsection{Solution 1}
The three types of magnetic materials are:

1.  **Ferromagnetic materials:** These materials exhibit strong attraction to magnets and can be permanently magnetized. Examples include iron, nickel, and cobalt.
2.  **Paramagnetic materials:** These materials are weakly attracted to magnets and are not easily magnetized. Examples include aluminum and platinum.
3.  **Diamagnetic materials:** These materials are weakly repelled by magnets. Examples include copper, silver, and gold.

The magnetic response of a material is determined by the behavior of its electrons in the presence of an external magnetic field.

*   **Ferromagnetic materials:** These materials have unpaired electrons that align their spins parallel to each other and the external field, creating a strong net magnetic moment. This alignment persists even after the external field is removed, resulting in permanent magnetization.

*   **Paramagnetic materials:** These materials also have unpaired electrons, but their spins are randomly oriented in the absence of an external field. When a magnetic field is applied, some of the spins align with the field, resulting in a weak net magnetic moment and attraction. This effect disappears when the external field is removed.

*   **Diamagnetic materials:** These materials have all their electrons paired, resulting in no net magnetic moment. When exposed to an external magnetic field, the electron orbits are altered, inducing a small magnetic moment that opposes the applied field. This leads to a weak repulsion.


\subsubsection{Solution 2}
The three types of magnetic materials are:

1.  **Ferromagnetic materials:** These materials exhibit strong attraction to magnets and can be permanently magnetized. Examples include iron, nickel, and cobalt.
2.  **Paramagnetic materials:** These materials are weakly attracted to magnets and are not easily magnetized. Examples include aluminum and platinum.
3.  **Diamagnetic materials:** These materials are weakly repelled by magnets. Examples include copper, silver, and gold.

The magnetic response of a material is determined by the behavior of its electrons in the presence of an external magnetic field.

*   **Ferromagnetic materials:** These materials have unpaired electrons that align their spins parallel to each other and the external field, resulting in a strong net magnetic moment and attraction to the magnet.

*   **Paramagnetic materials:** These materials have some unpaired electrons, but their spins are randomly oriented in the absence of an external field. When a magnet is brought nearby, a small fraction of the spins align with the field, resulting in a weak attraction.

*   **Diamagnetic materials:** These materials have all their electrons paired, resulting in no net magnetic moment. When exposed to an external field, a weak magnetic moment is induced in the opposite direction, leading to a weak repulsion.

\subsubsection{Solution 3}
The three types of magnetic materials are:

* **Ferromagnetic materials:** These materials exhibit strong attraction to magnets and can be permanently magnetized. Examples include iron, nickel, and cobalt.
* **Paramagnetic materials:** These materials are weakly attracted to magnets and do not retain their magnetism when the external magnetic field is removed. Examples include aluminum and platinum.
* **Diamagnetic materials:** These materials are weakly repelled by magnets. Examples include copper, silver, and gold.

The response of a material when brought near a bar magnet is determined by the **interaction between the external magnetic field and the magnetic dipoles within the material.**

* **Ferromagnetic materials:** These materials have a large number of unpaired electrons that act as tiny magnets (magnetic dipoles). In the absence of an external magnetic field, these dipoles are randomly oriented. However, when a ferromagnetic material is brought near a bar magnet, the external magnetic field causes the dipoles to align, resulting in a strong attraction. This alignment can persist even after the external field is removed, leading to permanent magnetization.

* **Paramagnetic materials:** These materials also have unpaired electrons, but their magnetic dipoles are much weaker and interact less strongly with each other. When a paramagnetic material is brought near a bar magnet, the external magnetic field causes some alignment of the dipoles, resulting in a weak attraction. However, this alignment is lost when the external field is removed.

* **Diamagnetic materials:** These materials have all their electrons paired up, resulting in no net magnetic dipole moment. When a diamagnetic material is brought near a bar magnet, the external magnetic field induces a weak magnetization in the opposite direction, leading to a weak repulsion. This effect is temporary and disappears when the external field is removed.

In summary, the magnetic behavior of a material depends on the nature and arrangement of its magnetic dipoles and their interaction with an external magnetic field.


\subsection{Question 2}

\begin{enumerate}
    \item Within the context of classical electromagnetism, briefly define what is 'free space'. In free space, the electric field, $\mathbf{E}$, satisfies 3 partial differential equations, one for each of its components:
    \[
    \nabla^2 \mathbf{E} = \epsilon_0 \mu_0 \frac{\partial^2 \mathbf{E}}{\partial t^2}
    \]
    \item Briefly describe the physical interpretation of this equation for showing how electric fields can propagate in space.
    \item Explain how this equation gives the speed of light $c$ and state its value.
\end{enumerate}


\paragraph{Prompt: \\} 
Answer the given question as a top physics student, giving step-by-step reasoning keeping your answer under 300 words. Format your answer as LaTeX code without using sub-sections. Available marks: [5 marks]

\subsubsection{Solution 1}
```latex
\textbf{Answer to Question 2} 

\begin{enumerate}
\item In classical electromagnetism, \textbf{free space} refers to a perfect vacuum, devoid of any matter or charge. It is characterized by the electric permittivity $\epsilon_0$ and magnetic permeability $\mu_0$. 

\item The equation $\nabla^2 \mathbf{E} = \epsilon_0 \mu_0 \frac{\partial^2 \mathbf{E}}{\partial t^2}$ is a wave equation. It shows that a change in the electric field in time ($\frac{\partial^2 \mathbf{E}}{\partial t^2}$) generates a spatial variation in the field ($\nabla^2 \mathbf{E}$). This coupling between temporal and spatial variations allows the electric field to propagate as a wave through space.

\item  This equation has the general form of a wave equation: $\nabla^2 \mathbf{E} = \frac{1}{v^2} \frac{\partial^2 \mathbf{E}}{\partial t^2}$, where $v$ is the wave's speed.  Comparing this with the given equation, we see that $v^2 = \frac{1}{\epsilon_0 \mu_0}$.  Therefore, the speed of light in free space, $c$, is given by:

$c = \frac{1}{\sqrt{\epsilon_0 \mu_0}}$

Substituting the values $\epsilon_0 = 8.854 \times 10^{-12} \, \text{F/m}$ and $\mu_0 = 4\pi \times 10^{-7} \, \text{H/m}$, we get:

$c \approx 3 \times 10^8 \, \text{m/s}$ 
\end{enumerate}
```

\subsubsection{Solution 2}
```latex
\textbf{Answer 2}

\begin{enumerate}
\item In classical electromagnetism, 'free space' refers to a perfect vacuum, devoid of any matter or charge. It is characterized by the electric permittivity $\epsilon_0$ and magnetic permeability $\mu_0$.

\item The equation $\nabla^2 \mathbf{E} = \epsilon_0 \mu_0 \frac{\partial^2 \mathbf{E}}{\partial t^2}$ is a wave equation. It describes how changes in the electric field  ($\frac{\partial^2 \mathbf{E}}{\partial t^2}$)  generate spatial variations in the field ($\nabla^2 \mathbf{E}$). This coupling between temporal and spatial variations allows the electric field to propagate as a wave through space.

\item This equation has the general form of a wave equation: $\nabla^2 \mathbf{E} = \frac{1}{v^2} \frac{\partial^2 \mathbf{E}}{\partial t^2}$, where $v$ is the wave's speed.  Comparing this to our equation, we see that $v^2 = \frac{1}{\epsilon_0 \mu_0}$.  Therefore, the speed of light in free space, $c$, is given by:

$c = \frac{1}{\sqrt{\epsilon_0 \mu_0}} \approx 3 \times 10^8 \, \text{m/s}$ 
\end{enumerate}
```

\subsubsection{Solution 3}
```latex
\textbf{Answer 2}

\begin{enumerate}
\item In classical electromagnetism, \textbf{free space} refers to a perfect vacuum, devoid of any matter or charge. It is characterized by the electric permittivity $\epsilon_0$ and magnetic permeability $\mu_0$. 

\item The equation $\nabla^2 \mathbf{E} = \epsilon_0 \mu_0 \frac{\partial^2 \mathbf{E}}{\partial t^2}$ is a wave equation that describes the propagation of electric fields in free space. 
    \begin{itemize}
        \item The left-hand side, $\nabla^2 \mathbf{E}$, represents the spatial variation of the electric field.
        \item The right-hand side, $\epsilon_0 \mu_0 \frac{\partial^2 \mathbf{E}}{\partial t^2}$, represents the time variation of the electric field. 
    \end{itemize}
    This equation shows that a change in the electric field at one point in space creates a disturbance that propagates outwards as a wave.

\item  This equation has the general form of a wave equation: $\nabla^2 \mathbf{E} = \frac{1}{v^2} \frac{\partial^2 \mathbf{E}}{\partial t^2}$, where $v$ is the speed of the wave.  Comparing this to our equation, we see that the speed of the electromagnetic wave (light) is given by:
    \[
    v = c = \frac{1}{\sqrt{\epsilon_0 \mu_0}} 
    \]
    Substituting the values for $\epsilon_0 = 8.854 \times 10^{-12} \, \text{F/m}$ and $\mu_0 = 4\pi \times 10^{-7} \, \text{H/m}$, we get:
    \[
    c \approx 3 \times 10^8 \, \text{m/s}
    \]
\end{enumerate}
```

\subsection{Question 3}
\begin{enumerate}
    \item Find the electric field a distance, \( d \), above one end of a straight line segment of length \( L \) that has a constant line charge density \( \lambda \) (as seen in the figure).
    \item In the limit of \( d \gg L \), what does the electric field reduce to and what is its physical interpretation?
\end{enumerate}


\paragraph{Prompt: \\} 
Answer the given question as a top physics student, giving step-by-step reasoning. Format your answer as LaTeX code without using sub-sections. Available marks: [14 marks]

\subsubsection{Solution 1}
\textbf{Question 3}

\begin{enumerate}
\item \textbf{Finding the Electric Field}

To find the electric field at a distance \(d\) above one end of the line segment, we'll follow these steps:

\begin{enumerate}
\item \textbf{Define the coordinate system:}  Let's place the line segment along the \(x\)-axis, with one end at the origin and the other end at \(x=L\). The point where we want to find the electric field is at \((0,d)\).

\item \textbf{Consider a small element of charge:} Let's take a small element of length \(dx\) at a distance \(x\) from the origin. The charge of this element is \(dq = \lambda dx\).

\item \textbf{Calculate the electric field due to this element:} The electric field due to this small charge element at the point \((0,d)\) is given by Coulomb's law:
    
    \[d\vec{E} = \frac{1}{4\pi\epsilon_0} \frac{dq}{r^2} \hat{r}\]

    where \(r\) is the distance from the charge element to the point \((0,d)\), and \(\hat{r}\) is the unit vector in that direction. We have \(r = \sqrt{x^2 + d^2}\) and 
    
    \[\hat{r} = \frac{-x\hat{x} + d\hat{y}}{\sqrt{x^2 + d^2}}\]

    Therefore,
    
    \[d\vec{E} = \frac{1}{4\pi\epsilon_0} \frac{\lambda dx}{(x^2 + d^2)} \frac{-x\hat{x} + d\hat{y}}{\sqrt{x^2 + d^2}}\]

\item \textbf{Integrate to find the total electric field:} To find the total electric field, we need to integrate the electric field due to all the small charge elements along the line segment:

    \[\vec{E} = \int d\vec{E} = \frac{\lambda}{4\pi\epsilon_0} \int_0^L \frac{-x\hat{x} + d\hat{y}}{(x^2 + d^2)^{3/2}} dx\]

    This integral can be solved by splitting it into two components:

    \begin{align*}
    E_x &= -\frac{\lambda}{4\pi\epsilon_0} \int_0^L \frac{x dx}{(x^2 + d^2)^{3/2}} \\
    &= \frac{\lambda}{4\pi\epsilon_0} \left[\frac{1}{\sqrt{x^2 + d^2}}\right]_0^L \\
    &= \frac{\lambda}{4\pi\epsilon_0} \left(\frac{1}{\sqrt{L^2 + d^2}} - \frac{1}{d}\right)
    \end{align*}

    \begin{align*}
    E_y &= \frac{\lambda d}{4\pi\epsilon_0} \int_0^L \frac{dx}{(x^2 + d^2)^{3/2}} \\
    &= \frac{\lambda d}{4\pi\epsilon_0} \left[\frac{x}{d^2 \sqrt{x^2 + d^2}}\right]_0^L \\
    &= \frac{\lambda L}{4\pi\epsilon_0 d \sqrt{L^2 + d^2}}
    \end{align*}

    Therefore, the total electric field is:

    \[\vec{E} = \frac{\lambda}{4\pi\epsilon_0} \left[\left(\frac{1}{\sqrt{L^2 + d^2}} - \frac{1}{d}\right)\hat{x} + \frac{L}{d \sqrt{L^2 + d^2}}\hat{y}\right]\]

\end{enumerate}

\item \textbf{Limiting Case: \(d \gg L\)}

In the limit \(d \gg L\), we can use the binomial approximation:

\[\frac{1}{\sqrt{L^2 + d^2}} \approx \frac{1}{d} \left(1 - \frac{L^2}{2d^2}\right)\]

Substituting this into the expression for the electric field, we get:

\begin{align*}
\vec{E} &\approx \frac{\lambda}{4\pi\epsilon_0} \left[\left(\frac{1}{d} \left(1 - \frac{L^2}{2d^2}\right) - \frac{1}{d}\right)\hat{x} + \frac{L}{d^2 \left(1 - \frac{L^2}{2d^2}\right)}\hat{y}\right] \\
&\approx \frac{\lambda}{4\pi\epsilon_0} \left[-\frac{L^2}{2d^3}\hat{x} + \frac{L}{d^2}\hat{y}\right] \\
&\approx \frac{1}{4\pi\epsilon_0} \frac{\lambda L}{d^2} \hat{y}
\end{align*}

Since \(\lambda L\) is the total charge \(Q\) on the line segment, the electric field reduces to:

\[\vec{E} \approx \frac{1}{4\pi\epsilon_0} \frac{Q}{d^2} \hat{y}\]

This is the electric field due to a point charge \(Q\) at a distance \(d\).

\textbf{Physical Interpretation:} When we are far away from the line segment (\(d \gg L\)), the line segment appears as a point charge. This is because the length of the line segment becomes negligible compared to the distance from the point of observation. Therefore, the electric field behaves like that of a point charge.
\end{enumerate}


\subsubsection{Solution 2}
\textbf{1. Electric Field Calculation}

\begin{enumerate}
\item \textbf{Define the coordinate system:}  We place the line segment along the \(x\)-axis, with one end at the origin and the other end at \(x = L\). The point where we want to find the electric field is at \((0, d)\).

\item \textbf{Consider an infinitesimal charge element:} Let \(dq = \lambda dx\) be the charge on an infinitesimal length \(dx\) of the line segment at position \(x\).

\item \textbf{Determine the electric field due to the infinitesimal charge:} The distance from this charge element to the point \((0, d)\) is \(r = \sqrt{x^2 + d^2}\). The magnitude of the electric field due to this charge element is:
    
    \[dE = \frac{1}{4\pi\epsilon_0} \frac{dq}{r^2} = \frac{1}{4\pi\epsilon_0} \frac{\lambda dx}{x^2 + d^2}\]

\item \textbf{Resolve the electric field into components:} The electric field \(dE\) has two components:
    \begin{itemize}
    \item \(dE_x = dE \cos \theta = dE \frac{x}{r} = \frac{1}{4\pi\epsilon_0} \frac{\lambda x dx}{(x^2 + d^2)^{3/2}}\)
    \item \(dE_y = dE \sin \theta = dE \frac{d}{r} = \frac{1}{4\pi\epsilon_0} \frac{\lambda d dx}{(x^2 + d^2)^{3/2}}\)
    \end{itemize}

\item \textbf{Integrate to find the total electric field:} To find the total electric field, we integrate the components over the length of the line segment:

    \begin{align*}
    E_x &= \int_0^L dE_x = \frac{\lambda}{4\pi\epsilon_0} \int_0^L \frac{x dx}{(x^2 + d^2)^{3/2}} \\
    &= \frac{\lambda}{4\pi\epsilon_0} \left[-\frac{1}{\sqrt{x^2 + d^2}}\right]_0^L \\
    &= \frac{\lambda}{4\pi\epsilon_0} \left(\frac{1}{d} - \frac{1}{\sqrt{L^2 + d^2}}\right)
    \end{align*}

    \begin{align*}
    E_y &= \int_0^L dE_y = \frac{\lambda d}{4\pi\epsilon_0} \int_0^L \frac{dx}{(x^2 + d^2)^{3/2}} \\
    &= \frac{\lambda d}{4\pi\epsilon_0} \left[\frac{x}{d^2 \sqrt{x^2 + d^2}}\right]_0^L \\
    &= \frac{\lambda L}{4\pi\epsilon_0 d \sqrt{L^2 + d^2}}
    \end{align*}

\item \textbf{Express the total electric field:} The total electric field is given by:

    \[\mathbf{E} = E_x \hat{x} + E_y \hat{y} = \frac{\lambda}{4\pi\epsilon_0} \left[\left(\frac{1}{d} - \frac{1}{\sqrt{L^2 + d^2}}\right) \hat{x} + \frac{L}{d \sqrt{L^2 + d^2}} \hat{y}\right]\]
\end{enumerate}

\textbf{2. Limiting Case}

\begin{enumerate}
\item \textbf{Apply the limit \(d \gg L\):} In this limit, we can use the binomial approximation:

    \[\frac{1}{\sqrt{L^2 + d^2}} = \frac{1}{d} \left(1 + \frac{L^2}{d^2}\right)^{-1/2} \approx \frac{1}{d} \left(1 - \frac{1}{2} \frac{L^2}{d^2}\right)\]

\item \textbf{Simplify the electric field expression:} Substituting this approximation into the expression for \(\mathbf{E}\) and neglecting higher-order terms in \(\frac{L}{d}\), we get:

    \begin{align*}
    \mathbf{E} &\approx \frac{\lambda}{4\pi\epsilon_0} \left[\left(\frac{1}{d} - \frac{1}{d} \left(1 - \frac{1}{2} \frac{L^2}{d^2}\right)\right) \hat{x} + \frac{L}{d^2} \left(1 - \frac{1}{2} \frac{L^2}{d^2}\right) \hat{y}\right] \\
    &\approx \frac{\lambda}{4\pi\epsilon_0} \left[\frac{L^2}{2d^3} \hat{x} + \frac{L}{d^2} \hat{y}\right] \\
    &\approx \frac{1}{4\pi\epsilon_0} \frac{\lambda L}{d^2} \hat{y} = \frac{1}{4\pi\epsilon_0} \frac{Q}{d^2} \hat{y}
    \end{align*}
    where \(Q = \lambda L\) is the total charge on the line segment.

\item \textbf{Physical interpretation:} This result is the electric field due to a point charge \(Q\) located at the origin. When the distance \(d\) is much larger than the length \(L\) of the line segment, the line segment appears as a point charge from the observation point.
\end{enumerate}

\subsubsection{Solution 3}
\textbf{Question 3}

\begin{enumerate}
\item To find the electric field a distance \(d\) above one end of a straight line segment of length \(L\) with constant line charge density \(\lambda\), we can use the principle of superposition. We'll integrate the electric field contributions from infinitesimal charge elements along the line segment.

\begin{itemize}
\item Consider an infinitesimal charge element \(dq = \lambda dx\) at a distance \(x\) from the end of the line segment.
\item The distance from this charge element to the point of interest is \(r = \sqrt{x^2 + d^2}\).
\item The electric field contribution from this charge element is 
\begin{equation*}
d\vec{E} = \frac{1}{4\pi\epsilon_0} \frac{dq}{r^2} \hat{r} = \frac{1}{4\pi\epsilon_0} \frac{\lambda dx}{(x^2 + d^2)} \hat{r},
\end{equation*}
where \(\hat{r}\) is the unit vector pointing from the charge element to the point of interest.
\item We need to resolve \(d\vec{E}\) into its horizontal and vertical components. Using the geometry of the problem, we find:
\begin{align*}
dE_x &= d\vec{E} \cdot \cos\theta = \frac{1}{4\pi\epsilon_0} \frac{\lambda dx}{(x^2 + d^2)} \frac{x}{\sqrt{x^2 + d^2}} = \frac{1}{4\pi\epsilon_0} \frac{\lambda x dx}{(x^2 + d^2)^{3/2}}, \\
dE_y &= d\vec{E} \cdot \sin\theta = \frac{1}{4\pi\epsilon_0} \frac{\lambda dx}{(x^2 + d^2)} \frac{d}{\sqrt{x^2 + d^2}} = \frac{1}{4\pi\epsilon_0} \frac{\lambda d dx}{(x^2 + d^2)^{3/2}}.
\end{align*}
\item Now we integrate these components over the length of the line segment:
\begin{align*}
E_x &= \int_0^L dE_x = \frac{1}{4\pi\epsilon_0} \lambda \int_0^L \frac{x dx}{(x^2 + d^2)^{3/2}} = \frac{\lambda}{4\pi\epsilon_0} \left[ -\frac{1}{\sqrt{x^2 + d^2}} \right]_0^L \\
&= \frac{\lambda}{4\pi\epsilon_0} \left( \frac{1}{d} - \frac{1}{\sqrt{L^2 + d^2}} \right), \\
E_y &= \int_0^L dE_y = \frac{1}{4\pi\epsilon_0} \lambda d \int_0^L \frac{dx}{(x^2 + d^2)^{3/2}} = \frac{\lambda d}{4\pi\epsilon_0} \left[ \frac{x}{d^2 \sqrt{x^2 + d^2}} \right]_0^L \\
&= \frac{\lambda L}{4\pi\epsilon_0 d \sqrt{L^2 + d^2}}.
\end{align*}
\item Therefore, the total electric field at the point of interest is
\begin{equation*}
\vec{E} = E_x \hat{x} + E_y \hat{y} = \frac{\lambda}{4\pi\epsilon_0} \left[ \left( \frac{1}{d} - \frac{1}{\sqrt{L^2 + d^2}} \right) \hat{x} + \frac{L}{d \sqrt{L^2 + d^2}} \hat{y} \right].
\end{equation*}
\end{itemize}

\item In the limit of \(d \gg L\), we can use the binomial approximation:
\begin{equation*}
\frac{1}{\sqrt{L^2 + d^2}} = \frac{1}{d} \left( 1 + \frac{L^2}{d^2} \right)^{-1/2} \approx \frac{1}{d} \left( 1 - \frac{1}{2} \frac{L^2}{d^2} \right).
\end{equation*}
Substituting this into the expression for \(\vec{E}\), we get
\begin{align*}
\vec{E} &\approx \frac{\lambda}{4\pi\epsilon_0} \left[ \left( \frac{1}{d} - \frac{1}{d} \left( 1 - \frac{1}{2} \frac{L^2}{d^2} \right) \right) \hat{x} + \frac{L}{d^2 \left( 1 + \frac{L^2}{d^2} \right)^{1/2}} \hat{y} \right] \\
&\approx \frac{\lambda}{4\pi\epsilon_0} \left[ \frac{L^2}{2d^3} \hat{x} + \frac{L}{d^2} \hat{y} \right].
\end{align*}
Since \(d \gg L\), the \(x\)-component of the electric field becomes negligible compared to the \(y\)-component. Thus, the electric field reduces to
\begin{equation*}
\vec{E} \approx \frac{\lambda L}{4\pi\epsilon_0 d^2} \hat{y} = \frac{Q}{4\pi\epsilon_0 d^2} \hat{y},
\end{equation*}
where \(Q = \lambda L\) is the total charge on the line segment. This is the electric field of a point charge, which makes sense because when viewed from a very large distance, the line segment effectively looks like a point charge.
\end{enumerate}


\subsection{Question 4}
A square loop of wire (sides of length $L$) lies a fixed distance $d$ from a very long straight wire, which carries a constant current $I$ directed downwards (as seen in the figure).

\begin{enumerate}
    \item[(a)] Find the flux of $\mathbf{B}$ through the loop.
    \item[(b)] If the loop is moved directly away from the wire to the right, at speed $v$, what emf is generated? In what direction (clockwise or counterclockwise) does the current flow?
    \item[(c)] What happens if the loop only moves downwards at speed $v$?
\end{enumerate}


\paragraph{Prompt: \\} 
Answer the given question as a top physics student, giving step-by-step reasoning. Format your answer as LaTeX code without using sub-sections. Available marks: a) [5 marks], b) [5 marks], c [1 mark].

\subsubsection{Solution 1}
\textbf{Question 4}

\begin{enumerate}
\item[(a)] \textbf{[5 marks]} 
To find the magnetic flux through the loop, we'll use the formula:

\begin{equation*}
\Phi_B = \int \mathbf{B} \cdot d\mathbf{A}
\end{equation*}

where $\mathbf{B}$ is the magnetic field and $d\mathbf{A}$ is an infinitesimal area vector.

First, we need to determine the magnetic field produced by the long straight wire. Using Ampère's law, the magnetic field at a distance $r$ from a long straight wire carrying current $I$ is given by:

\begin{equation*}
B = \frac{\mu_0 I}{2 \pi r}
\end{equation*}

where $\mu_0$ is the permeability of free space.

In our case, the magnetic field varies across the loop since the distance from the wire changes. Consider a thin vertical strip of width $dx$ at a distance $x$ from the wire. The magnetic field is constant within this strip and is given by:

\begin{equation*}
B = \frac{\mu_0 I}{2 \pi x}
\end{equation*}

The area of this strip is $dA = L dx$. The flux through this strip is:

\begin{equation*}
d\Phi_B = B dA = \frac{\mu_0 I L}{2 \pi x} dx
\end{equation*}

To find the total flux, we integrate over the entire loop, from $x = d$ to $x = d + L$:

\begin{align*}
\Phi_B &= \int_{d}^{d+L} \frac{\mu_0 I L}{2 \pi x} dx \\
&= \frac{\mu_0 I L}{2 \pi} \int_{d}^{d+L} \frac{1}{x} dx \\
&= \frac{\mu_0 I L}{2 \pi} \left[ \ln(x) \right]_{d}^{d+L} \\
&= \frac{\mu_0 I L}{2 \pi} \ln \left( \frac{d+L}{d} \right)
\end{align*}

Therefore, the flux of $\mathbf{B}$ through the loop is:

\begin{equation*}
\boxed{\Phi_B = \frac{\mu_0 I L}{2 \pi} \ln \left( \frac{d+L}{d} \right)}
\end{equation*}


\item[(b)] \textbf{[5 marks]} 
To find the emf generated, we use Faraday's law of induction:

\begin{equation*}
\mathcal{E} = -\frac{d\Phi_B}{dt}
\end{equation*}

Since the loop is moving to the right at speed $v$, the distance $d$ is changing with time: $d(t) = d + vt$.  We can now differentiate the flux with respect to time:

\begin{align*}
\mathcal{E} &= -\frac{d}{dt} \left[ \frac{\mu_0 I L}{2 \pi} \ln \left( \frac{d(t)+L}{d(t)} \right) \right] \\
&= -\frac{\mu_0 I L}{2 \pi} \frac{d}{dt} \left[ \ln \left( \frac{d + vt + L}{d + vt} \right) \right] \\
&= -\frac{\mu_0 I L}{2 \pi} \cdot \frac{d + vt}{d + vt + L} \cdot \frac{v(d+vt) - v(d+vt+L)}{(d+vt)^2} \\
&= \frac{\mu_0 I L^2 v}{2 \pi (d+vt)(d+vt+L)}
\end{align*}

Therefore, the emf generated is:

\begin{equation*}
\boxed{\mathcal{E} = \frac{\mu_0 I L^2 v}{2 \pi (d+vt)(d+vt+L)}}
\end{equation*}

To determine the direction of the induced current, we use Lenz's law. As the loop moves to the right, the magnetic flux through the loop decreases. To oppose this change, the induced current will create a magnetic field that points into the page within the loop. Using the right-hand rule, this corresponds to a \textbf{clockwise} current.


\item[(c)] \textbf{[1 mark]} 
If the loop moves downwards at speed $v$, the distance $d$ from the wire remains constant. Therefore, the magnetic flux through the loop does not change, and no emf is generated. 
\end{enumerate}

\subsubsection{Solution 2}
\textbf{(a) [5 marks]}

To find the magnetic flux through the loop, we'll follow these steps:

1. **Determine the magnetic field:** The magnetic field due to a long straight wire carrying current $I$ at a distance $r$ is given by Ampère's Law:
   
   $B = \frac{\mu_0 I}{2 \pi r}$

   where $\mu_0$ is the permeability of free space.

2. **Consider a differential area element:**  We'll take a thin vertical strip of width $dx$ and height $L$ within the loop at a distance $x$ from the wire. The area of this strip is $dA = L dx$.

3. **Calculate the flux through the differential area:** The magnetic field is perpendicular to this strip, so the flux through this small area is:

   $d\Phi = B dA = \frac{\mu_0 I}{2 \pi x} L dx$

4. **Integrate to find the total flux:** To find the total flux through the loop, we integrate $d\Phi$ over the entire width of the loop, from $x = d$ to $x = d + L$:

   $\Phi = \int_{d}^{d+L} \frac{\mu_0 I L}{2 \pi x} dx = \frac{\mu_0 I L}{2 \pi} \int_{d}^{d+L} \frac{1}{x} dx$

   $\Phi = \frac{\mu_0 I L}{2 \pi} \left[ \ln(x) \right]_{d}^{d+L} = \frac{\mu_0 I L}{2 \pi} \ln \left( \frac{d+L}{d} \right)$

Therefore, the magnetic flux through the loop is:

$\boxed{\Phi = \frac{\mu_0 I L}{2 \pi} \ln \left( \frac{d+L}{d} \right)}$


\textbf{(b) [5 marks]}

To find the emf and current direction, we'll use Faraday's Law and Lenz's Law:

1. **Apply Faraday's Law:** Faraday's Law states that the induced emf is equal to the negative rate of change of magnetic flux:

   $\mathcal{E} = -\frac{d\Phi}{dt}$

2. **Determine the change in flux:** As the loop moves to the right with speed $v$, the distance $d$ increases.  Therefore, we can write $d$ as a function of time: $d(t) = d_0 + vt$, where $d_0$ is the initial distance. Now, we can express the flux as a function of time:

   $\Phi(t) = \frac{\mu_0 I L}{2 \pi} \ln \left( \frac{d_0 + vt + L}{d_0 + vt} \right)$

3. **Calculate the derivative of the flux:** 

   $\frac{d\Phi}{dt} = \frac{\mu_0 I L}{2 \pi} \frac{d}{dt} \left[ \ln \left( \frac{d_0 + vt + L}{d_0 + vt} \right) \right]$

   $\frac{d\Phi}{dt} = \frac{\mu_0 I L}{2 \pi} \left[ \frac{v}{d_0 + vt + L} - \frac{v}{d_0 + vt} \right] = -\frac{\mu_0 I L^2 v}{2 \pi (d_0 + vt)(d_0 + vt + L)}$

4. **Find the emf:**

   $\mathcal{E} = -\frac{d\Phi}{dt} = \boxed{\frac{\mu_0 I L^2 v}{2 \pi (d_0 + vt)(d_0 + vt + L)}}$

5. **Determine the direction of the current using Lenz's Law:** As the loop moves to the right, the magnetic flux through it decreases.  To oppose this change, the induced current will create a magnetic field that points into the page within the loop. By the right-hand rule, this corresponds to a \textbf{clockwise} current.


\textbf{(c) [1 mark]}

If the loop moves downwards at speed $v$, the distance $d$ between the wire and the loop remains constant. Since the flux depends only on $d$, and $d$ is not changing, the flux remains constant.  Therefore, $\frac{d\Phi}{dt} = 0$, and no emf is generated.

$\boxed{\mathcal{E} = 0}$


\subsubsection{Solution 3}
\textbf{(a) [5 marks]}

To find the magnetic flux through the loop, we'll use the following steps:

1. **Determine the magnetic field due to the long wire:**
The magnetic field around a long straight wire carrying current $I$ is given by Ampère's Law:

$B = \frac{\mu_0 I}{2 \pi r}$

where:
* $B$ is the magnetic field strength
* $\mu_0$ is the permeability of free space
* $I$ is the current in the wire
* $r$ is the distance from the wire

2. **Consider a differential area element:**
Take a thin vertical strip of width $dx$ and height $L$ within the loop at a distance $x$ from the wire. The area of this strip is $dA = L dx$.

3. **Calculate the flux through the differential area:**
The magnetic field is constant across this strip (since it's thin), and perpendicular to it.  The flux through this strip is:

$d\Phi = B dA = \frac{\mu_0 I}{2 \pi x} L dx$

4. **Integrate to find the total flux:**
To find the total flux through the loop, we integrate $d\Phi$ over the entire width of the loop, from $x = d$ to $x = d + L$:

$\Phi = \int_{d}^{d+L} \frac{\mu_0 I L}{2 \pi x} dx = \frac{\mu_0 I L}{2 \pi} \int_{d}^{d+L} \frac{1}{x} dx$

$\Phi = \frac{\mu_0 I L}{2 \pi} \left[ \ln(x) \right]_{d}^{d+L} = \frac{\mu_0 I L}{2 \pi} \ln \left( \frac{d+L}{d} \right)$

Therefore, the flux of $\mathbf{B}$ through the loop is:

$\boxed{\Phi = \frac{\mu_0 I L}{2 \pi} \ln \left( \frac{d+L}{d} \right)}$


\textbf{(b) [5 marks]}

To find the emf generated and the direction of the induced current, we'll use Faraday's Law:

1. **Apply Faraday's Law:**
Faraday's Law states that the induced emf is equal to the negative rate of change of magnetic flux:

$\mathcal{E} = -\frac{d\Phi}{dt}$

2. **Determine the rate of change of flux:**
Since the loop is moving to the right at speed $v$, the distance $d$ is increasing with time: $d(t) = d + vt$.  We can now express the flux as a function of time:

$\Phi(t) = \frac{\mu_0 I L}{2 \pi} \ln \left( \frac{d + vt + L}{d + vt} \right)$

Taking the derivative with respect to time:

$\frac{d\Phi}{dt} = \frac{\mu_0 I L}{2 \pi} \left[ \frac{v}{d + vt + L} - \frac{v}{d + vt} \right] = -\frac{\mu_0 I L^2 v}{2 \pi (d + vt)(d + vt + L)}$

3. **Calculate the emf:**
Substituting the rate of change of flux into Faraday's Law:

$\mathcal{E} =  \frac{\mu_0 I L^2 v}{2 \pi (d + vt)(d + vt + L)}$

4. **Determine the direction of the induced current:**
The magnetic flux through the loop is decreasing as it moves away from the wire. By Lenz's Law, the induced current will flow in a direction to oppose this change. This means the induced current will create a magnetic field that points into the page within the loop, which requires a **clockwise** current.

Therefore, the emf generated is:

$\boxed{\mathcal{E} = \frac{\mu_0 I L^2 v}{2 \pi (d + vt)(d + vt + L)}}$

and the induced current flows **clockwise**.


\textbf{(c) [1 mark]}

If the loop moves downwards at speed $v$, the distance $d$ between the wire and the loop remains constant.  Since the flux depends only on $d$, and not on the loop's vertical position, the flux remains constant.  Therefore, $\frac{d\Phi}{dt} = 0$, and no emf is generated.

$\boxed{\mathcal{E} = 0}$


\subsection{Question 5}
Consider the circuit diagram seen in the figure.
\begin{enumerate}
    \item Calculate the current seen by the ammeter.
    \item Calculate the energy delivered by the 12 V battery in 4 seconds.
\end{enumerate}


\paragraph{Prompt: \\} 
Answer the given question as a top physics student, giving step-by-step reasoning. Format your answer as LaTeX code without using sub-sections. Available marks: a) [8 marks], b) [3 marks].

\subsubsection{Solution 1}
Okay, I've analyzed the circuit diagram. Here's the solution:

**a) Calculate the current seen by the ammeter. [8 marks]**

1. **Simplify the circuit:**
    * The two 1 $\Omega$ resistors are in series, so their equivalent resistance is $R_1 = 1 \Omega + 1 \Omega = 2 \Omega$.
    * This $2 \Omega$ resistance is in parallel with the $2 \Omega$ resistor next to the ammeter. Their equivalent resistance is $R_2 = \frac{2 \Omega \cdot 2 \Omega}{2 \Omega + 2 \Omega} = 1 \Omega$.
    * Now, $R_2$ is in series with the $2 \Omega$ resistor at the top. This gives us a total resistance of $R_T = 1 \Omega + 2 \Omega = 3 \Omega$.

2. **Calculate the total current:**
    * The total voltage in the circuit is $V_T = 12 V + 2 V = 14 V$.
    * Using Ohm's Law ($V = IR$), the total current is $I_T = \frac{V_T}{R_T} = \frac{14 V}{3 \Omega} = \frac{14}{3} A$.

3. **Calculate the voltage across $R_2$:**
    * Using Ohm's Law, the voltage across $R_2$ is $V_2 = I_T \cdot R_2 = \frac{14}{3} A \cdot 1 \Omega = \frac{14}{3} V$.

4. **Calculate the current through the ammeter:**
    * The voltage across the $2 \Omega$ resistor next to the ammeter is also $V_2 = \frac{14}{3} V$.
    * Using Ohm's Law, the current through the ammeter is $I_A = \frac{V_2}{2 \Omega} = \frac{14}{3} V / 2 \Omega = \frac{7}{3} A$.

Therefore, the current seen by the ammeter is $\boxed{\frac{7}{3} A}$.


**b) Calculate the energy delivered by the 12 V battery in 4 seconds. [3 marks]**

1. **Calculate the power delivered by the 12 V battery:**
    * Power is given by $P = IV$.
    * The current through the 12 V battery is the total current $I_T = \frac{14}{3} A$.
    * The power delivered by the 12 V battery is $P_{12V} = 12 V \cdot \frac{14}{3} A = 56 W$.

2. **Calculate the energy delivered:**
    * Energy is given by $E = Pt$.
    * The energy delivered by the 12 V battery in 4 seconds is $E_{12V} = 56 W \cdot 4 s = 224 J$.

Therefore, the energy delivered by the 12 V battery in 4 seconds is $\boxed{224 J}$.


\subsubsection{Solution 2}
Okay, I've analyzed the circuit diagram. Here's the solution:

**a) Calculate the current seen by the ammeter. [8 marks]**

1. **Simplify the circuit:**
    * The two 1\(\Omega\) resistors are in series, so their equivalent resistance is \(1\Omega + 1\Omega = 2\Omega\).
    * This \(2\Omega\) equivalent resistance is in parallel with the \(2\Omega\) resistor next to the ammeter. The equivalent resistance of this parallel combination is: 
       \begin{align*}
       \frac{1}{R_{eq}} &= \frac{1}{2\Omega} + \frac{1}{2\Omega} = \frac{1}{\Omega} \\
       R_{eq} &= 1\Omega
       \end{align*}

2. **Redraw the simplified circuit:**
    * The circuit now consists of a 12V battery, a \(2\Omega\) resistor, a 2V battery, and the \(1\Omega\) equivalent resistance we just calculated.

3. **Apply Kirchhoff's Voltage Law (KVL):**
    * Choose a loop direction (let's say clockwise). 
    * Starting from the 12V battery, write the KVL equation:
        \(12V - 2\Omega I - 2V - 1\Omega I = 0\)

4. **Solve for the current (I):**
    * Combine terms: \(10V = 3\Omega I\)
    * Isolate I: \(I = \frac{10V}{3\Omega} = 3.33A\)

Therefore, the current seen by the ammeter is \(\boxed{3.33A}\).


**b) Calculate the energy delivered by the 12V battery in 4 seconds. [3 marks]**

1. **Use the formula for electrical energy:**
    * Energy (E) = Power (P) x time (t)
    * Power (P) = Voltage (V) x Current (I)

2. **Calculate the power delivered by the 12V battery:**
    * \(P = 12V \times 3.33A = 40W\)

3. **Calculate the energy delivered in 4 seconds:**
    * \(E = 40W \times 4s = 160J\)

Therefore, the energy delivered by the 12V battery in 4 seconds is \(\boxed{160J}\).


\subsubsection{Solution 3}
Okay, I've analyzed the circuit diagram. Here's the solution:

**a) Calculate the current seen by the ammeter. [8 marks]**

1. **Simplify the circuit:**
    * The two 1 $\Omega$ resistors in series on the right-hand side can be combined into a single 2 $\Omega$ resistor. 
    * This 2 $\Omega$ resistor is in parallel with the 2 $\Omega$ resistor next to the ammeter. The equivalent resistance of these two parallel resistors is:
    
    $R_{eq} = \frac{2 \Omega \times 2 \Omega}{2 \Omega + 2 \Omega} = 1 \Omega$

    * Now the circuit can be redrawn with a single 1 $\Omega$ resistor in the right-hand branch.

2. **Apply Kirchhoff's Voltage Law (KVL):**
    * Define a clockwise loop around the entire circuit.
    * Starting from the 12V battery, apply KVL:

    $+12V - 2\Omega I_1 - 2V - 1\Omega I_2 = 0$

    * Where $I_1$ is the current flowing through the top 2 $\Omega$ resistor and $I_2$ is the current flowing through the equivalent 1 $\Omega$ resistor (and hence through the ammeter).

3. **Apply Kirchhoff's Current Law (KCL):**
    * At the junction between the two branches:

    $I_1 = I_2 + I_3$

    * Where $I_3$ is the current flowing through the 2 $\Omega$ resistor next to the 2V source.

4. **Apply Ohm's Law:**
    * For the branch with the 2V source:

    $2V = 2\Omega I_3$

    * Solving for $I_3$:

    $I_3 = \frac{2V}{2\Omega} = 1A$

5. **Solve the system of equations:**
    * Substitute $I_3 = 1A$ into the KCL equation:

    $I_1 = I_2 + 1A$

    * Substitute this expression for $I_1$ into the KVL equation:

    $+12V - 2\Omega (I_2 + 1A) - 2V - 1\Omega I_2 = 0$

    * Simplify and solve for $I_2$:

    $10V - 2\Omega I_2 - 2V - 1\Omega I_2 = 0$

    $8V = 3\Omega I_2$

    $I_2 = \frac{8V}{3\Omega} = \frac{8}{3} A$

    * Therefore, the current seen by the ammeter is $\frac{8}{3} A$.

**b) Calculate the energy delivered by the 12 V battery in 4 seconds. [3 marks]**

1. **Calculate the power delivered by the battery:**
    * Power is given by:

    $P = VI$

    * Where $V$ is the voltage of the battery and $I$ is the current flowing through it.  We know $V = 12V$ and we can find $I_1$ (the current through the battery) using the equations from part (a):

    $I_1 = I_2 + 1A = \frac{8}{3} A + 1A = \frac{11}{3} A$

    * Therefore, the power is:

    $P = 12V \times \frac{11}{3} A = 44W$

2. **Calculate the energy delivered:**
    * Energy is given by:

    $E = Pt$

    * Where $P$ is the power and $t$ is the time. 

    $E = 44W \times 4s = 176J$

    * Therefore, the energy delivered by the 12V battery in 4 seconds is 176J.


**Final Answers:**

* a) $I_2 = \frac{8}{3} A$
* b) $E = 176 J$ 


\subsection{Question 6}
Q6. Two infinite parallel planes have equal and opposite uniform surface charge density $-\sigma$ and $+\sigma$, where $\sigma$ is positive. The planes are separated by a distance of 15 cm, and the potential difference between the plates is 120 V.

\begin{enumerate}
    \item[(a)] Determine the electric field, $E$, between the plates.
    \item[(b)] An object with charge $+0.001 \, \text{C}$ and a mass of $23 \, \text{g}$ is held at rest at the positive plate, then let go. Determine the acceleration of the object.
\end{enumerate}


\paragraph{Prompt: \\} 
Answer the given question as a top physics student, giving step-by-step reasoning. Format your answer as LaTeX code without using sub-sections. Available marks: a) [3 marks], b) [2 marks].

\subsubsection{Solution 1}
\textbf{Question 6}

\textbf{(a)} 

The electric field between two infinite parallel plates with equal and opposite uniform surface charge density is given by:

\begin{equation*}
E = \frac{\sigma}{\epsilon_0}
\end{equation*}

where:

* $E$ is the electric field strength
* $\sigma$ is the surface charge density
* $\epsilon_0$ is the permittivity of free space ($8.85 \times 10^{-12} \, \text{C}^2/\text{N} \cdot \text{m}^2$)

We can relate the electric field to the potential difference ($V$) and the distance between the plates ($d$) using:

\begin{equation*}
E = \frac{V}{d}
\end{equation*}

Combining these equations to solve for $\sigma$, we get:

\begin{equation*}
\sigma = \frac{V \epsilon_0}{d} 
\end{equation*}

Substituting the given values:

\begin{equation*}
\sigma = \frac{120 \, \text{V} \cdot 8.85 \times 10^{-12} \, \text{C}^2/\text{N} \cdot \text{m}^2}{0.15 \, \text{m}} = 7.08 \times 10^{-9} \, \text{C/m}^2
\end{equation*}

Now we can find the electric field:

\begin{equation*}
E = \frac{7.08 \times 10^{-9} \, \text{C/m}^2}{8.85 \times 10^{-12} \, \text{C}^2/\text{N} \cdot \text{m}^2} = \boxed{8.00 \times 10^2 \, \text{N/C}}
\end{equation*}

\textbf{(b)} 

The force ($F$) experienced by a charge ($q$) in an electric field is given by:

\begin{equation*}
F = qE
\end{equation*}

Using Newton's second law, we can relate this force to the object's mass ($m$) and acceleration ($a$):

\begin{equation*}
F = ma
\end{equation*}

Combining these equations, we get:

\begin{equation*}
a = \frac{qE}{m}
\end{equation*}

Substituting the given values and the electric field calculated in part (a):

\begin{equation*}
a = \frac{0.001 \, \text{C} \cdot 8.00 \times 10^2 \, \text{N/C}}{0.023 \, \text{kg}} = \boxed{34.8 \, \text{m/s}^2}
\end{equation*}

\subsubsection{Solution 2}
**Question 6**

**(a) Determine the electric field, $E$, between the plates.** [3 marks]

**1. Relevant Equations:**

*   The electric field due to an infinite plane sheet of charge is given by:  $E = \frac{\sigma}{2\epsilon_0}$, where $\sigma$ is the surface charge density and $\epsilon_0$ is the permittivity of free space.
*   The electric field between two parallel plates with equal and opposite charge densities is the sum of the fields due to each plate.
*   The potential difference between two points in a uniform electric field is given by: $\Delta V = Ed$, where $d$ is the distance between the points.

**2.  Calculate the electric field due to each plate:**

Since the plates have equal and opposite charge densities, the magnitude of the electric field due to each plate is the same:

$E_{plate} = \frac{\sigma}{2\epsilon_0}$

**3. Calculate the total electric field between the plates:**

The electric fields due to each plate point in the same direction between the plates, so they add up:

$E_{total} = E_{plate} + E_{plate} = 2 \cdot \frac{\sigma}{2\epsilon_0} = \frac{\sigma}{\epsilon_0}$

**4.  Relate the electric field to the potential difference:**

We are given the potential difference between the plates, $\Delta V$, and the distance between them, $d$. We can use this information to find the electric field:

$E_{total} = \frac{\Delta V}{d}$

**5. Combine the expressions for the electric field:**

We now have two expressions for the total electric field. Setting them equal to each other, we get:

$\frac{\sigma}{\epsilon_0} = \frac{\Delta V}{d}$

**6. Solve for the electric field:**

$E_{total} = \frac{\Delta V}{d}$

**7. Substitute the given values:**

$E_{total} = \frac{120 \, \text{V}}{0.15 \, \text{m}} = 800 \, \text{V/m}$

**Therefore, the electric field between the plates is 800 V/m.**

**(b) Determine the acceleration of the object.** [2 marks]

**1. Relevant Equations:**

*   Force on a charged object in an electric field: $F = qE$, where $q$ is the charge and $E$ is the electric field.
*   Newton's second law of motion: $F = ma$, where $m$ is the mass and $a$ is the acceleration.

**2. Calculate the force on the object:**

$F = qE = (+0.001 \, \text{C})(800 \, \text{V/m}) = 0.8 \, \text{N}$

**3. Calculate the acceleration of the object:**

$a = \frac{F}{m} = \frac{0.8 \, \text{N}}{0.023 \, \text{kg}} \approx 34.78 \, \text{m/s}^2$

**Therefore, the acceleration of the object is approximately 34.78 m/s².** 

\subsubsection{Solution 3}
**a) Determine the electric field, $E$, between the plates. [3 marks]**

**1.  Gauss's Law:**

We start with Gauss's Law, which relates the electric flux through a closed surface to the enclosed charge:

\begin{equation*}
\Phi_E = \oint \mathbf{E} \cdot d\mathbf{A} = \frac{Q_{enc}}{\epsilon_0}
\end{equation*}

where:

* $\Phi_E$ is the electric flux
* $\mathbf{E}$ is the electric field
* $d\mathbf{A}$ is a differential area vector 
* $Q_{enc}$ is the enclosed charge
* $\epsilon_0$ is the permittivity of free space.

**2. Gaussian Surface:**

To find the electric field between the plates, we choose a cylindrical Gaussian surface with its axis perpendicular to the plates. The ends of the cylinder have an area $A$ and lie within the plates.

**3.  Symmetry and Flux:**

Due to the symmetry of the infinite planes, the electric field is uniform and perpendicular to the plates.  This means the electric field is parallel to the area vector on the ends of the cylinder and zero on the curved surface. Thus, the flux through the Gaussian surface is:

\begin{equation*}
\Phi_E = EA + EA = 2EA
\end{equation*}

**4. Enclosed Charge:**

The charge enclosed by the Gaussian surface is the charge density multiplied by the area of the end of the cylinder:

\begin{equation*}
Q_{enc} = \sigma A
\end{equation*}

**5. Applying Gauss's Law:**

Substituting the flux and enclosed charge into Gauss's Law, we get:

\begin{equation*}
2EA = \frac{\sigma A}{\epsilon_0}
\end{equation*}

The area $A$ cancels out, and we solve for the electric field:

\begin{equation*}
E = \frac{\sigma}{2 \epsilon_0} 
\end{equation*}

**6.  Relationship between Electric Field and Potential Difference:**

The electric field between two parallel plates is related to the potential difference, $V$, and the distance between the plates, $d$, by:

\begin{equation*}
E = \frac{V}{d}
\end{equation*}

**7.  Combining Equations and Solving:**

We can now combine our expression for the electric field from Gauss's law with the potential difference equation:

\begin{equation*}
\frac{\sigma}{2 \epsilon_0}  = \frac{V}{d}
\end{equation*}

Solving for the surface charge density, $\sigma$, we get:

\begin{equation*}
\sigma = \frac{2 \epsilon_0 V}{d}
\end{equation*}

Substituting the given values ($V = 120 \, \text{V}$ and $d = 0.15 \, \text{m}$):

\begin{equation*}
\sigma = \frac{2 (8.85 \times 10^{-12} \, \text{F/m})(120 \, \text{V})}{0.15 \, \text{m}} \approx 1.42 \times 10^{-8} \, \text{C/m}^2
\end{equation*}

Finally, substituting this value of $\sigma$ back into the expression for the electric field:

\begin{equation*}
E = \frac{1.42 \times 10^{-8} \, \text{C/m}^2}{2 (8.85 \times 10^{-12} \, \text{F/m})} \approx 800 \, \text{N/C}
\end{equation*}

Therefore, the electric field between the plates is approximately $800 \, \text{N/C}$.


**b) Determine the acceleration of the object. [2 marks]**

**1. Force on the charged object:**

The force on a charged object in an electric field is given by:

\begin{equation*}
\mathbf{F} = q\mathbf{E}
\end{equation*}

where:

* $\mathbf{F}$ is the force 
* $q$ is the charge of the object
* $\mathbf{E}$ is the electric field

Since the charge is positive, the force is in the same direction as the electric field.

**2. Newton's Second Law:**

Newton's second law relates force, mass, and acceleration:

\begin{equation*}
\mathbf{F} = m\mathbf{a}
\end{equation*}

where:

* $m$ is the mass
* $\mathbf{a}$ is the acceleration

**3. Calculating the acceleration:**

Combining the equations for force and Newton's second law, we get:

\begin{equation*}
q\mathbf{E} = m\mathbf{a}
\end{equation*}

Solving for the acceleration:

\begin{equation*}
\mathbf{a} = \frac{q\mathbf{E}}{m}
\end{equation*}

Substituting the given values ($q = +0.001 \, \text{C}$, $m = 0.023 \, \text{kg}$, and $E = 800 \, \text{N/C}$):

\begin{equation*}
a = \frac{(0.001 \, \text{C})(800 \, \text{N/C})}{0.023 \, \text{kg}} \approx 34.8 \, \text{m/s}^2
\end{equation*}

Therefore, the acceleration of the object is approximately $34.8 \, \text{m/s}^2$.


\subsection{Question 7}
An RLC circuit has a resistor with resistance $R = 600\,\Omega$, capacitor with capacitance $C = 1500\,\mathrm{pF}$, inductor with inductance $L = 20\,\mathrm{mH}$.  

\begin{enumerate}
    \item[(a)] What is the resonant frequency, $\omega_0$, of the circuit?  

    \item[(b)] The circuit is driven with an e.m.f source of the form  
    \[
    \mathcal{E} = \mathcal{E}_0 \cos(\omega t)
    \]  
    Find an expression for the voltage drop across the inductor, $\Delta V_L$, as a function of $\mathcal{E}_0$, the impedance $Z$, inductive reactance $X_L$, angular frequency $\omega$, and some phase shift $\phi$.  

    \item[(c)] Describe the phase relationship between the source e.m.f and voltage across the inductor in the limits of low frequency ($\omega \ll \omega_0$) and high frequency ($\omega \gg \omega_0$).
\end{enumerate}


\paragraph{Prompt: \\} 
Answer the given question as a top physics student, giving step-by-step reasoning. Format your answer as LaTeX code without using sub-sections. Available marks: a) [1 mark], b) [3 marks], c [2 marks].

\subsubsection{Solution 1}
\textbf{Answer:}

\begin{enumerate}
\item[(a)] The resonant angular frequency, $\omega_0$, of an RLC circuit is given by:

\begin{align*}
\omega_0 &= \frac{1}{\sqrt{LC}} \quad [1~\mathrm{mark}]\\
&= \frac{1}{\sqrt{(20 \times 10^{-3}~\mathrm{H})(1500 \times 10^{-12}~\mathrm{F})}} \\
&= 5.77 \times 10^4~\mathrm{rad/s} 
\end{align*}

\item[(b)] The voltage drop across the inductor, $\Delta V_L$, is given by:

\begin{align*}
\Delta V_L &= I X_L  \\
&= \frac{\mathcal{E}_0}{Z} X_L \\
&= \frac{\mathcal{E}_0}{\sqrt{R^2 + (X_L - X_C)^2}} X_L \\
&= \frac{\mathcal{E}_0}{\sqrt{R^2 + (\omega L - 1/(\omega C))^2}} \omega L \\
&= \mathcal{E}_0 \frac{\omega L}{\sqrt{R^2 + (\omega L - 1/(\omega C))^2}} \cos(\omega t + \phi) \quad [3~\mathrm{marks}]
\end{align*}

where $\phi$ is the phase shift between the source e.m.f. and the current in the circuit.

\item[(c)] 
\begin{itemize}
\item In the low frequency limit ($\omega \ll \omega_0$), the capacitive reactance $X_C = 1/(\omega C)$ is much greater than the inductive reactance $X_L = \omega L$.  The circuit is dominated by the capacitor, and the voltage across the inductor lags the source e.m.f. by $\pi/2$. [1 mark]
\item In the high frequency limit ($\omega \gg \omega_0$), the inductive reactance $X_L = \omega L$ is much greater than the capacitive reactance $X_C = 1/(\omega C)$. The circuit is dominated by the inductor, and the voltage across the inductor leads the source e.m.f. by $\pi/2$. [1 mark]
\end{itemize}
\end{enumerate}


\subsubsection{Solution 2}
\textbf{Answer:}

\begin{enumerate}
\item[(a)] The resonant angular frequency, $\omega_0$, of an RLC circuit is given by:

\begin{align*}
\omega_0 &= \frac{1}{\sqrt{LC}} \quad [1~\mathrm{mark}]\\
&= \frac{1}{\sqrt{(20 \times 10^{-3}~\mathrm{H})(1500 \times 10^{-12}~\mathrm{F})}} \\
&= 5.77 \times 10^4~\mathrm{rad/s}
\end{align*}

\item[(b)] The voltage drop across the inductor, $\Delta V_L$, is given by:

\begin{align*}
\Delta V_L &= I X_L  \\
&= \frac{\mathcal{E}_0}{Z} X_L \\
&= \frac{\mathcal{E}_0}{\sqrt{R^2 + (X_L - X_C)^2}} X_L \\
&= \frac{\mathcal{E}_0}{\sqrt{R^2 + (\omega L - 1/(\omega C))^2}} \omega L \\
&= \mathcal{E}_0 \frac{\omega L}{\sqrt{R^2 + (\omega L - 1/(\omega C))^2}} \cos(\omega t + \phi) \quad [3~\mathrm{marks}]
\end{align*}

where $\phi = \arctan \left( \frac{X_L - X_C}{R} \right)$ is the phase shift between the source e.m.f. and the current.

\item[(c)] In the limit of low frequency ($\omega \ll \omega_0$):

\begin{align*}
\Delta V_L &\approx \mathcal{E}_0 \frac{\omega L}{1/(\omega C)} \cos(\omega t + \phi) \\
&= \mathcal{E}_0 \omega^2 LC \cos(\omega t + \phi) \\
&\approx 0
\end{align*}

The voltage across the inductor is very small and nearly in phase with the source e.m.f. ($\phi \approx 0$).

In the limit of high frequency ($\omega \gg \omega_0$):

\begin{align*}
\Delta V_L &\approx \mathcal{E}_0 \frac{\omega L}{\omega L} \cos(\omega t + \phi) \\
&= \mathcal{E}_0 \cos(\omega t + \phi)
\end{align*}

The voltage across the inductor is approximately equal to the source e.m.f. and the phase shift approaches $\phi \approx \pi/2$, meaning the voltage across the inductor leads the source e.m.f. by $\pi/2$. \quad [2~\mathrm{marks}]
\end{enumerate}

\subsubsection{Solution 3}
\textbf{Answer:}

\begin{enumerate}
\item[(a)] The resonant angular frequency, $\omega_0$, of an RLC circuit is given by:

\begin{align*}
\omega_0 &= \frac{1}{\sqrt{LC}} \quad [1~\mathrm{mark}]\\
&= \frac{1}{\sqrt{(20 \times 10^{-3}~\mathrm{H})(1500 \times 10^{-12}~\mathrm{F})}} \\
&= 5.77 \times 10^4~\mathrm{rad/s}
\end{align*}

\item[(b)] The voltage drop across the inductor, $\Delta V_L$, is given by:

\begin{align*}
\Delta V_L &= I X_L \quad [\text{where } I \text{ is the current in the circuit}] \\
&= \frac{\mathcal{E}_0}{Z} X_L \\
&= \frac{\mathcal{E}_0}{\sqrt{R^2 + (X_L - X_C)^2}} X_L \quad [\text{where } X_C = \frac{1}{\omega C} \text{ is the capacitive reactance}] \\
&= \frac{\mathcal{E}_0}{\sqrt{R^2 + (\omega L - \frac{1}{\omega C})^2}} \omega L \\
&= \mathcal{E}_0 \frac{\omega L}{\sqrt{R^2 + (\omega L - \frac{1}{\omega C})^2}} \cos(\omega t + \phi) \quad [3~\mathrm{marks}]
\end{align*}

where $\phi$ is the phase shift between the source e.m.f. and the current in the circuit.

\item[(c)] In the limit of low frequency ($\omega \ll \omega_0$):

\begin{align*}
\Delta V_L &\approx \mathcal{E}_0 \frac{\omega L}{R} \cos(\omega t + \phi) \\
&\approx \mathcal{E}_0 \frac{\omega L}{R} \cos(\omega t) \quad [\text{since } \phi \approx 0 \text{ for } \omega \ll \omega_0]
\end{align*}

Thus, the voltage across the inductor is in phase with the source e.m.f.

In the limit of high frequency ($\omega \gg \omega_0$):

\begin{align*}
\Delta V_L &\approx \mathcal{E}_0 \frac{\omega L}{\omega L} \cos(\omega t + \phi) \\
&\approx \mathcal{E}_0 \cos(\omega t + \frac{\pi}{2}) \quad [\text{since } \phi \approx \frac{\pi}{2} \text{ for } \omega \gg \omega_0]
\end{align*}

Thus, the voltage across the inductor leads the source e.m.f. by $\frac{\pi}{2}$. \quad [2 marks]
\end{enumerate}


\subsection{Question 8}
 Two linear magnetic media occupy the half-spaces above and below the $xy$ plane. The space $z > 0$ is occupied by Material 1, with relative permeability $\mu_{r1} = 1.493$. The magnetic field $\mathbf{B}_1$ in this region is spatially uniform and static, with positive $x$ and $z$ components, and zero $y$ component. It is directed at an angle $\alpha_1$ with respect to the positive $z$ direction. The space $z < 0$ is occupied by Material 2, with relative permeability $\mu_{r2} = 3.012$. The magnetic field $\mathbf{B}_2$ in this region is spatially uniform and static, with positive $x$ and $z$ components, and zero $y$ component. It is directed at an angle of $\alpha_2 = 45^\circ$ with respect to the positive $z$ direction. There is no free current flowing anywhere in the system. Using the information above, calculate the value of the angle $\alpha_1$.

\paragraph{Prompt: \\} 
Answer the given question as a top physics student, giving step-by-step reasoning. Format your answer as LaTeX code without using sub-sections. Available marks: [6 marks]

\subsubsection{Solution 1}
To find the angle $\alpha_1$, we will use the boundary conditions for magnetic fields at the interface between two magnetic media. Since there is no free current, we will focus on the continuity of the normal component of the magnetic flux density $\mathbf{B}$ and the tangential component of the magnetic field intensity $\mathbf{H}$.

**1. Define the magnetic fields in each region:**

Given that the magnetic fields have only $x$ and $z$ components, we can write them as:

\begin{align*}
\mathbf{B}_1 &= B_{1x} \hat{x} + B_{1z} \hat{z} \\
\mathbf{B}_2 &= B_{2x} \hat{x} + B_{2z} \hat{z}
\end{align*}

We can also express the components in terms of the angles and magnitudes:

\begin{align*}
B_{1x} &= B_1 \sin \alpha_1, & B_{1z} &= B_1 \cos \alpha_1 \\
B_{2x} &= B_2 \sin \alpha_2, & B_{2z} &= B_2 \cos \alpha_2
\end{align*}

**2. Apply the boundary condition for the normal component of $\mathbf{B}$:**

The normal component of $\mathbf{B}$ must be continuous across the interface (the $xy$ plane). This means:

\begin{equation*}
B_{1z} = B_{2z}
\end{equation*}

Substituting the expressions from step 1:

\begin{equation*}
B_1 \cos \alpha_1 = B_2 \cos \alpha_2
\end{equation*}

**3. Apply the boundary condition for the tangential component of $\mathbf{H}$:**

The tangential component of $\mathbf{H}$ must be continuous across the interface. Since $\mathbf{H} = \frac{\mathbf{B}}{\mu}$, this gives:

\begin{equation*}
\frac{B_{1x}}{\mu_{r1} \mu_0} = \frac{B_{2x}}{\mu_{r2} \mu_0}
\end{equation*}

Simplifying and substituting the expressions from step 1:

\begin{equation*}
\frac{B_1 \sin \alpha_1}{\mu_{r1}} = \frac{B_2 \sin \alpha_2}{\mu_{r2}} 
\end{equation*}

**4. Solve for $\alpha_1$:**

Divide the equation from step 3 by the equation from step 2:

\begin{equation*}
\frac{\tan \alpha_1}{\mu_{r1}} = \frac{\tan \alpha_2}{\mu_{r2}}
\end{equation*}

Rearrange to solve for $\alpha_1$:

\begin{equation*}
\alpha_1 = \arctan \left( \frac{\mu_{r1} \tan \alpha_2}{\mu_{r2}} \right)
\end{equation*}

**5. Substitute the given values and calculate:**

Substitute the given values of $\mu_{r1} = 1.493$, $\mu_{r2} = 3.012$, and $\alpha_2 = 45^\circ$:

\begin{align*}
\alpha_1 &= \arctan \left( \frac{1.493 \cdot \tan 45^\circ}{3.012} \right) \\
&= \arctan \left( \frac{1.493}{3.012} \right) \\
&\approx 26.4^\circ
\end{align*}

**Therefore, the angle $\alpha_1$ is approximately $26.4^\circ$.**

\subsubsection{Solution 2}
This problem involves applying boundary conditions for magnetic fields at the interface between two magnetic media. Here's a step-by-step solution:

**1. Define the magnetic fields in each medium:**

Since the magnetic fields are uniform and static with zero $y$ components, we can write them as:

\begin{align*}
\mathbf{B}_1 &= B_{1x} \hat{x} + B_{1z} \hat{z} \\
\mathbf{B}_2 &= B_{2x} \hat{x} + B_{2z} \hat{z}
\end{align*}

where $B_{1x}$, $B_{1z}$, $B_{2x}$, and $B_{2z}$ are the $x$ and $z$ components of the magnetic fields in each medium.

**2. Apply the boundary conditions:**

At the interface between the two media ($z=0$), the following boundary conditions apply:

* **Tangential component of $\mathbf{H}$ is continuous:**
   $H_{1x} = H_{2x}$ 
* **Normal component of $\mathbf{B}$ is continuous:**
   $B_{1z} = B_{2z}$

**3. Relate $\mathbf{B}$ and $\mathbf{H}$:**

The magnetic field intensity $\mathbf{H}$ is related to the magnetic flux density $\mathbf{B}$ by:

\begin{align*}
\mathbf{B}_1 &= \mu_0 \mu_{r1} \mathbf{H}_1 \\
\mathbf{B}_2 &= \mu_0 \mu_{r2} \mathbf{H}_2
\end{align*}

where $\mu_0$ is the permeability of free space.

**4. Express the boundary conditions in terms of $\mathbf{B}$:**

Using the relations between $\mathbf{B}$ and $\mathbf{H}$, the boundary conditions become:

* $\frac{B_{1x}}{\mu_{r1}} = \frac{B_{2x}}{\mu_{r2}}$
* $B_{1z} = B_{2z}$

**5. Use the angles and trigonometry:**

We are given that $\alpha_2 = 45^\circ$.  From the geometry, we can write:

\begin{align*}
B_{2x} &= B_2 \sin \alpha_2 = B_2 \sin 45^\circ = \frac{B_2}{\sqrt{2}}\\
B_{2z} &= B_2 \cos \alpha_2 = B_2 \cos 45^\circ = \frac{B_2}{\sqrt{2}}
\end{align*}

Similarly, for $\mathbf{B}_1$:

\begin{align*}
B_{1x} &= B_1 \sin \alpha_1 \\
B_{1z} &= B_1 \cos \alpha_1 
\end{align*}

**6. Solve for $\alpha_1$:**

Substituting the expressions for the components of $\mathbf{B}_1$ and $\mathbf{B}_2$ into the boundary conditions, we get:

* $\frac{B_1 \sin \alpha_1}{\mu_{r1}} = \frac{B_2}{\sqrt{2} \mu_{r2}}$
* $B_1 \cos \alpha_1  = \frac{B_2}{\sqrt{2}}$

Dividing the first equation by the second equation, we eliminate $B_1$ and $B_2$:

\begin{align*}
\frac{\tan \alpha_1}{\mu_{r1}} &= \frac{1}{\mu_{r2}} \\
\tan \alpha_1 &= \frac{\mu_{r1}}{\mu_{r2}} = \frac{1.493}{3.012} \\
\alpha_1 &= \arctan \left( \frac{1.493}{3.012} \right) \approx 26.4^\circ
\end{align*}

**Therefore, the value of the angle $\alpha_1$ is approximately $26.4^\circ$.**

\subsubsection{Solution 3}
To find the angle $\alpha_1$, we will use the boundary conditions for magnetic fields at the interface between two linear magnetic media. Since there is no free current, we will focus on the continuity of the normal component of the magnetic flux density $\mathbf{B}$ and the tangential component of the magnetic field intensity $\mathbf{H}$.

**1. Define the magnetic fields in each medium:**

* **Material 1 ($z>0$):** 
   $\mathbf{B}_1 = B_{1x} \hat{x} + B_{1z} \hat{z}$ 
   Since $\mathbf{B}_1$ makes an angle $\alpha_1$ with the positive $z$ direction, we have:
   $B_{1x} = B_1 \sin \alpha_1$ and $B_{1z} = B_1 \cos \alpha_1$.

* **Material 2 ($z<0$):** 
   $\mathbf{B}_2 = B_{2x} \hat{x} + B_{2z} \hat{z}$
   Similarly, with $\alpha_2 = 45^\circ$:
   $B_{2x} = B_2 \sin 45^\circ = \frac{B_2}{\sqrt{2}}$ and $B_{2z} = B_2 \cos 45^\circ = \frac{B_2}{\sqrt{2}}$.

**2. Apply the boundary condition for the normal component of $\mathbf{B}$:**

The normal component of $\mathbf{B}$ is continuous across the boundary:
$B_{1z} = B_{2z}$
$B_1 \cos \alpha_1 = \frac{B_2}{\sqrt{2}}$  

**3. Apply the boundary condition for the tangential component of $\mathbf{H}$:**

The tangential component of $\mathbf{H}$ is continuous across the boundary. Recall that $\mathbf{B} = \mu_0 \mu_r \mathbf{H}$. Therefore:
$\frac{B_{1x}}{\mu_0 \mu_{r1}} = \frac{B_{2x}}{\mu_0 \mu_{r2}}$
$\frac{B_1 \sin \alpha_1}{\mu_{r1}} = \frac{B_2 \sin 45^\circ}{\mu_{r2}}$
$\frac{B_1 \sin \alpha_1}{\mu_{r1}} = \frac{B_2}{\sqrt{2} \mu_{r2}}$

**4. Solve for $\alpha_1$:**

Divide the equation from step 3 by the equation from step 2:

$\frac{\frac{B_1 \sin \alpha_1}{\mu_{r1}}}{B_1 \cos \alpha_1} = \frac{\frac{B_2}{\sqrt{2} \mu_{r2}}}{\frac{B_2}{\sqrt{2}}}$

This simplifies to:
$\frac{\tan \alpha_1}{\mu_{r1}} = \frac{1}{\mu_{r2}}$
$\tan \alpha_1 = \frac{\mu_{r1}}{\mu_{r2}}$

Substitute the given values for $\mu_{r1}$ and $\mu_{r2}$:
$\tan \alpha_1 = \frac{1.493}{3.012}$

Finally, calculate $\alpha_1$:
$\alpha_1 = \arctan \left( \frac{1.493}{3.012} \right) \approx 26.4^\circ$

Therefore, the angle $\alpha_1$ is approximately $26.4^\circ$.


\subsection{Question 9}
A transverse electromagnetic wave propagating in vacuum has an electric field which has complex representation:
\[
E(r, t) = E_0 \exp(i (k \cdot r - \omega t))
\]
With real valued vectors \(E_0, k\).

\begin{enumerate}
    \item[(a)] Apply Faraday’s Law and show magnetic intensity of the wave has magnitude given by
    \[
    H(r, t) = \sqrt{\frac{\epsilon_0}{\mu_0}} E(r, t)
    \]

    \item[(b)] Show the Poynting flux averaged over one period of oscillation is given by
    \[
    \langle N \rangle = \frac{1}{2} \sqrt{\frac{\mu_0}{\epsilon_0}} (H_0)^2 \hat{k}
    \]
    where \(H_0\) denotes the amplitude of the magnetic intensity.

    \item[(c)] The pressure exerted by radiation has intensity \(I\) on a perfect planar reflector and has the value
    \[
    \frac{2I}{c},
    \]
    where \(c\) is the speed of light. Show that when the angle of incidence of radiation is \(\alpha\), the radiation pressure becomes
    \[
    2I \cos^2 \alpha / c.
    \]
\end{enumerate}


\paragraph{Prompt: \\} 
Answer the given question as a top physics student, giving step-by-step reasoning. Format your answer as LaTeX code without using sub-sections. Available marks: a) [7 marks], b) [6 marks], c [2 marks].

\subsubsection{Solution 1}
\textbf{(a) [7 marks]}

Faraday's Law in differential form is given by:
\begin{equation*}
\nabla \times \mathbf{E} = -\frac{\partial \mathbf{B}}{\partial t}
\end{equation*}

Substituting the given electric field $\mathbf{E}(r, t) = E_0 \exp(i (k \cdot r - \omega t))$ into Faraday's Law:
\begin{align*}
\nabla \times \mathbf{E} &= \nabla \times (E_0 \exp(i (k \cdot r - \omega t))) \\
&= i  \exp(i (k \cdot r - \omega t)) (\nabla \times E_0) + i E_0 \nabla \times (\exp(i (k \cdot r - \omega t))) \\
&= i E_0 \nabla \times (\exp(i (k \cdot r - \omega t))) \quad [\because \nabla \times E_0 = 0 \text{ as } E_0 \text{ is constant}]\\
&= i E_0  \nabla (i (k \cdot r - \omega t)) \times \exp(i (k \cdot r - \omega t)) \\
&= i E_0 (i \mathbf{k}) \times \exp(i (k \cdot r - \omega t)) \\
&= - \mathbf{k} \times E_0 \exp(i (k \cdot r - \omega t)) \\
&= - \mathbf{k} \times \mathbf{E}
\end{align*}

Therefore,
\begin{equation*}
-\frac{\partial \mathbf{B}}{\partial t} = - \mathbf{k} \times \mathbf{E}
\end{equation*}

Integrating both sides with respect to time, we get:
\begin{equation*}
\mathbf{B} = \frac{1}{\omega} (\mathbf{k} \times \mathbf{E}) 
\end{equation*}

Since $\mathbf{B} = \mu_0 \mathbf{H}$, we have:
\begin{equation*}
\mathbf{H} = \frac{1}{\mu_0 \omega} (\mathbf{k} \times \mathbf{E})
\end{equation*}

The magnitude of the magnetic intensity is:
\begin{align*}
|\mathbf{H}| &= \frac{1}{\mu_0 \omega} |\mathbf{k} \times \mathbf{E}| \\
&= \frac{1}{\mu_0 \omega} |\mathbf{k}| |\mathbf{E}| \sin(\theta) \quad [\theta \text{ is the angle between } \mathbf{k} \text{ and } \mathbf{E}]
\end{align*}

For a transverse electromagnetic wave, $\mathbf{k}$ is perpendicular to $\mathbf{E}$, so $\theta = 90^\circ$ and $\sin(\theta) = 1$. Also, the dispersion relation for an electromagnetic wave in vacuum is $\omega = c |\mathbf{k}|$, where $c = 1/\sqrt{\mu_0 \epsilon_0}$ is the speed of light.

Substituting these values, we get:
\begin{align*}
|\mathbf{H}| &= \frac{1}{\mu_0 \omega} |\mathbf{k}| |\mathbf{E}| \\
&= \frac{1}{\mu_0 c |\mathbf{k}|} |\mathbf{k}| |\mathbf{E}| \\
&= \frac{1}{\mu_0 c} |\mathbf{E}| \\
&= \sqrt{\frac{\epsilon_0}{\mu_0}} |\mathbf{E}| \\
&= \sqrt{\frac{\epsilon_0}{\mu_0}} |E(r,t)|
\end{align*}

Therefore, the magnitude of the magnetic intensity of the wave is given by:
\begin{equation*}
H(r, t) = \sqrt{\frac{\epsilon_0}{\mu_0}} E(r, t)
\end{equation*}

\textbf{(b) [6 marks]}

The Poynting vector $\mathbf{N}$ is given by:
\begin{equation*}
\mathbf{N} = \mathbf{E} \times \mathbf{H}
\end{equation*}

Substituting the expressions for $\mathbf{E}$ and $\mathbf{H}$:
\begin{align*}
\mathbf{N} &= E_0 \exp(i (k \cdot r - \omega t)) \times \frac{1}{\mu_0 \omega} (\mathbf{k} \times E_0 \exp(i (k \cdot r - \omega t))) \\
&= \frac{1}{\mu_0 \omega} E_0 \exp(i (k \cdot r - \omega t)) \times (\mathbf{k} \times E_0 \exp(i (k \cdot r - \omega t))) 
\end{align*}

Using the vector triple product identity $\mathbf{a} \times (\mathbf{b} \times \mathbf{c}) = (\mathbf{a} \cdot \mathbf{c})\mathbf{b} - (\mathbf{a} \cdot \mathbf{b})\mathbf{c}$, we get:
\begin{align*}
\mathbf{N} &= \frac{1}{\mu_0 \omega}  \left[ (E_0 \exp(i (k \cdot r - \omega t)) \cdot E_0 \exp(i (k \cdot r - \omega t))) \mathbf{k} \right. \\ 
& \quad \left. - (E_0 \exp(i (k \cdot r - \omega t)) \cdot \mathbf{k}) E_0 \exp(i (k \cdot r - \omega t)) \right]
\end{align*}

Since $\mathbf{E}$ is perpendicular to $\mathbf{k}$, the second term vanishes:
\begin{align*}
\mathbf{N} &= \frac{1}{\mu_0 \omega} (E_0 \exp(i (k \cdot r - \omega t)) \cdot E_0 \exp(i (k \cdot r - \omega t))) \mathbf{k} \\
&= \frac{1}{\mu_0 \omega} |E_0|^2 \mathbf{k} \\
&= \frac{1}{\mu_0 \omega} E_0^2 \mathbf{k} \quad [\because E_0 \text{ is real}]
\end{align*}

To find the time-averaged Poynting flux, we average over one period of oscillation, $T = 2\pi/\omega$:
\begin{align*}
\langle \mathbf{N} \rangle &= \frac{1}{T} \int_0^T \mathbf{N} dt \\
&= \frac{1}{T} \int_0^T \frac{1}{\mu_0 \omega} E_0^2 \mathbf{k} dt \\
&= \frac{1}{\mu_0 \omega} E_0^2 \mathbf{k} \frac{1}{T} \int_0^T dt \\
&= \frac{1}{\mu_0 \omega} E_0^2 \mathbf{k} \\
&= \frac{c}{\omega} E_0^2 \sqrt{\frac{\epsilon_0}{\mu_0}} \mathbf{k} \quad [\because c = 1/\sqrt{\mu_0 \epsilon_0}] \\
&= \frac{c^2}{\omega^2} E_0^2 \sqrt{\frac{\epsilon_0}{\mu_0}} \mathbf{k} \frac{\omega}{c} \\
&= \frac{1}{2} \frac{c^2}{\omega^2} E_0^2 \sqrt{\frac{\epsilon_0}{\mu_0}} \mathbf{k} \frac{2\omega}{c} \\
&= \frac{1}{2} \sqrt{\frac{\epsilon_0}{\mu_0}} \left( \sqrt{\frac{\epsilon_0}{\mu_0}} E_0 \frac{c}{\omega} \right)^2 \mathbf{k} \frac{2\omega}{c} \\
&= \frac{1}{2} \sqrt{\frac{\epsilon_0}{\mu_0}} H_0^2 \hat{k} \frac{2\omega}{c} \quad [\because H_0 = \sqrt{\frac{\epsilon_0}{\mu_0}} E_0 \frac{c}{\omega}] \\
&= \frac{1}{2} \sqrt{\frac{\mu_0}{\epsilon_0}} H_0^2 \hat{k} \quad [\because c = \omega/|\mathbf{k}|]
\end{align*}


\textbf{(c) [2 marks]}

The radiation pressure on a perfect planar reflector is given by $P = 2I/c$, where $I$ is the intensity of the radiation and $c$ is the speed of light.

When the radiation is incident at an angle $\alpha$, the normal component of the momentum transferred to the reflector is reduced by a factor of $\cos \alpha$.  Since pressure is force per unit area, and force is the rate of change of momentum, the radiation pressure is also reduced by a factor of $\cos \alpha$.

Furthermore, the area over which the force is distributed increases by a factor of $1/\cos \alpha$. This reduces the pressure by a further factor of $\cos \alpha$.

Therefore, the radiation pressure when the angle of incidence is $\alpha$ becomes:

\begin{equation*}
P(\alpha) = \frac{2I}{c} \cos \alpha \cdot \cos \alpha = 2I \cos^2 \alpha / c
\end{equation*}


\subsubsection{Solution 2}
\textbf{(a)} [7 marks]

Faraday's Law in differential form is given by:
\begin{equation*}
\nabla \times \mathbf{E} = -\frac{\partial \mathbf{B}}{\partial t}
\end{equation*}
where $\mathbf{E}$ is the electric field and $\mathbf{B}$ is the magnetic flux density.

The given electric field is:
\begin{equation*}
\mathbf{E}(\mathbf{r}, t) = \mathbf{E}_0 \exp(i (\mathbf{k} \cdot \mathbf{r} - \omega t))
\end{equation*}

Taking the curl of $\mathbf{E}$:
\begin{align*}
\nabla \times \mathbf{E} &= \nabla \times (\mathbf{E}_0 \exp(i (\mathbf{k} \cdot \mathbf{r} - \omega t))) \\
&= i \mathbf{k} \times \mathbf{E}_0 \exp(i (\mathbf{k} \cdot \mathbf{r} - \omega t)) \\
&= i \mathbf{k} \times \mathbf{E}
\end{align*}

Substituting this into Faraday's Law:
\begin{equation*}
i \mathbf{k} \times \mathbf{E} = -\frac{\partial \mathbf{B}}{\partial t}
\end{equation*}

Integrating both sides with respect to time:
\begin{align*}
\int i \mathbf{k} \times \mathbf{E} \, dt &= -\int \frac{\partial \mathbf{B}}{\partial t} \, dt \\
\frac{i}{\omega} \mathbf{k} \times \mathbf{E} &= -\mathbf{B} \\
\mathbf{B} &= -\frac{i}{\omega} \mathbf{k} \times \mathbf{E}
\end{align*}

The magnetic intensity $\mathbf{H}$ is related to the magnetic flux density $\mathbf{B}$ by:
\begin{equation*}
\mathbf{B} = \mu_0 \mathbf{H}
\end{equation*}
where $\mu_0$ is the permeability of free space.

Therefore:
\begin{align*}
\mathbf{H} &= \frac{\mathbf{B}}{\mu_0} \\
&= -\frac{i}{\mu_0 \omega} \mathbf{k} \times \mathbf{E}
\end{align*}

Taking the magnitude of both sides:
\begin{align*}
|\mathbf{H}| &= \left| -\frac{i}{\mu_0 \omega} \mathbf{k} \times \mathbf{E} \right| \\
&= \frac{1}{\mu_0 \omega} |\mathbf{k}| |\mathbf{E}| |\sin \theta|
\end{align*}
where $\theta$ is the angle between $\mathbf{k}$ and $\mathbf{E}$.

For a transverse electromagnetic wave, $\mathbf{k}$ and $\mathbf{E}$ are perpendicular, so $\sin \theta = 1$. Also, the speed of light in vacuum is given by $c = \omega/|\mathbf{k}| = 1/\sqrt{\mu_0 \epsilon_0}$, where $\epsilon_0$ is the permittivity of free space.

Therefore:
\begin{align*}
|\mathbf{H}| &= \frac{1}{\mu_0 \omega} |\mathbf{k}| |\mathbf{E}| \\
&= \frac{1}{\mu_0 c} |\mathbf{E}| \\
&= \sqrt{\frac{\epsilon_0}{\mu_0}} |\mathbf{E}|
\end{align*}

Since $\mathbf{E}(\mathbf{r}, t) = \mathbf{E}_0 \exp(i (\mathbf{k} \cdot \mathbf{r} - \omega t))$, we have $|\mathbf{E}| = |\mathbf{E}_0| = E_0$.

Therefore:
\begin{equation*}
|\mathbf{H}| = \sqrt{\frac{\epsilon_0}{\mu_0}} E_0 = \sqrt{\frac{\epsilon_0}{\mu_0}} |\mathbf{E}| = \sqrt{\frac{\epsilon_0}{\mu_0}} |\mathbf{E}(\mathbf{r}, t)|
\end{equation*}

\textbf{(b)} [6 marks]

The Poynting vector $\mathbf{N}$ is given by:
\begin{equation*}
\mathbf{N} = \mathbf{E} \times \mathbf{H}
\end{equation*}

Substituting the expressions for $\mathbf{E}$ and $\mathbf{H}$:
\begin{align*}
\mathbf{N} &= \mathbf{E} \times \mathbf{H} \\
&= \mathbf{E} \times \left( -\frac{i}{\mu_0 \omega} \mathbf{k} \times \mathbf{E} \right) \\
&= -\frac{i}{\mu_0 \omega} \mathbf{E} \times (\mathbf{k} \times \mathbf{E}) \\
&= -\frac{i}{\mu_0 \omega} [(\mathbf{E} \cdot \mathbf{E}) \mathbf{k} - (\mathbf{E} \cdot \mathbf{k}) \mathbf{E}]
\end{align*}

For a transverse electromagnetic wave, $\mathbf{E} \cdot \mathbf{k} = 0$. Therefore:
\begin{align*}
\mathbf{N} &= -\frac{i}{\mu_0 \omega} (\mathbf{E} \cdot \mathbf{E}) \mathbf{k} \\
&= -\frac{i}{\mu_0 \omega} |\mathbf{E}|^2 \mathbf{k} \\
&= -\frac{i}{\mu_0 \omega} E_0^2 \mathbf{k} \exp(2i (\mathbf{k} \cdot \mathbf{r} - \omega t))
\end{align*}

The time-averaged Poynting vector $\langle \mathbf{N} \rangle$ is given by:
\begin{align*}
\langle \mathbf{N} \rangle &= \frac{1}{T} \int_0^T \mathbf{N} \, dt \\
&= \frac{1}{T} \int_0^T -\frac{i}{\mu_0 \omega} E_0^2 \mathbf{k} \exp(2i (\mathbf{k} \cdot \mathbf{r} - \omega t)) \, dt \\
&= -\frac{i E_0^2 \mathbf{k}}{\mu_0 \omega T} \int_0^T \exp(2i (\mathbf{k} \cdot \mathbf{r} - \omega t)) \, dt
\end{align*}
where $T = 2\pi/\omega$ is the period of oscillation.

Evaluating the integral:
\begin{align*}
\int_0^T \exp(2i (\mathbf{k} \cdot \mathbf{r} - \omega t)) \, dt &= \left[ -\frac{1}{2i\omega} \exp(2i (\mathbf{k} \cdot \mathbf{r} - \omega t)) \right]_0^T \\
&= -\frac{1}{2i\omega} [\exp(2i (\mathbf{k} \cdot \mathbf{r} - 2\pi)) - \exp(2i \mathbf{k} \cdot \mathbf{r})] \\
&= 0
\end{align*}

Therefore, the imaginary part of $\mathbf{N}$ averages to zero over one period. The real part of $\mathbf{N}$ is:
\begin{align*}
\text{Re}(\mathbf{N}) &= \frac{1}{2} \left( \mathbf{N} + \mathbf{N}^* \right) \\
&= \frac{1}{2} \left( -\frac{i}{\mu_0 \omega} E_0^2 \mathbf{k} \exp(2i (\mathbf{k} \cdot \mathbf{r} - \omega t)) + \frac{i}{\mu_0 \omega} E_0^2 \mathbf{k} \exp(-2i (\mathbf{k} \cdot \mathbf{r} - \omega t)) \right) \\
&= \frac{E_0^2 \mathbf{k}}{\mu_0 \omega} \sin(2 (\mathbf{k} \cdot \mathbf{r} - \omega t))
\end{align*}

The time-averaged value of $\sin(2 (\mathbf{k} \cdot \mathbf{r} - \omega t))$ over one period is zero. Therefore:
\begin{align*}
\langle \mathbf{N} \rangle &= \langle \text{Re}(\mathbf{N}) \rangle \\
&= \frac{1}{2} \frac{E_0^2}{\mu_0 \omega} \mathbf{k} \\
&= \frac{1}{2} \sqrt{\frac{\epsilon_0}{\mu_0}} E_0^2 \hat{k} \\
&= \frac{1}{2} \sqrt{\frac{\mu_0}{\epsilon_0}} H_0^2 \hat{k}
\end{align*}
where $H_0 = \sqrt{\epsilon_0/\mu_0} E_0$ is the amplitude of the magnetic intensity and $\hat{k} = \mathbf{k}/|\mathbf{k}|$ is the unit vector in the direction of propagation.

\textbf{(c)} [2 marks]

The radiation pressure $P$ on a perfect planar reflector is given by:
\begin{equation*}
P = \frac{2I}{c}
\end{equation*}
where $I$ is the intensity of the radiation and $c$ is the speed of light.

When the radiation is incident at an angle $\alpha$ to the normal to the reflector, the component of the momentum transferred to the reflector per unit area per unit time is reduced by a factor of $\cos \alpha$. The intensity of the radiation is also reduced by a factor of $\cos \alpha$ due to the increased area over which the radiation is spread.

Therefore, the radiation pressure becomes:
\begin{align*}
P &= \frac{2 (I \cos \alpha)}{c} \cos \alpha \\
&= \frac{2I \cos^2 \alpha}{c}
\end{align*}

\subsubsection{Solution 3}
\textbf{(a) [7 marks]}

Faraday's Law in differential form is given by:
\begin{equation*}
\nabla \times \mathbf{E} = -\frac{\partial \mathbf{B}}{\partial t}
\end{equation*}
where $\mathbf{E}$ is the electric field and $\mathbf{B}$ is the magnetic flux density.

The given electric field is:
\begin{equation*}
\mathbf{E}(\mathbf{r}, t) = \mathbf{E}_0 \exp(i (\mathbf{k} \cdot \mathbf{r} - \omega t))
\end{equation*}

Taking the curl of $\mathbf{E}$:
\begin{align*}
\nabla \times \mathbf{E} &= \nabla \times (\mathbf{E}_0 \exp(i (\mathbf{k} \cdot \mathbf{r} - \omega t))) \\
&= i \mathbf{k} \times \mathbf{E}_0 \exp(i (\mathbf{k} \cdot \mathbf{r} - \omega t)) \\
&= i \mathbf{k} \times \mathbf{E}
\end{align*}

Substituting this into Faraday's Law:
\begin{equation*}
i \mathbf{k} \times \mathbf{E} = -\frac{\partial \mathbf{B}}{\partial t}
\end{equation*}

Integrating both sides with respect to time:
\begin{align*}
\int i \mathbf{k} \times \mathbf{E} \, dt &= -\int \frac{\partial \mathbf{B}}{\partial t} \, dt \\
\frac{i}{\omega} \mathbf{k} \times \mathbf{E} &= -\mathbf{B} \\
\mathbf{B} &= -\frac{i}{\omega} \mathbf{k} \times \mathbf{E}
\end{align*}

The magnetic intensity $\mathbf{H}$ is related to the magnetic flux density $\mathbf{B}$ by:
\begin{equation*}
\mathbf{B} = \mu_0 \mathbf{H}
\end{equation*}
where $\mu_0$ is the permeability of free space.

Therefore:
\begin{align*}
\mathbf{H} &= \frac{\mathbf{B}}{\mu_0} \\
&= -\frac{i}{\mu_0 \omega} \mathbf{k} \times \mathbf{E}
\end{align*}

Since $\mathbf{E}$, $\mathbf{k}$, and $\mathbf{H}$ are mutually perpendicular, the magnitude of $\mathbf{H}$ is given by:
\begin{align*}
|\mathbf{H}| &= \frac{1}{\mu_0 \omega} |\mathbf{k}| |\mathbf{E}| \\
&= \frac{k}{\mu_0 \omega} |\mathbf{E}|
\end{align*}

For an electromagnetic wave in vacuum, the speed of light $c$ is related to the angular frequency $\omega$ and the wave number $k$ by:
\begin{equation*}
c = \frac{\omega}{k} = \frac{1}{\sqrt{\mu_0 \epsilon_0}}
\end{equation*}
where $\epsilon_0$ is the permittivity of free space.

Therefore:
\begin{align*}
|\mathbf{H}| &= \frac{k}{\mu_0 \omega} |\mathbf{E}| \\
&= \frac{1}{\mu_0 c} |\mathbf{E}| \\
&= \sqrt{\frac{\epsilon_0}{\mu_0}} |\mathbf{E}|
\end{align*}

Thus, the magnitude of the magnetic intensity of the wave is given by:
\begin{equation*}
H(\mathbf{r}, t) = \sqrt{\frac{\epsilon_0}{\mu_0}} E(\mathbf{r}, t)
\end{equation*}

\textbf{(b) [6 marks]}

The Poynting vector $\mathbf{N}$ is given by:
\begin{equation*}
\mathbf{N} = \mathbf{E} \times \mathbf{H}
\end{equation*}

Substituting the expressions for $\mathbf{E}$ and $\mathbf{H}$:
\begin{align*}
\mathbf{N} &= \mathbf{E} \times \mathbf{H} \\
&= \mathbf{E} \times \left( -\frac{i}{\mu_0 \omega} \mathbf{k} \times \mathbf{E} \right) \\
&= \frac{i}{\mu_0 \omega} (\mathbf{E} \cdot \mathbf{E}) \mathbf{k} - \frac{i}{\mu_0 \omega} (\mathbf{E} \cdot \mathbf{k}) \mathbf{E}
\end{align*}

Since $\mathbf{E}$ and $\mathbf{k}$ are perpendicular for a transverse electromagnetic wave, $\mathbf{E} \cdot \mathbf{k} = 0$. Therefore:
\begin{equation*}
\mathbf{N} = \frac{i}{\mu_0 \omega} (\mathbf{E} \cdot \mathbf{E}) \mathbf{k}
\end{equation*}

The time-averaged Poynting flux $\langle \mathbf{N} \rangle$ is given by:
\begin{align*}
\langle \mathbf{N} \rangle &= \frac{1}{T} \int_0^T \mathbf{N} \, dt \\
&= \frac{1}{T} \int_0^T \frac{i}{\mu_0 \omega} (\mathbf{E} \cdot \mathbf{E}) \mathbf{k} \, dt \\
&= \frac{i}{\mu_0 \omega T} \mathbf{k} \int_0^T (\mathbf{E} \cdot \mathbf{E}) \, dt
\end{align*}
where $T = \frac{2\pi}{\omega}$ is the period of oscillation.

Substituting the expression for $\mathbf{E}$:
\begin{align*}
\langle \mathbf{N} \rangle &= \frac{i}{\mu_0 \omega T} \mathbf{k} \int_0^T (\mathbf{E}_0 \exp(i (\mathbf{k} \cdot \mathbf{r} - \omega t)) \cdot \mathbf{E}_0 \exp(i (\mathbf{k} \cdot \mathbf{r} - \omega t))) \, dt \\
&= \frac{i}{\mu_0 \omega T} \mathbf{k} \int_0^T |\mathbf{E}_0|^2 \exp(-2i \omega t) \, dt \\
&= \frac{i |\mathbf{E}_0|^2}{2 \mu_0 \omega^2} \mathbf{k} [\exp(-2i \omega t)]_0^T \\
&= \frac{i |\mathbf{E}_0|^2}{2 \mu_0 \omega^2} \mathbf{k} (\exp(-4i \pi) - 1) \\
&= \frac{|\mathbf{E}_0|^2}{2 \mu_0 \omega^2} \mathbf{k}
\end{align*}

Using the relation $c = \frac{\omega}{k} = \frac{1}{\sqrt{\mu_0 \epsilon_0}}$:
\begin{align*}
\langle \mathbf{N} \rangle &= \frac{|\mathbf{E}_0|^2}{2 \mu_0 \omega^2} \mathbf{k} \\
&= \frac{1}{2} \mu_0 c^2 |\mathbf{E}_0|^2 \frac{\mathbf{k}}{k} \\
&= \frac{1}{2} \sqrt{\frac{\mu_0}{\epsilon_0}} |\mathbf{E}_0|^2 \hat{k} \\
&= \frac{1}{2} \sqrt{\frac{\mu_0}{\epsilon_0}} (H_0)^2 \hat{k}
\end{align*}
where $H_0 = \sqrt{\frac{\epsilon_0}{\mu_0}} |\mathbf{E}_0|$ is the amplitude of the magnetic intensity and $\hat{k}$ is the unit vector in the direction of $\mathbf{k}$.

\textbf{(c) [2 marks]}

The radiation pressure $P$ on a perfect planar reflector is given by:
\begin{equation*}
P = \frac{2I}{c}
\end{equation*}
where $I$ is the intensity of the radiation and $c$ is the speed of light.

When the radiation is incident at an angle $\alpha$ to the normal of the reflector, the component of the momentum transferred to the reflector per unit area per unit time is reduced by a factor of $\cos \alpha$. The intensity of the radiation is also reduced by a factor of $\cos \alpha$ due to the increased area over which the radiation is spread.

Therefore, the radiation pressure becomes:
\begin{align*}
P &= \frac{2 (I \cos \alpha)}{c} \cos \alpha \\
&= \frac{2I \cos^2 \alpha}{c}
\end{align*}

\subsection{Question 10}
Q10. Consider 2 inertial reference frames \( S \) and \( S' \). The frames are aligned in such a way that origins of both frames coincide at time zero within both frames (\( t = t' = 0 \)). The frame \( S' \) moves with velocity \( v \) in the \( x \)-direction as seen by \( S \). The transformation of electric and magnetic fields from frame \( S \) to \( S' \) is given by 

\[
E'_x = E_x, \quad E'_y = \gamma (E_y - v B_z), \quad E'_z = \gamma (E_z + v B_y)
\]
\[
B'_x = B_x, \quad B'_y = \gamma \left( B_y + \frac{v}{c^2} E_z \right), \quad B'_z = \gamma \left( B_z - \frac{v}{c^2} E_y \right)
\]

\begin{enumerate}
    \item[(a)] Using the transformations given, show that the scalar product, \( (\mathbf{E} \cdot \mathbf{B}) \), is invariant under transformation.
    
    \item[(b)] A plane electromagnetic wave observed in the reference frame \( S \) propagates in a vacuum along the \( x \)-direction. In frame \( S \), it is represented by the Cartesian representation:
    \[
    \mathbf{E} = E_0 \hat{y} \exp \left( i (k x - \omega t) \right)
    \]
    with \( E_0 \), a real value denoting the amplitude, \( k \) is the wavevector, and \( \omega \) the angular frequency. \( (kx - \omega t) \) defines the phase of the wave in frame \( S \). Show that this phase in reference frame \( S' \) is written \( (k' x' - \omega' t') \), where primed coordinates correspond to the frame \( S' \) and 
    \[
    \omega' = \gamma (\omega - k v), \quad k' = \gamma \left( k - \frac{\omega v}{c^2} \right)
    \]

    \item[(c)] Examine the relationship between the wave frequencies \( \omega \) and \( \omega' \) in the limit \( v / c \to 0 \). What physical phenomenon does this represent? Justify your answer.
\end{enumerate}


\paragraph{Prompt: \\} 
Answer the given question as a top physics student, giving step-by-step reasoning. Format your answer as LaTeX code without using sub-sections. Available marks: a) [4 marks], b) [2 marks], c [1 mark].

\subsubsection{Solution 1}
\textbf{Q10.}

\textbf{(a) [4 marks]}

The scalar product $\mathbf{E} \cdot \mathbf{B}$ is given by:

\begin{align*}
\mathbf{E} \cdot \mathbf{B} &= E_x B_x + E_y B_y + E_z B_z \\
&= E'_x B'_x + \gamma (E_y - v B_z) \gamma \left( B_y + \frac{v}{c^2} E_z \right) + \gamma (E_z + v B_y) \gamma \left( B_z - \frac{v}{c^2} E_y \right) \\
&= E'_x B'_x + \gamma^2 \left( E_y B_y + \frac{v}{c^2} E_y E_z - v B_y B_z - \frac{v^2}{c^2} B_z E_z \right) \\
&\quad + \gamma^2 \left( E_z B_z - \frac{v}{c^2} E_y E_z + v B_y B_z - \frac{v^2}{c^2} E_y B_y \right) \\
&= E'_x B'_x + \gamma^2 \left( E_y B_y \left( 1 - \frac{v^2}{c^2} \right) + E_z B_z \left( 1 - \frac{v^2}{c^2} \right) \right) \\
&= E'_x B'_x + E_y B_y + E_z B_z \\
&= \mathbf{E'} \cdot \mathbf{B'}
\end{align*}

Therefore, the scalar product $(\mathbf{E} \cdot \mathbf{B})$ is invariant under transformation.


\textbf{(b) [2 marks]}

The phase of the wave in frame $S$ is given by $(kx - \omega t)$.  Using the Lorentz transformations:

\begin{align*}
x &= \gamma (x' + vt') \\
t &= \gamma \left( t' + \frac{vx'}{c^2} \right)
\end{align*}

we can write the phase in terms of the primed coordinates:

\begin{align*}
kx - \omega t &= k \gamma (x' + vt') - \omega \gamma \left( t' + \frac{vx'}{c^2} \right) \\
&= \gamma \left( k - \frac{\omega v}{c^2} \right) x' - \gamma (\omega - kv) t' \\
&= k'x' - \omega' t'
\end{align*}

where

\begin{align*}
\omega' &= \gamma (\omega - k v) \\
k' &= \gamma \left( k - \frac{\omega v}{c^2} \right)
\end{align*}


\textbf{(c) [1 mark]}

In the limit $v/c \to 0$, $\gamma \to 1$, and the relationship between the wave frequencies becomes:

\[\omega' = \omega - kv\]

This represents the \textbf{Doppler effect} for light. When the relative velocity between the source and observer is small compared to the speed of light, the observed frequency is shifted by an amount proportional to the relative velocity. This is analogous to the Doppler effect for sound waves, where the observed frequency of a sound wave changes depending on the relative motion of the source and observer.


\subsubsection{Solution 2}
```latex
\textbf{Question 10}

\textbf{(a)} [4 marks] 

To show that the scalar product $(\mathbf{E} \cdot \mathbf{B})$ is invariant under the given transformation, we need to calculate  $(\mathbf{E} \cdot \mathbf{B})$ in frame $S$ and $(\mathbf{E'} \cdot \mathbf{B'})$ in frame $S'$ and show they are equal.

In frame $S$:

\begin{align*}
(\mathbf{E} \cdot \mathbf{B}) &= E_x B_x + E_y B_y + E_z B_z
\end{align*}

In frame $S'$:

\begin{align*}
(\mathbf{E'} \cdot \mathbf{B'}) &= E'_x B'_x + E'_y B'_y + E'_z B'_z \\
&= E_x B_x + \gamma (E_y - v B_z) \gamma \left( B_y + \frac{v}{c^2} E_z \right) + \gamma (E_z + v B_y) \gamma \left( B_z - \frac{v}{c^2} E_y \right) \\
&= E_x B_x + \gamma^2 \left[ E_y B_y + \frac{v}{c^2} E_y E_z - v B_y B_z - \frac{v^2}{c^2} B_z E_z  \right. \\
&\qquad \qquad \qquad \left. + E_z B_z - \frac{v}{c^2} E_y E_z + v B_y B_z - \frac{v^2}{c^2} B_y E_y \right] \\
&= E_x B_x + \gamma^2 \left[ E_y B_y \left( 1 - \frac{v^2}{c^2} \right) + E_z B_z \left( 1 - \frac{v^2}{c^2} \right) \right] \\
&= E_x B_x + E_y B_y + E_z B_z \qquad \left( \text{since } \gamma^2 = \frac{1}{1 - v^2/c^2} \right) \\
&= (\mathbf{E} \cdot \mathbf{B})
\end{align*}

Therefore, $(\mathbf{E} \cdot \mathbf{B}) = (\mathbf{E'} \cdot \mathbf{B'})$, demonstrating that the scalar product is invariant under the transformation.

\textbf{(b)} [2 marks]

The phase of the wave in frame $S$ is given by $(kx - \omega t)$. To find the phase in frame $S'$, we need to express $x$ and $t$ in terms of $x'$ and $t'$ using the Lorentz transformations:

\begin{align*}
x &= \gamma (x' + vt') \\
t &= \gamma \left( t' + \frac{vx'}{c^2} \right)
\end{align*}

Substituting these into the phase expression:

\begin{align*}
kx - \omega t &= k \gamma (x' + vt') - \omega \gamma \left( t' + \frac{vx'}{c^2} \right) \\
&= \gamma \left( k - \frac{\omega v}{c^2} \right) x' - \gamma (\omega - kv) t' \\
&= k' x' - \omega' t'
\end{align*}

where 
\begin{align*}
\omega' &= \gamma (\omega - k v) \\
k' &= \gamma \left( k - \frac{\omega v}{c^2} \right)
\end{align*}

This shows that the phase of the wave in frame $S'$ can be written as $(k'x' - \omega' t')$ with the given expressions for $\omega'$ and $k'$.

\textbf{(c)} [1 mark]

In the limit $v/c \to 0$, the expressions for $\omega'$ and $k'$ become:

\begin{align*}
\omega' &= \gamma (\omega - k v) \approx \omega  \\
k' &= \gamma \left( k - \frac{\omega v}{c^2} \right) \approx k
\end{align*}

This implies that $\omega' \approx \omega$ when the relative velocity between the frames is much smaller than the speed of light. This represents the \textbf{Doppler effect} for light. Specifically, since the source (the wave) and the observer (frame $S'$) are moving relative to each other, there is a shift in the observed frequency. In this non-relativistic limit, the effect is analogous to the Doppler effect observed for sound waves. 

\subsubsection{Solution 3}
\textbf{Answer:}

\textbf{(a) [4 marks]}

The scalar product of $\mathbf{E}$ and $\mathbf{B}$ is given by:

\begin{align*}
\mathbf{E} \cdot \mathbf{B} &= E_x B_x + E_y B_y + E_z B_z \\
&= E'_x B'_x + \gamma (E_y - v B_z) \gamma \left( B_y + \frac{v}{c^2} E_z \right) + \gamma (E_z + v B_y) \gamma \left( B_z - \frac{v}{c^2} E_y \right) \\
&= E'_x B'_x + \gamma^2 \left( E_y B_y + \frac{v}{c^2} E_y E_z - v B_y B_z - \frac{v^2}{c^2} B_z E_z \right) \\
&\quad + \gamma^2 \left( E_z B_z - \frac{v}{c^2} E_y E_z + v B_y B_z - \frac{v^2}{c^2} E_y B_y \right) \\
&= E'_x B'_x + \gamma^2 \left( E_y B_y \left( 1 - \frac{v^2}{c^2} \right) + E_z B_z \left( 1 - \frac{v^2}{c^2} \right) \right) \\
&= E'_x B'_x + E_y B_y + E_z B_z \\
&= \mathbf{E'} \cdot \mathbf{B'}
\end{align*}

Therefore, the scalar product $(\mathbf{E} \cdot \mathbf{B})$ is invariant under transformation.

\textbf{(b) [2 marks]}

The phase of the wave in frame $S$ is given by:

\[
\phi = kx - \omega t
\]

Using the Lorentz transformations:

\begin{align*}
x &= \gamma (x' + vt') \\
t &= \gamma \left( t' + \frac{vx'}{c^2} \right)
\end{align*}

we can express the phase in terms of the primed coordinates:

\begin{align*}
\phi &= k \gamma (x' + vt') - \omega \gamma \left( t' + \frac{vx'}{c^2} \right) \\
&= \gamma \left( k - \frac{\omega v}{c^2} \right) x' - \gamma (\omega - kv) t' \\
&= k' x' - \omega' t'
\end{align*}

where

\[
\omega' = \gamma (\omega - k v), \quad k' = \gamma \left( k - \frac{\omega v}{c^2} \right)
\]

\textbf{(c) [1 mark]}

In the limit $v/c \to 0$, $\gamma \to 1$, and the relationship between the wave frequencies becomes:

\[
\omega' = \omega - kv
\]

This represents the \textbf{Doppler effect} for light. When the relative velocity between the source and observer is small compared to the speed of light, the observed frequency is shifted by an amount proportional to the relative velocity. This is a classic example of how relative motion affects the observation of waves, in this case, light waves.

\end{document}
