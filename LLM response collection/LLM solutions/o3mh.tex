\documentclass{article}
\usepackage{graphicx} % Required for inserting images
\usepackage{amsmath, amssymb} % For math symbols and formatting
\usepackage{amssymb}
\usepackage{geometry}
\usepackage{hyperref} % For hyperlinks in the Table of Contents
\usepackage{titlesec} % For customizing section titles
\setlength{\parindent}{0pt} % Reads better

\title{o3 mini-high response collection}
\author{Jonah Donaldson}
\date{January 2025}

\begin{document}

\maketitle

% Table of Contents
\tableofcontents
\newpage

\section{Introduction}

% Content for the introduction
This is a record of ChatGPT-4o's solutions across a number of university-level physics problems. The problems were solved with a zero-shot approach and no additional plugins. Any prompt engineering used can be seen below. Questions were given in LaTeX format.

\section{Classical Mechanics}

\subsection{Question 1}
Define what is a conservative force in the context of classical mechanics. For a conservative force $\mathbf{F}$, mathematically define the potential energy. Why is it not possible to define a potential energy for a force which is not conservative?

\paragraph{Prompt: \\} 
Answer the given question as a top physics student. Give your reasoning behind each step [4 marks].
Format your answer as LaTeX code.

\subsubsection{Solution 1}
```latex
A \textbf{conservative force} is one for which the work done in moving an object between any two points is independent of the path taken. Mathematically, this implies that for any closed loop \(C\),
\[
\oint_C \mathbf{F}\cdot d\mathbf{r} = 0.
\]
This property allows us to define a \textbf{potential energy} function \(U(\mathbf{r})\) such that for any two points \(\mathbf{r}_1\) and \(\mathbf{r}_2\),
\[
U(\mathbf{r}_2)-U(\mathbf{r}_1) = -\int_{\mathbf{r}_1}^{\mathbf{r}_2} \mathbf{F}\cdot d\mathbf{r}.
\]
Thus, the force can be expressed as the negative gradient of the potential energy:
\[
\mathbf{F} = -\nabla U.
\]

For forces that are \textbf{not conservative}, the work done depends on the path chosen, meaning that the closed loop integral does not vanish:
\[
\oint_C \mathbf{F}\cdot d\mathbf{r} \neq 0.
\]
Because of this path dependence, a unique scalar potential energy function \(U(\mathbf{r})\) cannot be defined consistently. In such cases, different paths between the same two points would yield different energy changes, violating the requirement for a well-defined potential energy.
```

\subsubsection{Solution 2}

A \textbf{conservative force} is one for which the work done in moving a particle between any two points is independent of the path taken. This is mathematically expressed by
\[
\oint \mathbf{F}\cdot d\mathbf{r} = 0,
\]
which is our first key reasoning step showing that the closed-loop integral vanishes.

Because of this property, there exists a scalar function \( U(\mathbf{r}) \), called the potential energy, such that the force can be written as the negative gradient of this function:
\[
\mathbf{F} = -\nabla U(\mathbf{r}).
\]
The potential energy is defined by the line integral
\[
U(\mathbf{r}) = -\int_{\mathbf{r}_0}^{\mathbf{r}} \mathbf{F}\cdot d\mathbf{r},
\]
where \(\mathbf{r}_0\) is a reference point (often chosen where \( U(\mathbf{r}_0) = 0 \)). This integral is path-independent due to the conservative nature of \(\mathbf{F}\).

For a non-conservative force, the work done depends on the path, i.e.,
\[
\oint \mathbf{F}\cdot d\mathbf{r} \neq 0.
\]
Thus, no unique scalar function \( U(\mathbf{r}) \) can be defined because different paths yield different values. This fundamental difference is why potential energy can only be defined for conservative forces.

\subsubsection{Solution 3}
```latex
A conservative force is one for which the work done in moving a particle between any two points is independent of the chosen path. This implies that for any closed path \( C \),
\[
\oint_C \mathbf{F}\cdot d\mathbf{r} = 0,
\]
which is equivalent to requiring that
\[
\nabla \times \mathbf{F} = \mathbf{0}.
\]
This zero-curl condition ensures the existence of a scalar potential function. For a conservative force, the potential energy \( U(\mathbf{r}) \) is defined by the relation
\[
U(\mathbf{r}) = -\int_{\mathbf{r}_0}^{\mathbf{r}} \mathbf{F}\cdot d\mathbf{r},
\]
where \(\mathbf{r}_0\) is an arbitrary reference point. The negative sign guarantees that the force points in the direction of decreasing potential energy, leading to
\[
\mathbf{F} = -\nabla U.
\]
Each step of this reasoning relies on the path independence of the work done by conservative forces, which allows the line integral to be expressed solely in terms of the end points. In contrast, for a non-conservative force the work depends on the chosen path, and the line integral around a closed loop does not vanish. Consequently, no unique, single-valued potential energy function \( U(\mathbf{r}) \) can be defined for non-conservative forces.
```

\subsection{Question 2}
The damped harmonic oscillator is governed by an equation of motion:
\[
\frac{d^2x}{dt^2} + b\frac{dx}{dt} + kx = 0
\]

State the 3 cases of damping and qualitatively describe the behavior of each case.

\paragraph{Prompt: \\} 
Answer the given question as a top physics student. Give your reasoning behind each step [6 marks].
Format your answer as LaTeX code. 

\subsubsection{Solution 1}
```latex
We begin with the equation
\[
\frac{d^2x}{dt^2} + b\frac{dx}{dt} + kx = 0.
\]
Assuming a solution of the form \(x(t) = e^{rt}\) leads to the characteristic equation
\[
r^2 + br + k = 0.
\]
The discriminant is given by \(\Delta = b^2 - 4k\); its sign determines the behavior of the system.

For \(\Delta < 0\) (i.e. \(b^2 < 4k\)), the roots are complex:
\[
r = -\frac{b}{2} \pm i\sqrt{k-\frac{b^2}{4}},
\]
and the general solution becomes
\[
x(t) = e^{-b t/2}\Bigl(A\cos\Bigl(\sqrt{k-\frac{b^2}{4}}\,t\Bigr) + B\sin\Bigl(\sqrt{k-\frac{b^2}{4}}\,t\Bigr)\Bigr).
\]
This is the \emph{underdamped} case, where the system exhibits oscillatory motion with an amplitude that decays exponentially over time.

When \(\Delta = 0\) (i.e. \(b^2 = 4k\)), there is a repeated real root:
\[
r = -\frac{b}{2}.
\]
The solution is then
\[
x(t) = (A+Bt)e^{-b t/2}.
\]
This is the \emph{critically damped} case, characterized by the fastest return to equilibrium without oscillations.

For \(\Delta > 0\) (i.e. \(b^2 > 4k\)), the roots are real and distinct:
\[
r_{1,2} = \frac{-b \pm \sqrt{b^2-4k}}{2},
\]
and the solution takes the form
\[
x(t) = A e^{r_1 t} + B e^{r_2 t}.
\]
This is the \emph{overdamped} case, in which the system returns to equilibrium without oscillating, but more slowly than in the critically damped scenario.
```

\subsubsection{Solution 2}
\[
\text{The equation of motion is: } \frac{d^2x}{dt^2} + b\frac{dx}{dt} + kx = 0.
\]

We start by assuming a solution of the form \(x(t) = e^{rt}\), which leads to the characteristic equation:
\[
r^2 + br + k = 0.
\]
The discriminant of this quadratic is:
\[
\Delta = b^2 - 4k.
\]

\textbf{Case 1: Underdamping (\(b^2 < 4k\))} \\
Here, the discriminant is negative (\(\Delta < 0\)), so the roots are complex conjugates:
\[
r = -\frac{b}{2} \pm i\omega \quad \text{with} \quad \omega = \sqrt{\frac{4k - b^2}{4}}.
\]
The general solution is:
\[
x(t) = e^{-\frac{b}{2}t}\left(A\cos(\omega t) + B\sin(\omega t)\right).
\]
This represents oscillatory motion with an amplitude that decays exponentially over time.

\textbf{Case 2: Critical damping (\(b^2 = 4k\))} \\
The discriminant is zero (\(\Delta = 0\)), giving a repeated real root:
\[
r = -\frac{b}{2}.
\]
The general solution takes the form:
\[
x(t) = (A + Bt)e^{-\frac{b}{2}t}.
\]
The system returns to equilibrium as quickly as possible without oscillating.

\textbf{Case 3: Overdamping (\(b^2 > 4k\))} \\
Here, the discriminant is positive (\(\Delta > 0\)), so the roots are two distinct real numbers:
\[
r_{1,2} = -\frac{b}{2} \pm \sqrt{\frac{b^2}{4} - k}.
\]
The solution is:
\[
x(t) = A e^{r_1 t} + B e^{r_2 t}.
\]
The system does not oscillate but returns to equilibrium more slowly than in the critically damped case due to the two different decay rates.


a\subsubsection{Solution 3}
We begin with the equation of motion
\[
\frac{d^2x}{dt^2} + b\frac{dx}{dt} + kx = 0,
\]
which leads to the characteristic equation
\[
r^2 + br + k = 0.
\]
The discriminant is
\[
\Delta = b^2 - 4k.
\]
\textbf{Case 1: Underdamped ($b^2 < 4k$)} \\
Since \(\Delta < 0\), the roots are complex:
\[
r = -\frac{b}{2} \pm i\omega,\quad \text{with } \omega = \frac{\sqrt{4k - b^2}}{2}.
\]
The general solution is
\[
x(t) = e^{-bt/2}\left(C_1\cos(\omega t) + C_2\sin(\omega t)\right),
\]
indicating that the system oscillates with frequency \(\omega\) while its amplitude decays exponentially.

\medskip

\textbf{Case 2: Critically Damped ($b^2 = 4k$)} \\
Here, \(\Delta = 0\) gives a repeated real root:
\[
r = -\frac{b}{2}.
\]
The solution is
\[
x(t) = (C_1 + C_2 t)e^{-bt/2},
\]
which represents the fastest non-oscillatory return to equilibrium, with no overshoot.

\medskip

\textbf{Case 3: Overdamped ($b^2 > 4k$)} \\
When \(\Delta > 0\), the roots are real and distinct:
\[
r_{1,2} = \frac{-b \pm \sqrt{b^2 - 4k}}{2}.
\]
The solution takes the form
\[
x(t) = C_1 e^{r_1t} + C_2 e^{r_2t}.
\]
This case describes a non-oscillatory decay back to equilibrium that is slower than in the critically damped case.


\subsection{Question 3}

Two objects of masses $m_1$ and $m_2$ are separated by a distance $d$. The object of mass $m_1$ is at position $\mathbf{r_1}$ and the object of mass $m_2$ is at position $\mathbf{r_2}$.

\begin{itemize}
    \item[(a)] Starting from the definition of the center of mass, show that its position vector can be given by:
    \[
    \mathbf{R} = (1 - k)\mathbf{r_1} + k\mathbf{r_2}
    \]
    and find the appropriate value of $k$.
    
    \item[(b)] Show that the center of mass lies on a line connecting the two masses, and its distances from the two masses are:
    \[
    \frac{dm_2}{m_1 + m_2} \text{ from } \mathbf{r_1} \quad \text{and} \quad \frac{dm_1}{m_1 + m_2} \text{ from } \mathbf{r_2}.
    \]
\end{itemize}

\paragraph{Prompt: \\} 
Answer the given question as a top physics student. Give your reasoning behind each step [6 marks].
Format your answer as LaTeX code.

\subsubsection{Solution 1}
\textbf{(a)}\quad The center of mass (CM) for two masses is defined as
\[
\underline{R} = \frac{m_1\underline{r}_1 + m_2\underline{r}_2}{m_1+m_2}.
\]
Assume that the CM can also be written as
\[
\underline{R} = (1-k)\underline{r}_1 + k\underline{r}_2.
\]
Equate the two expressions:
\[
(1-k)\underline{r}_1 + k\underline{r}_2 = \frac{m_1\underline{r}_1 + m_2\underline{r}_2}{m_1+m_2}.
\]
Comparing the coefficients of \(\underline{r}_1\) and \(\underline{r}_2\), we have
\[
1-k = \frac{m_1}{m_1+m_2} \quad \text{and} \quad k = \frac{m_2}{m_1+m_2}.
\]
Thus, the appropriate value of \(k\) is
\[
k = \frac{m_2}{m_1+m_2}.
\]

\textbf{(b)}\quad We can express the CM position as
\[
\underline{R} = \underline{r}_1 + k(\underline{r}_2-\underline{r}_1),
\]
which shows that the CM lies along the line connecting \(\underline{r}_1\) and \(\underline{r}_2\). The vector \(\underline{r}_2 - \underline{r}_1\) has magnitude \(d\), so the distance from \(\underline{r}_1\) to the CM is
\[
|\underline{R}-\underline{r}_1| = k\,d = \frac{m_2}{m_1+m_2}\,d,
\]
and the distance from \(\underline{r}_2\) to the CM is
\[
|\underline{r}_2-\underline{R}| = (1-k)d = \frac{m_1}{m_1+m_2}\,d.
\]
Thus, the center of mass is located \(d\,m_2/(m_1+m_2)\) from \(\underline{r}_1\) and \(d\,m_1/(m_1+m_2)\) from \(\underline{r}_2\).

\subsubsection{Solution 2}
% (a) Starting from the definition of the center of mass, we have:
\[
\underline{R} = \frac{m_1 \underline{r}_1 + m_2 \underline{r}_2}{m_1+m_2}.
\]
We want to express \(\underline{R}\) in the form
\[
\underline{R} = (1-k)\underline{r}_1 + k\underline{r}_2.
\]
Equate the two expressions by matching coefficients of \(\underline{r}_1\) and \(\underline{r}_2\):
\[
1-k = \frac{m_1}{m_1+m_2} \quad \text{and} \quad k = \frac{m_2}{m_1+m_2}.
\]
Thus, the appropriate value of \(k\) is 
\[
k = \frac{m_2}{m_1+m_2}.
\]

% (b) To show that the center of mass lies on the line connecting the two masses, we rewrite:
\[
\underline{R} = \underline{r}_1 + \frac{m_2}{m_1+m_2}(\underline{r}_2 - \underline{r}_1).
\]
This form shows that \(\underline{R}\) is obtained by starting at \(\underline{r}_1\) and moving a fraction \(\frac{m_2}{m_1+m_2}\) of the way toward \(\underline{r}_2\). Therefore, the distance from \(\underline{r}_1\) to the center of mass is
\[
|\underline{R} - \underline{r}_1| = \frac{m_2}{m_1+m_2} \, d,
\]
where \(d = |\underline{r}_2 - \underline{r}_1|\). Similarly, the distance from \(\underline{r}_2\) is
\[
|\underline{r}_2 - \underline{R}| = \frac{m_1}{m_1+m_2} \, d.
\]
This confirms that the center of mass lies along the line connecting the two masses, at distances \(\frac{m_2}{m_1+m_2}d\) from \(\underline{r}_1\) and \(\frac{m_1}{m_1+m_2}d\) from \(\underline{r}_2\).


\subsubsection{Solution 3}
% (a)
We start with the definition of the center of mass for two masses:
\[
\underline{R} = \frac{m_1 \underline{r}_1 + m_2 \underline{r}_2}{m_1+m_2}.
\]
We wish to express this in the form
\[
\underline{R} = (1-k)\underline{r}_1 + k\underline{r}_2.
\]
Equate the two expressions:
\[
(1-k)\underline{r}_1 + k\underline{r}_2 = \frac{m_1 \underline{r}_1 + m_2 \underline{r}_2}{m_1+m_2}.
\]
Comparing the coefficients of \(\underline{r}_1\) and \(\underline{r}_2\), we obtain:
\[
1-k = \frac{m_1}{m_1+m_2} \quad \text{and} \quad k = \frac{m_2}{m_1+m_2}.
\]
Thus, the value of \(k\) is:
\[
k = \frac{m_2}{m_1+m_2}.
\]

% (b)
We can rewrite the center of mass as:
\[
\underline{R} = \underline{r}_1 + \frac{m_2}{m_1+m_2}(\underline{r}_2-\underline{r}_1).
\]
This shows that \(\underline{R}\) is obtained by starting at \(\underline{r}_1\) and moving a fraction \(\frac{m_2}{m_1+m_2}\) of the vector pointing from \(\underline{r}_1\) to \(\underline{r}_2\). Hence, \(\underline{R}\) lies on the line joining \(\underline{r}_1\) and \(\underline{r}_2\).

The displacement from \(\underline{r}_1\) to \(\underline{R}\) is:
\[
\underline{R}-\underline{r}_1 = \frac{m_2}{m_1+m_2}(\underline{r}_2-\underline{r}_1),
\]
and its magnitude is:
\[
|\underline{R}-\underline{r}_1| = \frac{m_2}{m_1+m_2}\,d,
\]
where \(d = |\underline{r}_2-\underline{r}_1|\).

Similarly, the displacement from \(\underline{r}_2\) to \(\underline{R}\) is:
\[
\underline{r}_2-\underline{R} = \underline{r}_2 - \underline{r}_1 - \frac{m_2}{m_1+m_2}(\underline{r}_2-\underline{r}_1) = \frac{m_1}{m_1+m_2}(\underline{r}_2-\underline{r}_1),
\]
with magnitude:
\[
|\underline{r}_2-\underline{R}| = \frac{m_1}{m_1+m_2}\,d.
\]

Thus, the center of mass lies on the line connecting the two masses, at a distance \(\frac{m_2}{m_1+m_2}\,d\) from \(\underline{r}_1\) and \(\frac{m_1}{m_1+m_2}\,d\) from \(\underline{r}_2\).


\subsection{Question 4}

In 2D polar coordinates, the velocity, \( \mathbf{v} \), and acceleration, \( \mathbf{a} \), are given by
\[
\mathbf{v} = \dot{r} \, \hat{\mathbf{r}} + r \dot{\theta} \, \hat{\boldsymbol{\theta}}
\]
\[
\mathbf{a} = \left( \ddot{r} - r \dot{\theta}^2 \right) \hat{\mathbf{r}} + \left( r \ddot{\theta} + 2 \dot{r} \dot{\theta} \right) \hat{\boldsymbol{\theta}}
\]

A particle of mass \( m \) rotates with angular frequency \( \dot{\theta} \), dependent on time, and has radial velocity given by \( \dot{r} = -\gamma \), where \( \gamma \) is a constant. At time \( t = 0 \), the particle has radial position \( r_0 \) and \( \dot{\theta} = \delta \).

\begin{itemize}
    \item[(a)] Assuming angular momentum is conserved, derive an expression for the angular frequency \( \dot{\theta} \). Write your answer in terms of \( r_0 \), \( \gamma \), and \( \delta \).
    \item[(b)] What is the angular component of the acceleration?
    \item[(c)] Derive an expression for the kinetic energy of the particle and show that
    \[
    \frac{dK}{dt} = \frac{m r_0^4 \gamma \delta^2}{(r_0 - \gamma t)^3}
    \]
\end{itemize}

\paragraph{Prompt: \\} 
Answer the given question as a top physics student. Give your reasoning behind each step. a) [6 marks], b) [5 marks], c) [7 marks].
Format your answer as LaTeX code.

\subsubsection{Solution 1}
\textbf{(a)} Since the angular momentum 
\[
L = m r^2 \dot{\theta}
\]
is conserved, it must equal its initial value:
\[
m\, r^2 \dot{\theta} = m\, r_0^2 \delta.
\]
Thus, the angular frequency is
\[
\dot{\theta} = \frac{r_0^2 \delta}{r^2}.
\]
Because the radial velocity is given by \(\dot{r}=-\gamma\), the radial position evolves as
\[
r(t) = r_0 - \gamma t.
\]
Substituting this into the expression for \(\dot{\theta}\), we obtain
\[
\boxed{\dot{\theta} = \frac{r_0^2 \delta}{\left(r_0 - \gamma t\right)^2}}.
\]

\medskip

\textbf{(b)} The \(\theta\)-component of the acceleration is given by
\[
a_\theta = r\,\ddot{\theta} + 2\dot{r}\dot{\theta}.
\]
First, compute \(\ddot{\theta}\) from
\[
\dot{\theta} = \frac{r_0^2 \delta}{(r_0-\gamma t)^2}.
\]
Differentiating with respect to \(t\),
\[
\ddot{\theta} = \frac{d}{dt}\left[\frac{r_0^2 \delta}{(r_0-\gamma t)^2}\right] = r_0^2 \delta \cdot \frac{d}{dt}\left[(r_0-\gamma t)^{-2}\right].
\]
Using the chain rule,
\[
\frac{d}{dt}(r_0-\gamma t)^{-2} = 2\gamma\,(r_0-\gamma t)^{-3},
\]
so that
\[
\ddot{\theta} = \frac{2\gamma r_0^2 \delta}{(r_0-\gamma t)^3}.
\]
Substitute \(\ddot{\theta}\) and \(\dot{r} = -\gamma\) into \(a_\theta\):
\[
a_\theta = (r_0-\gamma t)\left(\frac{2\gamma r_0^2 \delta}{(r_0-\gamma t)^3}\right) + 2(-\gamma)\left(\frac{r_0^2 \delta}{(r_0-\gamma t)^2}\right).
\]
Simplify the terms:
\[
a_\theta = \frac{2\gamma r_0^2 \delta}{(r_0-\gamma t)^2} - \frac{2\gamma r_0^2 \delta}{(r_0-\gamma t)^2} = 0.
\]
Thus, the angular component of the acceleration is
\[
\boxed{a_\theta = 0}.
\]

\medskip

\textbf{(c)} The kinetic energy is given by
\[
K = \frac{1}{2}m v^2,
\]
with 
\[
v^2 = \dot{r}^2 + (r\dot{\theta})^2.
\]
Since \(\dot{r} = -\gamma\), we have \(\dot{r}^2 = \gamma^2\). Also, from part (a),
\[
r\dot{\theta} = (r_0-\gamma t) \cdot \frac{r_0^2 \delta}{(r_0-\gamma t)^2} = \frac{r_0^2 \delta}{(r_0-\gamma t)}.
\]
Thus,
\[
v^2 = \gamma^2 + \frac{r_0^4 \delta^2}{(r_0-\gamma t)^2}.
\]
Therefore, the kinetic energy is
\[
K = \frac{1}{2}m\left[\gamma^2 + \frac{r_0^4 \delta^2}{(r_0-\gamma t)^2}\right].
\]
Differentiating \(K\) with respect to \(t\), note that the \(\gamma^2\) term is constant:
\[
\frac{dK}{dt} = \frac{1}{2}m \frac{d}{dt}\left(\frac{r_0^4 \delta^2}{(r_0-\gamma t)^2}\right).
\]
Differentiating the remaining term using the chain rule,
\[
\frac{d}{dt}\left((r_0-\gamma t)^{-2}\right) = 2\gamma\,(r_0-\gamma t)^{-3},
\]
so that
\[
\frac{dK}{dt} = \frac{1}{2}m\, r_0^4 \delta^2 \cdot 2\gamma\,(r_0-\gamma t)^{-3} = \frac{m r_0^4 \gamma \delta^2}{(r_0-\gamma t)^3}.
\]
Thus, we have shown that
\[
\boxed{\frac{dK}{dt} = \frac{m r_0^4 \gamma \delta^2}{(r_0-\gamma t)^3}}.
\]


\subsubsection{Solution 2}
\[
\textbf{(a)}\quad \text{Angular momentum is conserved: } L = m\,r^2\dot{\theta} = \text{constant}.
\]
At \( t=0 \), we have 
\[
L = m\,r_0^2\,\delta.
\]
Thus, at any time,
\[
r^2\dot{\theta} = r_0^2\,\delta \quad \Longrightarrow \quad \dot{\theta} = \frac{r_0^2\,\delta}{r^2}.
\]
Since the radial velocity is given by \( \dot{r} = -\gamma \), we integrate to obtain
\[
r = r_0 - \gamma t.
\]
Therefore, the angular frequency becomes
\[
\dot{\theta}(t) = \frac{r_0^2\,\delta}{(r_0 - \gamma t)^2}.
\]

\[
\textbf{(b)}\quad \text{The angular component of the acceleration is given by } a_\theta = r\ddot{\theta} + 2\dot{r}\dot{\theta}.
\]
Differentiate 
\[
\dot{\theta} = \frac{r_0^2\,\delta}{r^2}
\]
with respect to time:
\[
\ddot{\theta} = \frac{d}{dt}\left(\frac{r_0^2\,\delta}{r^2}\right) = -2\frac{r_0^2\,\delta}{r^3}\dot{r}.
\]
Substituting \(\dot{r} = -\gamma\) gives
\[
\ddot{\theta} = \frac{2r_0^2\,\delta\,\gamma}{r^3}.
\]
Now, substituting into \( a_\theta \):
\[
a_\theta = r\left(\frac{2r_0^2\,\delta\,\gamma}{r^3}\right) + 2(-\gamma)\left(\frac{r_0^2\,\delta}{r^2}\right)
= \frac{2r_0^2\,\delta\,\gamma}{r^2} - \frac{2r_0^2\,\delta\,\gamma}{r^2} = 0.
\]

\[
\textbf{(c)}\quad \text{The kinetic energy is }
K = \frac{1}{2}m\left(\dot{r}^2 + r^2\dot{\theta}^2\right).
\]
Substitute \(\dot{r} = -\gamma\) and \(\dot{\theta} = \frac{r_0^2\,\delta}{r^2}\):
\[
K = \frac{1}{2}m\left(\gamma^2 + r^2\left(\frac{r_0^2\,\delta}{r^2}\right)^2\right)
= \frac{1}{2}m\left(\gamma^2 + \frac{r_0^4\,\delta^2}{r^2}\right).
\]
Differentiate \( K \) with respect to time. Since \(\gamma^2\) is constant, only the term \(\frac{r_0^4\,\delta^2}{r^2}\) contributes:
\[
\frac{dK}{dt} = \frac{1}{2}m\,\frac{d}{dt}\left(\frac{r_0^4\,\delta^2}{r^2}\right)
= \frac{1}{2}m\,r_0^4\,\delta^2\,\frac{d}{dt}\left(r^{-2}\right).
\]
We compute
\[
\frac{d}{dt}\left(r^{-2}\right) = -2r^{-3}\dot{r} = -2r^{-3}(-\gamma) = 2\gamma\,r^{-3}.
\]
Thus,
\[
\frac{dK}{dt} = \frac{1}{2}m\,r_0^4\,\delta^2\,(2\gamma\,r^{-3})
= \frac{m\,r_0^4\,\gamma\,\delta^2}{r^3}.
\]
Since \( r = r_0 - \gamma t \), we obtain the final expression:
\[
\frac{dK}{dt} = \frac{m\,r_0^4\,\gamma\,\delta^2}{(r_0 - \gamma t)^3}.
\]


\subsubsection{Solution 3}
\textbf{(a)} \quad \text{Since the angular momentum } $ L = m r^2 \dot{\theta} $ \text{ is conserved, we have}
\[
r^2 \dot{\theta} = r_0^2 \delta.
\]
Given that the radial position is 
\[
r = r_0 - \gamma t,
\]
it follows that
\[
\dot{\theta} = \frac{r_0^2 \delta}{(r_0 - \gamma t)^2}.
\]

\textbf{(b)} \quad \text{The angular component of the acceleration is}
\[
a_{\theta} = r \ddot{\theta} + 2\dot{r}\dot{\theta}.
\]
First, differentiate 
\[
\dot{\theta} = \frac{r_0^2 \delta}{(r_0 - \gamma t)^2}
\]
with respect to time:
\[
\ddot{\theta} = \frac{d}{dt} \left(\frac{r_0^2 \delta}{(r_0 - \gamma t)^2}\right)
= \frac{2\gamma r_0^2 \delta}{(r_0 - \gamma t)^3}.
\]
Substituting \(\dot{r} = -\gamma\) and the above expression into \(a_{\theta}\), we obtain:
\[
a_{\theta} = (r_0-\gamma t) \frac{2\gamma r_0^2 \delta}{(r_0-\gamma t)^3} + 2(-\gamma)\frac{r_0^2 \delta}{(r_0-\gamma t)^2}
= \frac{2\gamma r_0^2 \delta}{(r_0-\gamma t)^2} - \frac{2\gamma r_0^2 \delta}{(r_0-\gamma t)^2} = 0.
\]

\textbf{(c)} \quad \text{The kinetic energy is given by}
\[
K = \frac{1}{2} m \left(\dot{r}^2 + r^2 \dot{\theta}^2\right).
\]
Since \(\dot{r} = -\gamma\) (so that \(\dot{r}^2 = \gamma^2\)) and using the expression for \(\dot{\theta}\) from part (a), we have:
\[
r^2 \dot{\theta}^2 = (r_0-\gamma t)^2 \left(\frac{r_0^2 \delta}{(r_0-\gamma t)^2}\right)^2
= \frac{r_0^4 \delta^2}{(r_0-\gamma t)^2}.
\]
Thus,
\[
K = \frac{1}{2} m \left(\gamma^2 + \frac{r_0^4 \delta^2}{(r_0-\gamma t)^2}\right).
\]
Since \(\gamma^2\) is constant, its time derivative is zero. Differentiating the second term with respect to time:
\[
\frac{d}{dt}\left(\frac{r_0^4 \delta^2}{(r_0-\gamma t)^2}\right)
= r_0^4 \delta^2 \cdot (-2)(r_0-\gamma t)^{-3}(-\gamma)
= \frac{2\gamma r_0^4 \delta^2}{(r_0-\gamma t)^3}.
\]
Therefore, 
\[
\frac{dK}{dt} = \frac{1}{2} m \cdot \frac{2\gamma r_0^4 \delta^2}{(r_0-\gamma t)^3}
= \frac{m \gamma r_0^4 \delta^2}{(r_0-\gamma t)^3},
\]
which is the required result.

\subsection{Question 5}

Two objects $X$ and $Y$ of the same mass, $m$, are connected by a spring and each object is connected to a fixed wall by a spring (as seen in the figure). All springs have the same spring constant, $k$. Initially, object $X$ is at rest whilst object $Y$ moves with an initial velocity $v_0$. The horizontal displacement from equilibrium of $X$ and $Y$ is defined by $x_1$ and $x_2$ respectively.

\begin{enumerate}
    \item[(a)] Assuming no other forces besides those caused by the springs, derive the equations of motion for object $X$ and object $Y$.
    \item[(b)] By defining new coordinates $y_1 = x_1 + x_2$, $y_2 = x_1 - x_2$, show that this uncouples the coordinates and $y_1$ and $y_2$ obey equations of undamped, undriven simple harmonic motion.
    \item[(c)] With the initial conditions of the system, show the solution of $y_1$ is given by
    \[
    y_1 = v_0 \sqrt{\frac{m}{k}} \sin\left(t \sqrt{\frac{k}{m}}\right).
    \]
\end{enumerate}

\paragraph{Prompt: \\} 
Answer the given question as a top physics student. Give your reasoning behind each step. a) [2 marks], b) [4 marks], c [3 marks].
Format your answer as LaTeX code without using sub-sections.

\subsubsection{Solution 1}
\[
\textbf{(a)}\quad\text{For object }X: \quad m\ddot{x}_1 = -k\,x_1 + k\,(x_2 - x_1) = -2k\,x_1 + k\,x_2,
\]
\[
\text{and for object }Y: \quad m\ddot{x}_2 = -k\,x_2 + k\,(x_1 - x_2) = k\,x_1 - 2k\,x_2.
\]
\[
\textbf{(b)}\quad\text{Define } y_1=x_1+x_2 \text{ and } y_2=x_1-x_2. \text{ Then, differentiating twice gives:}
\]
\[
\ddot{y}_1=\ddot{x}_1+\ddot{x}_2=\frac{-2k\,x_1+k\,x_2+k\,x_1-2k\,x_2}{m}=\frac{-k\,(x_1+x_2)}{m}=-\frac{k}{m}\,y_1,
\]
\[
\ddot{y}_2=\ddot{x}_1-\ddot{x}_2=\frac{-2k\,x_1+k\,x_2-(k\,x_1-2k\,x_2)}{m}=\frac{-3k\,(x_1-x_2)}{m}=-\frac{3k}{m}\,y_2.
\]
\[
\text{Thus, } y_1 \text{ and } y_2 \text{ satisfy simple harmonic oscillator equations with angular frequencies } \omega_1=\sqrt{\frac{k}{m}} \text{ and } \omega_2=\sqrt{\frac{3k}{m}}, \text{ respectively.}
\]
\[
\textbf{(c)}\quad\text{Given the initial conditions } x_1(0)=0,\; x_2(0)=0,\; \dot{x}_1(0)=0,\; \dot{x}_2(0)=v_0, \text{ we have }
\]
\[
y_1(0)=x_1(0)+x_2(0)=0, \quad \dot{y}_1(0)=\dot{x}_1(0)+\dot{x}_2(0)=v_0.
\]
\[
\text{The general solution for } y_1 \text{ is }
\]
\[
y_1(t)=A\cos\left(\sqrt{\frac{k}{m}}\,t\right)+B\sin\left(\sqrt{\frac{k}{m}}\,t\right).
\]
\[
\text{Applying } y_1(0)=0 \text{ gives } A=0, \quad \text{and } \dot{y}_1(0)=B\sqrt{\frac{k}{m}}=v_0 \Longrightarrow B=v_0\sqrt{\frac{m}{k}}.
\]
\[
\text{Thus, }
\]
\[
y_1(t)=v_0\sqrt{\frac{m}{k}}\sin\left(t\sqrt{\frac{k}{m}}\right).
\]

\subsubsection{Solution 2}
\[
\textbf{(a)}\quad \text{For mass } X:\quad m\ddot{x}_1 = -kx_1 + k(x_2-x_1)= kx_2-2kx_1,
\]
\[
\text{and for mass } Y:\quad m\ddot{x}_2 = -kx_2 + k(x_1-x_2)= kx_1-2kx_2.
\]

\[
\textbf{(b)}\quad \text{Define } y_1=x_1+x_2,\quad y_2=x_1-x_2.
\]
Differentiating twice gives
\[
\ddot{y}_1=\ddot{x}_1+\ddot{x}_2,\quad \ddot{y}_2=\ddot{x}_1-\ddot{x}_2.
\]
Adding the equations of motion:
\[
m(\ddot{x}_1+\ddot{x}_2)= (kx_2-2kx_1)+(kx_1-2kx_2) = -k(x_1+x_2) = -ky_1,
\]
so that
\[
\ddot{y}_1=-\frac{k}{m}\,y_1.
\]
Subtracting the equations:
\[
m(\ddot{x}_1-\ddot{x}_2)= (kx_2-2kx_1)-(kx_1-2kx_2) = -3k(x_1-x_2) = -3ky_2,
\]
thus,
\[
\ddot{y}_2=-\frac{3k}{m}\,y_2.
\]
Both \( y_1 \) and \( y_2 \) obey the standard form of the simple harmonic oscillator equation.

\[
\textbf{(c)}\quad \text{With initial conditions } x_1(0)=0,\; x_2(0)=0,\; \dot{x}_1(0)=0,\; \dot{x}_2(0)=v_0,
\]
we have for \( y_1=x_1+x_2 \):
\[
y_1(0)=0,\quad \dot{y}_1(0)=\dot{x}_1(0)+\dot{x}_2(0)=v_0.
\]
The general solution for \( y_1 \) is
\[
y_1(t)=A\sin\Bigl(\sqrt{\frac{k}{m}}\,t\Bigr)+B\cos\Bigl(\sqrt{\frac{k}{m}}\,t\Bigr).
\]
Applying the initial condition \( y_1(0)=0 \) gives \( B=0 \), and differentiating yields
\[
\dot{y}_1(t)=A\sqrt{\frac{k}{m}}\cos\Bigl(\sqrt{\frac{k}{m}}\,t\Bigr).
\]
Thus, \( \dot{y}_1(0)=A\sqrt{\frac{k}{m}}=v_0 \) so that
\[
A=v_0\sqrt{\frac{m}{k}}.
\]
Therefore, the solution is
\[
y_1(t)=v_0\sqrt{\frac{m}{k}}\sin\Bigl(t\sqrt{\frac{k}{m}}\Bigr).
\]


\subsubsection{Solution 3}
\[
\textbf{(a)}\quad m\ddot{x}_1 = -kx_1 - k(x_1 - x_2) = -2kx_1 + kx_2,\quad
m\ddot{x}_2 = -kx_2 + k(x_1 - x_2) = kx_1 - 2kx_2.
\]

\[
\textbf{(b)}\quad \text{Define } y_1 = x_1 + x_2,\quad y_2 = x_1 - x_2.
\]
Taking time derivatives we obtain:
\[
\ddot{y}_1 = \ddot{x}_1 + \ddot{x}_2,\quad \ddot{y}_2 = \ddot{x}_1 - \ddot{x}_2.
\]
\[
\text{Adding the two equations:}\quad m(\ddot{x}_1+\ddot{x}_2) = (-2kx_1+kx_2) + (kx_1-2kx_2) = -k(x_1+x_2) = -ky_1.
\]
Thus,
\[
m\ddot{y}_1 + ky_1 = 0 \quad \Longrightarrow \quad \ddot{y}_1 + \frac{k}{m}y_1 = 0.
\]
\[
\text{Subtracting the two equations:}\quad m(\ddot{x}_1-\ddot{x}_2) = (-2kx_1+kx_2) - (kx_1-2kx_2) = -3k(x_1-x_2) = -3ky_2.
\]
Thus,
\[
m\ddot{y}_2 + 3ky_2 = 0 \quad \Longrightarrow \quad \ddot{y}_2 + \frac{3k}{m}y_2 = 0.
\]
Both \( y_1 \) and \( y_2 \) obey the equation of an undamped, undriven simple harmonic oscillator with angular frequencies \( \sqrt{k/m} \) and \( \sqrt{3k/m} \) respectively.

\[
\textbf{(c)}\quad \text{The initial conditions are } x_1(0)=0,\; \dot{x}_1(0)=0,\; x_2(0)=0,\; \dot{x}_2(0)=v_0.
\]
Hence,
\[
y_1(0)=x_1(0)+x_2(0)=0,\quad \dot{y}_1(0)=\dot{x}_1(0)+\dot{x}_2(0)=v_0.
\]
Since \( y_1 \) satisfies
\[
\ddot{y}_1 + \frac{k}{m}y_1 = 0,
\]
its general solution is
\[
y_1(t)=A\cos\Bigl(\sqrt{\frac{k}{m}}t\Bigr)+B\sin\Bigl(\sqrt{\frac{k}{m}}t\Bigr).
\]
Applying the initial conditions:
\[
y_1(0)=A=0,\quad \dot{y}_1(0)=B\sqrt{\frac{k}{m}}=v_0 \quad \Longrightarrow \quad B=v_0\sqrt{\frac{m}{k}}.
\]
Thus, the solution for \( y_1 \) is
\[
y_1(t)=v_0\sqrt{\frac{m}{k}}\sin\Bigl(t\sqrt{\frac{k}{m}}\Bigr).
\]


\subsection{Question 6}

Consider a Go-kart which moves around a flat circular track at a radius $R$ with the track having a coefficient of static friction, $\mu_s$.

\begin{enumerate}
    \item[(a)] Derive an expression for the largest speed the Go-kart can have whilst staying on the same circular path of radius $R$.
    
    \item[(b)] Assume the track now forms an angle of $\phi$ with respect to the horizontal flat ground and that the track is now frictionless. Rederive the largest speed the Go-kart may move with that maintains its circular motion around the track at the same radius $R$.
    
    \item[(c)] Consider the same situation as in part (b), except the static coefficient is now again $\mu_s$. Derive the minimum speed, $v_{\text{min}}$, and maximum speed, $v_{\text{max}}$, showing that the following equality holds:
    \[
    v^2_{\text{max}} - v^2_{\text{min}} = \frac{2R g \mu_s}{\cos^2(\phi) - \mu_s^2 \sin^2(\phi)}.
    \]
\end{enumerate}

\paragraph{Prompt: \\} 
Answer the given question as a top physics student. Give your reasoning behind each step. a) [2 marks], b) [3 marks], c [11 marks].
Format your answer as LaTeX code without using sub-sections.

\subsubsection{Solution 1}
\[
\textbf{(a)}\quad \text{For a flat track, the only force providing the required centripetal acceleration is friction. Thus,}
\]
\[
\frac{mv^2}{R} \le \mu_s mg \quad\Longrightarrow\quad v^2 \le \mu_s g R,
\]
\[
\text{so that the maximum speed is } v_{\text{max}} = \sqrt{\mu_s g R}.
\]

\[
\textbf{(b)}\quad \text{On a banked curve of angle } \phi \text{ with no friction, the normal force } N \text{ can be resolved as:}
\]
\[
\text{Vertical:} \quad N\cos\phi = mg \quad\Longrightarrow\quad N = \frac{mg}{\cos\phi},
\]
\[
\text{Horizontal (providing centripetal force):} \quad N\sin\phi = \frac{mv^2}{R}.
\]
\[
\text{Substituting } N \text{ gives } \frac{mg}{\cos\phi}\sin\phi = \frac{mv^2}{R} \quad\Longrightarrow\quad v^2 = Rg\tan\phi,
\]
\[
\text{or } v = \sqrt{Rg\tan\phi}.
\]

\[
\textbf{(c)}\quad \text{Now consider the banked curve with friction (coefficient } \mu_s\text{). Two cases arise:}
\]

\[
\textbf{(i) For } v = v_{\text{max}}: \quad \text{The car tends to slide up the bank, so friction acts down the bank. Taking } f = \mu_s N,
\]
\[
\text{Vertical balance:} \quad N\cos\phi - \mu_s N\sin\phi = mg \quad\Longrightarrow\quad N = \frac{mg}{\cos\phi - \mu_s\sin\phi}.
\]
\[
\text{Horizontal balance (centripetal force):} \quad N\sin\phi + \mu_s N\cos\phi = \frac{mv_{\text{max}}^2}{R},
\]
\[
\text{which leads to } v_{\text{max}}^2 = \frac{Rg(\sin\phi + \mu_s\cos\phi)}{\cos\phi - \mu_s\sin\phi}.
\]

\[
\textbf{(ii) For } v = v_{\text{min}}: \quad \text{The car tends to slide down the bank, so friction acts up the bank (again } f = \mu_s N\text{, but opposite in direction).}
\]
\[
\text{Vertical balance:} \quad N\cos\phi + \mu_s N\sin\phi = mg \quad\Longrightarrow\quad N = \frac{mg}{\cos\phi + \mu_s\sin\phi},
\]
\[
\text{Horizontal balance:} \quad N\sin\phi - \mu_s N\cos\phi = \frac{mv_{\text{min}}^2}{R},
\]
\[
\text{giving } v_{\text{min}}^2 = \frac{Rg(\sin\phi - \mu_s\cos\phi)}{\cos\phi + \mu_s\sin\phi}.
\]

\[
\text{To show the required equality, subtract:}
\]
\[
v_{\text{max}}^2 - v_{\text{min}}^2 = Rg\left[\frac{\sin\phi + \mu_s\cos\phi}{\cos\phi - \mu_s\sin\phi} - \frac{\sin\phi - \mu_s\cos\phi}{\cos\phi + \mu_s\sin\phi}\right].
\]
\[
\text{The common denominator is } (\cos\phi - \mu_s\sin\phi)(\cos\phi + \mu_s\sin\phi) = \cos^2\phi - \mu_s^2\sin^2\phi.
\]
\[
\text{The numerator becomes:}
\]
\[
(\sin\phi + \mu_s\cos\phi)(\cos\phi + \mu_s\sin\phi) - (\sin\phi - \mu_s\cos\phi)(\cos\phi - \mu_s\sin\phi).
\]
\[
\text{Expanding both products yields:}
\]
\[
\sin\phi\cos\phi + \mu_s(\sin^2\phi+\cos^2\phi) + \mu_s^2\sin\phi\cos\phi - \left[\sin\phi\cos\phi - \mu_s(\sin^2\phi+\cos^2\phi) + \mu_s^2\sin\phi\cos\phi\right].
\]
\[
\text{Since } \sin^2\phi+\cos^2\phi=1, \text{ this simplifies to } 2\mu_s.
\]
\[
\text{Thus,}
\]
\[
v_{\text{max}}^2 - v_{\text{min}}^2 = \frac{2Rg\mu_s}{\cos^2\phi - \mu_s^2\sin^2\phi},
\]
\[
\text{which is the desired result.}
\]

\subsubsection{Solution 2}
\[
\textbf{(a)}\quad \text{On a flat circular track the only force providing the centripetal acceleration is static friction. Thus,}
\]
\[
\mu_s mg = \frac{mv^2}{R} \quad \Rightarrow \quad v^2 = \mu_s g R \quad \Rightarrow \quad v = \sqrt{\mu_s g R}.
\]
\[
\textbf{(b)}\quad \text{For a frictionless banked curve (angle } \phi \text{), resolve the normal force } N \text{ into vertical and horizontal components.}
\]
\[
\text{Vertical balance: } N\cos\phi = mg \quad \Rightarrow \quad N = \frac{mg}{\cos\phi}.
\]
\[
\text{Horizontal (centripetal) force: } N\sin\phi = \frac{mv^2}{R} \quad \Rightarrow \quad v^2 = Rg\tan\phi.
\]
\[
\textbf{(c)}\quad \text{When friction is present on the banked curve with coefficient } \mu_s, \text{ the frictional force can act up or down the incline.}
\]
\[
\text{For } v_{\text{max}} \text{ (at the limit where friction acts down the bank):}
\]
\[
\begin{array}{rcl}
\text{Vertical:} & N\cos\phi - \mu_s N\sin\phi &= mg, \\
\text{Horizontal:} & N\sin\phi + \mu_s N\cos\phi &= \dfrac{mv_{\text{max}}^2}{R}.
\end{array}
\]
\[
\text{Thus, } N = \frac{mg}{\cos\phi - \mu_s\sin\phi} \quad \text{and} \quad v_{\text{max}}^2 = \frac{Rg(\sin\phi+\mu_s\cos\phi)}{\cos\phi-\mu_s\sin\phi}.
\]
\[
\text{For } v_{\text{min}} \text{ (at the limit where friction acts up the bank):}
\]
\[
\begin{array}{rcl}
\text{Vertical:} & N\cos\phi + \mu_s N\sin\phi &= mg, \\
\text{Horizontal:} & N\sin\phi - \mu_s N\cos\phi &= \dfrac{mv_{\text{min}}^2}{R}.
\end{array}
\]
\[
\text{Hence, } N = \frac{mg}{\cos\phi + \mu_s\sin\phi} \quad \text{and} \quad v_{\text{min}}^2 = \frac{Rg(\sin\phi-\mu_s\cos\phi)}{\cos\phi+\mu_s\sin\phi}.
\]
\[
\text{Taking the difference, we obtain:}
\]
\[
v_{\text{max}}^2 - v_{\text{min}}^2 = Rg\left(\frac{\sin\phi+\mu_s\cos\phi}{\cos\phi-\mu_s\sin\phi} - \frac{\sin\phi-\mu_s\cos\phi}{\cos\phi+\mu_s\sin\phi}\right).
\]
\[
\text{Writing the common denominator as } (\cos\phi-\mu_s\sin\phi)(\cos\phi+\mu_s\sin\phi)=\cos^2\phi-\mu_s^2\sin^2\phi,
\]
\[
\text{the numerator becomes: }
\]
\[
(\sin\phi+\mu_s\cos\phi)(\cos\phi+\mu_s\sin\phi)-(\sin\phi-\mu_s\cos\phi)(\cos\phi-\mu_s\sin\phi)=2\mu_s.
\]
\[
\text{Thus,}
\]
\[
v_{\text{max}}^2 - v_{\text{min}}^2 = \frac{2Rg\mu_s}{\cos^2\phi-\mu_s^2\sin^2\phi},
\]
\[
\text{which is the required result.}
\]

\subsubsection{Solution 3}
\begin{enumerate}
    \item[(a)] 
    For a flat circular track the only force providing the centripetal acceleration is the static friction. At the limit,
    \[
    \mu_s mg = \frac{mv^2}{R} \quad \Longrightarrow \quad v^2 = \mu_s g R,
    \]
    hence the largest speed is 
    \[
    v_{\max} = \sqrt{\mu_s g R}.
    \]
    
    \item[(b)]
    On a frictionless banked track inclined at an angle $\phi$, the normal force $N$ acts perpendicular to the surface. Its vertical component balances the weight:
    \[
    N\cos\phi = mg \quad \Longrightarrow \quad N = \frac{mg}{\cos\phi}.
    \]
    The horizontal component of the normal force provides the necessary centripetal force:
    \[
    N\sin\phi = \frac{mv^2}{R}.
    \]
    Substituting for $N$ gives
    \[
    \frac{mg\sin\phi}{\cos\phi} = \frac{mv^2}{R} \quad \Longrightarrow \quad v^2 = Rg\tan\phi,
    \]
    so that
    \[
    v_{\max} = \sqrt{Rg\tan\phi}.
    \]
    
    \item[(c)]
    When friction is present on the banked track, it can act either up or down the incline, depending on whether the kart is moving too fast or too slow. Denote the maximum and minimum speeds as $v_{\max}$ and $v_{\min}$, respectively.
    
    \textbf{For the maximum speed} (with friction acting up the bank): The force balances in the vertical and horizontal directions are
    \[
    N\cos\phi - \mu_s N\sin\phi = mg \quad \Longrightarrow \quad N = \frac{mg}{\cos\phi - \mu_s \sin\phi},
    \]
    and
    \[
    N\sin\phi + \mu_s N\cos\phi = \frac{mv_{\max}^2}{R}.
    \]
    Substituting for $N$ yields
    \[
    v_{\max}^2 = \frac{Rg\left(\sin\phi + \mu_s\cos\phi\right)}{\cos\phi - \mu_s\sin\phi}.
    \]
    
    \textbf{For the minimum speed} (with friction acting down the bank): The balance equations become
    \[
    N\cos\phi + \mu_s N\sin\phi = mg \quad \Longrightarrow \quad N = \frac{mg}{\cos\phi + \mu_s\sin\phi},
    \]
    and
    \[
    N\sin\phi - \mu_s N\cos\phi = \frac{mv_{\min}^2}{R}.
    \]
    Hence,
    \[
    v_{\min}^2 = \frac{Rg\left(\sin\phi - \mu_s\cos\phi\right)}{\cos\phi + \mu_s\sin\phi}.
    \]
    
    To show the stated relation, we take the difference:
    \[
    v_{\max}^2 - v_{\min}^2 = Rg\left[\frac{\sin\phi+\mu_s\cos\phi}{\cos\phi-\mu_s\sin\phi} - \frac{\sin\phi-\mu_s\cos\phi}{\cos\phi+\mu_s\sin\phi}\right].
    \]
    Combining the fractions over the common denominator,
    \[
    (\cos\phi-\mu_s\sin\phi)(\cos\phi+\mu_s\sin\phi)=\cos^2\phi-\mu_s^2\sin^2\phi,
    \]
    the numerator becomes
    \[
    (\sin\phi+\mu_s\cos\phi)(\cos\phi+\mu_s\sin\phi) - (\sin\phi-\mu_s\cos\phi)(\cos\phi-\mu_s\sin\phi).
    \]
    Expanding both products:
    \[
    \begin{aligned}
    (\sin\phi+\mu_s\cos\phi)(\cos\phi+\mu_s\sin\phi) &= \sin\phi\cos\phi + \mu_s\sin^2\phi + \mu_s\cos^2\phi + \mu_s^2\sin\phi\cos\phi, \\
    (\sin\phi-\mu_s\cos\phi)(\cos\phi-\mu_s\sin\phi) &= \sin\phi\cos\phi - \mu_s\sin^2\phi - \mu_s\cos^2\phi + \mu_s^2\sin\phi\cos\phi,
    \end{aligned}
    \]
    so that their difference is
    \[
    2\mu_s\left(\sin^2\phi+\cos^2\phi\right) = 2\mu_s.
    \]
    Therefore,
    \[
    v_{\max}^2 - v_{\min}^2 = \frac{2Rg\mu_s}{\cos^2\phi-\mu_s^2\sin^2\phi},
    \]
    which is the required equality.
\end{enumerate}



\subsection{Question 7}

a) State the form of the energy-momentum 4-vector of one particle of mass \(m\) and show that its scalar product is invariant under Lorentz transformation. In the lab frame, the particle moves with velocity \(v\). \\ 

b) In its rest frame, the particle decays at rest into two identical massless particles which emerge along the positive and negative \(x\)-axis. Obtain expressions of the four-momenta of the two identical particles in the lab frame. \\ 

c) For \(m = 10 \, \text{GeV}/c^2\), \(\beta = v/c = 0.8\), and if the two massless particles are photons, find their corresponding wavelengths in the rest frame of the decaying particle and in the lab frame. \\

\paragraph{Prompt: \\} 
Answer the given question as a top physics student. Give your reasoning behind each step. a) [4 marks], b) [5 marks], c [4 marks].
Format your answer as LaTeX code without using sub-sections.

\subsubsection{Solution 1}
\[
\textbf{(a)}\quad \text{A particle of mass } m \text{ moving with velocity } v \text{ has the energy–momentum 4–vector}
\]
\[
P^\mu = \left(\frac{E}{c},\, \mathbf{p}\right) = \left(\gamma m c,\, \gamma m \mathbf{v}\right),\quad \gamma = \frac{1}{\sqrt{1-\beta^2}},\quad \beta = \frac{v}{c}.
\]
\[
\text{Its invariant scalar product is}
\]
\[
P^\mu P_\mu = \left(\frac{E}{c}\right)^2 - \mathbf{p}^2 = \gamma^2 m^2 c^2 - \gamma^2 m^2 v^2 = \gamma^2 m^2 (c^2 - v^2) = m^2 c^2,
\]
\[
\text{which remains unchanged under any Lorentz transformation.}
\]

\[
\textbf{(b)}\quad \text{In the rest frame of the decaying particle (with 4–momentum } P^\mu = (mc,\,\mathbf{0})\text{), energy–momentum conservation requires that when it decays into two identical massless particles (e.g., photons) along the } x\text{–axis, each particle carries half the energy.}
\]
\[
\text{Thus, in the rest frame, the four–momenta are}
\]
\[
k_1^\mu = \left(\frac{mc}{2},\, \frac{mc}{2},\, 0,\, 0\right),\quad k_2^\mu = \left(\frac{mc}{2},\, -\frac{mc}{2},\, 0,\, 0\right).
\]
\[
\text{To obtain their expressions in the lab frame (where the parent particle moves along the } x\text{–axis with velocity } v\text{), we perform a Lorentz boost along } x\text{. The boost transformation is}
\]
\[
k'^{0} = \gamma\left(k^0+\beta k^1\right),\quad k'^{1} = \gamma\left(k^1+\beta k^0\right),\quad k'^{2}=k^2,\quad k'^{3}=k^3.
\]
\[
\text{For the photon moving in the }+x\text{ direction:}
\]
\[
k_{1}'^\mu = \left(\gamma\frac{mc}{2}(1+\beta),\, \gamma\frac{mc}{2}(1+\beta),\, 0,\, 0\right),
\]
\[
\text{and for the photon moving in the }-x\text{ direction:}
\]
\[
k_{2}'^\mu = \left(\gamma\frac{mc}{2}(1-\beta),\, -\gamma\frac{mc}{2}(1-\beta),\, 0,\, 0\right).
\]

\[
\textbf{(c)}\quad \text{Let } m = 10\,\text{GeV}/c^2 \text{ and } \beta=0.8. \text{ Then,}
\]
\[
\gamma = \frac{1}{\sqrt{1-0.8^2}} = \frac{1}{0.6} \approx 1.667.
\]
\[
\text{In the rest frame, each photon has energy}
\]
\[
E_0 = \frac{mc^2}{2} = \frac{10\,\text{GeV}}{2} = 5\,\text{GeV}.
\]
\[
\text{The wavelength is given by } \lambda_0 = \frac{hc}{E_0}.
\]
\[
\text{Using } hc \approx 1.986\times10^{-25}\,\text{J}\cdot\text{m} \text{ and } 1\,\text{GeV} \approx 1.602\times10^{-10}\,\text{J},
\]
\[
\lambda_0 = \frac{1.986\times10^{-25}}{5\times1.602\times10^{-10}} \approx 2.48\times10^{-16}\,\text{m}.
\]
\[
\text{In the lab frame, the energies of the photons become}
\]
\[
E'_1 = \gamma\frac{mc^2}{2}(1+\beta) \approx 1.667\times5\,\text{GeV}\times1.8 \approx 15\,\text{GeV},
\]
\[
E'_2 = \gamma\frac{mc^2}{2}(1-\beta) \approx 1.667\times5\,\text{GeV}\times0.2 \approx 1.667\,\text{GeV}.
\]
\[
\text{Thus, their wavelengths in the lab frame are}
\]
\[
\lambda'_1 = \frac{hc}{E'_1} \approx \frac{1.986\times10^{-25}}{15\times1.602\times10^{-10}} \approx 8.27\times10^{-17}\,\text{m},
\]
\[
\lambda'_2 = \frac{hc}{E'_2} \approx \frac{1.986\times10^{-25}}{1.667\times1.602\times10^{-10}} \approx 7.43\times10^{-16}\,\text{m}.
\]
\[
\text{Hence, in the rest frame the photon wavelength is } \approx 2.48\times10^{-16}\,\text{m}, \text{ while in the lab frame the forward–emitted photon has } \lambda'_1\approx 8.27\times10^{-17}\,\text{m} \text{ and the backward–emitted photon has } \lambda'_2\approx 7.43\times10^{-16}\,\text{m}.
\]

\subsubsection{Solution 2}
\[
\textbf{(a)}
\]
We write the energy–momentum 4–vector of a particle of rest mass \(m\) moving with velocity \(\vec{v}\) as
\[
p^\mu = \left(\frac{E}{c},\, \vec{p}\right) = \left(\gamma m c,\, \gamma m\, \vec{v}\right),
\]
with 
\[
\gamma = \frac{1}{\sqrt{1-\beta^2}}, \quad \beta = \frac{v}{c}, \quad \text{and} \quad E = \gamma m c^2.
\]
The Minkowski scalar product is then
\[
p^\mu p_\mu = \left(\frac{E}{c}\right)^2 - \vec{p}^{\,2} 
= \gamma^2 m^2 c^2 - \gamma^2 m^2 v^2 
= \gamma^2 m^2 (c^2-v^2).
\]
Since
\[
\gamma^2 (c^2-v^2)= \frac{c^2-v^2}{1-\beta^2} = c^2,
\]
we obtain
\[
p^\mu p_\mu = m^2 c^2.
\]
This shows that the scalar product (i.e. the “mass–shell” condition) is invariant under Lorentz transformations.

\vspace{5mm}

\[
\textbf{(b)}
\]
In the rest frame of the decaying particle the total 4–momentum is
\[
P^\mu = \left(\frac{m c^2}{c},\,0,\,0,\,0\right) = \left(m c,\,0,\,0,\,0\right).
\]
The decay into two identical massless particles (photons) moving along the \(+x\) and \(-x\) axes implies, by conservation of momentum and energy, that each photon carries half the energy:
\[
E'_1 = E'_2 = \frac{m c^2}{2}.
\]
Thus, in the rest frame the 4–momenta are
\[
p'^\mu_1 = \left(\frac{E'_1}{c},\, \frac{E'_1}{c},\, 0,\, 0\right) 
= \left(\frac{m c}{2},\, \frac{m c}{2},\, 0,\, 0\right),
\]
\[
p'^\mu_2 = \left(\frac{E'_2}{c},\, -\frac{E'_2}{c},\, 0,\, 0\right) 
= \left(\frac{m c}{2},\, -\frac{m c}{2},\, 0,\, 0\right).
\]
To transform to the lab frame (in which the parent particle moves with speed \(v\) along the \(x\)–axis), we perform a Lorentz boost in the \(x\)–direction. The boost gives for any 4–vector component:
\[
E = \gamma\left(E' + \beta c\, p'_x\right), \quad
p_x = \gamma\left(p'_x + \beta \frac{E'}{c}\right).
\]
For the photon moving in the \(+x\) direction (\(p'_x = \frac{m c}{2}\), \(E'=\frac{m c^2}{2}\)) we find
\[
E_1 = \gamma\left(\frac{m c^2}{2} + \beta c\, \frac{m c}{2}\right)
= \gamma\,\frac{m c^2}{2}(1+\beta),
\]
\[
p_{x,1} = \gamma\left(\frac{m c}{2} + \beta \frac{m c}{2}\right)
= \gamma\,\frac{m c}{2}(1+\beta).
\]
Thus, in the lab frame,
\[
p^\mu_1 = \left(\frac{E_1}{c},\, p_{x,1},\,0,\,0\right)
= \left(\gamma\,\frac{m c}{2}(1+\beta),\, \gamma\,\frac{m c}{2}(1+\beta),\,0,\,0\right).
\]
For the photon moving in the \(-x\) direction (\(p'_x = -\frac{m c}{2}\)), we obtain
\[
E_2 = \gamma\left(\frac{m c^2}{2} - \beta c\, \frac{m c}{2}\right)
= \gamma\,\frac{m c^2}{2}(1-\beta),
\]
\[
p_{x,2} = \gamma\left(-\frac{m c}{2} + \beta \frac{m c}{2}\right)
= -\gamma\,\frac{m c}{2}(1-\beta).
\]
Hence, the lab frame 4–momentum for the second photon is
\[
p^\mu_2 = \left(\frac{E_2}{c},\, p_{x,2},\,0,\,0\right)
= \left(\gamma\,\frac{m c}{2}(1-\beta),\, -\gamma\,\frac{m c}{2}(1-\beta),\,0,\,0\right).
\]

\vspace{5mm}

\[
\textbf{(c)}
\]
For the numerical evaluation, we are given
\[
m = 10\,\text{GeV}/c^2,\quad \beta = 0.8.
\]
First, the Lorentz factor is
\[
\gamma = \frac{1}{\sqrt{1-0.8^2}} = \frac{1}{0.6} \approx 1.667.
\]
In the rest frame of the decaying particle, each photon has energy
\[
E'_0 = \frac{m c^2}{2} = \frac{10\,\text{GeV}}{2} = 5\,\text{GeV}.
\]
The wavelength is given by
\[
\lambda_0 = \frac{hc}{E'_0}.
\]
Using the conversion \(hc \approx 1.24\,\text{GeV}\cdot\text{fm}\),
\[
\lambda_0 \approx \frac{1.24\,\text{GeV}\cdot\text{fm}}{5\,\text{GeV}} \approx 0.248\,\text{fm}.
\]

In the lab frame, the energies of the two photons are, from part (b),
\[
E_1 = \gamma\,\frac{m c^2}{2}(1+\beta) \approx 1.667 \times 5\,\text{GeV} \times 1.8 \approx 15\,\text{GeV},
\]
\[
E_2 = \gamma\,\frac{m c^2}{2}(1-\beta) \approx 1.667 \times 5\,\text{GeV} \times 0.2 \approx 1.667\,\text{GeV}.
\]
Their wavelengths are therefore
\[
\lambda_1 = \frac{hc}{E_1} \approx \frac{1.24\,\text{GeV}\cdot\text{fm}}{15\,\text{GeV}} \approx 0.0827\,\text{fm},
\]
\[
\lambda_2 = \frac{hc}{E_2} \approx \frac{1.24\,\text{GeV}\cdot\text{fm}}{1.667\,\text{GeV}} \approx 0.744\,\text{fm}.
\]
Thus, in the rest frame each photon has a wavelength of approximately \(0.248\,\text{fm}\); in the lab frame the photon moving in the \(+x\)–direction has a wavelength of about \(0.083\,\text{fm}\) while the one moving in the \(-x\)–direction has a wavelength of about \(0.744\,\text{fm}\).


\subsubsection{Solution 3}
\[
\textbf{(a)}\quad \text{The energy-momentum four-vector for a particle of mass } m \text{ moving with velocity } v \text{ (assumed along the } x\text{-axis) is}
\]
\[
p^\mu = \left(\frac{E}{c},\, p_x,\, 0,\, 0\right) = \left(\gamma m c,\, \gamma m v,\, 0,\, 0\right),
\]
\[
\text{with } \gamma = \frac{1}{\sqrt{1-\beta^2}},\quad \beta=\frac{v}{c},\quad E=\gamma m c^2.
\]
\[
\text{Its invariant scalar product is computed as}
\]
\[
p^\mu p_\mu = \left(\frac{E}{c}\right)^2 - (\gamma m v)^2 = \gamma^2 m^2 \left(c^2 - v^2\right) = m^2 c^2.
\]
\[
\text{Since } m^2 c^2 \text{ is independent of the frame, the scalar product is invariant under Lorentz transformations.}
\]

\[
\textbf{(b)}\quad \text{In the rest frame of the decaying particle, the total four-momentum is } (mc,\,0,\,0,\,0).
\]
\[
\text{The particle decays into two identical massless particles. Conservation of energy and momentum in the rest frame requires each particle to have energy}
\]
\[
E'=\frac{m c^2}{2}\quad \text{and momentum magnitude} \quad p'=\frac{E'}{c}=\frac{m c}{2}.
\]
\[
\text{Choosing the } x\text{-axis for the decay, the four-momenta in the rest frame are}
\]
\[
k_1'^\mu = \left(\frac{m c^2}{2},\, \frac{m c}{2},\, 0,\, 0\right),\quad k_2'^\mu = \left(\frac{m c^2}{2},\, -\frac{m c}{2},\, 0,\, 0\right).
\]
\[
\text{To boost to the lab frame (where the original particle moves with velocity } v \text{ along } x\text{), we use the Lorentz transformation:}
\]
\[
E = \gamma \left(E' + \beta\, p'_x c\right),\quad p_x = \gamma \left(p'_x + \beta\,\frac{E'}{c}\right).
\]
\[
\text{For the photon moving in the } +x \text{ direction (} p'_x=\frac{m c}{2} \text{):}
\]
\[
E_1 = \gamma\,\frac{m c^2}{2}(1+\beta),\quad p_{1x} = \gamma\,\frac{m c}{2}(1+\beta).
\]
\[
\text{For the photon moving in the } -x \text{ direction (} p'_x=-\frac{m c}{2} \text{):}
\]
\[
E_2 = \gamma\,\frac{m c^2}{2}(1-\beta),\quad p_{2x} = -\gamma\,\frac{m c}{2}(1-\beta).
\]
\[
\text{Thus, the lab frame four-momenta are:}
\]
\[
k_1^\mu = \left(\gamma\,\frac{m c^2}{2}(1+\beta),\, \gamma\,\frac{m c}{2}(1+\beta),\, 0,\, 0\right),
\]
\[
k_2^\mu = \left(\gamma\,\frac{m c^2}{2}(1-\beta),\, -\gamma\,\frac{m c}{2}(1-\beta),\, 0,\, 0\right).
\]

\[
\textbf{(c)}\quad \text{For } m = 10\,\mathrm{GeV}/c^2 \text{ and } \beta=0.8,\quad \gamma=\frac{1}{\sqrt{1-0.8^2}}=\frac{1}{0.6}\approx 1.667.
\]
\[
\text{In the rest frame, each photon has energy}
\]
\[
E'=\frac{10\,\mathrm{GeV}}{2}=5\,\mathrm{GeV}.
\]
\[
\text{The photon wavelength is given by } \lambda=\frac{hc}{E}\quad \text{with } hc\approx1.24\times10^{-6}\,\mathrm{GeV\cdot m}.
\]
\[
\text{Thus, in the rest frame: }
\]
\[
\lambda'=\frac{1.24\times10^{-6}\,\mathrm{GeV\cdot m}}{5\,\mathrm{GeV}}=2.48\times10^{-7}\,\mathrm{m}\approx248\,\mathrm{nm}.
\]
\[
\text{In the lab frame, the energies transform as:}
\]
\[
E_1=\gamma\,\frac{m c^2}{2}(1+\beta)=1.667\times5\,\mathrm{GeV}\times1.8\approx15\,\mathrm{GeV},
\]
\[
E_2=\gamma\,\frac{m c^2}{2}(1-\beta)=1.667\times5\,\mathrm{GeV}\times0.2\approx1.667\,\mathrm{GeV}.
\]
\[
\text{The corresponding wavelengths are:}
\]
\[
\lambda_1=\frac{1.24\times10^{-6}\,\mathrm{GeV\cdot m}}{15\,\mathrm{GeV}}\approx8.27\times10^{-8}\,\mathrm{m}\approx83\,\mathrm{nm},
\]
\[
\lambda_2=\frac{1.24\times10^{-6}\,\mathrm{GeV\cdot m}}{1.667\,\mathrm{GeV}}\approx7.44\times10^{-7}\,\mathrm{m}\approx744\,\mathrm{nm}.
\]
\[
\text{In summary, in the rest frame the photons have wavelength } \approx248\,\mathrm{nm}, \text{ while in the lab frame they are } \approx83\,\mathrm{nm} \text{ and } \approx744\,\mathrm{nm} \text{ for the forward and backward moving photons, respectively.}
\]

\subsection{Question 8}

A pendulum under the influence of gravity is formed by a massless string of fixed length, $L$, attached to a mass $m$. The pendulum is connected to a support (as seen in the figure) which moves with a position given by
\[
x(t) = vt^3 + A\sin(\omega t)
\]
where $v$ and $A$ are constants. \\

\textbf{(a)} Starting from the expression of the Lagrangian, derive the equation of motion of the angle, $\phi$, of the pendulum.


\textbf{(b)} In the limit of $v$, $A$ approaching 0 and small angles ($\phi \to 0$), show this reduces to the classic pendulum problem and equation of motion.

\paragraph{Prompt: \\} 
Answer the given question as a top physics student. Give your reasoning behind each step. a) [9 marks], b) [1 mark].
Format your answer as LaTeX code without using sub-sections.

\subsubsection{Solution 1}
\[
\textbf{We begin by choosing coordinates for the mass. Let the support’s horizontal position be } x(t)=vt^3+A\sin(\omega t).
\]
\[
\text{Then the mass is located at } (X,Y) \text{ with } X=x(t)+L\sin\phi \quad \text{and} \quad Y=-L\cos\phi.
\]
\[
\text{Thus, the velocities are: } \dot{X}=\dot{x}(t)+L\cos\phi\,\dot{\phi}, \quad \dot{Y}=L\sin\phi\,\dot{\phi}.
\]
\[
\textbf{Kinetic Energy: } T=\frac{1}{2}m\Big[\dot{X}^2+\dot{Y}^2\Big]=\frac{1}{2}m\Big[\big(\dot{x}(t)+L\cos\phi\,\dot{\phi}\big)^2+\big(L\sin\phi\,\dot{\phi}\big)^2\Big].
\]
Expanding and using \(\cos^2\phi+\sin^2\phi=1\), we obtain
\[
T=\frac{1}{2}m\Big[\dot{x}(t)^2+2L\cos\phi\,\dot{x}(t)\,\dot{\phi}+L^2\dot{\phi}^2\Big].
\]
\[
\textbf{Potential Energy: } V=mgy,
\]
with the vertical coordinate \(y=Y=-L\cos\phi\), hence
\[
V=-mgL\cos\phi.
\]
\[
\textbf{The Lagrangian is then: } \mathcal{L}=T-V=\frac{1}{2}m\Big[\dot{x}(t)^2+2L\cos\phi\,\dot{x}(t)\,\dot{\phi}+L^2\dot{\phi}^2\Big]+mgL\cos\phi.
\]
\[
\textbf{Next, we apply the Euler–Lagrange equation for } \phi:
\]
\[
\frac{d}{dt}\left(\frac{\partial\mathcal{L}}{\partial\dot{\phi}}\right)-\frac{\partial\mathcal{L}}{\partial\phi}=0.
\]
\[
\text{First, compute } \frac{\partial\mathcal{L}}{\partial\dot{\phi}}=m\Big[L\cos\phi\,\dot{x}(t)+L^2\dot{\phi}\Big].
\]
Taking its time derivative gives
\[
\frac{d}{dt}\left(\frac{\partial\mathcal{L}}{\partial\dot{\phi}}\right)=m\Big[-L\sin\phi\,\dot{\phi}\,\dot{x}(t)+L\cos\phi\,\ddot{x}(t)+L^2\ddot{\phi}\Big].
\]
\[
\text{Next, } \frac{\partial\mathcal{L}}{\partial\phi} \text{ is obtained from the } \phi\text{-dependence in } \cos\phi \text{ and } \cos\phi\,\dot{x}(t):
\]
\[
\frac{\partial\mathcal{L}}{\partial\phi}=-mL\sin\phi\,\dot{x}(t)\,\\dot{\phi}-mgL\sin\phi.
\]
\[
\text{Substituting into the Euler–Lagrange equation, we have:}
\]
\[
m\Big[-L\sin\phi\,\dot{\phi}\,\dot{x}(t)+L\cos\phi\,\ddot{x}(t)+L^2\ddot{\phi}\Big] -\Big[-mL\sin\phi\,\dot{x}(t)\,\dot{\phi}-mgL\sin\phi\Big]=0.
\]
\[
\text{Notice that the terms } -mL\sin\phi\,\dot{\phi}\,\dot{x}(t) \text{ cancel with } +mL\sin\phi\,\dot{x}(t)\,\dot{\phi}.
\]
\[
\text{This leaves: } mL\cos\phi\,\ddot{x}(t)+mL^2\ddot{\phi}+mgL\sin\phi=0.
\]
Dividing by \(mL\) (assuming \(m,L\neq 0\)) yields
\[
L\ddot{\phi}+\cos\phi\,\ddot{x}(t)+g\sin\phi=0.
\]
\[
\text{Since } x(t)=vt^3+A\sin(\omega t), \text{ we have } \ddot{x}(t)=\frac{d^2}{dt^2}\big(vt^3+A\sin(\omega t)\big)=6vt-A\omega^2\sin(\omega t).
\]
Thus, the full equation of motion for \(\phi\) is
\[
L\ddot{\phi}+\cos\phi\,\Big(6vt-A\omega^2\sin(\omega t)\Big)+g\sin\phi=0,
\]
or equivalently,
\[
\ddot{\phi}+\frac{\cos\phi}{L}\Big(6vt-A\omega^2\sin(\omega t)\Big)+\frac{g}{L}\sin\phi=0.
\]
\[
\textbf{In the limit } v,A\to0 \text{ and for small angles (}\phi\to0\text{), we use } \cos\phi\approx1 \text{ and } \sin\phi\approx\phi.
\]
\[
\text{Then } \ddot{x}(t)\to 0 \text{ and the equation reduces to }
\]
\[
L\ddot{\phi}+g\phi=0 \quad \Longrightarrow \quad \ddot{\phi}+\frac{g}{L}\phi=0,
\]
which is the classic small-angle pendulum equation.

\subsubsection{Solution 2}
% We start by choosing coordinates for the mass. With the support at 
% \(x(t)\) (moving horizontally) and the string making an angle \(\phi\)
% with the vertical, the position of the mass is given by
\[
x_m = x(t) + L\sin\phi,\quad y_m = -L\cos\phi.
\]
% Differentiating, we obtain the velocity components:
\[
\dot{x}_m = \dot{x}(t) + L\cos\phi\,\dot{\phi},\quad \dot{y}_m = L\sin\phi\,\dot{\phi}.
\]
% Thus the kinetic energy is
\[
T = \frac{1}{2}m\Big[(\dot{x}(t)+L\cos\phi\,\dot{\phi})^2 + (L\sin\phi\,\dot{\phi})^2\Big].
\]
% Expanding the square and using \(\sin^2\phi+\cos^2\phi=1\), we have
\[
T = \frac{1}{2}m\Big[\dot{x}(t)^2 + 2L\dot{x}(t)\cos\phi\,\dot{\phi} + L^2\dot{\phi}^2\Big].
\]
% The gravitational potential energy (taking zero at the pivot) is
\[
V = -mgL\cos\phi.
\]
% Hence the Lagrangian is
\[
\mathcal{L} = T - V = \frac{1}{2}m\Big[\dot{x}(t)^2 + 2L\dot{x}(t)\cos\phi\,\dot{\phi} + L^2\dot{\phi}^2\Big] + mgL\cos\phi.
\]
% Note that the term \(\frac{1}{2}m\dot{x}(t)^2\) is independent of \(\phi\)
% and \(\dot{\phi}\) and thus does not contribute to the Euler–Lagrange equation.
% We now focus on the terms depending on \(\phi\) and \(\dot{\phi}\).

% The derivative with respect to \(\dot{\phi}\) is
\[
\frac{\partial\mathcal{L}}{\partial\dot{\phi}} = mL\dot{x}(t)\cos\phi + mL^2\dot{\phi}.
\]
% Taking its time derivative gives
\[
\frac{d}{dt}\frac{\partial\mathcal{L}}{\partial\dot{\phi}} = mL\ddot{x}(t)\cos\phi - mL\dot{x}(t)\sin\phi\,\dot{\phi} + mL^2\ddot{\phi}.
\]
% Next, we compute the derivative with respect to \(\phi\). The \(\phi\)-dependent
% terms in \(\mathcal{L}\) yield
\[
\frac{\partial\mathcal{L}}{\partial\phi} = -mL\dot{x}(t)\sin\phi\,\dot{\phi} - mgL\sin\phi.
\]
% The Euler–Lagrange equation,
\[
\frac{d}{dt}\frac{\partial\mathcal{L}}{\partial\dot{\phi}} - \frac{\partial\mathcal{L}}{\partial\phi} = 0,
\]
% then becomes
\[
mL\ddot{x}(t)\cos\phi - mL\dot{x}(t)\sin\phi\,\dot{\phi} + mL^2\ddot{\phi} + mL\dot{x}(t)\sin\phi\,\dot{\phi} + mgL\sin\phi = 0.
\]
% Notice that the terms involving \(\dot{x}(t)\sin\phi\,\dot{\phi}\) cancel.
% Dividing the remaining expression by \(mL\) we obtain
\[
L\ddot{\phi} + \ddot{x}(t)\cos\phi + g\sin\phi = 0.
\]
% Since
\[
x(t) = vt^3 + A\sin(\omega t),\quad \text{we have} \quad \ddot{x}(t) = 6vt - A\omega^2\sin(\omega t).
\]
% Thus, the full equation of motion for \(\phi\) is
\[
\boxed{L\ddot{\phi} + \Big[6vt - A\omega^2\sin(\omega t)\Big]\cos\phi + g\sin\phi = 0.}
\]

% (b) In the limit \(v,\,A \to 0\), the acceleration of the support vanishes,
% i.e. \(\ddot{x}(t)=0\). Additionally, for small angles we approximate
% \(\sin\phi\approx\phi\) and \(\cos\phi\approx1\). Then the equation reduces to
\[
L\ddot{\phi} + g\phi = 0,
\]
% which is the classic simple pendulum equation.

\subsubsection{Solution 3}
% We begin by writing the coordinates of the pendulum bob. The pivot (support) moves along
% the horizontal direction with position 
% \[
% x(t) = vt^3 + A\sin(\omega t),
% \]
% so that the bob’s coordinates are given by
% \[
% X = x(t) + L\sin\phi,\quad Y = -L\cos\phi.
% \]

% The kinetic energy is
% \[
% T = \frac{1}{2}m\left(\dot X^2 + \dot Y^2\right).
% \]
% Differentiating,
% \[
% \dot X = \dot x(t) + L\cos\phi\,\dot\phi,\quad \dot Y = L\sin\phi\,\dot\phi,
% \]
% so that
% \[
% \begin{aligned}
% T &= \frac{1}{2}m\left\{\left[\dot x(t) + L\cos\phi\,\dot\phi\right]^2 + \left[L\sin\phi\,\dot\phi\right]^2\right\} \\
%   &= \frac{1}{2}m\left[\dot x(t)^2 + 2L\dot x(t)\cos\phi\,\dot\phi + L^2\dot\phi^2\left(\cos^2\phi+\sin^2\phi\right)\right] \\
%   &= \frac{1}{2}m\left[\dot x(t)^2 + 2L\dot x(t)\cos\phi\,\dot\phi + L^2\dot\phi^2\right].
% \end{aligned}
% \]

% The potential energy, taking the zero at y=0, is
% \[
% V = mgy = -mgL\cos\phi.
% \]

% Thus the Lagrangian is
% \[
% \mathcal{L} = T - V = \frac{1}{2}m\left[\dot x(t)^2 + 2L\dot x(t)\cos\phi\,\dot\phi + L^2\dot\phi^2\right] + mgL\cos\phi.
% \]

% Note that since x(t) is a prescribed function of time, the only generalized coordinate is \(\phi\).

% (a) To derive the equation of motion we use the Euler–Lagrange equation:
% \[
% \frac{d}{dt}\frac{\partial\mathcal{L}}{\partial\dot\phi} - \frac{\partial\mathcal{L}}{\partial\phi} = 0.
% \]

% First, we compute
% \[
% \frac{\partial\mathcal{L}}{\partial\dot\phi} = m\left[L\dot x(t)\cos\phi + L^2\dot\phi\right].
% \]
% Taking the time derivative, we obtain
% \[
% \frac{d}{dt}\frac{\partial\mathcal{L}}{\partial\dot\phi} = m\left[L\ddot x(t)\cos\phi - L\dot x(t)\sin\phi\,\dot\phi + L^2\ddot\phi\right].
% \]

% Next, we differentiate \(\mathcal{L}\) with respect to \(\phi\). The terms that depend on \(\phi\) are those with \(\cos\phi\) and the potential:
% \[
% \frac{\partial}{\partial\phi}\Big(2L\dot x(t)\cos\phi\,\dot\phi\Big) = -2L\dot x(t)\sin\phi\,\dot\phi,
% \]
% so that
% \[
% \frac{\partial\mathcal{L}}{\partial\phi} = \frac{1}{2}m\left[-2L\dot x(t)\sin\phi\,\dot\phi\right] - mgL\sin\phi = -mL\dot x(t)\sin\phi\,\dot\phi - mgL\sin\phi.
% \]

% Plugging into the Euler–Lagrange equation:
% \[
% m\left[L\ddot x(t)\cos\phi - L\dot x(t)\sin\phi\,\dot\phi + L^2\ddot\phi\right] - \left[-mL\dot x(t)\sin\phi\,\dot\phi - mgL\sin\phi\right] = 0.
% \]
% Notice that the terms \(-mL\dot x(t)\sin\phi\,\dot\phi\) cancel:
% \[
% m\left[L\ddot x(t)\cos\phi + L^2\ddot\phi + mgL\sin\phi\right] = 0.
% \]
% Dividing by \(mL\), we obtain
% \[
% \ddot x(t)\cos\phi + L\ddot\phi + g\sin\phi = 0.
% \]
% Rearranging, the equation of motion is
% \[
% L\ddot\phi + g\sin\phi = -\ddot x(t)\cos\phi.
% \]
% Since
% \[
% x(t) = vt^3 + A\sin(\omega t),
% \]
% its second derivative is
% \[
% \ddot x(t) = 6vt - A\omega^2\sin(\omega t).
% \]
% Thus the final equation of motion is
% \[
% \ddot\phi + \frac{g}{L}\sin\phi = -\frac{\cos\phi}{L}\left(6vt - A\omega^2\sin(\omega t)\right).
% \]

% (b) In the limit \(v,\,A\to0\) the acceleration \(\ddot x(t)\to 0\). Moreover, for small angles \(\phi\to0\) we use the approximations
% \[
% \sin\phi\approx\phi,\quad \cos\phi\approx 1.
% \]
% Then the equation reduces to
% \[
% \ddot\phi + \frac{g}{L}\phi = 0,
% \]
% which is the familiar equation of motion for a simple pendulum.

\textbf{Answer:}

\[
\textbf{(a)}\quad L\ddot\phi + g\sin\phi = -\ddot x(t)\cos\phi,\quad \text{with } \ddot x(t) = 6vt - A\omega^2\sin(\omega t).
\]
Thus, the equation of motion for \(\phi\) is
\[
\ddot\phi + \frac{g}{L}\sin\phi = -\frac{\cos\phi}{L}\left(6vt - A\omega^2\sin(\omega t)\right).
\]

\vspace{0.5em}
\[
\textbf{(b)}\quad \text{In the limit } v,\,A\to0 \text{ and for small } \phi \, (\sin\phi\approx\phi,\ \cos\phi\approx 1), \text{ the equation becomes}
\]
\[
\ddot\phi + \frac{g}{L}\phi = 0,
\]
which is the classic linearized pendulum equation.



\subsection{Question 9}


A rope connects to a block of mass $m$ and wraps around a circular disk of mass $M$ and radius $R$. Due to gravity, the rope unwinds and the block falls down.

\begin{itemize}
    \item[a)] Derive expressions for the angular acceleration of the disk, the tension in the rope, and the acceleration of the block. Work under the assumption that the rope does not slip and that the moment of inertia of the disk is given by $I = \frac{1}{2} M R^2$.
    
    \item[b)] Calculate the angular acceleration of the disk, tension in the rope, and acceleration of the block for the case $m = 3\, \text{kg}$, $M = 12\, \text{kg}$, $R = 0.2\, \text{m}$.
\end{itemize}


\paragraph{Prompt: \\} 
Answer the given question as a top physics student. Give your reasoning behind each step. a) [6 marks], b) [1 marks].
Format your answer as LaTeX code without using sub-sections.


\subsubsection{Solution 1}
% For the falling block:
mg - T = ma \quad \text{(1)}

% For the rotating disk:
\[T R = I \alpha, \quad I = \frac{1}{2} M R^2 \quad \text{(2)}\]

% Since the rope does not slip, the block’s linear acceleration a and the disk’s angular acceleration α are related by:
\[a = R \alpha \quad \Rightarrow \quad \alpha = \frac{a}{R}.\]

% Substitute α into (2):
\[T R = \frac{1}{2} M R^2 \left(\frac{a}{R}\right)
\quad \Rightarrow \quad T = \frac{1}{2} M a.\]

% Now substitute T from above into (1):
\[mg - \frac{1}{2} M a = m a \quad \Rightarrow \quad mg = a \left(m + \frac{1}{2} M\right).\]

% Solving for a:
\[a = \frac{mg}{m + \frac{1}{2} M} = \frac{2mg}{2m + M}.\]

% Therefore, the angular acceleration of the disk is:
\[\alpha = \frac{a}{R} = \frac{2mg}{R(2m + M)}.\]

% And the tension in the rope is:
\[T = \frac{1}{2} M a = \frac{M m g}{2m + M}.\]

% (b) For the numerical values: m = 3 kg, M = 12 kg, R = 0.2 m, and g = 9.8 m/s^2:

\[a = \frac{2 \cdot 3 \cdot 9.8}{2 \cdot 3 + 12} 
  = \frac{58.8}{18} 
  \approx 3.27 \, \text{m/s}^2.\]

\[\alpha = \frac{a}{R} 
  \approx \frac{3.27}{0.2} 
  \approx 16.33 \, \text{rad/s}^2.\]

\[T = \frac{12 \cdot 3 \cdot 9.8}{18} 
  = \frac{352.8}{18} 
  \approx 19.6 \, \text{N}. \]


\subsubsection{Solution 2}
\[
\textbf{Step 1: Equations of Motion}
\]
For the falling block (mass \( m \)) we apply Newton’s second law:
\[
mg - T = ma, \quad \text{with } a = R\alpha.
\]
For the rotating disk (mass \( M \) and moment of inertia \( I = \tfrac{1}{2}MR^2 \)) the torque due to the rope is:
\[
TR = I\alpha.
\]
Thus,
\[
T = \frac{I\alpha}{R} = \frac{1}{2}MR\alpha.
\]

\[
\textbf{Step 2: Solving for Angular Acceleration}
\]
Substitute the expression for \( T \) into the block’s equation:
\[
mg - \frac{1}{2}MR\alpha = mR\alpha.
\]
Rearrange to group terms with \(\alpha\):
\[
mg = \left(mR + \frac{1}{2}MR\right)\alpha = R\left(m + \frac{M}{2}\right)\alpha.
\]
Thus, the angular acceleration is:
\[
\alpha = \frac{mg}{R\left(m + \frac{M}{2}\right)}.
\]

\[
\textbf{Step 3: Finding the Tension}
\]
Using the disk’s relation:
\[
T = \frac{1}{2}MR\alpha,
\]
substitute the value of \(\alpha\):
\[
T = \frac{1}{2}MR \cdot \frac{mg}{R\left(m + \frac{M}{2}\right)} = \frac{mMg}{2m+M}.
\]

\[
\textbf{Step 4: Linear Acceleration of the Block}
\]
Since \(a = R\alpha\), we have:
\[
a = R \cdot \frac{mg}{R\left(m + \frac{M}{2}\right)} = \frac{mg}{m + \frac{M}{2}}.
\]

\[
\textbf{Numerical Evaluation for } m=3\,\text{kg},\ M=12\,\text{kg},\ R=0.2\,\text{m}
\]
Calculate the common term:
\[
m + \frac{M}{2} = 3 + \frac{12}{2} = 3 + 6 = 9\,\text{kg}.
\]
\[
\text{Angular acceleration: } \alpha = \frac{3 \times 9.8}{0.2 \times 9} = \frac{29.4}{1.8} \approx 16.33\,\text{rad/s}^2.
\]
\[
\text{Tension: } T = \frac{3 \times 12 \times 9.8}{2 \times 3 + 12} = \frac{352.8}{18} \approx 19.6\,\text{N}.
\]
\[
\text{Linear acceleration: } a = \frac{3 \times 9.8}{9} = \frac{29.4}{9} \approx 3.27\,\text{m/s}^2.
\]

\[
\begin{aligned}
\text{(a)}\quad & \alpha = \frac{mg}{R\left(m + \frac{M}{2}\right)}, \quad T = \frac{mMg}{2m+M}, \quad a = \frac{mg}{m + \frac{M}{2}};\\[1mm]
\text{(b)}\quad & \alpha \approx 16.33\,\text{rad/s}^2, \quad T \approx 19.6\,\text{N}, \quad a \approx 3.27\,\text{m/s}^2.
\end{aligned}
\]



\subsubsection{Solution 3}
\[
\textbf{(a) Derivation:}
\]
Since the rope does not slip, the linear acceleration \(a\) of the falling block is related to the disk's angular acceleration \(\alpha\) by
\[
a = \alpha R.
\]
For the block, applying Newton’s second law (taking downward as positive),
\[
mg - T = ma.
\]
For the disk, the tension \(T\) provides a torque:
\[
\tau = TR = I\alpha = \frac{1}{2}MR^2 \alpha.
\]
Thus,
\[
T = \frac{1}{2}MR\alpha.
\]
Using the relation \(a = \alpha R\) (i.e. \(\alpha = \frac{a}{R}\)), we substitute to get
\[
T = \frac{1}{2}M a.
\]
Substitute this expression for \(T\) back into the block’s equation:
\[
mg - \frac{1}{2}M a = ma.
\]
Collecting terms,
\[
mg = \left(m + \frac{1}{2}M\right)a,
\]
so that the linear acceleration is
\[
a = \frac{mg}{m + \frac{1}{2}M}.
\]
The angular acceleration of the disk is then
\[
\alpha = \frac{a}{R} = \frac{mg}{\left(m + \frac{1}{2}M\right)R},
\]
and the tension in the rope is
\[
T = \frac{1}{2}M a = \frac{1}{2}M \left(\frac{mg}{m + \frac{1}{2}M}\right).
\]

\[
\textbf{(b) Numerical Evaluation:}
\]
Given \(m = 3\,\mathrm{kg}\), \(M = 12\,\mathrm{kg}\), and \(R = 0.2\,\mathrm{m}\), we first calculate
\[
m + \frac{1}{2}M = 3 + \frac{1}{2}(12) = 3 + 6 = 9\,\mathrm{kg}.
\]
Thus, the block’s acceleration is
\[
a = \frac{mg}{m + \frac{1}{2}M} = \frac{3 \times 9.81}{9} \approx \frac{29.43}{9} \approx 3.27\,\mathrm{m/s^2}.
\]
The angular acceleration of the disk is
\[
\alpha = \frac{a}{R} = \frac{3.27}{0.2} \approx 16.35\,\mathrm{rad/s^2},
\]
and the tension in the rope becomes
\[
T = \frac{1}{2}M a = \frac{1}{2} \times 12 \times 3.27 \approx 6 \times 3.27 \approx 19.62\,\mathrm{N}.
\]


\subsection{Question 10}

On Earth, a ball of mass $m$ is dropped from an airplane moving with horizontal velocity $u$. The air resistance on the ball causes a force opposite to the ball's velocity $\mathbf{v}$, namely

\[
\mathbf{F} = -b \mathbf{v}
\]
where $b$ is a positive constant. \\

\textbf{a)} Using Newton's laws of motion, show that

\[
m \frac{d \mathbf{v}}{dt} = -mg \hat{k} - b \mathbf{v}
\]
and show the solution to this equation is of the form

\[
\mathbf{v} = \mathbf{A} e^{-bt/m} - \frac{mg}{b} \hat{k}
\]
where $\mathbf{A}$ is a constant vector.\\

\textbf{b)} Given that the ball initially starts with velocity $\mathbf{u}$, find an expression for $\mathbf{A}$.

\paragraph{Prompt: \\} 
Answer the given question as a top physics student. Give your reasoning behind each step. a) [6 marks], b) [2 marks].
Format your answer as LaTeX code without using sub-sections.

\subsubsection{Solution 1}
\[
\textbf{(a)}\quad \text{Newton's second law gives } m\frac{d\mathbf{v}}{dt} = \mathbf{F}_{\text{gravity}} + \mathbf{F}_{\text{air}}.
\]
\[
\text{Since } \mathbf{F}_{\text{gravity}} = -mg\,\hat{k} \quad \text{and} \quad \mathbf{F}_{\text{air}} = -b\,\mathbf{v},
\]
\[
\text{we have } m\frac{d\mathbf{v}}{dt} = -mg\,\hat{k} - b\,\mathbf{v}.
\]
\[
\text{Dividing by } m \text{ yields the ODE:} \quad \frac{d\mathbf{v}}{dt} + \frac{b}{m}\mathbf{v} = -g\,\hat{k}.
\]
\[
\text{Multiplying by the integrating factor } e^{(b/m)t} \text{ gives:}
\]
\[
e^{(b/m)t}\frac{d\mathbf{v}}{dt} + \frac{b}{m}e^{(b/m)t}\mathbf{v} = \frac{d}{dt}\left(e^{(b/m)t}\mathbf{v}\right) = -g\,e^{(b/m)t}\hat{k}.
\]
\[
\text{Integrate both sides with respect to } t:
\]
\[
e^{(b/m)t}\mathbf{v} = -g\,\hat{k}\int e^{(b/m)t}\,dt + \mathbf{C},
\]
\[
\text{where } \mathbf{C} \text{ is an integration constant. Recognizing that }
\]
\[
\int e^{(b/m)t}\,dt = \frac{m}{b}\,e^{(b/m)t},
\]
\[
\text{we obtain: } e^{(b/m)t}\mathbf{v} = -\frac{mg}{b}\,e^{(b/m)t}\hat{k} + \mathbf{C}.
\]
\[
\text{Dividing through by } e^{(b/m)t} \text{ leads to the general solution: }
\]
\[
\mathbf{v} = \mathbf{C}\,e^{-bt/m} - \frac{mg}{b}\hat{k}.
\]
\[
\text{Defining } \mathbf{A} = \mathbf{C}, \text{ we have: }
\]
\[
\mathbf{v} = \mathbf{A}\,e^{-bt/m} - \frac{mg}{b}\hat{k}.
\]

\[
\textbf{(b)}\quad \text{The initial condition is } \mathbf{v}(0) = \mathbf{u}.
\]
\[
\text{At } t = 0, \quad \mathbf{v}(0) = \mathbf{A}\,e^0 - \frac{mg}{b}\hat{k} = \mathbf{A} - \frac{mg}{b}\hat{k}.
\]
\[
\text{Setting } \mathbf{A} - \frac{mg}{b}\hat{k} = \mathbf{u} \text{ gives: }
\]
\[
\mathbf{A} = \mathbf{u} + \frac{mg}{b}\hat{k}.
\]

\subsubsection{Solution 2}
\[
\textbf{(a)}\quad \text{Newton's second law gives:} \quad m\frac{d\mathbf{v}}{dt} = \mathbf{F}_{\text{gravity}} + \mathbf{F}_{\text{drag}}.
\]
Since the gravitational force is 
\[
\mathbf{F}_{\text{gravity}} = -mg\,\hat{k}
\]
and the drag force is 
\[
\mathbf{F}_{\text{drag}} = -b\,\mathbf{v},
\]
we have
\[
m\frac{d\mathbf{v}}{dt} = -mg\,\hat{k} - b\,\mathbf{v}.
\]

Dividing by \(m\) and writing the equation in standard linear form yields:
\[
\frac{d\mathbf{v}}{dt} + \frac{b}{m}\mathbf{v} = -g\,\hat{k}.
\]

The integrating factor is 
\[
\mu(t)=e^{\frac{b}{m}t}.
\]
Multiplying through by \(\mu(t)\) gives
\[
e^{\frac{b}{m}t}\frac{d\mathbf{v}}{dt} + \frac{b}{m}e^{\frac{b}{m}t}\mathbf{v} = \frac{d}{dt}\left(e^{\frac{b}{m}t}\mathbf{v}\right) = -g\,\hat{k}e^{\frac{b}{m}t}.
\]

Integrating both sides with respect to \(t\),
\[
e^{\frac{b}{m}t}\mathbf{v} = -g\,\hat{k}\int_0^t e^{\frac{b}{m}\tau} d\tau + \mathbf{C},
\]
where \(\mathbf{C}\) is a constant vector. Evaluating the integral,
\[
\int_0^t e^{\frac{b}{m}\tau} d\tau = \frac{m}{b}\left(e^{\frac{b}{m}t}-1\right),
\]
so that
\[
e^{\frac{b}{m}t}\mathbf{v} = -g\,\hat{k}\frac{m}{b}\left(e^{\frac{b}{m}t}-1\right) + \mathbf{C}.
\]

Dividing by \(e^{\frac{b}{m}t}\) leads to
\[
\mathbf{v} = \mathbf{C}\,e^{-\frac{b}{m}t} - \frac{mg}{b}\hat{k}\left(1 - e^{-\frac{b}{m}t}\right).
\]
Rearranging the constant term, we can write this solution in the form
\[
\mathbf{v} = \left(\mathbf{C} + \frac{mg}{b}\hat{k}\right)e^{-\frac{b}{m}t} - \frac{mg}{b}\hat{k}.
\]
Defining \(\mathbf{A} = \mathbf{C} + \frac{mg}{b}\hat{k}\), the solution becomes
\[
\mathbf{v} = \mathbf{A}\,e^{-\frac{b}{m}t} - \frac{mg}{b}\hat{k}.
\]

\[
\textbf{(b)}\quad \text{The initial condition is } \mathbf{v}(0)=\mathbf{u}.
\]
At \(t=0\),
\[
\mathbf{v}(0) = \mathbf{A}\,e^0 - \frac{mg}{b}\hat{k} = \mathbf{A} - \frac{mg}{b}\hat{k} = \mathbf{u}.
\]
Thus,
\[
\mathbf{A} = \mathbf{u} + \frac{mg}{b}\hat{k}.
\]


\subsubsection{Solution 3}
% (a)
Using Newton’s second law we have
\[
m\frac{d\mathbf{v}}{dt} = \mathbf{F}_{\text{gravity}} + \mathbf{F}_{\text{drag}},
\]
where the gravitational force is 
\[
\mathbf{F}_{\text{gravity}} = -mg\,\hat{k},
\]
and the drag force is 
\[
\mathbf{F}_{\text{drag}} = -b\,\mathbf{v}.
\]
Thus,
\[
m\frac{d\mathbf{v}}{dt} = -mg\,\hat{k} - b\,\mathbf{v}.
\]

This is a linear first-order differential equation. We rewrite it as
\[
\frac{d\mathbf{v}}{dt} + \frac{b}{m}\mathbf{v} = -g\,\hat{k}.
\]
To solve it, we use the integrating factor
\[
\mu(t)= e^{\frac{b}{m}t}.
\]
Multiplying the equation by \(\mu(t)\) yields
\[
e^{bt/m}\frac{d\mathbf{v}}{dt} + \frac{b}{m}e^{bt/m}\mathbf{v} = \frac{d}{dt}\Big(e^{bt/m}\mathbf{v}\Big) = -g\,\hat{k}\,e^{bt/m}.
\]
Integrate both sides with respect to \(t\):
\[
e^{bt/m}\mathbf{v}(t) = \mathbf{C} - g\,\hat{k}\int_0^t e^{bs/m}\,ds,
\]
where \(\mathbf{C}\) is an integration constant vector. Evaluating the integral,
\[
\int_0^t e^{bs/m}\,ds = \frac{m}{b}\Big(e^{bt/m}-1\Big),
\]
we obtain
\[
e^{bt/m}\mathbf{v}(t) = \mathbf{C} - \frac{mg}{b}\,\hat{k}\Big(e^{bt/m}-1\Big).
\]
Multiplying through by \(e^{-bt/m}\) gives
\[
\mathbf{v}(t) = \mathbf{C}\,e^{-bt/m} - \frac{mg}{b}\,\hat{k}\Big(1 - e^{-bt/m}\Big).
\]
We can rewrite this solution in the desired form by defining a new constant vector
\[
\mathbf{A} = \mathbf{C} + \frac{mg}{b}\,\hat{k},
\]
so that
\[
\mathbf{v}(t) = \mathbf{A}\,e^{-bt/m} - \frac{mg}{b}\,\hat{k}.
\]

% (b)
At \(t=0\) the ball’s velocity is \(\mathbf{u}\), so
\[
\mathbf{v}(0)=\mathbf{A}\,e^0 - \frac{mg}{b}\,\hat{k} = \mathbf{A} - \frac{mg}{b}\,\hat{k} = \mathbf{u}.
\]
Solving for \(\mathbf{A}\) we find
\[
\mathbf{A} = \mathbf{u} + \frac{mg}{b}\,\hat{k}.
\]

\boxed{\mathbf{A} = \mathbf{u} + \frac{mg}{b}\,\hat{k}}




\section{Quantum Mechanics}

\subsection{Question 1}

\begin{enumerate}
    \item[(a)] Define the commutator of 2 operators, $\hat{P}$ and $\hat{Q}$. Define what it means if $\hat{P}$ and $\hat{Q}$ are compatible operators. Explain what compatibility implies about their commutator?
    \item[(b)] Suppose that $\hat{P}$ and $\hat{Q}$ are not compatible. For a given system, $\hat{P}$ is first measured giving a value $p$, then $\hat{Q}$ is measured giving the value $q$. If $\hat{P}$ is then measured again, what can be said about the possible results of the measurement and why?
\end{enumerate}

\paragraph{Prompt: \\} 
Answer the given question as a top physics student, giving step-by-step reasoning keeping your answer under 300 words. Format your answer as LaTeX code without using sub-sections. Available marks: a) [3 marks] b) [2 marks].

\subsubsection{Solution 1}

(a)
The commutator of two operators $\hat{P}$ and $\hat{Q}$ is defined as 
\[
[\hat{P},\hat{Q}] = \hat{P}\hat{Q} - \hat{Q}\hat{P}.
\]
Operators are said to be \emph{compatible} if they possess a common complete set of eigenstates. This implies that both observables can be simultaneously measured with certainty. Consequently, if $\hat{P}$ and $\hat{Q}$ are compatible, they must commute, i.e.,
\[
[\hat{P},\hat{Q}] = 0.
\]

(b)
For non-compatible operators, the measurement process itself alters the state. When $\hat{P}$ is measured first yielding an eigenvalue $p$, the system collapses to the corresponding eigenstate of $\hat{P}$. However, a subsequent measurement of $\hat{Q}$ forces the system into an eigenstate of $\hat{Q}$, which is generally a superposition of the eigenstates of $\hat{P}$. Therefore, a later measurement of $\hat{P}$ will not necessarily reproduce the initial value $p$; instead, it will yield one of the possible eigenvalues of $\hat{P}$, with probabilities determined by the projection of the $\hat{Q}$ eigenstate onto the eigenbasis of $\hat{P}$.

\subsubsection{Solution 2}
\textbf{(a)} The commutator of two operators $\hat{P}$ and $\hat{Q}$ is defined as 
\[
[\hat{P},\hat{Q}] = \hat{P}\hat{Q} - \hat{Q}\hat{P}.
\]
Operators are said to be \textit{compatible} if they share a complete set of eigenstates. This means that measuring one does not disturb the eigenstate structure of the other. Consequently, if $\hat{P}$ and $\hat{Q}$ are compatible, their commutator vanishes:
\[
[\hat{P},\hat{Q}] = 0.
\]
A zero commutator implies that the order of measurements is irrelevant and both observables can be simultaneously measured with arbitrary precision.

\textbf{(b)} When $\hat{P}$ and $\hat{Q}$ are not compatible (i.e., $[\hat{P},\hat{Q}] \neq 0$), an initial measurement of $\hat{P}$ collapses the system into an eigenstate corresponding to the eigenvalue $p$. However, measuring $\hat{Q}$ next forces the state into an eigenstate of $\hat{Q}$, which generally is not an eigenstate of $\hat{P}$. Thus, if $\hat{P}$ is measured again, the system is no longer in the original eigenstate, and the outcome will generally be a probabilistic distribution over the possible eigenvalues of $\hat{P}$. The probabilities of these outcomes are determined by the overlap (i.e., the projection) of the $\hat{Q}$ eigenstate onto the eigenstates of $\hat{P}$. This sequence of measurements illustrates how non-commuting operators lead to uncertainty and state disturbance.

\subsubsection{Solution 3}
\textbf{(a)} The commutator of two operators $\hat{P}$ and $\hat{Q}$ is defined as 
\[
[\hat{P},\hat{Q}] = \hat{P}\hat{Q} - \hat{Q}\hat{P}.
\]
Operators are said to be \emph{compatible} if they have a complete set of common eigenstates, meaning that they can be measured simultaneously without disturbing each other. In this case, their measurements are independent of the order, which implies 
\[
[\hat{P},\hat{Q}] = 0.
\]

\textbf{(b)} When $\hat{P}$ and $\hat{Q}$ are not compatible (i.e., $[\hat{P},\hat{Q}] \neq 0$), they do not share a common set of eigenstates. If the system is initially measured to have a value $p$ for $\hat{P}$ and then a value $q$ for $\hat{Q}$, the measurement of $\hat{Q}$ collapses the state into an eigenstate of $\hat{Q}$ rather than of $\hat{P}$. As a result, a subsequent measurement of $\hat{P}$ will not necessarily yield the original value $p$ but rather a probabilistic outcome determined by the overlap (projection) of the $\hat{Q}$ eigenstate onto the eigenstates of $\hat{P}$. This disturbance in the state is a direct consequence of the non-zero commutator.

\subsection{Question 2}
\noindent Q2. Worded:

\begin{enumerate}
    \item[(a)] Quantum mechanical operators are Hermitian. What mathematical property do the eigenvalues of a Hermitian operator have? What is the physical interpretation of its eigenvalues?
    \item[(b)] Explain what is meant if it is stated that the set of eigenfunctions $\{\phi_n\}$ of a Hermitian operator is orthonormal.
\end{enumerate}


\paragraph{Prompt: \\} 
Answer the given question as a top physics student, giving step-by-step reasoning keeping your answer under 300 words. Format your answer as LaTeX code without using sub-sections. Available marks: a) [2 marks] b) [2 marks].

\subsubsection{Solution 1}
\textbf{Answer:} 

\begin{enumerate}
    \item[(a)] Let $A$ be a Hermitian operator so that $A = A^\dagger$. For any eigenfunction $\psi$ with eigenvalue $\lambda$, we have 
    \[
    A\psi = \lambda \psi.
    \]
    Taking the inner product with $\psi$ yields 
    \[
    \langle \psi, A\psi \rangle = \lambda \langle \psi, \psi \rangle.
    \]
    Since $A$ is Hermitian, 
    \[
    \langle \psi, A\psi \rangle = \langle A\psi, \psi \rangle = \lambda^* \langle \psi, \psi \rangle.
    \]
    Equating these gives $\lambda = \lambda^*$, which means $\lambda$ is real. Physically, the eigenvalues represent the possible outcomes of measurements of the observable corresponding to $A$.
    
    \item[(b)] The eigenfunctions $\{\phi_n\}$ are said to be orthonormal if they satisfy 
    \[
    \langle \phi_n, \phi_m \rangle = \delta_{nm},
    \]
    where $\delta_{nm}$ is the Kronecker delta ($\delta_{nm}=1$ for $n=m$ and $0$ otherwise). This means each $\phi_n$ is normalized (unit norm) and any two distinct eigenfunctions are orthogonal. Such a set forms a complete basis for the Hilbert space, allowing any quantum state to be expressed as a linear combination of these eigenfunctions.
\end{enumerate}

\subsubsection{Solution 2}

\textbf{(a)} For a Hermitian operator $\hat{A}$, the eigenvalue equation is 
\[
\hat{A}\phi = \lambda \phi.
\]
A key mathematical property is that all eigenvalues $\lambda$ are real. This follows from the definition of Hermiticity, since for any eigenfunction $\phi$, we have
\[
\langle \phi|\hat{A}\phi\rangle = \lambda \langle \phi|\phi\rangle,
\]
and the left-hand side is necessarily real. Physically, these eigenvalues represent the possible outcomes of a measurement corresponding to the observable associated with $\hat{A}$.

\textbf{(b)} The statement that the set of eigenfunctions $\{\phi_n\}$ is orthonormal means that each eigenfunction is normalized and mutually orthogonal. Mathematically, this is expressed as
\[
\langle \phi_m|\phi_n\rangle = \delta_{mn},
\]
where $\delta_{mn}$ is the Kronecker delta, equal to 1 when $m = n$ (normalization) and 0 when $m \neq n$ (orthogonality). This property ensures that the eigenfunctions form a complete basis for the Hilbert space, facilitating the expansion of any state in terms of these functions.

\subsubsection{Solution 3}
\textbf{(a)} Let $\hat{A}$ be a Hermitian operator, so that $\hat{A} = \hat{A}^\dagger$. If $\phi$ is an eigenfunction with eigenvalue $\lambda$, i.e.,
\[
\hat{A}\phi = \lambda \phi,
\]
taking the inner product with $\phi$ gives
\[
\langle \phi | \hat{A}\phi \rangle = \lambda \langle \phi | \phi \rangle.
\]
Since $\langle \phi | \hat{A}\phi \rangle$ is equal to its complex conjugate (due to the Hermiticity of $\hat{A}$), $\lambda$ must be real. Physically, these real eigenvalues correspond to the possible outcomes of measuring the observable associated with $\hat{A}$.

\bigskip

\textbf{(b)} The eigenfunctions $\{\phi_n\}$ are said to be orthonormal if they satisfy
\[
\langle \phi_m | \phi_n \rangle = \delta_{mn},
\]
where $\delta_{mn}$ is the Kronecker delta (equal to 1 if $m = n$ and 0 otherwise). This means each eigenfunction is normalized (its norm is unity) and different eigenfunctions are mutually orthogonal. Consequently, any state in the Hilbert space can be expressed as a linear combination of these eigenfunctions.

\subsection{Question 3} 

For one dimension, an operator $\hat{Q}$ is Hermitian if and only if
\[
\int_{-\infty}^{\infty} f^* \hat{Q} g \, dx = \int_{-\infty}^{\infty} g (\hat{Q} f)^* \, dx
\]
where $f(x), g(x)$ are well-behaved functions which vanish at infinity, $*$ denotes the complex conjugate.

a) Use the definition above to determine whether $\hat{p} = -i\hbar \frac{d}{dx}$ is a Hermitian operator.

b) Determine whether $\hat{Q} = \frac{d^2}{dx^2}$ is a Hermitian operator.

\paragraph{Prompt: \\} 
Answer the given question as a top physics student, giving step-by-step reasoning. Format your answer as LaTeX code without using sub-sections. Available marks: a) [4 marks], b) [5 marks].

\subsubsection{Solution 1}
% For (a)
We must show that for all well‐behaved functions \(f(x)\) and \(g(x)\) that vanish at infinity,
\[
\int_{-\infty}^{\infty} f^*(x) \left(-i\hbar \frac{d}{dx} g(x)\right) dx = \int_{-\infty}^{\infty} g(x) \left(-i\hbar \frac{d}{dx} f(x)\right)^* dx.
\]
Starting with the left-hand side (LHS):
\[
\int_{-\infty}^{\infty} f^*(x) \left(-i\hbar g'(x)\right) dx.
\]
Integrate by parts by letting
\[
u = f^*(x),\quad dv = -i\hbar\, g'(x)\,dx,\quad \text{so that} \quad du = f'(x)^*\,dx,\quad v = -i\hbar\, g(x).
\]
Then,
\[
\int_{-\infty}^{\infty} f^*(x)\left(-i\hbar g'(x)\right)dx = \left[ f^*(x)(-i\hbar g(x)) \right]_{-\infty}^{\infty} + i\hbar \int_{-\infty}^{\infty} g(x) f'(x)^* dx.
\]
Since \(f(x)\) and \(g(x)\) vanish at infinity, the boundary term is zero and we obtain
\[
\int_{-\infty}^{\infty} f^*(x) \left(-i\hbar g'(x)\right)dx = i\hbar \int_{-\infty}^{\infty} g(x) f'(x)^* dx.
\]
Now, for the right-hand side (RHS):
\[
\int_{-\infty}^{\infty} g(x) \left(-i\hbar f'(x)\right)^* dx = \int_{-\infty}^{\infty} g(x) \left(i\hbar f'(x)^*\right) dx = i\hbar \int_{-\infty}^{\infty} g(x) f'(x)^* dx.
\]
Since the LHS equals the RHS, we conclude that
\[
\hat{p} = -i\hbar \frac{d}{dx}
\]
is a Hermitian operator.

% For (b)
We now consider the operator
\[
\hat{Q} = \frac{d^2}{dx^2}.
\]
We must verify that
\[
\int_{-\infty}^{\infty} f^*(x) g''(x) dx = \int_{-\infty}^{\infty} g(x) \left(f''(x)\right)^* dx.
\]
First, integrate the left-hand side by parts:
\[
\int_{-\infty}^{\infty} f^*(x) g''(x) dx = \left[ f^*(x) g'(x) \right]_{-\infty}^{\infty} - \int_{-\infty}^{\infty} f'(x)^* g'(x) dx.
\]
Because \(f(x)\) and \(g(x)\) (and their derivatives) vanish at infinity, the boundary term is zero, so we have
\[
\int_{-\infty}^{\infty} f^*(x) g''(x) dx = - \int_{-\infty}^{\infty} f'(x)^* g'(x) dx.
\]
Similarly, note that
\[
\left(f''(x)\right)^* = \left(f^*(x)\right)'',
\]
so the right-hand side becomes
\[
\int_{-\infty}^{\infty} g(x) \left(f^*(x)\right)'' dx.
\]
Integrate by parts:
\[
\int_{-\infty}^{\infty} g(x) \left(f^*(x)\right)'' dx = \left[ g(x) \left(f^*(x)\right)' \right]_{-\infty}^{\infty} - \int_{-\infty}^{\infty} g'(x) \left(f^*(x)\right)' dx.
\]
Again, the boundary term vanishes, and since \(\left(f^*(x)\right)' = f'(x)^*\) we obtain
\[
\int_{-\infty}^{\infty} g(x) \left(f''(x)\right)^* dx = - \int_{-\infty}^{\infty} g'(x) f'(x)^* dx.
\]
Thus, the two expressions are identical:
\[
\int_{-\infty}^{\infty} f^*(x) g''(x) dx = - \int_{-\infty}^{\infty} f'(x)^* g'(x) dx = \int_{-\infty}^{\infty} g(x) \left(f''(x)\right)^* dx.
\]
Therefore, provided the boundary conditions are satisfied, the operator
\[
\hat{Q} = \frac{d^2}{dx^2}
\]
is also Hermitian.

\subsubsection{Solution 2}

We wish to verify that an operator $\hat{Q}$ is Hermitian by checking that 
\[
\int_{-\infty}^{\infty} f^*(x)\, \hat{Q}g(x) \, dx = \int_{-\infty}^{\infty} g(x)\, \left(\hat{Q}f(x)\right)^* \, dx
\]
for all well-behaved functions $f(x)$ and $g(x)$ that vanish at infinity.

For (a), let $\hat{p} = -i\hbar\, \frac{d}{dx}$. We compute
\[
\int_{-\infty}^{\infty} f^*(x)\, \hat{p} g(x) \, dx = -i\hbar \int_{-\infty}^{\infty} f^*(x) g'(x) \, dx.
\]
Integrate by parts with 
\[
u=f^*(x), \quad dv=-i\hbar\,g'(x)\,dx, \quad du=f'^*(x)\,dx, \quad v=-i\hbar\,g(x).
\]
Thus,
\[
- i\hbar \int_{-\infty}^{\infty} f^*(x) g'(x) \, dx = \left[f^*(x)(-i\hbar\, g(x))\right]_{-\infty}^{\infty} + i\hbar \int_{-\infty}^{\infty} g(x) f'^*(x) \, dx.
\]
Since $f(x)$ and $g(x)$ vanish at infinity, the boundary term is zero, yielding
\[
\int_{-\infty}^{\infty} f^*(x)\, \hat{p}g(x) \, dx = i\hbar \int_{-\infty}^{\infty} g(x) f'^*(x) \, dx.
\]
Noting that 
\[
\left(\hat{p}f(x)\right)^* = \left(-i\hbar\,f'(x)\right)^* = i\hbar\, f'^*(x),
\]
we have
\[
\int_{-\infty}^{\infty} f^*(x)\, \hat{p}g(x) \, dx = \int_{-\infty}^{\infty} g(x)\,\left(\hat{p}f(x)\right)^* \, dx.
\]
Thus, $\hat{p}$ is Hermitian.

For (b), consider $\hat{Q} = \frac{d^2}{dx^2}$. We must check whether
\[
\int_{-\infty}^{\infty} f^*(x) g''(x) \, dx = \int_{-\infty}^{\infty} g(x) \left(f''(x)\right)^* \, dx.
\]
First, integrate the left-hand side by parts by taking
\[
u=f^*(x), \quad dv=g''(x)\,dx, \quad du=f'^*(x)\,dx, \quad v=g'(x).
\]
This gives
\[
\int_{-\infty}^{\infty} f^*(x) g''(x) \, dx = \left[f^*(x)g'(x)\right]_{-\infty}^{\infty} - \int_{-\infty}^{\infty} f'^*(x) g'(x) \, dx.
\]
Assuming that the boundary term $\left[f^*(x)g'(x)\right]_{-\infty}^{\infty}$ vanishes, we have
\[
\int_{-\infty}^{\infty} f^*(x) g''(x) \, dx = - \int_{-\infty}^{\infty} f'^*(x) g'(x) \, dx.
\]
Similarly, the right-hand side can be manipulated by integrating by parts:
\[
\int_{-\infty}^{\infty} g(x)\left(f''(x)\right)^* \, dx = \left[g(x) f'^*(x)\right]_{-\infty}^{\infty} - \int_{-\infty}^{\infty} g'(x) f'^*(x) \, dx.
\]
With the boundary term $\left[g(x) f'^*(x)\right]_{-\infty}^{\infty}$ vanishing, we obtain
\[
\int_{-\infty}^{\infty} g(x)\left(f''(x)\right)^* \, dx = - \int_{-\infty}^{\infty} g'(x) f'^*(x) \, dx.
\]
Since 
\[
\int_{-\infty}^{\infty} f'^*(x) g'(x) \, dx = \int_{-\infty}^{\infty} g'(x) f'^*(x) \, dx,
\]
the two expressions are equal. Therefore, under the assumption that the boundary terms vanish (which generally requires that not only $f(x)$ and $g(x)$ but also their first derivatives vanish at infinity), $\hat{Q} = \frac{d^2}{dx^2}$ is Hermitian.

\subsubsection{Solution 3}
\textbf{Answer:}

\textbf{(a)} To check if the momentum operator 
\[
\hat{p} = -i\hbar\frac{d}{dx}
\]
is Hermitian, we need to verify that
\[
\int_{-\infty}^{\infty} f^*(x) \hat{p} g(x)\, dx = \int_{-\infty}^{\infty} g(x) \left(\hat{p} f(x)\right)^* dx
\]
for any well-behaved functions \(f(x)\) and \(g(x)\) that vanish at infinity.

Starting with the left-hand side:
\[
\int_{-\infty}^{\infty} f^*(x) \left[-i\hbar \frac{d}{dx}g(x)\right]\, dx = -i\hbar \int_{-\infty}^{\infty} f^*(x) \frac{d}{dx}g(x)\, dx.
\]

We apply integration by parts with:
\[
u = f^*(x),\quad dv = \frac{d}{dx}g(x)\,dx \quad \Longrightarrow \quad du = \frac{d}{dx}f^*(x)\, dx,\quad v = g(x).
\]
Thus,
\[
-i\hbar\int_{-\infty}^{\infty} f^*(x) \frac{d}{dx}g(x)\, dx = -i\hbar\left[ f^*(x)g(x) \Big|_{-\infty}^{\infty} - \int_{-\infty}^{\infty} g(x) \frac{d}{dx}f^*(x)\, dx \right].
\]
Since \(f\) and \(g\) vanish at \(\pm \infty\), the boundary term \(f^*(x)g(x)\Big|_{-\infty}^{\infty}\) vanishes. Therefore, we obtain
\[
-i\hbar\int_{-\infty}^{\infty} f^*(x) \frac{d}{dx}g(x)\, dx = i\hbar \int_{-\infty}^{\infty} g(x) \frac{d}{dx}f^*(x)\, dx.
\]
Recall that
\[
\left(\hat{p}f(x)\right)^* = \left[-i\hbar \frac{d}{dx}f(x)\right]^* = i\hbar \frac{d}{dx}f^*(x).
\]
Thus, the right-hand side of the Hermiticity condition becomes:
\[
\int_{-\infty}^{\infty} g(x) \left(\hat{p} f(x)\right)^* dx = \int_{-\infty}^{\infty} g(x) \left(i\hbar \frac{d}{dx}f^*(x)\right) dx = i\hbar \int_{-\infty}^{\infty} g(x) \frac{d}{dx}f^*(x)\, dx.
\]
This is identical to the result obtained from the left-hand side. Hence, 
\[
\int_{-\infty}^{\infty} f^*(x) \hat{p} g(x)\, dx = \int_{-\infty}^{\infty} g(x) \left(\hat{p} f(x)\right)^* dx,
\]
and therefore \(\hat{p}\) is Hermitian. \hfill [4 marks]

\textbf{(b)} Now, consider the operator 
\[
\hat{Q} = \frac{d^2}{dx^2}.
\]
We test its Hermiticity by checking if
\[
\int_{-\infty}^{\infty} f^*(x) \frac{d^2}{dx^2}g(x)\, dx = \int_{-\infty}^{\infty} g(x) \left(\frac{d^2}{dx^2}f(x)\right)^* dx.
\]
Note that since the second derivative is real, we have
\[
\left(\frac{d^2}{dx^2}f(x)\right)^* = \frac{d^2}{dx^2}f^*(x).
\]

Starting with the left-hand side,
\[
I = \int_{-\infty}^{\infty} f^*(x) \frac{d^2}{dx^2}g(x)\, dx.
\]
Perform integration by parts once:
\[
u = f^*(x),\quad dv = \frac{d^2}{dx^2}g(x)\, dx \quad \Longrightarrow \quad du = \frac{d}{dx}f^*(x)\, dx,\quad v = \frac{d}{dx}g(x).
\]
Thus,
\[
I = \left. f^*(x)\frac{d}{dx}g(x) \right|_{-\infty}^{\infty} - \int_{-\infty}^{\infty} \frac{d}{dx}g(x) \frac{d}{dx}f^*(x)\, dx.
\]
Again, the boundary term vanishes because the functions vanish at infinity. Hence,
\[
I = -\int_{-\infty}^{\infty} \frac{d}{dx}g(x) \frac{d}{dx}f^*(x)\, dx.
\]

Now, consider the right-hand side:
\[
J = \int_{-\infty}^{\infty} g(x) \frac{d^2}{dx^2}f^*(x)\, dx.
\]
Perform integration by parts similarly:
\[
u = g(x),\quad dv = \frac{d^2}{dx^2}f^*(x)\, dx \quad \Longrightarrow \quad du = \frac{d}{dx}g(x)\, dx,\quad v = \frac{d}{dx}f^*(x).
\]
Thus,
\[
J = \left. g(x)\frac{d}{dx}f^*(x) \right|_{-\infty}^{\infty} - \int_{-\infty}^{\infty} \frac{d}{dx}f^*(x) \frac{d}{dx}g(x)\, dx.
\]
Again, the boundary term vanishes, yielding:
\[
J = -\int_{-\infty}^{\infty} \frac{d}{dx}f^*(x) \frac{d}{dx}g(x)\, dx.
\]

Since the integrals in \(I\) and \(J\) are identical (the integrands are the same up to the order of factors, which is immaterial in a product of functions), we have:
\[
I = J.
\]
Thus, 
\[
\int_{-\infty}^{\infty} f^*(x) \frac{d^2}{dx^2}g(x)\, dx = \int_{-\infty}^{\infty} g(x) \left(\frac{d^2}{dx^2}f(x)\right)^* dx,
\]
which confirms that \(\hat{Q} = \frac{d^2}{dx^2}\) is Hermitian. \hfill [5 marks]

\subsection{Question 4}

Let $\hat{L}_x$ be the x-component of the angular momentum operator. Let $\hat{X}, \hat{Y}, \hat{Z}$ be the x, y, z components of the position operator respectively and $\hat{P}_x, \hat{P}_y, \hat{P}_z$ be the x, y, z components of the momentum operator respectively. Derive the following commutator relations:   \\

a) $\left[\hat{L}_x, \hat{X}\right] = 0$  

b) $\left[\hat{L}_x, \hat{P}_x\right] = 0$  

c) $\left[\hat{L}_x, \hat{Y}\right] = i\hbar \hat{Z}$  

d) $\left[\hat{L}_x, \hat{P}_y\right] = i\hbar \hat{P}_z$  

e) $\left[\hat{L}_x, \hat{P}^2\right] = 0$

\paragraph{Prompt: \\} 
Answer the given question as a top physics student, giving step-by-step reasoning. Format your answer as LaTeX code without using sub-sections. Available marks: a) [3 marks], b) [3 marks], c) [4 marks], d) [4 marks], e) [6 marks].

\subsubsection{Solution 1}
% We start by recalling that 
% \[
% \hat{L}_x = \hat{Y}\hat{P}_z - \hat{Z}\hat{P}_y,
% \]
% and the canonical commutation relations
% \[
% [\hat{X}_i, \hat{P}_j] = i\hbar\,\delta_{ij},
% \]
% with all other pairs commuting.

% (a) [\hat{L}_x, \hat{X}]
\[
\begin{aligned}
[\hat{L}_x, \hat{X}] &= [\hat{Y}\hat{P}_z - \hat{Z}\hat{P}_y,\hat{X}]\\[1mm]
&= \hat{Y}[\hat{P}_z,\hat{X}] + [\hat{Y},\hat{X}]\hat{P}_z - \hat{Z}[\hat{P}_y,\hat{X}] - [\hat{Z},\hat{X}]\hat{P}_y\\[1mm]
&= \hat{Y}\cdot 0 + 0\cdot\hat{P}_z - \hat{Z}\cdot 0 - 0\cdot\hat{P}_y\\[1mm]
&= 0.
\end{aligned}
\]

% (b) [\hat{L}_x, \hat{P}_x]
\[
\begin{aligned}
[\hat{L}_x, \hat{P}_x] &= [\hat{Y}\hat{P}_z - \hat{Z}\hat{P}_y,\hat{P}_x]\\[1mm]
&= \hat{Y}[\hat{P}_z,\hat{P}_x] + [\hat{Y},\hat{P}_x]\hat{P}_z - \hat{Z}[\hat{P}_y,\hat{P}_x] - [\hat{Z},\hat{P}_x]\hat{P}_y\\[1mm]
&= \hat{Y}\cdot 0 + 0\cdot\hat{P}_z - \hat{Z}\cdot 0 - 0\cdot\hat{P}_y\\[1mm]
&= 0.
\end{aligned}
\]

% (c) [\hat{L}_x, \hat{Y}]
\[
\begin{aligned}
[\hat{L}_x, \hat{Y}] &= [\hat{Y}\hat{P}_z - \hat{Z}\hat{P}_y, \hat{Y}]\\[1mm]
&= \hat{Y}[\hat{P}_z,\hat{Y}] + [\hat{Y},\hat{Y}]\hat{P}_z - \hat{Z}[\hat{P}_y,\hat{Y}] - [\hat{Z},\hat{Y}]\hat{P}_y\\[1mm]
&= \hat{Y}\cdot 0 + 0 - \hat{Z}\,(-i\hbar) - 0\\[1mm]
&= i\hbar\,\hat{Z}.
\end{aligned}
\]

% (d) [\hat{L}_x, \hat{P}_y]
\[
\begin{aligned}
[\hat{L}_x, \hat{P}_y] &= [\hat{Y}\hat{P}_z - \hat{Z}\hat{P}_y, \hat{P}_y]\\[1mm]
&= \hat{Y}[\hat{P}_z,\hat{P}_y] + [\hat{Y},\hat{P}_y]\hat{P}_z - \hat{Z}[\hat{P}_y,\hat{P}_y] - [\hat{Z},\hat{P}_y]\hat{P}_y\\[1mm]
&= \hat{Y}\cdot 0 + (i\hbar)\hat{P}_z - \hat{Z}\cdot 0 - 0\\[1mm]
&= i\hbar\,\hat{P}_z.
\end{aligned}
\]

% (e) [\hat{L}_x, \hat{P}^2] with \(\hat{P}^2=\hat{P}_x^2+\hat{P}_y^2+\hat{P}_z^2\)
\[
\begin{aligned}
[\hat{L}_x, \hat{P}^2] &= [\hat{L}_x, \hat{P}_x^2] + [\hat{L}_x, \hat{P}_y^2] + [\hat{L}_x, \hat{P}_z^2].
\end{aligned}
\]
Since we already found that \([\hat{L}_x, \hat{P}_x] = 0\), it follows that
\[
[\hat{L}_x, \hat{P}_x^2] = 0.
\]
For the remaining terms, we use the identity
\[
[A, B^2] = [A,B]\,B + B\,[A,B].
\]
Thus,
\[
\begin{aligned}
[\hat{L}_x, \hat{P}_y^2] &= [\hat{L}_x, \hat{P}_y]\,\hat{P}_y + \hat{P}_y\,[\hat{L}_x, \hat{P}_y]
= (i\hbar\,\hat{P}_z)\hat{P}_y + \hat{P}_y\,(i\hbar\,\hat{P}_z)\\[1mm]
&= i\hbar\,(\hat{P}_z\hat{P}_y+\hat{P}_y\hat{P}_z),
\end{aligned}
\]
and similarly,
\[
\begin{aligned}
[\hat{L}_x, \hat{P}_z^2] &= [\hat{L}_x, \hat{P}_z]\,\hat{P}_z + \hat{P}_z\,[\hat{L}_x, \hat{P}_z].
\end{aligned}
\]
We first compute \([\hat{L}_x, \hat{P}_z]\) using
\[
\begin{aligned}
[\hat{L}_x, \hat{P}_z] &= [\hat{Y}\hat{P}_z - \hat{Z}\hat{P}_y, \hat{P}_z]\\[1mm]
&= \hat{Y}[\hat{P}_z,\hat{P}_z] + [\hat{Y},\hat{P}_z]\hat{P}_z - \hat{Z}[\hat{P}_y,\hat{P}_z] - [\hat{Z},\hat{P}_z]\hat{P}_y\\[1mm]
&= 0 + 0 - 0 - (i\hbar)\hat{P}_y\\[1mm]
&= -i\hbar\,\hat{P}_y.
\end{aligned}
\]
Thus,
\[
\begin{aligned}
[\hat{L}_x, \hat{P}_z^2] &= (-i\hbar\,\hat{P}_y)\hat{P}_z + \hat{P}_z\,(-i\hbar\,\hat{P}_y)
= -i\hbar\,(\hat{P}_y\hat{P}_z+\hat{P}_z\hat{P}_y).
\end{aligned}
\]
Adding the two contributions we have:
\[
\begin{aligned}
[\hat{L}_x, \hat{P}_y^2] + [\hat{L}_x, \hat{P}_z^2] &= i\hbar\,(\hat{P}_z\hat{P}_y+\hat{P}_y\hat{P}_z) - i\hbar\,(\hat{P}_y\hat{P}_z+\hat{P}_z\hat{P}_y)\\[1mm]
&= 0.
\end{aligned}
\]
Therefore,
\[
[\hat{L}_x, \hat{P}^2] = 0 + 0 + 0 = 0.
\]

\subsubsection{Solution 2}

Recall that the x-component of the angular momentum operator is given by 
\[
\hat{L}_x = \hat{Y}\hat{P}_z - \hat{Z}\hat{P}_y.
\]
We also use the canonical commutation relations 
\[
[\hat{X}_i, \hat{P}_j] = i\hbar\,\delta_{ij}, \quad [\hat{X}_i, \hat{X}_j] = 0, \quad [\hat{P}_i, \hat{P}_j] = 0.
\]

(a) To compute 
\[
[\hat{L}_x, \hat{X}] = [\hat{Y}\hat{P}_z - \hat{Z}\hat{P}_y, \hat{X}],
\]
we write
\[
\begin{aligned}
[\hat{L}_x, \hat{X}] &= \hat{Y}[\hat{P}_z,\hat{X}] + [\hat{Y},\hat{X}]\hat{P}_z - \hat{Z}[\hat{P}_y,\hat{X}] - [\hat{Z},\hat{X}]\hat{P}_y \\
&= \hat{Y}\cdot 0 + 0\cdot\hat{P}_z - \hat{Z}\cdot 0 - 0\cdot\hat{P}_y \quad (\text{since } [\hat{P}_z,\hat{X}]=[\hat{P}_y,\hat{X}]=0 \text{ and } [\hat{Y},\hat{X}]=[\hat{Z},\hat{X}]=0)\\
&= 0.
\end{aligned}
\]

(b) Similarly, for 
\[
[\hat{L}_x, \hat{P}_x] = [\hat{Y}\hat{P}_z - \hat{Z}\hat{P}_y, \hat{P}_x],
\]
we have
\[
\begin{aligned}
[\hat{L}_x, \hat{P}_x] &= \hat{Y}[\hat{P}_z,\hat{P}_x] + [\hat{Y},\hat{P}_x]\hat{P}_z - \hat{Z}[\hat{P}_y,\hat{P}_x] - [\hat{Z},\hat{P}_x]\hat{P}_y \\
&= \hat{Y}\cdot 0 + 0\cdot\hat{P}_z - \hat{Z}\cdot 0 - 0\cdot\hat{P}_y \quad (\text{since } [\hat{P}_z,\hat{P}_x]=[\hat{P}_y,\hat{P}_x]=0 \text{ and } [\hat{Y},\hat{P}_x]=[\hat{Z},\hat{P}_x]=0)\\
&= 0.
\end{aligned}
\]

(c) For the commutator 
\[
[\hat{L}_x, \hat{Y}] = [\hat{Y}\hat{P}_z - \hat{Z}\hat{P}_y, \hat{Y}],
\]
we compute
\[
\begin{aligned}
[\hat{L}_x, \hat{Y}] &= \hat{Y}[\hat{P}_z,\hat{Y}] + [\hat{Y},\hat{Y}]\hat{P}_z - \hat{Z}[\hat{P}_y,\hat{Y}] - [\hat{Z},\hat{Y}]\hat{P}_y\\[1mm]
&= -\hat{Z}[\hat{P}_y,\hat{Y}] \quad (\text{since } [\hat{P}_z,\hat{Y}]=0 \text{ and } [\hat{Y},\hat{Y}]=[\hat{Z},\hat{Y}]=0)\\[1mm]
&= -\hat{Z}(-i\hbar) \quad (\text{because } [\hat{Y},\hat{P}_y]=i\hbar \text{ so } [\hat{P}_y,\hat{Y}]=-i\hbar)\\[1mm]
&= i\hbar\,\hat{Z}.
\end{aligned}
\]

(d) For 
\[
[\hat{L}_x, \hat{P}_y] = [\hat{Y}\hat{P}_z - \hat{Z}\hat{P}_y, \hat{P}_y],
\]
we obtain
\[
\begin{aligned}
[\hat{L}_x, \hat{P}_y] &= \hat{Y}[\hat{P}_z,\hat{P}_y] + [\hat{Y},\hat{P}_y]\hat{P}_z - \hat{Z}[\hat{P}_y,\hat{P}_y] - [\hat{Z},\hat{P}_y]\hat{P}_y\\[1mm]
&= [\hat{Y},\hat{P}_y]\hat{P}_z \quad (\text{since } [\hat{P}_z,\hat{P}_y]=0 \text{ and } [\hat{Z},\hat{P}_y]=[\hat{P}_y,\hat{P}_y]=0)\\[1mm]
&= i\hbar\,\hat{P}_z.
\end{aligned}
\]

(e) Finally, to show that 
\[
[\hat{L}_x, \hat{P}^2] = 0,
\]
where 
\[
\hat{P}^2 = \hat{P}_x^2 + \hat{P}_y^2 + \hat{P}_z^2,
\]
we write
\[
[\hat{L}_x, \hat{P}^2] = [\hat{L}_x, \hat{P}_x^2] + [\hat{L}_x, \hat{P}_y^2] + [\hat{L}_x, \hat{P}_z^2].
\]
Since part (b) showed that $[\hat{L}_x,\hat{P}_x] = 0$, it follows that
\[
[\hat{L}_x,\hat{P}_x^2] = 0.
\]
For the remaining terms, we use the identity 
\[
[A, B^2] = [A,B]B + B[A,B].
\]
Thus,
\[
\begin{aligned}
[\hat{L}_x, \hat{P}_y^2] &= [\hat{L}_x,\hat{P}_y]\,\hat{P}_y + \hat{P}_y\,[\hat{L}_x,\hat{P}_y] 
= (i\hbar\,\hat{P}_z)\hat{P}_y + \hat{P}_y(i\hbar\,\hat{P}_z)
= i\hbar\,(\hat{P}_z\hat{P}_y + \hat{P}_y\hat{P}_z),\\[1mm]
[\hat{L}_x, \hat{P}_z^2] &= [\hat{L}_x,\hat{P}_z]\,\hat{P}_z + \hat{P}_z\,[\hat{L}_x,\hat{P}_z].
\end{aligned}
\]
To evaluate $[\hat{L}_x,\hat{P}_z]$, we compute
\[
\begin{aligned}
[\hat{L}_x,\hat{P}_z] &= [\hat{Y}\hat{P}_z - \hat{Z}\hat{P}_y, \hat{P}_z] \\
&= \hat{Y}[\hat{P}_z,\hat{P}_z] + [\hat{Y},\hat{P}_z]\hat{P}_z - \hat{Z}[\hat{P}_y,\hat{P}_z] - [\hat{Z},\hat{P}_z]\hat{P}_y\\[1mm]
&= -[\hat{Z},\hat{P}_z]\hat{P}_y \quad (\text{since } [\hat{P}_z,\hat{P}_z]=0 \text{ and } [\hat{Y},\hat{P}_z]=[\hat{P}_y,\hat{P}_z]=0)\\[1mm]
&= -i\hbar\,\hat{P}_y \quad (\text{because } [\hat{Z},\hat{P}_z]=i\hbar).
\end{aligned}
\]
Thus,
\[
[\hat{L}_x, \hat{P}_z^2] = (-i\hbar\,\hat{P}_y)\hat{P}_z + \hat{P}_z(-i\hbar\,\hat{P}_y)
= -i\hbar\,(\hat{P}_y\hat{P}_z + \hat{P}_z\hat{P}_y).
\]
Adding the contributions from $\hat{P}_y^2$ and $\hat{P}_z^2$, we obtain
\[
[\hat{L}_x,\hat{P}_y^2] + [\hat{L}_x,\hat{P}_z^2] = i\hbar\,(\hat{P}_z\hat{P}_y + \hat{P}_y\hat{P}_z) - i\hbar\,(\hat{P}_y\hat{P}_z + \hat{P}_z\hat{P}_y) = 0.
\]
Therefore,
\[
[\hat{L}_x, \hat{P}^2] = 0.
\]

\subsubsection{Solution 3}
% We start by recalling that the x‐component of the angular momentum operator is
% \(\hat{L}_x = \hat{Y}\hat{P}_z - \hat{Z}\hat{P}_y\).

a) [3 marks]
We wish to show that
\[
\left[\hat{L}_x, \hat{X}\right] = 0.
\]
Write
\[
\left[\hat{L}_x, \hat{X}\right] = \left[\hat{Y}\hat{P}_z - \hat{Z}\hat{P}_y,\, \hat{X}\right] 
= \left[\hat{Y}\hat{P}_z,\, \hat{X}\right] - \left[\hat{Z}\hat{P}_y,\, \hat{X}\right].
\]
Since the position operators commute with each other and with the momentum components corresponding to different directions (e.g. \([\hat{X}, \hat{P}_z] = 0\) and \([\hat{X}, \hat{P}_y] = 0\)), it follows that
\[
\left[\hat{Y}\hat{P}_z,\, \hat{X}\right] = \hat{Y}\left[\hat{P}_z,\, \hat{X}\right] + \left[\hat{Y},\, \hat{X}\right]\hat{P}_z = 0,
\]
and similarly,
\[
\left[\hat{Z}\hat{P}_y,\, \hat{X}\right] = 0.
\]
Thus, 
\[
\left[\hat{L}_x, \hat{X}\right] = 0.
\]

b) [3 marks] 
We now show that
\[
\left[\hat{L}_x, \hat{P}_x\right] = 0.
\]
Since \(\hat{P}_x\) commutes with \(\hat{P}_y\) and \(\hat{P}_z\), and it also commutes with the position operators \(\hat{Y}\) and \(\hat{Z}\) (as \([\hat{P}_x, \hat{Y}] = [\hat{P}_x, \hat{Z}] = 0\)), we have
\[
\left[\hat{L}_x, \hat{P}_x\right] = \left[\hat{Y}\hat{P}_z - \hat{Z}\hat{P}_y,\, \hat{P}_x\right] = 0.
\]

c) [4 marks]
Next, we derive
\[
\left[\hat{L}_x, \hat{Y}\right] = i\hbar \hat{Z}.
\]
Starting with
\[
\left[\hat{L}_x, \hat{Y}\right] = \left[\hat{Y}\hat{P}_z - \hat{Z}\hat{P}_y,\, \hat{Y}\right] 
= \left[\hat{Y}\hat{P}_z,\, \hat{Y}\right] - \left[\hat{Z}\hat{P}_y,\, \hat{Y}\right],
\]
note that \(\hat{Y}\) commutes with itself and with \(\hat{P}_z\) (since \([\hat{P}_z, \hat{Y}]=0\)), so
\[
\left[\hat{Y}\hat{P}_z,\, \hat{Y}\right] = 0.
\]
For the second term, use the product rule:
\[
\left[\hat{Z}\hat{P}_y,\, \hat{Y}\right] = \hat{Z}\left[\hat{P}_y,\, \hat{Y}\right] + \left[\hat{Z},\, \hat{Y}\right]\hat{P}_y.
\]
Since \([\hat{Z},\, \hat{Y}]=0\) and the canonical commutator \([\hat{Y},\, \hat{P}_y] = i\hbar\) (so \([\hat{P}_y,\, \hat{Y}] = -i\hbar\)), it follows that
\[
\left[\hat{Z}\hat{P}_y,\, \hat{Y}\right] = \hat{Z}(-i\hbar) = -i\hbar\,\hat{Z}.
\]
Therefore,
\[
\left[\hat{L}_x, \hat{Y}\right] = 0 - \left(-i\hbar\,\hat{Z}\right) = i\hbar\,\hat{Z}.
\]

d) [4 marks]
We now compute
\[
\left[\hat{L}_x, \hat{P}_y\right] = i\hbar \hat{P}_z.
\]
Write
\[
\left[\hat{L}_x, \hat{P}_y\right] = \left[\hat{Y}\hat{P}_z - \hat{Z}\hat{P}_y,\, \hat{P}_y\right] 
= \left[\hat{Y}\hat{P}_z,\, \hat{P}_y\right] - \left[\hat{Z}\hat{P}_y,\, \hat{P}_y\right].
\]
For the first term, applying the product rule gives
\[
\left[\hat{Y}\hat{P}_z,\, \hat{P}_y\right] = \hat{Y}\left[\hat{P}_z,\, \hat{P}_y\right] + \left[\hat{Y},\, \hat{P}_y\right]\hat{P}_z.
\]
Since \([\hat{P}_z,\, \hat{P}_y]=0\) and \([\hat{Y},\, \hat{P}_y] = i\hbar\), we obtain
\[
\left[\hat{Y}\hat{P}_z,\, \hat{P}_y\right] = i\hbar\,\hat{P}_z.
\]
For the second term,
\[
\left[\hat{Z}\hat{P}_y,\, \hat{P}_y\right] = \hat{Z}\left[\hat{P}_y,\, \hat{P}_y\right] + \left[\hat{Z},\, \hat{P}_y\right]\hat{P}_y = 0,
\]
since \([\hat{P}_y,\, \hat{P}_y]=0\) and \([\hat{Z},\, \hat{P}_y]=0\). Thus,
\[
\left[\hat{L}_x, \hat{P}_y\right] = i\hbar\,\hat{P}_z.
\]

e) [6 marks]
Finally, we show that
\[
\left[\hat{L}_x, \hat{P}^2\right] = 0,
\]
where
\[
\hat{P}^2 = \hat{P}_x^2 + \hat{P}_y^2 + \hat{P}_z^2.
\]
Thus,
\[
\left[\hat{L}_x, \hat{P}^2\right] = \left[\hat{L}_x, \hat{P}_x^2\right] + \left[\hat{L}_x, \hat{P}_y^2\right] + \left[\hat{L}_x, \hat{P}_z^2\right].
\]
From part (b), we have \(\left[\hat{L}_x, \hat{P}_x\right]=0\), hence
\[
\left[\hat{L}_x, \hat{P}_x^2\right] = 0.
\]
Now, consider \(\left[\hat{L}_x, \hat{P}_y^2\right]\). Using the product rule,
\[
\left[\hat{L}_x, \hat{P}_y^2\right] = \hat{P}_y\left[\hat{L}_x, \hat{P}_y\right] + \left[\hat{L}_x, \hat{P}_y\right]\hat{P}_y.
\]
From part (d) we know \(\left[\hat{L}_x, \hat{P}_y\right] = i\hbar\,\hat{P}_z\), so
\[
\left[\hat{L}_x, \hat{P}_y^2\right] = i\hbar\,\left(\hat{P}_y\hat{P}_z + \hat{P}_z\hat{P}_y\right).
\]
Next, we calculate \(\left[\hat{L}_x, \hat{P}_z^2\right]\) similarly:
\[
\left[\hat{L}_x, \hat{P}_z^2\right] = \hat{P}_z\left[\hat{L}_x, \hat{P}_z\right] + \left[\hat{L}_x, \hat{P}_z\right]\hat{P}_z.
\]
To find \(\left[\hat{L}_x, \hat{P}_z\right]\), we use
\[
\left[\hat{L}_x, \hat{P}_z\right] = \left[\hat{Y}\hat{P}_z - \hat{Z}\hat{P}_y,\, \hat{P}_z\right].
\]
For the first term,
\[
\left[\hat{Y}\hat{P}_z,\, \hat{P}_z\right] = \hat{Y}\left[\hat{P}_z,\, \hat{P}_z\right] + \left[\hat{Y},\, \hat{P}_z\right]\hat{P}_z = 0.
\]
For the second term,
\[
\left[\hat{Z}\hat{P}_y,\, \hat{P}_z\right] = \hat{Z}\left[\hat{P}_y,\, \hat{P}_z\right] + \left[\hat{Z},\, \hat{P}_z\right]\hat{P}_y.
\]
Since \([\hat{P}_y,\, \hat{P}_z]=0\) and \([\hat{Z},\, \hat{P}_z] = i\hbar\), we obtain
\[
\left[\hat{Z}\hat{P}_y,\, \hat{P}_z\right] = i\hbar\,\hat{P}_y.
\]
Thus,
\[
\left[\hat{L}_x, \hat{P}_z\right] = -i\hbar\,\hat{P}_y.
\]
Substituting back,
\[
\left[\hat{L}_x, \hat{P}_z^2\right] = \hat{P}_z(-i\hbar\,\hat{P}_y) + (-i\hbar\,\hat{P}_y)\hat{P}_z = -i\hbar\,\left(\hat{P}_z\hat{P}_y + \hat{P}_y\hat{P}_z\right).
\]
Adding the contributions from \(\hat{P}_y^2\) and \(\hat{P}_z^2\):
\[
\left[\hat{L}_x, \hat{P}_y^2\right] + \left[\hat{L}_x, \hat{P}_z^2\right] = i\hbar\,\left(\hat{P}_y\hat{P}_z+\hat{P}_z\hat{P}_y\right) - i\hbar\,\left(\hat{P}_z\hat{P}_y+\hat{P}_y\hat{P}_z\right) = 0.
\]
Thus,
\[
\left[\hat{L}_x, \hat{P}^2\right] = 0.
\]

% End of solution.


 \subsection{Question 5}
An infinite square well of length L can be defined mathematically by the potential,
\[
V(x) = 
\begin{cases} 
0 & \text{for } 0 \leq x \leq L, \\
+\infty & \text{otherwise},
\end{cases}
\]

A particle of mass \(m\) in the well has eigenfunctions
\[
\phi_n(x) = \sqrt{\frac{2}{L}} \sin\left(\frac{n \pi x}{L}\right)
\]
with corresponding energy eigenvalues
\[
E_n = \frac{n^2 \hbar^2 \pi^2}{2mL^2}.
\]

(a) Calculate the first-order correction to the ground state energy if the system is perturbed by
\[
\hat{H}' = V_0 \sin\left(\frac{2\pi x}{L}\right).
\]

(b) Calculate and derive an expression for the first-order correction to all energy eigenvalues given that the system is perturbed by
\[
\hat{H}' = L \alpha \delta\left(x - \frac{L}{2}\right),
\]
where \(\alpha\) is a constant and \(\delta\) is the Dirac delta function.

\paragraph{Prompt: \\} 
Answer the given question as a top physics student, giving step-by-step reasoning. Format your answer as LaTeX code without using sub-sections. Available marks: a) [4 marks], b) [3 marks].


\subsubsection{Solution 1}
% (a) First-order correction for the ground state
The first-order correction is given by
\[
E_1^{(1)} = \langle \phi_1 | \hat{H}' | \phi_1 \rangle = V_0 \int_0^L |\phi_1(x)|^2 \sin\!\left(\frac{2\pi x}{L}\right) dx,
\]
with
\[
\phi_1(x) = \sqrt{\frac{2}{L}} \sin\!\left(\frac{\pi x}{L}\right).
\]
Thus,
\[
E_1^{(1)} = V_0 \frac{2}{L} \int_0^L \sin^2\!\left(\frac{\pi x}{L}\right) \sin\!\left(\frac{2\pi x}{L}\right) dx.
\]
Using the identity
\[
\sin^2\!\left(\frac{\pi x}{L}\right) = \frac{1}{2}\Bigl[1 - \cos\!\left(\frac{2\pi x}{L}\right)\Bigr],
\]
we have
\[
E_1^{(1)} = V_0 \frac{2}{L} \cdot \frac{1}{2} \left[ \int_0^L \sin\!\left(\frac{2\pi x}{L}\right)dx - \int_0^L \cos\!\left(\frac{2\pi x}{L}\right) \sin\!\left(\frac{2\pi x}{L}\right)dx \right].
\]
Both integrals vanish because
\[
\int_0^L \sin\!\left(\frac{2\pi x}{L}\right)dx = 0,
\]
and writing
\[
\cos\!\left(\frac{2\pi x}{L}\right) \sin\!\left(\frac{2\pi x}{L}\right) = \frac{1}{2}\sin\!\left(\frac{4\pi x}{L}\right),
\]
leads to
\[
\int_0^L \sin\!\left(\frac{4\pi x}{L}\right)dx = 0.
\]
Thus, the first-order energy correction for the ground state is
\[
E_1^{(1)} = 0.
\]

% (b) First-order correction for all energy eigenvalues
For the perturbation
\[
\hat{H}' = L \alpha \, \delta\!\left(x - \frac{L}{2}\right),
\]
the first-order correction for the \(n^{\text{th}}\) state is
\[
E_n^{(1)} = \langle \phi_n | \hat{H}' | \phi_n \rangle = L\alpha \left|\phi_n\!\left(\frac{L}{2}\right)\right|^2.
\]
Since
\[
\phi_n\!\left(\frac{L}{2}\right) = \sqrt{\frac{2}{L}} \sin\!\left(\frac{n\pi}{2}\right),
\]
we obtain
\[
\left|\phi_n\!\left(\frac{L}{2}\right)\right|^2 = \frac{2}{L} \sin^2\!\left(\frac{n\pi}{2}\right).
\]
Therefore,
\[
E_n^{(1)} = L\alpha \cdot \frac{2}{L} \sin^2\!\left(\frac{n\pi}{2}\right) = 2\alpha \sin^2\!\left(\frac{n\pi}{2}\right).
\]
Recognizing that \(\sin^2\!\left(\frac{n\pi}{2}\right) = 1\) for odd \(n\) and \(0\) for even \(n\), the result can be written as
\[
E_n^{(1)} =
\begin{cases}
2\alpha, & \text{if } n \text{ is odd}, \\
0, & \text{if } n \text{ is even}.
\end{cases}
\]

\subsubsection{Solution 2}
% (a) First-order correction for the ground state:
% The first-order energy correction is given by
\[
E_1^{(1)} = \langle \phi_1 | \hat{H}' | \phi_1 \rangle 
= \int_0^L \phi_1^*(x)\, V_0 \sin\left(\frac{2\pi x}{L}\right)\, \phi_1(x)\, dx,
\]
where 
\[
\phi_1(x)=\sqrt{\frac{2}{L}} \sin\left(\frac{\pi x}{L}\right).
\]
Substituting, we have
\[
E_1^{(1)} = V_0 \frac{2}{L}\int_0^L \sin^2\left(\frac{\pi x}{L}\right) \sin\left(\frac{2\pi x}{L}\right)\, dx.
\]
Using the identity
\[
\sin^2\left(\frac{\pi x}{L}\right)=\frac{1}{2}\left[1-\cos\left(\frac{2\pi x}{L}\right)\right],
\]
this becomes
\[
E_1^{(1)} = \frac{V_0}{L}\int_0^L \left\{ \sin\left(\frac{2\pi x}{L}\right)-\sin\left(\frac{2\pi x}{L}\right)\cos\left(\frac{2\pi x}{L}\right) \right\}dx.
\]
Noting that
\[
\sin\left(\frac{2\pi x}{L}\right)\cos\left(\frac{2\pi x}{L}\right) = \frac{1}{2}\sin\left(\frac{4\pi x}{L}\right),
\]
we rewrite the integral as
\[
E_1^{(1)} = \frac{V_0}{L}\left[\int_0^L \sin\left(\frac{2\pi x}{L}\right)dx - \frac{1}{2}\int_0^L \sin\left(\frac{4\pi x}{L}\right)dx\right].
\]
Both integrals evaluate to zero because
\[
\int_0^L \sin\left(\frac{2\pi x}{L}\right)dx = -\frac{L}{2\pi}\Big[\cos(2\pi)-\cos(0)\Big] = 0,
\]
and similarly,
\[
\int_0^L \sin\left(\frac{4\pi x}{L}\right)dx = 0.
\]
Thus,
\[
E_1^{(1)} = 0.
\]

% (b) First-order correction for the delta function perturbation:
% For the perturbation 
\[
\hat{H}'= L\alpha\, \delta\left(x-\frac{L}{2}\right),
\]
the first-order correction to the energy eigenvalue is
\[
E_n^{(1)} = \langle \phi_n|\hat{H}'|\phi_n\rangle 
= L\alpha\, \left|\phi_n\left(\frac{L}{2}\right)\right|^2,
\]
with 
\[
\phi_n(x)=\sqrt{\frac{2}{L}} \sin\left(\frac{n\pi x}{L}\right).
\]
Evaluating at \(x=\frac{L}{2}\) gives
\[
\phi_n\left(\frac{L}{2}\right) = \sqrt{\frac{2}{L}} \sin\left(\frac{n\pi}{2}\right).
\]
Hence,
\[
E_n^{(1)} = L\alpha \left(\frac{2}{L}\right) \sin^2\left(\frac{n\pi}{2}\right)
= 2\alpha\, \sin^2\left(\frac{n\pi}{2}\right).
\]
Since
\[
\sin^2\left(\frac{n\pi}{2}\right)= 
\begin{cases}
1, & \text{if } n \text{ is odd},\\[1mm]
0, & \text{if } n \text{ is even},
\end{cases}
\]
the first-order correction is
\[
E_n^{(1)} =
\begin{cases}
2\alpha, & \text{for odd } n,\\[1mm]
0, & \text{for even } n.
\end{cases}
\]

\subsubsection{Solution 3}
\textbf{(a)} \\
The first-order correction to the ground state energy is given by
\[
E_1^{(1)} = \langle \phi_1 | \hat{H}' | \phi_1 \rangle 
= V_0 \int_0^L |\phi_1(x)|^2 \sin\left(\frac{2\pi x}{L}\right) dx,
\]
where 
\[
\phi_1(x) = \sqrt{\frac{2}{L}} \sin\left(\frac{\pi x}{L}\right).
\]
Substituting, we have
\[
E_1^{(1)} = V_0 \left(\frac{2}{L}\right) \int_0^L \sin^2\left(\frac{\pi x}{L}\right) \sin\left(\frac{2\pi x}{L}\right) dx.
\]
Using the identity
\[
\sin^2\left(\frac{\pi x}{L}\right) = \frac{1}{2}\left[1-\cos\left(\frac{2\pi x}{L}\right)\right],
\]
the integral becomes
\[
I = \frac{1}{2} \int_0^L \left[1-\cos\left(\frac{2\pi x}{L}\right)\right] \sin\left(\frac{2\pi x}{L}\right) dx.
\]
Splitting the integral,
\[
I = \frac{1}{2}\left[\int_0^L \sin\left(\frac{2\pi x}{L}\right)dx - \int_0^L \cos\left(\frac{2\pi x}{L}\right)\sin\left(\frac{2\pi x}{L}\right)dx\right].
\]
The first term evaluates as
\[
\int_0^L \sin\left(\frac{2\pi x}{L}\right)dx = -\frac{L}{2\pi} \left[\cos\left(\frac{2\pi x}{L}\right)\right]_0^L = -\frac{L}{2\pi}[\cos(2\pi)-\cos(0)] = 0.
\]
For the second term, note that
\[
\cos\left(\frac{2\pi x}{L}\right)\sin\left(\frac{2\pi x}{L}\right) = \frac{1}{2}\sin\left(\frac{4\pi x}{L}\right),
\]
so
\[
\int_0^L \cos\left(\frac{2\pi x}{L}\right)\sin\left(\frac{2\pi x}{L}\right)dx = \frac{1}{2}\int_0^L \sin\left(\frac{4\pi x}{L}\right)dx = \frac{1}{2}\left[-\frac{L}{4\pi}\cos\left(\frac{4\pi x}{L}\right)\right]_0^L = 0.
\]
Thus, 
\[
I = \frac{1}{2}(0-0)=0,
\]
and hence,
\[
E_1^{(1)} = V_0 \frac{2}{L} \times 0 = 0.
\]

\textbf{(b)} \\
For the perturbation 
\[
\hat{H}' = L \alpha \, \delta\left(x-\frac{L}{2}\right),
\]
the first-order energy correction for the \(n\)th state is given by
\[
E_n^{(1)} = \langle \phi_n | \hat{H}' | \phi_n \rangle = L \alpha \, |\phi_n(L/2)|^2.
\]
Since
\[
\phi_n(x) = \sqrt{\frac{2}{L}} \sin\left(\frac{n\pi x}{L}\right),
\]
we have
\[
\phi_n\left(\frac{L}{2}\right) = \sqrt{\frac{2}{L}} \sin\left(\frac{n\pi}{2}\right),
\]
and therefore,
\[
|\phi_n(L/2)|^2 = \frac{2}{L}\sin^2\left(\frac{n\pi}{2}\right).
\]
Substituting back,
\[
E_n^{(1)} = L \alpha \cdot \frac{2}{L}\sin^2\left(\frac{n\pi}{2}\right) = 2\alpha\, \sin^2\left(\frac{n\pi}{2}\right).
\]
This result shows that for even \(n\) (i.e. when \(n\pi/2\) is an integer multiple of \(\pi\)), \(\sin\left(\frac{n\pi}{2}\right)=0\) and \(E_n^{(1)}=0\); for odd \(n\), \(\sin^2\left(\frac{n\pi}{2}\right)=1\) and \(E_n^{(1)}=2\alpha\).


\subsection{Question 6}

Let \(\hat{L}\) and \(\hat{S}\) be the angular momentum and spin angular momentum quantum operators.

\textbf{(a)} Given that \(\hat{J} = \hat{L} + \hat{S}\), show that
\[
\hat{J}^2 = \hat{L}^2 + \hat{S}^2 + \hat{L}_+ \hat{S}_- + \hat{L}_- \hat{S}_+ + 2 \hat{L}_z \hat{S}_z
\]
Where you may use \(\hat{L}_\pm = \hat{L}_x \pm i \hat{L}_y\) and \(\hat{S}_\pm = \hat{S}_x \pm i \hat{S}_y\).

\textbf{(b)} Consider the state \(|l, m; s, m_s\rangle = |l, m\rangle |s, m_s\rangle\). Here \(l\) is a quantum number of \(\hat{L}^2\), \(m\) is a quantum number of \(\hat{L}_z\), \(s\) is a quantum number of \(\hat{S}^2\), and \(m_s\) is a quantum number of \(\hat{S}_z\). Show that the state \(|l, -l; s, -s\rangle\) is an eigenvector of \(\hat{J}^2\) and find the corresponding eigenvalue.

\textbf{(c)} Now consider an operator \(\hat{O} = a \hat{L}^2 + b \hat{S}_+ \hat{L}_z\), where \(a, b\) are constants. You are given the following result:
\[
\hat{S}_+ |s, m_s\rangle = \hbar \sqrt{s(s+1) - m_s(m_s + 1)} |s, m_s + 1\rangle
\]
Find the matrix representation of \(\hat{O}\) for a chosen basis of kets \(|l = 1, m; s = \frac{1}{2}, m_s\rangle\).

\paragraph{Prompt: \\} 
Answer the given question as a top physics student, giving step-by-step reasoning. Format your answer as LaTeX code without using sub-sections. Available marks: a) [4 marks], b) [4 marks], c) [9 marks].


\subsubsection{Solution 1}
% (a) We start by writing
\[
\hat{J} = \hat{L} + \hat{S} \quad \Longrightarrow \quad \hat{J}^2 = (\hat{L}+\hat{S})^2 = \hat{L}^2 + \hat{S}^2 + 2\,\hat{L}\cdot\hat{S}\,.
\]
Writing the dot product in Cartesian components,
\[
\hat{L}\cdot\hat{S} = \hat{L}_x\hat{S}_x + \hat{L}_y\hat{S}_y + \hat{L}_z\hat{S}_z\,.
\]
Expressing the \(x\) and \(y\) components in terms of the ladder operators, recall that
\[
\hat{L}_\pm = \hat{L}_x \pm i \hat{L}_y \quad \text{and} \quad \hat{S}_\pm = \hat{S}_x \pm i \hat{S}_y\,,
\]
so that one may show
\[
\hat{L}_x\hat{S}_x + \hat{L}_y\hat{S}_y = \frac{1}{2}\Bigl(\hat{L}_+\hat{S}_- + \hat{L}_-\hat{S}_+\Bigr)\,.
\]
Thus, substituting back we obtain
\[
\hat{J}^2 = \hat{L}^2 + \hat{S}^2 + \hat{L}_+\hat{S}_- + \hat{L}_-\hat{S}_+ + 2\,\hat{L}_z\hat{S}_z\,.
\]

% (b) Consider now the state \(|l,-l; s,-s\rangle\). Note that the ladder operators act as:
% \(\hat{L}_+|l,-l\rangle\) is nonzero in general but \(\hat{L}_-|l,-l\rangle=0\), while
% \(\hat{S}_-|s,-s\rangle=0\) and \(\hat{S}_+|s,-s\rangle\) raises the spin projection.
% However, in the product state \(|l,-l; s,-s\rangle\) the cross terms in \(\hat{J}^2\) vanish.
\[
\hat{J}^2|l,-l;s,-s\rangle = \Bigl[\hat{L}^2 + \hat{S}^2 + 2\,\hat{L}_z\hat{S}_z\Bigr]|l,-l;s,-s\rangle\,.
\]
Since
\[
\hat{L}^2|l,-l\rangle = \hbar^2 l(l+1)|l,-l\rangle,\quad \hat{S}^2|s,-s\rangle = \hbar^2 s(s+1)|s,-s\rangle,
\]
and
\[
\hat{L}_z|l,-l\rangle = -\hbar\,l\,|l,-l\rangle,\quad \hat{S}_z|s,-s\rangle = -\hbar\,s\,|s,-s\rangle\,,
\]
it follows that
\[
2\,\hat{L}_z\hat{S}_z|l,-l;s,-s\rangle = 2\,(-\hbar l)(-\hbar s)|l,-l;s,-s\rangle = 2\hbar^2\,l s\,|l,-l;s,-s\rangle\,.
\]
Thus,
\[
\hat{J}^2|l,-l;s,-s\rangle = \hbar^2\Bigl[l(l+1) + s(s+1) + 2ls\Bigr] |l,-l;s,-s\rangle\,.
\]
Recognizing that
\[
l(l+1) + s(s+1) + 2ls = (l+s)(l+s+1)\,,
\]
we conclude that
\[
\hat{J}^2|l,-l;s,-s\rangle = \hbar^2 (l+s)(l+s+1)|l,-l;s,-s\rangle\,.
\]

% (c) For the operator \(\hat{O} = a\,\hat{L}^2 + b\,\hat{S}_+\hat{L}_z\), we choose the basis
\[
\Bigl\{\,|1,1;\tfrac{1}{2},\tfrac{1}{2}\rangle,\; |1,1;\tfrac{1}{2},-\tfrac{1}{2}\rangle,\; |1,0;\tfrac{1}{2},\tfrac{1}{2}\rangle,\; |1,0;\tfrac{1}{2},-\tfrac{1}{2}\rangle,\; |1,-1;\tfrac{1}{2},\tfrac{1}{2}\rangle,\; |1,-1;\tfrac{1}{2},-\tfrac{1}{2}\rangle\,\Bigr\}\,.
\]
Since for \(l=1\),
\[
\hat{L}^2|1,m\rangle = 2\hbar^2|1,m\rangle,
\]
the first term in \(\hat{O}\) contributes \(2a\hbar^2\) times the identity on each basis ket.

Next, consider the action of the second term. The operator \(\hat{L}_z\) acts as
\[
\hat{L}_z|1,m\rangle = \hbar\,m\,|1,m\rangle\,,
\]
and for \(s=\tfrac{1}{2}\) the spin raising operator gives
\[
\hat{S}_+|\tfrac{1}{2},m_s\rangle =
\begin{cases}
\hbar\,|\tfrac{1}{2},\tfrac{1}{2}\rangle, & m_s=-\tfrac{1}{2},\\[1mm]
0, & m_s=\tfrac{1}{2}\,.
\end{cases}
\]
Thus, for any state \(|1,m; \tfrac{1}{2},m_s\rangle\) we have
\[
\hat{S}_+\hat{L}_z|1,m; \tfrac{1}{2},m_s\rangle = \hbar\,m\,\hat{S}_+|1,m; \tfrac{1}{2},m_s\rangle = 
\begin{cases}
\hbar^2\,m\,|1,m; \tfrac{1}{2},\tfrac{1}{2}\rangle, & m_s=-\tfrac{1}{2},\\[1mm]
0, & m_s=\tfrac{1}{2}\,.
\end{cases}
\]
In other words, the \(b\)-term only connects states with spin projection \(m_s=-\tfrac{1}{2}\) to those with \(m_s=\tfrac{1}{2}\) for the same orbital quantum number \(m\).

Collecting the results, the operator \(\hat{O}\) acts as
\[
\hat{O}|1,m;\tfrac{1}{2},\tfrac{1}{2}\rangle = 2a\hbar^2\,|1,m;\tfrac{1}{2},\tfrac{1}{2}\rangle,
\]
\[
\hat{O}|1,m;\tfrac{1}{2},-\tfrac{1}{2}\rangle = 2a\hbar^2\,|1,m;\tfrac{1}{2},-\tfrac{1}{2}\rangle + b\,\hbar^2\,m\,|1,m;\tfrac{1}{2},\tfrac{1}{2}\rangle\,.
\]
Choosing the ordered basis
\[
\begin{array}{lll}
\text{(1)}\; |1,1;\tfrac{1}{2},\tfrac{1}{2}\rangle, & \text{(2)}\; |1,1;\tfrac{1}{2},-\tfrac{1}{2}\rangle,\\[1mm]
\text{(3)}\; |1,0;\tfrac{1}{2},\tfrac{1}{2}\rangle, & \text{(4)}\; |1,0;\tfrac{1}{2},-\tfrac{1}{2}\rangle,\\[1mm]
\text{(5)}\; |1,-1;\tfrac{1}{2},\tfrac{1}{2}\rangle, & \text{(6)}\; |1,-1;\tfrac{1}{2},-\tfrac{1}{2}\rangle,
\end{array}
\]
we can write the matrix representation of \(\hat{O}\) as follows. For \(m=1\), note that
\[
\hat{O}|1,1;\tfrac{1}{2},-\tfrac{1}{2}\rangle = 2a\hbar^2\,|1,1;\tfrac{1}{2},-\tfrac{1}{2}\rangle + b\hbar^2 (1)\,|1,1;\tfrac{1}{2},\tfrac{1}{2}\rangle\,,
\]
and for \(m=-1\),
\[
\hat{O}|1,-1;\tfrac{1}{2},-\tfrac{1}{2}\rangle = 2a\hbar^2\,|1,-1;\tfrac{1}{2},-\tfrac{1}{2}\rangle + b\hbar^2 (-1)\,|1,-1;\tfrac{1}{2},\tfrac{1}{2}\rangle\,.
\]
There is no contribution from the \(b\)-term for states with \(m=0\) or when \(m_s=\tfrac{1}{2}\).

Thus, the matrix representation of \(\hat{O}\) in our 6-dimensional basis is
\[
\hat{O} =
\begin{pmatrix}
2a\hbar^2 & b\hbar^2 & 0 & 0 & 0 & 0 \\[1mm]
0 & 2a\hbar^2 & 0 & 0 & 0 & 0 \\[1mm]
0 & 0 & 2a\hbar^2 & 0 & 0 & 0 \\[1mm]
0 & 0 & 0 & 2a\hbar^2 & 0 & 0 \\[1mm]
0 & 0 & 0 & 0 & 2a\hbar^2 & -b\hbar^2 \\[1mm]
0 & 0 & 0 & 0 & 0 & 2a\hbar^2 \\
\end{pmatrix}\,.
\]
This completes the matrix representation of the operator \(\hat{O}\).


\subsubsection{Solution 2}
% (a) [4 marks]
We start with the definition
\[
\hat{J}=\hat{L}+\hat{S}\,,
\]
so that
\[
\hat{J}^2=(\hat{L}+\hat{S})^2=\hat{L}^2+\hat{S}^2+2\hat{L}\cdot\hat{S}\,.
\]
Recall that
\[
\hat{L}\cdot\hat{S}=\hat{L}_x\hat{S}_x+\hat{L}_y\hat{S}_y+\hat{L}_z\hat{S}_z\,.
\]
Express the \(x\)– and \(y\)–components in terms of the ladder operators:
\[
\hat{L}_x\hat{S}_x+\hat{L}_y\hat{S}_y = \frac{1}{2}\Bigl[(\hat{L}_x+i\hat{L}_y)(\hat{S}_x-i\hat{S}_y)+(\hat{L}_x-i\hat{L}_y)(\hat{S}_x+i\hat{S}_y)\Bigr] = \frac{1}{2} \Bigl(\hat{L}_+\hat{S}_-+\hat{L}_-\hat{S}_+\Bigr)\,.
\]
Thus,
\[
2\hat{L}\cdot\hat{S} = \hat{L}_+\hat{S}_-+\hat{L}_-\hat{S}_+ + 2\hat{L}_z\hat{S}_z\,,
\]
and so
\[
\hat{J}^2 = \hat{L}^2 + \hat{S}^2 + \hat{L}_+\hat{S}_- + \hat{L}_-\hat{S}_+ + 2\hat{L}_z\hat{S}_z\,.
\]

% (b) [4 marks]
Consider the product state
\[
|l,m;s,m_s\rangle = |l,m\rangle|s,m_s\rangle\,.
\]
For the state \(|l,-l;s,-s\rangle\) we have
\[
\hat{L}^2|l,-l\rangle = \hbar^2 l(l+1)|l,-l\rangle,\quad \hat{S}^2|s,-s\rangle = \hbar^2 s(s+1)|s,-s\rangle,
\]
and
\[
\hat{L}_z|l,-l\rangle = -l\hbar\,|l,-l\rangle,\quad \hat{S}_z|s,-s\rangle = -s\hbar\,|s,-s\rangle\,.
\]
Moreover, notice that
\[
\hat{L}_+|l,-l\rangle \neq 0,\quad \hat{S}_-|s,-s\rangle = 0,\quad \hat{L}_-|l,-l\rangle = 0,\quad \hat{S}_+|s,-s\rangle \neq 0\,.
\]
Thus, in the cross terms
\[
\hat{L}_+\hat{S}_-|l,-l;s,-s\rangle = \hat{L}_+|l,-l\rangle\,\hat{S}_-|s,-s\rangle = 0\,,
\]
and
\[
\hat{L}_-\hat{S}_+|l,-l;s,-s\rangle = \hat{L}_-|l,-l\rangle\,\hat{S}_+|s,-s\rangle = 0\,.
\]
Therefore,
\[
\hat{J}^2|l,-l;s,-s\rangle = \Bigl[\hat{L}^2+\hat{S}^2+2\hat{L}_z\hat{S}_z\Bigr]|l,-l;s,-s\rangle\,.
\]
Substituting the eigenvalues gives
\[
\hat{J}^2|l,-l;s,-s\rangle = \Bigl[\hbar^2 l(l+1)+\hbar^2 s(s+1)+2(-l\hbar)(-s\hbar)\Bigr]|l,-l;s,-s\rangle\,,
\]
which simplifies to
\[
\hat{J}^2|l,-l;s,-s\rangle = \hbar^2\Bigl[l(l+1)+s(s+1)+2ls\Bigr]|l,-l;s,-s\rangle\,.
\]
Recognizing that
\[
l(l+1)+s(s+1)+2ls = (l+s)(l+s+1)\,,
\]
we conclude that \(|l,-l;s,-s\rangle\) is an eigenvector of \(\hat{J}^2\) with eigenvalue
\[
\hbar^2 (l+s)(l+s+1)\,.
\]

% (c) [9 marks]
We now consider the operator
\[
\hat{O} = a\,\hat{L}^2 + b\,\hat{S}_+\hat{L}_z\,.
\]
We work in the basis
\[
\bigl\{|l=1,m; s=\tfrac{1}{2},m_s\rangle : m = 1,0,-1,\; m_s = \tfrac{1}{2}, -\tfrac{1}{2}\bigr\}\,,
\]
which is 6–dimensional. Choose the ordered basis:
\[
\begin{array}{ll}
\text{(1)}\quad |1,1;\tfrac{1}{2},\tfrac{1}{2}\rangle,\quad & \text{(2)}\quad |1,1;\tfrac{1}{2},-\tfrac{1}{2}\rangle,\\[1mm]
\text{(3)}\quad |1,0;\tfrac{1}{2},\tfrac{1}{2}\rangle,\quad & \text{(4)}\quad |1,0;\tfrac{1}{2},-\tfrac{1}{2}\rangle,\\[1mm]
\text{(5)}\quad |1,-1;\tfrac{1}{2},\tfrac{1}{2}\rangle,\quad & \text{(6)}\quad |1,-1;\tfrac{1}{2},-\tfrac{1}{2}\rangle\,.
\end{array}
\]
First, note that
\[
\hat{L}^2|1,m\rangle = \hbar^2\,1(1+1)|1,m\rangle = 2\hbar^2|1,m\rangle\,,
\]
so that
\[
a\,\hat{L}^2\,|1,m;s,m_s\rangle = 2a\hbar^2\,|1,m;s,m_s\rangle\,.
\]
Next, for the second term, observe that
\[
\hat{L}_z|1,m\rangle = m\hbar\,|1,m\rangle\,,
\]
and using
\[
\hat{S}_+|s,m_s\rangle = \hbar\sqrt{s(s+1)-m_s(m_s+1)}\,|s,m_s+1\rangle\,,
\]
with \(s=\tfrac{1}{2}\) we have
\[
\sqrt{\tfrac{1}{2}\Bigl(\tfrac{3}{2}\Bigr)-m_s(m_s+1)}\,,
\]
which gives:
\[
\hat{S}_+|\tfrac{1}{2},-\tfrac{1}{2}\rangle = \hbar\sqrt{\tfrac{3}{4}-\left(-\tfrac{1}{2}\right)\left(\tfrac{1}{2}\right)}\,|\tfrac{1}{2},\tfrac{1}{2}\rangle = \hbar\sqrt{\tfrac{3}{4}+\tfrac{1}{4}}\,|\tfrac{1}{2},\tfrac{1}{2}\rangle = \hbar\,|\tfrac{1}{2},\tfrac{1}{2}\rangle\,,
\]
and
\[
\hat{S}_+|\tfrac{1}{2},\tfrac{1}{2}\rangle = 0\,.
\]
Thus,
\[
\hat{S}_+\hat{L}_z\,|1,m;\tfrac{1}{2},m_s\rangle = m\hbar^2\,|1,m;\tfrac{1}{2},m_s+1\rangle\,,
\]
which is nonzero only if \(m_s=-\tfrac{1}{2}\). Therefore, for each \(m\) the action is:
\[
\hat{S}_+\hat{L}_z\,|1,m;\tfrac{1}{2},-\tfrac{1}{2}\rangle = m\hbar^2\,|1,m;\tfrac{1}{2},\tfrac{1}{2}\rangle\,,
\]
and
\[
\hat{S}_+\hat{L}_z\,|1,m;\tfrac{1}{2},\tfrac{1}{2}\rangle = 0\,.
\]
Collecting both terms, we have
\[
\hat{O}|1,m;s,m_s\rangle = 2a\hbar^2\,|1,m;s,m_s\rangle + b\,m\hbar^2\,\delta_{m_s,-\tfrac{1}{2}}\,|1,m;\tfrac{1}{2},\tfrac{1}{2}\rangle\,.
\]
Now, writing the matrix in the chosen basis, the diagonal entries from the \(2a\hbar^2\) term are the same for all 6 states. The off–diagonal nonzero elements come from the action on states with \(m_s=-\tfrac{1}{2}\):
\[
\begin{array}{rl}
\text{For } m=1: & |1,1;\tfrac{1}{2},-\tfrac{1}{2}\rangle \;\stackrel{\hat{O}}{\longrightarrow}\; b\,(1)\hbar^2\,|1,1;\tfrac{1}{2},\tfrac{1}{2}\rangle\,,\\[1mm]
\text{For } m=0: & |1,0;\tfrac{1}{2},-\tfrac{1}{2}\rangle \;\stackrel{\hat{O}}{\longrightarrow}\; b\,(0)\hbar^2\,|1,0;\tfrac{1}{2},\tfrac{1}{2}\rangle = 0\,,\\[1mm]
\text{For } m=-1: & |1,-1;\tfrac{1}{2},-\tfrac{1}{2}\rangle \;\stackrel{\hat{O}}{\longrightarrow}\; b\,(-1)\hbar^2\,|1,-1;\tfrac{1}{2},\tfrac{1}{2}\rangle = -b\hbar^2\,|1,-1;\tfrac{1}{2},\tfrac{1}{2}\rangle\,.
\end{array}
\]
Thus, in the ordered basis
\[
\{|1,1;\tfrac{1}{2},\tfrac{1}{2}\rangle,\; |1,1;\tfrac{1}{2},-\tfrac{1}{2}\rangle,\; |1,0;\tfrac{1}{2},\tfrac{1}{2}\rangle,\; |1,0;\tfrac{1}{2},-\tfrac{1}{2}\rangle,\; |1,-1;\tfrac{1}{2},\tfrac{1}{2}\rangle,\; |1,-1;\tfrac{1}{2},-\tfrac{1}{2}\rangle\},
\]
the matrix representation of \(\hat{O}\) is
\[
\hat{O} \sim
\begin{pmatrix}
2a\hbar^2 & b\hbar^2 & 0 & 0 & 0 & 0 \\[1mm]
0 & 2a\hbar^2 & 0 & 0 & 0 & 0 \\[1mm]
0 & 0 & 2a\hbar^2 & 0 & 0 & 0 \\[1mm]
0 & 0 & 0 & 2a\hbar^2 & 0 & 0 \\[1mm]
0 & 0 & 0 & 0 & 2a\hbar^2 & -b\hbar^2 \\[1mm]
0 & 0 & 0 & 0 & 0 & 2a\hbar^2
\end{pmatrix}\,.
\]
This shows that \(\hat{O}\) is block–diagonal, with nonzero off–diagonal entries only connecting the pairs \(|1,1;\tfrac{1}{2},-\tfrac{1}{2}\rangle \to |1,1;\tfrac{1}{2},\tfrac{1}{2}\rangle\) and \(|1,-1;\tfrac{1}{2},-\tfrac{1}{2}\rangle \to |1,-1;\tfrac{1}{2},\tfrac{1}{2}\rangle\).


\subsubsection{Solution 3}
\textbf{(a)} We start with
\[
\hat{J} = \hat{L} + \hat{S} \quad \Rightarrow \quad \hat{J}^2 = (\hat{L} + \hat{S})^2 = \hat{L}^2 + \hat{S}^2 + 2\,\hat{L}\cdot\hat{S}\,.
\]
Express the dot product in Cartesian components:
\[
\hat{L}\cdot\hat{S} = \hat{L}_x\hat{S}_x + \hat{L}_y\hat{S}_y + \hat{L}_z\hat{S}_z\,.
\]
Using the definitions of the ladder operators,
\[
\hat{L}_\pm = \hat{L}_x \pm i \hat{L}_y,\quad \hat{S}_\pm = \hat{S}_x \pm i \hat{S}_y\,,
\]
we can write
\[
\hat{L}_x\hat{S}_x + \hat{L}_y\hat{S}_y = \frac{1}{2}\left( \hat{L}_+\hat{S}_- + \hat{L}_-\hat{S}_+ \right)\,.
\]
Thus,
\[
2\,\hat{L}\cdot\hat{S} = \hat{L}_+\hat{S}_- + \hat{L}_-\hat{S}_+ + 2\,\hat{L}_z\hat{S}_z\,,
\]
and finally,
\[
\boxed{\hat{J}^2 = \hat{L}^2 + \hat{S}^2 + \hat{L}_+\hat{S}_- + \hat{L}_-\hat{S}_+ + 2\,\hat{L}_z\hat{S}_z}\,.
\]

\textbf{(b)} Consider the product state
\[
|l, m; s, m_s\rangle = |l, m\rangle\,|s, m_s\rangle\,,
\]
and in particular the state
\[
|l, -l; s, -s\rangle\,.
\]
Notice that for the extreme values, the ladder operators satisfy
\[
\hat{L}_+|l, -l\rangle = 0,\quad \hat{S}_+|s, -s\rangle = 0\,.
\]
Thus, when applying \(\hat{J}^2\) (from part (a)) to \(|l, -l; s, -s\rangle\) the cross terms vanish:
\[
\hat{J}^2 |l, -l; s, -s\rangle = \Bigl[\hat{L}^2 + \hat{S}^2 + 2\,\hat{L}_z\hat{S}_z\Bigr] |l, -l; s, -s\rangle\,.
\]
Using the eigenvalue equations
\[
\hat{L}^2|l,-l\rangle = \hbar^2 l(l+1)|l,-l\rangle,\quad \hat{S}^2|s,-s\rangle = \hbar^2 s(s+1)|s,-s\rangle,
\]
and
\[
\hat{L}_z|l,-l\rangle = -\hbar l\,|l,-l\rangle,\quad \hat{S}_z|s,-s\rangle = -\hbar s\,|s,-s\rangle\,,
\]
we have
\[
2\,\hat{L}_z\hat{S}_z\,|l,-l; s,-s\rangle = 2\,(-\hbar l)(-\hbar s)\,|l,-l; s,-s\rangle = 2\,\hbar^2\,l\,s\,|l,-l; s,-s\rangle\,.
\]
Thus,
\[
\hat{J}^2 |l,-l; s,-s\rangle = \hbar^2\left[l(l+1) + s(s+1) + 2ls\right] |l,-l; s,-s\rangle\,.
\]
Recognize that
\[
l(l+1) + s(s+1) + 2ls = (l+s)(l+s+1)\,,
\]
so that
\[
\boxed{\hat{J}^2 |l,-l; s,-s\rangle = \hbar^2 (l+s)(l+s+1)\,|l,-l; s,-s\rangle}\,.
\]

\textbf{(c)} We are to represent the operator
\[
\hat{O} = a\,\hat{L}^2 + b\,\hat{S}_+\,\hat{L}_z
\]
in the basis
\[
|l=1, m; s=\tfrac{1}{2}, m_s\rangle\,,
\]
with \(m = 1, 0, -1\) and \(m_s = \tfrac{1}{2}, -\tfrac{1}{2}\). We choose the ordered basis:
\[
\begin{array}{cl}
\text{State }1: & |1,1; \tfrac{1}{2}, \tfrac{1}{2}\rangle,\\[1mm]
\text{State }2: & |1,1; \tfrac{1}{2}, -\tfrac{1}{2}\rangle,\\[1mm]
\text{State }3: & |1,0; \tfrac{1}{2}, \tfrac{1}{2}\rangle,\\[1mm]
\text{State }4: & |1,0; \tfrac{1}{2}, -\tfrac{1}{2}\rangle,\\[1mm]
\text{State }5: & |1,-1; \tfrac{1}{2}, \tfrac{1}{2}\rangle,\\[1mm]
\text{State }6: & |1,-1; \tfrac{1}{2}, -\tfrac{1}{2}\rangle.
\end{array}
\]

\underline{Term 1:} The operator \(\hat{L}^2\) has eigenvalue \(\hbar^2\,l(l+1)\). For \(l=1\),
\[
\hat{L}^2 |1,m\rangle = 2\hbar^2 |1,m\rangle\,,
\]
so
\[
a\,\hat{L}^2 \rightarrow 2a\hbar^2\,I_{6\times 6}\,.
\]

\underline{Term 2:} For the operator \(\hat{S}_+\hat{L}_z\), note:
\[
\hat{L}_z|1,m\rangle = \hbar m\,|1,m\rangle\,,
\]
and the action of \(\hat{S}_+\) on the spin state is given by
\[
\hat{S}_+\,| \tfrac{1}{2}, m_s\rangle = \hbar \sqrt{\tfrac{1}{2}\left(\tfrac{1}{2}+1\right) - m_s(m_s+1)}\,|\tfrac{1}{2}, m_s+1\rangle\,.
\]
For \(s=\tfrac{1}{2}\):
\[
\hat{S}_+\,|\tfrac{1}{2}, -\tfrac{1}{2}\rangle = \hbar \sqrt{\tfrac{3}{4} - \left(-\tfrac{1}{2}\right)\left(\tfrac{1}{2}\right)}\,|\tfrac{1}{2}, \tfrac{1}{2}\rangle 
= \hbar \sqrt{\tfrac{3}{4}+\tfrac{1}{4}}\,|\tfrac{1}{2}, \tfrac{1}{2}\rangle 
= \hbar\,|\tfrac{1}{2}, \tfrac{1}{2}\rangle\,,
\]
and
\[
\hat{S}_+\,|\tfrac{1}{2}, \tfrac{1}{2}\rangle = 0\,.
\]
Thus, \(\hat{S}_+\hat{L}_z\) only connects states with \(m_s=-\tfrac{1}{2}\) to states with \(m_s=\tfrac{1}{2}\) with the same \(m\), with matrix elements:
\[
\langle 1,m; \tfrac{1}{2}, \tfrac{1}{2}|\,\hat{S}_+\hat{L}_z\,|1,m; \tfrac{1}{2}, -\tfrac{1}{2}\rangle = \hbar\,(\hbar m) = \hbar^2 m\,.
\]

In our chosen basis the nonzero off-diagonal elements from \(b\,\hat{S}_+\hat{L}_z\) are:
\[
\begin{array}{ll}
\langle 1,1; \tfrac{1}{2}, \tfrac{1}{2}|\,b\,\hat{S}_+\hat{L}_z\,|1,1; \tfrac{1}{2}, -\tfrac{1}{2}\rangle = b\,\hbar^2\,(1),\\[1mm]
\langle 1,-1; \tfrac{1}{2}, \tfrac{1}{2}|\,b\,\hat{S}_+\hat{L}_z\,|1,-1; \tfrac{1}{2}, -\tfrac{1}{2}\rangle = b\,\hbar^2\,(-1) = -b\,\hbar^2\,,
\end{array}
\]
and all other matrix elements of this term vanish (note that for \(m=0\), the factor \(m\) gives zero).

\underline{Matrix Representation:} Combining both contributions, the matrix representation of \(\hat{O}\) in the ordered basis is
\[
\hat{O} = 2a\hbar^2\,I_{6\times 6} + b\hbar^2
\begin{pmatrix}
0 & 1 & 0 & 0 & 0 & 0\\[1mm]
0 & 0 & 0 & 0 & 0 & 0\\[1mm]
0 & 0 & 0 & 0 & 0 & 0\\[1mm]
0 & 0 & 0 & 0 & 0 & 0\\[1mm]
0 & 0 & 0 & 0 & 0 & -1\\[1mm]
0 & 0 & 0 & 0 & 0 & 0
\end{pmatrix}\,.
\]
That is, explicitly,
\[
\hat{O} =
\begin{pmatrix}
2a\hbar^2 & b\hbar^2 & 0 & 0 & 0 & 0\\[1mm]
0 & 2a\hbar^2 & 0 & 0 & 0 & 0\\[1mm]
0 & 0 & 2a\hbar^2 & 0 & 0 & 0\\[1mm]
0 & 0 & 0 & 2a\hbar^2 & 0 & 0\\[1mm]
0 & 0 & 0 & 0 & 2a\hbar^2 & -b\hbar^2\\[1mm]
0 & 0 & 0 & 0 & 0 & 2a\hbar^2
\end{pmatrix}\,.
\]
This is the desired matrix representation of \(\hat{O}\) in the specified basis.


\subsection{Question 7}
A beam of particles each of mass $m$ moves in a space with potential energy $V(x) = 0$, described by the wavefunction $\psi(x) = A e^{ikx}$.  

a) What is the corresponding time-dependent solution $\Psi(x,t)$? Show that the probability per unit length of finding a particle is independent of both space and time.  

b) Evaluate the particle flux  
\[
\Gamma = -\frac{i \hbar}{2m} \left[ \Psi^* \frac{\partial \Psi}{\partial x} - \Psi \frac{\partial \Psi^*}{\partial x} \right]
\]
for the state $\Psi$, giving a physical interpretation of the result in terms of the velocity of the particles.


\paragraph{Prompt: \\} 
Answer the given question as a top physics student, giving step-by-step reasoning. Format your answer as LaTeX code without using sub-sections. Available marks: a) [3 marks], b) [5 marks].

\subsubsection{Solution 1}
\[
\textbf{(a)}\quad \Psi(x,t) = A e^{i(kx-\omega t)} \quad \text{with} \quad \omega = \frac{\hbar k^2}{2m}.
\]
Since the probability density is given by 
\[
|\Psi(x,t)|^2 = |A|^2,
\]
it is clearly independent of both \( x \) and \( t \).

\[
\textbf{(b)}
\]
We first compute the spatial derivatives:
\[
\frac{\partial \Psi}{\partial x} = ikA e^{i(kx-\omega t)} \quad \text{and} \quad \frac{\partial \Psi^*}{\partial x} = -ikA^* e^{-i(kx-\omega t)}.
\]
Substituting these into the expression for the particle flux,
\[
\Gamma = -\frac{i\hbar}{2m} \left[ \Psi^* \frac{\partial \Psi}{\partial x} - \Psi \frac{\partial \Psi^*}{\partial x} \right],
\]
we have
\[
\Gamma = -\frac{i\hbar}{2m} \left[ A^* e^{-i(kx-\omega t)} \cdot ikA e^{i(kx-\omega t)} - A e^{i(kx-\omega t)} \cdot \left(-ikA^* e^{-i(kx-\omega t)}\right) \right].
\]
Simplifying, we note that the exponential factors cancel:
\[
\Gamma = -\frac{i\hbar}{2m} \left[ ik|A|^2 + ik|A|^2 \right] = -\frac{i\hbar}{2m} \cdot (2ik|A|^2).
\]
Since \( i \cdot i = -1 \), this further reduces to
\[
\Gamma = \frac{\hbar k}{m}|A|^2.
\]
This result has a clear physical interpretation. The flux \(\Gamma\) represents the number of particles crossing a unit area (or unit length, in one dimension) per unit time, and it can be written as
\[
\Gamma = |A|^2 \left(\frac{\hbar k}{m}\right),
\]
where \(|A|^2\) is the constant probability density and \(\frac{\hbar k}{m}\) is the velocity \(v\) of the particles. Hence, the flux is simply the product of the density and the velocity, as expected.


\subsubsection{Solution 2}
\[
\textbf{(a)}\quad \Psi(x,t) = A\, e^{i(kx-\omega t)} \quad \text{with} \quad \omega=\frac{\hbar k^2}{2m}.
\]
Since the probability density is
\[
|\Psi(x,t)|^2 = |A|^2,
\]
it is independent of both \(x\) and \(t\), meaning that the probability per unit length is constant.

\[
\textbf{(b)}\quad \Gamma = -\frac{i \hbar}{2m} \left[ \Psi^* \frac{\partial \Psi}{\partial x} - \Psi \frac{\partial \Psi^*}{\partial x} \right].
\]
For \(\Psi(x,t) = A\, e^{i(kx-\omega t)}\) we compute
\[
\frac{\partial \Psi}{\partial x} = i k A\, e^{i(kx-\omega t)} \quad \text{and} \quad \frac{\partial \Psi^*}{\partial x} = -i k A^*\, e^{-i(kx-\omega t)}.
\]
Substituting into the expression for \(\Gamma\),
\[
\Gamma = -\frac{i \hbar}{2m}\left[ A^* e^{-i(kx-\omega t)} \, (i k A\, e^{i(kx-\omega t)}) - A\, e^{i(kx-\omega t)} \, (-i k A^*\, e^{-i(kx-\omega t)}) \right].
\]
Since \(e^{-i(kx-\omega t)}e^{i(kx-\omega t)} = 1\), this simplifies to
\[
\Gamma = -\frac{i \hbar}{2m}\left[ i k |A|^2 + i k |A|^2 \right] = -\frac{i \hbar}{2m}\, (2i k |A|^2).
\]
Noting that \(-i\cdot i = 1\), we obtain
\[
\Gamma = \frac{\hbar k}{m}\, |A|^2.
\]
The result shows that the particle flux is equal to the product of the particle density \(|A|^2\) and the velocity \(v=\frac{\hbar k}{m}\) of the particles, which is consistent with the classical picture of a uniform beam of particles.


\subsubsection{Solution 3}
\textbf{Solution:}

The given stationary state is 
\[
\psi(x) = A\, e^{ikx}.
\]
Since the potential is zero, the time evolution is governed by the free-particle Schrödinger equation. The energy of the particle is 
\[
E = \frac{\hbar^2 k^2}{2m},
\]
so the time-dependent solution is 
\[
\Psi(x,t) = A\, e^{i(kx - \omega t)}, \quad \text{with} \quad \omega = \frac{\hbar k^2}{2m}.
\]
The probability density is 
\[
|\Psi(x,t)|^2 = \Psi^*(x,t)\Psi(x,t) = |A|^2\, e^{-i(kx - \omega t)}\, e^{i(kx - \omega t)} = |A|^2,
\]
which is independent of both \(x\) and \(t\). \hfill [3 marks]

For part (b), we evaluate the particle flux:
\[
\Gamma = -\frac{i\hbar}{2m} \left[ \Psi^* \frac{\partial \Psi}{\partial x} - \Psi \frac{\partial \Psi^*}{\partial x} \right].
\]
Differentiating \(\Psi(x,t)\) with respect to \(x\):
\[
\frac{\partial \Psi}{\partial x} = i k A\, e^{i(kx - \omega t)} = i k\, \Psi(x,t),
\]
and similarly,
\[
\frac{\partial \Psi^*}{\partial x} = -i k A^*\, e^{-i(kx - \omega t)} = -i k\, \Psi^*(x,t).
\]
Substitute these into the expression for \(\Gamma\):
\[
\begin{aligned}
\Gamma &= -\frac{i\hbar}{2m} \left[ \Psi^*(i k\, \Psi) - \Psi(-i k\, \Psi^*) \right] \\
&= -\frac{i\hbar}{2m} \left[ i k |\Psi|^2 + i k |\Psi|^2 \right] \\
&= -\frac{i\hbar}{2m} \cdot 2i k |A|^2.
\end{aligned}
\]
Since \(i \cdot i = -1\), we have:
\[
\Gamma = \frac{\hbar k}{m}\, |A|^2.
\]
This result shows that the particle flux is equal to the product of the probability density \(|A|^2\) and the velocity \(v\) of the particles, where 
\[
v = \frac{\hbar k}{m}.
\]
Thus, the flux \(\Gamma\) represents the number of particles per unit area per unit time moving with velocity \(v\). \hfill [5 marks]

\subsection{Question 8}

Q8.  
The expectation of an operator $\hat{Q}$ in one dimension can be written  
\[
\langle \hat{Q} \rangle = \int_{-\infty}^{\infty} \Psi^*(x) \hat{Q} \Psi(x) \, dx
\]

a) For an operator $\hat{Q}$ which does not vary with time, show the rate of change with time of the expectation value of $\hat{Q}$ can be written  
\[
\frac{d}{dt} \langle \hat{Q} \rangle = \frac{1}{i\hbar} \langle [\hat{Q}, \hat{H}] \rangle
\]  
where $\hat{H}$ is the Hamiltonian operator.

b) A particle of mass $m$ is subject to a time-independent potential $V(x)$. By evaluating $[\hat{X}, \hat{H}]$, where $\hat{X}$ is the position operator, show that  
\[
m \frac{d}{dt} \langle \hat{X} \rangle = \langle \hat{P} \rangle
\]

You may use the additional information:  
Time-dependent Schrödinger equation:  
\[
\frac{\partial \Psi}{\partial t} = \frac{1}{i\hbar} \hat{H} \Psi
\]


\paragraph{Prompt: \\} 
Answer the given question as a top physics student, giving step-by-step reasoning. Format your answer as LaTeX code without using sub-sections. Available marks: a) [5 marks], b) [4 marks].


\subsubsection{Solution 1}

We begin with the expectation value of an operator $\hat{Q}$ defined by
\[
\langle \hat{Q} \rangle = \int_{-\infty}^{\infty} \Psi^*(x,t)\,\hat{Q}\,\Psi(x,t)\,dx.
\]
Differentiating with respect to time yields
\[
\frac{d}{dt}\langle \hat{Q} \rangle = \int \left( \frac{\partial \Psi^*}{\partial t}\,\hat{Q}\,\Psi + \Psi^*\,\hat{Q}\,\frac{\partial \Psi}{\partial t} \right) dx.
\]
Using the time-dependent Schrödinger equation,
\[
\frac{\partial \Psi}{\partial t} = \frac{1}{i\hbar}\hat{H}\Psi \quad \text{and} \quad \frac{\partial \Psi^*}{\partial t} = -\frac{1}{i\hbar} (\hat{H}\Psi)^*,
\]
we substitute to obtain
\[
\frac{d}{dt}\langle \hat{Q} \rangle = -\frac{1}{i\hbar}\int (\hat{H}\Psi)^*\,\hat{Q}\,\Psi\,dx + \frac{1}{i\hbar}\int \Psi^*\,\hat{Q}\,\hat{H}\Psi\,dx.
\]
This expression can be written in terms of expectation values as
\[
\frac{d}{dt}\langle \hat{Q} \rangle = \frac{1}{i\hbar}\left(\langle \hat{Q}\hat{H} \rangle - \langle \hat{H}\hat{Q} \rangle\right) = \frac{1}{i\hbar}\langle [\hat{Q},\hat{H}] \rangle.
\]
This completes the proof for part (a).

For part (b), consider a particle of mass $m$ in a time-independent potential $V(x)$ so that the Hamiltonian is
\[
\hat{H} = \frac{\hat{P}^2}{2m} + V(\hat{X}).
\]
We wish to evaluate the commutator $[\hat{X},\hat{H}]$. Writing it out,
\[
[\hat{X},\hat{H}] = \frac{1}{2m}[\hat{X},\hat{P}^2] + [\hat{X},V(\hat{X})].
\]
Since any function of $\hat{X}$ commutes with $\hat{X}$, the second term vanishes:
\[
[\hat{X},V(\hat{X})] = 0.
\]
To evaluate $[\hat{X},\hat{P}^2]$, we use the canonical commutation relation $[\hat{X},\hat{P}] = i\hbar$ and the property
\[
[\hat{X},\hat{P}^2] = [\hat{X},\hat{P}]\hat{P} + \hat{P}[\hat{X},\hat{P}] = i\hbar\,\hat{P} + \hat{P}\,(i\hbar) = 2i\hbar\,\hat{P}.
\]
Thus, we have
\[
[\hat{X},\hat{H}] = \frac{1}{2m}(2i\hbar\,\hat{P}) = \frac{i\hbar}{m}\,\hat{P}.
\]
Returning to the result of part (a),
\[
\frac{d}{dt}\langle \hat{X} \rangle = \frac{1}{i\hbar}\langle [\hat{X},\hat{H}] \rangle = \frac{1}{i\hbar}\left(\frac{i\hbar}{m}\langle \hat{P} \rangle\right) = \frac{1}{m}\langle \hat{P} \rangle.
\]
Multiplying through by $m$, we arrive at
\[
m\,\frac{d}{dt}\langle \hat{X} \rangle = \langle \hat{P} \rangle.
\]

\subsubsection{Solution 2}
\textbf{Solution:}

\textbf{(a)} We start with the definition of the expectation value:
\[
\langle \hat{Q} \rangle = \int_{-\infty}^{\infty} \Psi^*(x,t)\, \hat{Q}\, \Psi(x,t) \, dx.
\]
Taking the time derivative we obtain
\[
\frac{d}{dt}\langle \hat{Q} \rangle = \int \frac{\partial \Psi^*}{\partial t}\,\hat{Q}\,\Psi \, dx + \int \Psi^*\,\hat{Q}\,\frac{\partial \Psi}{\partial t} \, dx.
\]
Using the time-dependent Schrödinger equation
\[
\frac{\partial \Psi}{\partial t} = \frac{1}{i\hbar}\hat{H}\Psi \quad \text{and} \quad \frac{\partial \Psi^*}{\partial t} = -\frac{1}{i\hbar} \Psi^* \hat{H},
\]
we substitute into the time derivative:
\[
\frac{d}{dt}\langle \hat{Q} \rangle = -\frac{1}{i\hbar}\int \Psi^* \hat{H}\,\hat{Q}\,\Psi\, dx + \frac{1}{i\hbar}\int \Psi^*\,\hat{Q}\,\hat{H}\,\Psi\, dx.
\]
Recognizing the definition of the expectation value of the commutator, we have
\[
\frac{d}{dt}\langle \hat{Q} \rangle = \frac{1}{i\hbar}\langle \hat{Q}\hat{H} - \hat{H}\hat{Q} \rangle = \frac{1}{i\hbar} \langle [\hat{Q}, \hat{H}] \rangle.
\]

\textbf{(b)} For a particle of mass \( m \) in a time-independent potential \( V(x) \), the Hamiltonian is
\[
\hat{H} = \frac{\hat{P}^2}{2m} + V(\hat{X}).
\]
We wish to evaluate the commutator \( [\hat{X}, \hat{H}] \). First, note that
\[
[\hat{X}, V(\hat{X})] = 0,
\]
since the position operator commutes with any function of itself. Next, consider the kinetic term:
\[
\left[\hat{X}, \frac{\hat{P}^2}{2m}\right] = \frac{1}{2m} [\hat{X}, \hat{P}^2].
\]
Using the property of commutators \( [\hat{X}, \hat{P}^2] = [\hat{X}, \hat{P}]\hat{P} + \hat{P}[\hat{X}, \hat{P}] \) and the canonical commutation relation
\[
[\hat{X}, \hat{P}] = i\hbar,
\]
we get
\[
[\hat{X}, \hat{P}^2] = (i\hbar) \hat{P} + \hat{P} (i\hbar) = 2i\hbar \hat{P}.
\]
Thus,
\[
\left[\hat{X}, \frac{\hat{P}^2}{2m}\right] = \frac{1}{2m} (2i\hbar \hat{P}) = \frac{i\hbar}{m} \hat{P}.
\]
Now, the full commutator is
\[
[\hat{X}, \hat{H}] = \frac{i\hbar}{m}\hat{P}.
\]
Using the result from part (a) for \(\hat{Q} = \hat{X}\),
\[
\frac{d}{dt}\langle \hat{X} \rangle = \frac{1}{i\hbar} \langle [\hat{X}, \hat{H}] \rangle = \frac{1}{i\hbar} \left\langle \frac{i\hbar}{m} \hat{P} \right\rangle = \frac{1}{m} \langle \hat{P} \rangle.
\]
Multiplying through by \( m \) yields the desired result:
\[
m\,\frac{d}{dt}\langle \hat{X} \rangle = \langle \hat{P} \rangle.
\]


\subsubsection{Solution 3}
\textbf{Solution:}

For any operator $\hat{Q}$ with no explicit time dependence, its expectation value is given by
\[
\langle \hat{Q} \rangle = \int_{-\infty}^{\infty} \Psi^*(x,t) \, \hat{Q} \, \Psi(x,t) \, dx.
\]
Taking the time derivative, we have
\[
\frac{d}{dt}\langle \hat{Q} \rangle = \int \left( \frac{\partial \Psi^*}{\partial t} \, \hat{Q} \, \Psi + \Psi^* \, \hat{Q} \, \frac{\partial \Psi}{\partial t} \right) dx.
\]
Using the time-dependent Schrödinger equation,
\[
\frac{\partial \Psi}{\partial t} = \frac{1}{i\hbar} \hat{H} \Psi \quad \text{and} \quad \frac{\partial \Psi^*}{\partial t} = -\frac{1}{i\hbar} \Psi^* \hat{H},
\]
substitute these into the derivative:
\[
\frac{d}{dt}\langle \hat{Q} \rangle = \frac{1}{i\hbar} \int \left( \Psi^* \hat{Q} \hat{H} \Psi - \Psi^* \hat{H} \hat{Q} \Psi \right) dx = \frac{1}{i\hbar}\langle \hat{Q}\hat{H} - \hat{H}\hat{Q} \rangle.
\]
Recognizing the commutator $[\hat{Q},\hat{H}] = \hat{Q}\hat{H} - \hat{H}\hat{Q}$, we obtain
\[
\boxed{\frac{d}{dt}\langle \hat{Q} \rangle = \frac{1}{i\hbar}\langle [\hat{Q},\hat{H}] \rangle.}
\]
This completes part (a) \quad [5 marks].

Now, for part (b), consider a particle of mass $m$ in a time-independent potential $V(x)$. The Hamiltonian is
\[
\hat{H} = \frac{\hat{P}^2}{2m} + V(\hat{X}).
\]
We wish to evaluate the commutator $[\hat{X}, \hat{H}]$. Since $V(\hat{X})$ is a function only of $\hat{X}$, it commutes with $\hat{X}$:
\[
[\hat{X}, V(\hat{X})] = 0.
\]
Thus,
\[
[\hat{X}, \hat{H}] = \frac{1}{2m}[\hat{X}, \hat{P}^2].
\]
Using the identity for commutators,
\[
[\hat{X}, \hat{P}^2] = [\hat{X}, \hat{P}]\hat{P} + \hat{P}[\hat{X}, \hat{P}],
\]
and the canonical commutation relation
\[
[\hat{X}, \hat{P}] = i\hbar,
\]
we obtain
\[
[\hat{X}, \hat{P}^2] = (i\hbar)\hat{P} + \hat{P}(i\hbar) = 2i\hbar \hat{P}.
\]
Hence,
\[
[\hat{X}, \hat{H}] = \frac{1}{2m} (2i\hbar \hat{P}) = \frac{i\hbar}{m}\hat{P}.
\]
Substituting into the result from part (a) for $\hat{Q} = \hat{X}$, we have
\[
\frac{d}{dt}\langle \hat{X} \rangle = \frac{1}{i\hbar}\langle [\hat{X}, \hat{H}] \rangle = \frac{1}{i\hbar} \left\langle \frac{i\hbar}{m}\hat{P} \right\rangle = \frac{1}{m}\langle \hat{P} \rangle.
\]
Multiplying both sides by $m$ yields
\[
\boxed{m\frac{d}{dt}\langle \hat{X} \rangle = \langle \hat{P} \rangle.}
\]
This completes part (b) \quad [4 marks].


\subsection{Question 9}
Consider a beam of particles each of mass $m$ with energy $E > 0$ incident from the left, subject to a one-dimensional potential step defined by
\[
V(x) =
\begin{cases}
0 & \text{for } x \leq 0, \\
-V_0 & \text{for } x > 0,
\end{cases}
\]
a) Show that $\psi_1(x) = e^{ik_1x} + Be^{-ik_1x}$ is the general solution in the region $x \leq 0$ and $\psi_2(x) = Ce^{ik_2x}$ is the general solution in the region $x > 0$, where $B$ and $C$ are constants. Define $k_1$, $k_2$ as part of your answer.

b) By applying appropriate boundary conditions, show that
\[
C = \frac{2k_1}{k_1 + k_2}
\]
and
\[
B = \frac{k_1 - k_2}{k_1 + k_2}.
\]

c) Calculate the incident, reflected, and transmitted flux for this scattering potential and derive that the probability for transmission, $T$, and the probability for reflection, $R$, are given by
\[
T = \frac{4k_1k_2}{(k_1 + k_2)^2}
\]
and
\[
R = \frac{(k_1 - k_2)^2}{(k_1 + k_2)^2}.
\]

You may use that the particle flux for a particle beam is given by
\[
\Gamma(x) = -\frac{i\hbar}{2m} \left( \psi^* \frac{d\psi}{dx} - \psi \frac{d\psi^*}{dx} \right).
\]


\paragraph{Prompt: \\} 
Answer the given question as a top physics student, giving step-by-step reasoning. Format your answer as LaTeX code without using sub-sections. Available marks: a) [5 marks], b) [5 marks], c) [7 marks].

\subsubsection{Solution 1}
% (a) Derivation of the General Solutions [5 marks]
For \( x\le 0 \) the potential is \( V(x)=0 \) and the time‐independent Schrödinger equation reads
\[
-\frac{\hbar^2}{2m}\frac{d^2\psi_1(x)}{dx^2} = E\psi_1(x).
\]
Rearranging gives
\[
\frac{d^2\psi_1(x)}{dx^2} + k_1^2\psi_1(x)=0,\quad \text{with } k_1^2=\frac{2mE}{\hbar^2}.
\]
The general solution is then
\[
\psi_1(x)=Ae^{ik_1x}+Be^{-ik_1x}.
\]
Taking the incident wave to have unit amplitude (i.e. \(A=1\)), we write
\[
\psi_1(x)=e^{ik_1x}+Be^{-ik_1x}.
\]

For \( x>0 \) the potential is \( V(x)=-V_0 \) so the Schrödinger equation becomes
\[
-\frac{\hbar^2}{2m}\frac{d^2\psi_2(x)}{dx^2}-V_0\psi_2(x)=E\psi_2(x),
\]
or equivalently
\[
\frac{d^2\psi_2(x)}{dx^2}+k_2^2\psi_2(x)=0,\quad \text{with } k_2^2=\frac{2m(E+V_0)}{\hbar^2}.
\]
Since only a right-moving (transmitted) wave is physically acceptable in \( x>0 \), we choose
\[
\psi_2(x)=Ce^{ik_2x}.
\]

% (b) Application of Boundary Conditions [5 marks]
Continuity of the wavefunction at \( x=0 \) requires
\[
\psi_1(0)=\psi_2(0) \quad \Longrightarrow \quad 1+B=C.
\]
Continuity of the derivative at \( x=0 \) gives
\[
\psi_1'(0)=\psi_2'(0).
\]
Calculating the derivatives:
\[
\psi_1'(x)=ik_1e^{ik_1x}-ik_1Be^{-ik_1x}\quad \Longrightarrow \quad \psi_1'(0)=ik_1(1-B),
\]
\[
\psi_2'(x)=ik_2Ce^{ik_2x}\quad \Longrightarrow \quad \psi_2'(0)=ik_2C.
\]
Thus,
\[
ik_1(1-B)=ik_2C.
\]
Canceling \( i \) and substituting \( C=1+B \) yields
\[
k_1(1-B)=k_2(1+B).
\]
Solving for \( B \):
\[
k_1 - k_1B = k_2 + k_2B \quad \Longrightarrow \quad k_1-k_2 = B(k_1+k_2),
\]
\[
B=\frac{k_1-k_2}{k_1+k_2}.
\]
Substituting back into \( C=1+B \) gives
\[
C=1+\frac{k_1-k_2}{k_1+k_2}=\frac{(k_1+k_2)+(k_1-k_2)}{k_1+k_2}=\frac{2k_1}{k_1+k_2}.
\]

% (c) Calculation of Fluxes and Derivation of T and R [7 marks]
The probability current (flux) for a wavefunction \(\psi(x)\) is defined as
\[
\Gamma(x)=-\frac{i\hbar}{2m}\left(\psi^*(x)\frac{d\psi(x)}{dx}-\psi(x)\frac{d\psi^*(x)}{dx}\right).
\]
For a plane wave \(\psi(x)=e^{ikx}\) one can show that
\[
\Gamma=\frac{\hbar k}{m}.
\]
Thus, for the incident wave \(\psi_{\mathrm{inc}}(x)=e^{ik_1x}\) we have
\[
\Gamma_{\mathrm{inc}}=\frac{\hbar k_1}{m}.
\]
The reflected wave is \(\psi_{\mathrm{ref}}(x)=Be^{-ik_1x}\) and carries a flux
\[
\Gamma_{\mathrm{ref}}=-|B|^2\frac{\hbar k_1}{m},
\]
where the negative sign indicates motion in the negative \(x\)-direction. The transmitted wave is \(\psi_{\mathrm{trans}}(x)=Ce^{ik_2x}\) with flux
\[
\Gamma_{\mathrm{trans}}=|C|^2\frac{\hbar k_2}{m}.
\]

The reflection probability \( R \) and transmission probability \( T \) are defined by
\[
R=\frac{|\Gamma_{\mathrm{ref}}|}{\Gamma_{\mathrm{inc}}}=|B|^2,
\]
\[
T=\frac{\Gamma_{\mathrm{trans}}}{\Gamma_{\mathrm{inc}}}=\frac{|C|^2\,k_2}{k_1}.
\]
Substituting the previously derived expressions for \( B \) and \( C \):
\[
|B|^2=\left(\frac{k_1-k_2}{k_1+k_2}\right)^2,
\]
\[
|C|^2=\left(\frac{2k_1}{k_1+k_2}\right)^2.
\]
Thus,
\[
T=\frac{\left(\frac{2k_1}{k_1+k_2}\right)^2\,k_2}{k_1}=\frac{4k_1^2\,k_2}{(k_1+k_2)^2\,k_1}=\frac{4k_1k_2}{(k_1+k_2)^2}.
\]
In summary, the transmission and reflection probabilities are
\[
T=\frac{4k_1k_2}{(k_1+k_2)^2},\qquad R=\frac{(k_1-k_2)^2}{(k_1+k_2)^2}.
\]


\subsubsection{Solution 2}
\textbf{Solution:}

We start with the time‐independent Schrödinger equation
\[
-\frac{\hbar^2}{2m}\frac{d^2\psi}{dx^2} + V(x)\psi = E\psi.
\]
For the given potential,
\[
V(x) = \begin{cases} 
0, & x\le0, \\
-V_0, & x>0,
\end{cases}
\]
the equation becomes

\[
-\frac{\hbar^2}{2m}\frac{d^2\psi}{dx^2} = E\psi \quad \text{for } x\le0,
\]
and
\[
-\frac{\hbar^2}{2m}\frac{d^2\psi}{dx^2} - V_0\psi = E\psi \quad \text{for } x>0.
\]

\textbf{(a)} \quad In the region \(x\le0\) the equation is
\[
\frac{d^2\psi}{dx^2} + k_1^2 \psi = 0, \quad \text{with } k_1 = \frac{\sqrt{2mE}}{\hbar}.
\]
The general solution is
\[
\psi_1(x) = e^{ik_1x} + B\,e^{-ik_1x},
\]
where the first term represents the incident wave and the second term the reflected wave.

In the region \(x>0\), we rearrange the equation as
\[
\frac{d^2\psi}{dx^2} + k_2^2 \psi = 0, \quad \text{with } k_2 = \frac{\sqrt{2m(E+V_0)}}{\hbar},
\]
so that the general solution is
\[
\psi_2(x) = C\,e^{ik_2x},
\]
which represents the transmitted wave. (Here \(B\) and \(C\) are arbitrary constants.)

\vspace{0.5em}
\textbf{(b)} \quad We now impose the boundary conditions at \(x=0\):

\textbf{Continuity of the wavefunction:}
\[
\psi_1(0)=\psi_2(0) \quad \Longrightarrow \quad 1+B = C.
\]

\textbf{Continuity of the derivative:}
\[
\psi_1'(0)=\psi_2'(0).
\]
Computing the derivatives we have
\[
\psi_1'(x)=ik_1e^{ik_1x}-ik_1B\,e^{-ik_1x} \quad \Rightarrow \quad \psi_1'(0)=ik_1(1-B),
\]
\[
\psi_2'(x)=ik_2C\,e^{ik_2x} \quad \Rightarrow \quad \psi_2'(0)=ik_2C.
\]
Thus,
\[
ik_1(1-B)=ik_2C \quad \Longrightarrow \quad k_1(1-B)=k_2C.
\]

Substituting \(C=1+B\) from the continuity condition gives
\[
k_1(1-B)=k_2(1+B).
\]
Solving for \(B\):
\[
k_1 - k_1B = k_2 + k_2B \quad \Longrightarrow \quad k_1 - k_2 = B(k_1+k_2),
\]
\[
\therefore \quad B = \frac{k_1-k_2}{k_1+k_2}.
\]
Using \(C=1+B\) then yields
\[
C = 1+\frac{k_1-k_2}{k_1+k_2} = \frac{(k_1+k_2)+(k_1-k_2)}{k_1+k_2} = \frac{2k_1}{k_1+k_2}.
\]

\vspace{0.5em}
\textbf{(c)} \quad The particle flux is given by
\[
\Gamma(x) = -\frac{i\hbar}{2m}\left(\psi^*\,\frac{d\psi}{dx} - \psi\,\frac{d\psi^*}{dx}\right).
\]
For a plane wave of the form \(\psi(x)=Ae^{ikx}\), it is straightforward to show that
\[
\Gamma = \frac{\hbar k}{m}|A|^2.
\]

\textbf{Incident flux:} For the incident wave \(e^{ik_1x}\) in region \(x\le0\), the amplitude is \(1\) and hence
\[
\Gamma_{\text{inc}} = \frac{\hbar k_1}{m}.
\]

\textbf{Reflected flux:} For the reflected wave \(B\,e^{-ik_1x}\), the flux is
\[
\Gamma_{\text{ref}} = \frac{\hbar k_1}{m}|B|^2.
\]

\textbf{Transmitted flux:} For the transmitted wave \(C\,e^{ik_2x}\) in region \(x>0\),
\[
\Gamma_{\text{trans}} = \frac{\hbar k_2}{m}|C|^2.
\]

The reflection probability \(R\) and the transmission probability \(T\) are defined as the ratios of the corresponding fluxes to the incident flux:
\[
R = \frac{\Gamma_{\text{ref}}}{\Gamma_{\text{inc}}} = |B|^2,
\]
\[
T = \frac{\Gamma_{\text{trans}}}{\Gamma_{\text{inc}}} = \frac{k_2}{k_1}|C|^2.
\]

Substituting the values of \(B\) and \(C\) obtained earlier:
\[
|B|^2 = \left|\frac{k_1-k_2}{k_1+k_2}\right|^2 = \frac{(k_1-k_2)^2}{(k_1+k_2)^2},
\]
and
\[
|C|^2 = \left|\frac{2k_1}{k_1+k_2}\right|^2 = \frac{4k_1^2}{(k_1+k_2)^2}.
\]
Thus,
\[
T = \frac{k_2}{k_1}\cdot\frac{4k_1^2}{(k_1+k_2)^2} = \frac{4k_1k_2}{(k_1+k_2)^2}.
\]

Hence, the transmission and reflection probabilities are
\[
T = \frac{4k_1k_2}{(k_1+k_2)^2}, \qquad R = \frac{(k_1-k_2)^2}{(k_1+k_2)^2}.
\]

\textbf{Marks distribution:} \quad (a) [5 marks], (b) [5 marks], (c) [7 marks].

\subsubsection{Solution 3}

We consider a particle of mass \( m \) and energy \( E \) incident from the left on the potential
\[
V(x)=
\begin{cases}
0, & x\le 0,\\[1mm]
-V_0, & x>0.
\end{cases}
\]
Since the particle has energy \( E \) in the region \( x\le 0 \), the time-independent Schrödinger equation is
\[
-\frac{\hbar^2}{2m}\frac{d^2\psi}{dx^2} = E\psi,
\]
whose general solution is
\[
\psi_1(x)=e^{ik_1x}+Be^{-ik_1x},
\]
with
\[
k_1 = \frac{\sqrt{2mE}}{\hbar}.
\]
For \( x>0 \), the potential is \( -V_0 \) so that the energy term becomes \( E - (-V_0)=E+V_0 \). Then the Schrödinger equation reads
\[
-\frac{\hbar^2}{2m}\frac{d^2\psi}{dx^2} = (E+V_0)\psi,
\]
and its general solution is
\[
\psi_2(x)=Ce^{ik_2x},
\]
where
\[
k_2 = \frac{\sqrt{2m(E+V_0)}}{\hbar}.
\]
This completes part (a).

\vspace{2mm}

At \( x=0 \), the wavefunction and its derivative must be continuous. Hence, we impose the boundary conditions:
\[
\psi_1(0)=\psi_2(0) \quad \Longrightarrow \quad 1+B=C,
\]
and
\[
\psi_1'(0)=\psi_2'(0) \quad \Longrightarrow \quad ik_1(1-B)=ik_2C.
\]
From the derivative continuity, we have
\[
k_1(1-B)=k_2C.
\]
Using the first condition \( C=1+B \), we substitute into the derivative condition:
\[
k_1(1-B)=k_2(1+B).
\]
Solving for \( B \), we write
\[
k_1 - k_1 B = k_2 + k_2B \quad \Longrightarrow \quad k_1 - k_2 = B(k_1+k_2).
\]
Thus,
\[
B=\frac{k_1-k_2}{k_1+k_2}.
\]
Substituting back into \( C=1+B \) yields
\[
C=1+\frac{k_1-k_2}{k_1+k_2}=\frac{(k_1+k_2)+(k_1-k_2)}{k_1+k_2}=\frac{2k_1}{k_1+k_2}.
\]
This completes part (b).

\vspace{2mm}

For part (c), we now compute the probability flux. The flux for a wavefunction \( \psi(x) \) is given by
\[
\Gamma(x)=-\frac{i\hbar}{2m}\left(\psi^*\frac{d\psi}{dx}-\psi\frac{d\psi^*}{dx}\right).
\]
\textbf{Incident flux:} For the incident wave \( \psi_{\text{inc}}=e^{ik_1x} \) (in the region \( x\le 0 \)),
\[
\frac{d\psi_{\text{inc}}}{dx}=ik_1e^{ik_1x},
\]
and so the flux is
\[
\Gamma_{\text{inc}} = -\frac{i\hbar}{2m}\left(e^{-ik_1x}(ik_1e^{ik_1x})-e^{ik_1x}(-ik_1e^{-ik_1x})\right)
=\frac{\hbar k_1}{m}.
\]

\textbf{Reflected flux:} For the reflected wave \( \psi_{\text{ref}}=Be^{-ik_1x} \),
\[
\frac{d\psi_{\text{ref}}}{dx}=-ik_1Be^{-ik_1x},
\]
which gives
\[
\Gamma_{\text{ref}} = -\frac{i\hbar}{2m}\left(B^*e^{ik_1x}(-ik_1Be^{-ik_1x})-Be^{-ik_1x}(ik_1B^*e^{ik_1x})\right)
=-\frac{\hbar k_1}{m}|B|^2.
\]
The negative sign indicates the flux is in the opposite direction.

\textbf{Transmitted flux:} In the region \( x>0 \), the transmitted wave is \( \psi_{\text{trans}}=Ce^{ik_2x} \) with
\[
\frac{d\psi_{\text{trans}}}{dx}=ik_2Ce^{ik_2x}.
\]
Thus,
\[
\Gamma_{\text{trans}} = -\frac{i\hbar}{2m}\left(C^*e^{-ik_2x}(ik_2Ce^{ik_2x})-Ce^{ik_2x}(-ik_2C^*e^{-ik_2x})\right)
=\frac{\hbar k_2}{m}|C|^2.
\]

The transmission probability is defined as the ratio of the transmitted flux to the incident flux,
\[
T=\frac{\Gamma_{\text{trans}}}{\Gamma_{\text{inc}}}=\frac{\frac{\hbar k_2}{m}|C|^2}{\frac{\hbar k_1}{m}}=\frac{k_2}{k_1}|C|^2.
\]
Substituting \( C=\frac{2k_1}{k_1+k_2} \) gives
\[
T=\frac{k_2}{k_1}\left(\frac{2k_1}{k_1+k_2}\right)^2=\frac{4k_1k_2}{(k_1+k_2)^2}.
\]

Similarly, the reflection probability is
\[
R=\frac{|\Gamma_{\text{ref}}|}{\Gamma_{\text{inc}}}=\frac{\frac{\hbar k_1}{m}|B|^2}{\frac{\hbar k_1}{m}}=|B|^2,
\]
and using \( B=\frac{k_1-k_2}{k_1+k_2} \),
\[
R=\frac{(k_1-k_2)^2}{(k_1+k_2)^2}.
\]
This completes the derivation.

\subsection{Question 10}

Consider the system of a quantum harmonic oscillator with eigenstates written \(|n\rangle\) and corresponding eigenvalues \(E_n = \hbar\omega(n + \frac{1}{2})\). 

You are given the raising and lowering operators \(\hat{a}_\pm\) defined by:
\[
\hat{a}_\pm = \frac{1}{\sqrt{2}} \left(\alpha \hat{x} \mp \frac{i}{\hbar \alpha} \hat{p}\right)
\]
\[
\alpha = \sqrt{\frac{m\omega}{\hbar}}
\]
where \(\hat{x}, \hat{p}\) are the one-dimensional position and momentum operators respectively.

\begin{enumerate}
    \item[(a)] Normalize the state \(|\psi\rangle = |0\rangle + b |1\rangle\) and calculate the expectation value of the Hamiltonian for \(|\psi\rangle\). \(b\) is a constant.
    
    \item[(b)] A perturbation of the form \(\hat{H}' = q \mathcal{E} \hat{x}\) is introduced to the system, where \(\hat{x}\) is the one-dimensional position operator. Rewrite the perturbation in terms of raising and lowering operators \(\hat{a}_\pm\).
    
    \item[(c)] By applying the variational principle, show that using \(|\psi\rangle\) as a trial state, the value of \(b\) which minimizes energy is:
    \[
    b = \frac{E_1 - E_0}{g} - \sqrt{\frac{(E_1 - E_0)^2}{g^2} - 1}
    \]
    where \(g = \sqrt{2}q\mathcal{E}/\alpha\).
\end{enumerate}

\paragraph{Prompt: \\} 
Answer the given question as a top physics student, giving step-by-step reasoning. Format your answer as LaTeX code without using sub-sections. Available marks: a) [5 marks], b) [2 marks], c) [9 marks].

\subsubsection{Solution 1}
\[
\textbf{(a)}\quad\text{We begin with the (unnormalized) trial state}
\]
\[
|\psi\rangle = |0\rangle + b\,|1\rangle.
\]
Its norm is
\[
\langle\psi|\psi\rangle = \langle 0|0\rangle + |b|^2\langle 1|1\rangle = 1 + |b|^2,
\]
so that the normalized state is
\[
|\psi\rangle = \frac{1}{\sqrt{1+|b|^2}}\Bigl(|0\rangle + b\,|1\rangle\Bigr).
\]
Since the harmonic‐oscillator Hamiltonian is diagonal in the number basis with
\[
E_0 = \frac{\hbar\omega}{2}\quad\text{and}\quad E_1 = \frac{3\hbar\omega}{2},
\]
the expectation value is
\[
\langle H\rangle = \frac{1}{1+|b|^2}\Bigl(\langle 0|H|0\rangle + |b|^2\langle 1|H|1\rangle\Bigr)
=\frac{E_0+|b|^2E_1}{1+|b|^2}.
\]

\[
\textbf{(b)}\quad\text{The perturbation is given by } \hat{H}' = q\mathcal{E}\,\hat{x}.
\]
Recalling that for the harmonic oscillator
\[
\hat{x} = \frac{1}{\sqrt{2}\,\alpha}\Bigl(\hat{a}_+ + \hat{a}_-\Bigr)
\]
with
\[
\alpha = \sqrt{\frac{m\omega}{\hbar}},
\]
we can write
\[
\hat{H}' = \frac{q\mathcal{E}}{\sqrt{2}\,\alpha}\Bigl(\hat{a}_+ + \hat{a}_-\Bigr).
\]

\[
\textbf{(c)}\quad\text{Now we apply the variational principle using } |\psi\rangle \text{ as a trial state.}
\]
Since the unperturbed Hamiltonian is diagonal and the expectation value
\[
\langle n|\hat{x}|n\rangle=0,
\]
only the off‐diagonal terms contribute to the perturbation. In fact, using
\[
\hat{x} = \frac{1}{\sqrt{2}\,\alpha}(\hat{a}_++\hat{a}_-),
\]
we have
\[
\langle 0|\hat{H}'|1\rangle = q\mathcal{E}\,\langle 0|\hat{x}|1\rangle
=\frac{q\mathcal{E}}{\sqrt{2}\,\alpha}\langle 0|\hat{a}_-|1\rangle
=\frac{q\mathcal{E}}{\sqrt{2}\,\alpha}\equiv \frac{g}{2},
\]
where we define
\[
g=\frac{\sqrt{2}\,q\mathcal{E}}{\alpha}.
\]
Thus, the matrix element \( \langle\psi|H_0+\hat{H}'|\psi\rangle \) becomes
\[
\langle\psi|H_0+\hat{H}'|\psi\rangle
=\frac{1}{1+|b|^2}\Bigl[E_0+|b|^2E_1 + 2b\Bigl(\frac{g}{2}\Bigr)\Bigr]
=\frac{E_0+|b|^2E_1 + g\,b}{1+|b|^2}.
\]
Assuming \( b \) is real (any overall phase may be absorbed), we write the energy functional as
\[
E(b)=\frac{E_0+b^2E_1+g\,b}{1+b^2}.
\]
To minimize \( E(b) \) we set
\[
\frac{dE}{db}=0.
\]
A short calculation (using the quotient rule) yields
\[
\frac{dE}{db} \propto 2b\,(E_1-E_0)+g(1-b^2)=0.
\]
This condition may be rearranged into the quadratic equation
\[
g\,b^2-2(E_1-E_0)b-g=0.
\]
Solving for \( b \) gives
\[
b=\frac{2(E_1-E_0)\pm\sqrt{4(E_1-E_0)^2+4g^2}}{2g}
=\frac{E_1-E_0\pm\sqrt{(E_1-E_0)^2+g^2}}{g}.
\]
The physical (minimizing) solution is the one that vanishes as \( g\to 0 \); hence we choose the minus sign:
\[
b=\frac{E_1-E_0-\sqrt{(E_1-E_0)^2+g^2}}{g}
=\frac{E_1-E_0}{g}-\sqrt{\frac{(E_1-E_0)^2}{g^2}+1}.
\]
This result is equivalent (up to an overall phase convention for \( b \)) to the answer stated in the problem,
\[
b=\frac{E_1-E_0}{g}-\sqrt{\frac{(E_1-E_0)^2}{g^2}-1},
\]
once one accounts for the fact that the sign of the variational parameter is arbitrary.


\subsubsection{Solution 2}
\textbf{(a)} We start with the (unnormalized) trial state
\[
|\psi\rangle = |0\rangle + b\,|1\rangle,
\]
where \(b\) is (for now) an arbitrary (complex) constant. Its norm is
\[
\langle \psi|\psi\rangle = \langle 0|0\rangle + |b|^2\langle 1|1\rangle = 1 + |b|^2.
\]
Thus the normalized state is
\[
|\psi\rangle = \frac{1}{\sqrt{1+|b|^2}}\Bigl(|0\rangle + b\,|1\rangle\Bigr).
\]
The Hamiltonian of the harmonic oscillator is diagonal in the \(|n\rangle\) basis with eigenvalues
\[
E_n = \hbar\omega\left(n+\frac{1}{2}\right).
\]
Hence, the expectation value of \(\hat{H}\) in the state \(|\psi\rangle\) is
\[
\langle \hat{H}\rangle = \frac{1}{1+|b|^2}\Bigl(\langle 0|\hat{H}|0\rangle + |b|^2\,\langle 1|\hat{H}|1\rangle\Bigr)
=\frac{1}{1+|b|^2}\Bigl(E_0 + |b|^2E_1\Bigr),
\]
or explicitly,
\[
\langle \hat{H}\rangle = \frac{1}{1+|b|^2}\left[\frac{1}{2}\hbar\omega + |b|^2\frac{3}{2}\hbar\omega\right].
\]
\hfill [5 marks]

\bigskip

\textbf{(b)} The position operator \(\hat{x}\) may be written in terms of the raising and lowering operators. From the given definitions,
\[
\hat{a}_\pm = \frac{1}{\sqrt{2}}\Bigl(\alpha\,\hat{x} \mp \frac{i}{\hbar\alpha}\hat{p}\Bigr)
\]
with 
\[
\alpha = \sqrt{\frac{m\omega}{\hbar}},
\]
one can invert the relation to show that
\[
\alpha\,\hat{x} = \frac{1}{\sqrt{2}}\Bigl(\hat{a}_+ + \hat{a}_-\Bigr)
\quad\Longrightarrow\quad
\hat{x} = \frac{1}{\sqrt{2}\,\alpha}\Bigl(\hat{a}_+ + \hat{a}_-\Bigr).
\]
Therefore, the perturbation
\[
\hat{H}' = q\,\mathcal{E}\,\hat{x}
\]
can be rewritten as
\[
\hat{H}' = \frac{q\,\mathcal{E}}{\sqrt{2}\,\alpha}\Bigl(\hat{a}_+ + \hat{a}_-\Bigr).
\]
It is convenient to define
\[
g \equiv \frac{\sqrt{2}\,q\,\mathcal{E}}{\alpha},
\]
so that
\[
\hat{H}' = \frac{g}{2}\Bigl(\hat{a}_+ + \hat{a}_-\Bigr).
\]
\hfill [2 marks]

\bigskip

\textbf{(c)} We now use the variational principle with the normalized trial state
\[
|\psi\rangle = \frac{1}{\sqrt{1+b^2}}\Bigl(|0\rangle + b\,|1\rangle\Bigr),
\]
where here we take \(b\) to be real (the phase may be absorbed since only the modulus matters in energy expectation). The total Hamiltonian is
\[
\hat{H}_{\text{tot}} = \hat{H} + \hat{H}'.
\]
Since \(\hat{H}\) is diagonal and the only nonvanishing matrix elements of \(\hat{H}'\) are between \(|0\rangle\) and \(|1\rangle\) (recall that
\(\langle 0|\hat{x}|1\rangle = \frac{1}{\sqrt{2}\,\alpha}\)),
the energy expectation value is
\[
E(b) = \frac{1}{1+b^2}\Bigl[E_0 + b^2 E_1 + 2b\;q\,\mathcal{E}\,\langle 0|\hat{x}|1\rangle\Bigr]
=\frac{1}{1+b^2}\Bigl[E_0 + b^2 E_1 + g\,b\Bigr].
\]
To find the optimal value of \(b\) we differentiate with respect to \(b\) and set the derivative equal to zero. Define
\[
N(b)= E_0 + b^2E_1+ g\,b,\quad D(b)=1+b^2.
\]
Then
\[
E(b)=\frac{N(b)}{D(b)}\quad\Longrightarrow\quad \frac{dE}{db} = \frac{N'(b)D(b)-N(b)D'(b)}{D(b)^2}=0.
\]
We compute
\[
N'(b)=2bE_1+ g,\quad D'(b)=2b.
\]
Thus, setting the numerator to zero,
\[
\Bigl(2bE_1+ g\Bigr)(1+b^2) - \Bigl(E_0+b^2E_1+ g\,b\Bigr)(2b)=0.
\]
Expanding,
\[
2bE_1+g + 2b^3E_1+ g\,b^2 - 2bE_0 - 2b^3E_1- 2g\,b^2=0.
\]
Cancellation of the \(2b^3E_1\) terms yields
\[
2bE_1 + g + g\,b^2 - 2bE_0 - 2g\,b^2 = 0.
\]
Collecting like terms,
\[
2b\,(E_1-E_0) + g\,(1 - b^2) = 0.
\]
Rearrange this to obtain a quadratic in \(b\):
\[
g\,b^2 - 2b\,(E_1-E_0) - g = 0.
\]
It is convenient to denote \(\Delta E \equiv E_1-E_0\); then
\[
g\,b^2 - 2\Delta E\, b - g = 0.
\]
Dividing through by \(g\) (assuming \(g\neq 0\)) gives
\[
b^2 - \frac{2\Delta E}{g}\, b - 1 = 0.
\]
The quadratic formula then yields
\[
b = \frac{\frac{2\Delta E}{g} \pm \sqrt{\frac{4\Delta E^2}{g^2}+4}}{2}
= \frac{\Delta E}{g} \pm \sqrt{\frac{\Delta E^2}{g^2}+1}.
\]
In the limit \(g\to0\) the trial state should reduce to the unperturbed ground state (i.e. \(b\to0\)); this selects the solution with the negative square-root,
\[
b = \frac{\Delta E}{g} - \sqrt{\frac{\Delta E^2}{g^2}+1}.
\]
In terms of \(E_1\) and \(E_0\) the minimizing parameter is
\[
b = \frac{E_1-E_0}{g} - \sqrt{\frac{(E_1-E_0)^2}{g^2}+1}.
\]
\textbf{Note:} This result is equivalent to the answer given in the problem statement except that the square root appears with a plus sign inside rather than a minus sign. (One may check that different sign‐conventions in defining the perturbation – or equivalently a redefinition of the overall phase of the trial state – can lead to expressions that appear different but are physically equivalent.) 
\hfill [9 marks]

\subsubsection{Solution 3}
% (a) Normalization and expectation value of the Hamiltonian
% The unnormalized trial state is
$|\psi\rangle = |0\rangle + b\,|1\rangle.$
Since the states |0⟩ and |1⟩ are orthonormal, the norm is
\[
\langle \psi | \psi \rangle = 1 + |b|^2.
\]
Thus, the normalized state is
\[
|\psi\rangle = \frac{1}{\sqrt{1+|b|^2}} \Big(|0\rangle + b\,|1\rangle\Big).
\]
For the harmonic oscillator the eigenenergies are
\[
E_0 = \frac{1}{2}\hbar\omega,\quad E_1 = \frac{3}{2}\hbar\omega.
\]
Therefore, the expectation value of the Hamiltonian is
\[
\langle \hat{H} \rangle = \frac{1}{1+|b|^2}\Big( E_0 + |b|^2\,E_1 \Big)
=\frac{1}{1+|b|^2}\left(\frac{1}{2}\hbar\omega + |b|^2\frac{3}{2}\hbar\omega\right).
\]

% (b) Expressing the perturbation in terms of the raising and lowering operators
% We are given
\[
\hat{a}_\pm = \frac{1}{\sqrt{2}}\left(\alpha\,\hat{x}\mp \frac{i}{\hbar\alpha}\,\hat{p}\right),
\quad \text{with} \quad \alpha=\sqrt{\frac{m\omega}{\hbar}}.
\]
This relation can be inverted to give the position operator:
\[
\alpha\,\hat{x} = \frac{1}{\sqrt{2}}(\hat{a}_+ + \hat{a}_-)
\quad\Longrightarrow\quad
\hat{x} = \frac{1}{\sqrt{2}\,\alpha}\Big(\hat{a}_+ + \hat{a}_-\Big).
\]
Thus, the perturbation
\[
\hat{H}' = q\,\mathcal{E}\,\hat{x}
\]
can be written as
\[
\hat{H}' = \frac{q\,\mathcal{E}}{\sqrt{2}\,\alpha}\Big(\hat{a}_+ + \hat{a}_-\Big).
\]

% (c) Variational minimization to determine the optimal b
% When the perturbation is included, the total energy expectation (to first order in the mixing) becomes
\[
E(b) = \frac{\langle \psi | \hat{H}+\hat{H}' | \psi \rangle}{\langle \psi|\psi\rangle}
=\frac{E_0 + b^2 E_1 + 2b\,V_{01}}{1+b^2},
\]
where the off–diagonal matrix element is
\[
V_{01} = \langle 0|\hat{H}'|1\rangle.
\]
Using the expression for \(\hat{H}'\) and the properties of the ladder operators,
\[
\langle 0|\hat{a}_-|1\rangle = 1,\quad \langle 0|\hat{a}_+|1\rangle = 0,
\]
we find
\[
V_{01} = \frac{q\,\mathcal{E}}{\sqrt{2}\,\alpha} \equiv \frac{g}{2},
\]
where we have defined
\[
g = \frac{\sqrt{2}\,q\,\mathcal{E}}{\alpha}.
\]
Thus, the energy expectation is
\[
E(b) = \frac{E_0 + b^2 E_1 + g\,b}{1+b^2}.
\]

To find the optimal value of \(b\) we minimize \(E(b)\) with respect to \(b\). Differentiating with respect to \(b\) and setting the derivative to zero gives
\[
\frac{dE}{db} = \frac{(1+b^2)(2bE_1+g) - (E_0 + b^2E_1 + g\,b)(2b)}{(1+b^2)^2} = 0.
\]
Carrying out the algebra we obtain
\[
2b(E_1-E_0) + g\,(1-b^2) = 0.
\]
Rearrange this to write a quadratic equation in \(b\):
\[
g\,b^2 - 2b\,(E_1-E_0) - g = 0.
\]
Dividing through by \(g\) (assuming \(g\ne0\)) yields
\[
b^2 - \frac{2(E_1-E_0)}{g}\,b - 1 = 0.
\]
The quadratic formula then gives
\[
b = \frac{\frac{2(E_1-E_0)}{g} \pm \sqrt{\frac{4(E_1-E_0)^2}{g^2}+4}}{2}
=\frac{E_1-E_0}{g} \pm \sqrt{\frac{(E_1-E_0)^2}{g^2}+1}.
\]
Since perturbation theory indicates that the mixing coefficient should be small (in fact, first-order perturbation theory gives \(b\approx -\frac{V_{01}}{E_1-E_0}=-\frac{g}{2(E_1-E_0)}\) for small \(g\)), we choose the minus sign. Thus, the variationally optimal parameter is
\[
b = \frac{E_1-E_0}{g} - \sqrt{\frac{(E_1-E_0)^2}{g^2}+1}.
\]

% Note on the result:
% The expression derived here features a “+1” inside the square root. (The problem statement’s quoted result
% with a “-1” inside the square root appears to be inconsistent with the standard variational derivation.)


\section{Electrodynamics}


\subsection{Question 1}

Name the three types of magnetic materials. Explain what determines their response when brought near a bar magnet. 

\paragraph{Prompt: \\} 
Answer the given question as a top physics student, giving step-by-step reasoning keeping your answer under 300 words. Format your answer as LaTeX code without using sub-sections. Available marks: [8 marks]

\subsubsection{Solution 1}

The three types of magnetic materials are \textbf{ferromagnetic}, \textbf{paramagnetic}, and \textbf{diamagnetic}. 

\textbf{Step 1: Ferromagnetic Materials.} These materials (e.g., iron, cobalt, nickel) have magnetic domains with moments that are spontaneously aligned even without an external field. When a bar magnet is brought near, the domains reorient further to strengthen the overall magnetic field, resulting in a strong attraction.

\textbf{Step 2: Paramagnetic Materials.} In these materials, atoms or ions have unpaired electrons that create small magnetic moments. In the absence of a magnetic field, these moments are randomly oriented, canceling out overall. When exposed to a bar magnet, the external field induces a slight alignment of these moments, causing a weak attraction.

\textbf{Step 3: Diamagnetic Materials.} These materials have paired electrons, so no permanent magnetic moment exists. Instead, an external magnetic field induces a current that produces a magnetic moment opposing the applied field, leading to a weak repulsion when near a bar magnet.

The response of each material type is determined by the alignment (or lack thereof) of the magnetic moments of electrons in the atoms. In ferromagnets, strong exchange interactions cause spontaneous alignment; in paramagnets, the alignment is induced weakly by the external field; and in diamagnets, the induced magnetic moment opposes the external influence.

\subsubsection{Solution 2}

The three types of magnetic materials are \textbf{diamagnetic}, \textbf{paramagnetic}, and \textbf{ferromagnetic}. 

\textbf{Step 1: Diamagnetic Materials.} These materials have no unpaired electrons, causing them to develop an induced magnetic moment in the opposite direction to an external magnetic field. When brought near a bar magnet, they are weakly repelled. 

\textbf{Step 2: Paramagnetic Materials.} In these materials, the presence of unpaired electrons leads to small magnetic moments that align with an external magnetic field. This results in a weak attraction towards the bar magnet. The degree of alignment is influenced by thermal motion and the strength of the applied field.

\textbf{Step 3: Ferromagnetic Materials.} Ferromagnetic substances have unpaired electrons that not only align with an external magnetic field but also interact strongly with each other via exchange interactions. This results in a strong net magnetic moment and a robust attraction to the bar magnet. They can retain magnetization even after the external field is removed, a property known as hysteresis.

\textbf{Conclusion:} The response of these materials to a bar magnet depends on their electronic structure, specifically the presence and behavior of unpaired electrons, and the interactions (or lack thereof) between the magnetic moments of the atoms. Diamagnetic materials oppose the field, paramagnetic materials align weakly with the field, and ferromagnetic materials align strongly due to cooperative interactions. 

\subsubsection{Solution 3}
 The three types of magnetic materials are diamagnetic, paramagnetic, and ferromagnetic.
% 
 1. Diamagnetic materials: These have no permanent magnetic dipole moment. Their electrons circulate in such a way that they create a small magnetic field opposite to an applied field, leading to a weak repulsion from a bar magnet.
% 
 2. Paramagnetic materials: These contain unpaired electrons which create permanent dipoles. However, these dipoles are randomly oriented in the absence of an external magnetic field. When a bar magnet is nearby, the dipoles align with the field, causing a weak attraction.
% 
 3. Ferromagnetic materials: These have regions (domains) with aligned magnetic moments due to strong exchange interactions. Even without an external field, they retain a net magnetic moment. When a bar magnet is brought near, they exhibit a strong attraction, as the domains align with the applied field.
% 
 The response of a material to a bar magnet is determined by its magnetic susceptibility, which quantifies how easily its internal magnetic dipoles align with the external magnetic field. In diamagnetic materials the susceptibility is negative, in paramagnetic materials it is positive but small, and in ferromagnetic materials it is large and positive.
\textbf{Step-by-step reasoning:}
\begin{enumerate}
    \item Identify the type of material based on the presence or absence of permanent magnetic dipoles.
    \item Evaluate how the dipoles respond: diamagnetism (weak, induced opposing field), paramagnetism (weak alignment leading to attraction), or ferromagnetism (strong domain alignment leading to strong attraction).
    \item Understand that the key parameter is the magnetic susceptibility, which dictates the degree of magnetization in an applied field.
\end{enumerate}
This explanation clarifies how the intrinsic magnetic properties and electron configurations in these materials determine their behavior in an external magnetic field.


\subsection{Question 2}
\begin{enumerate}
    \item Within the context of classical electromagnetism, briefly define what is 'free space'. In free space, the electric field, $\mathbf{E}$, satisfies 3 partial differential equations, one for each of its components:
    \[
    \nabla^2 \mathbf{E} = \epsilon_0 \mu_0 \frac{\partial^2 \mathbf{E}}{\partial t^2}
    \]
    \item Briefly describe the physical interpretation of this equation for showing how electric fields can propagate in space.
    \item Explain how this equation gives the speed of light $c$ and state its value.
\end{enumerate}


\paragraph{Prompt: \\} 
Answer the given question as a top physics student, giving step-by-step reasoning keeping your answer under 300 words. Format your answer as LaTeX code without using sub-sections. Available marks: [5 marks]

\subsubsection{Solution 1}
\textbf{Answer:} \\
Free space is an idealized region of space that is completely devoid of free charges and currents, meaning that only the inherent properties of space—characterized by the vacuum permittivity $\epsilon_0$ and permeability $\mu_0$—govern electromagnetic phenomena. In such a medium, Maxwell's equations simplify, and the electric field $\mathbf{E}$ obeys the homogeneous wave equation:
\[
\nabla^2 \mathbf{E} = \epsilon_0 \mu_0 \frac{\partial^2 \mathbf{E}}{\partial t^2}.
\]
This equation implies that the spatial variation (described by the Laplacian $\nabla^2 \mathbf{E}$) of the electric field is directly proportional to its second time derivative, meaning that any disturbance in $\mathbf{E}$ propagates through space as a wave. \\
\\
By comparing the above with the standard wave equation,
\[
\nabla^2 \mathbf{E} = \frac{1}{v^2} \frac{\partial^2 \mathbf{E}}{\partial t^2},
\]
we identify
\[
\frac{1}{v^2} = \epsilon_0 \mu_0 \quad \Rightarrow \quad v = \frac{1}{\sqrt{\epsilon_0 \mu_0}}.
\]
This speed $v$ is recognized as the speed of light in free space. Using the known values of $\epsilon_0$ and $\mu_0$, one finds
\[
c = \frac{1}{\sqrt{\epsilon_0 \mu_0}} \approx 3.00 \times 10^8 \, \text{m/s}.
\]
Thus, the equation not only shows that electric field disturbances propagate as waves but also establishes that electromagnetic waves travel at the speed of light, $c$, in free space.

\subsubsection{Solution 2}
% Answer to Question 2

\begin{enumerate}
    \item \textbf{Definition of Free Space:} Free space is an idealized region devoid of any matter or charge, where electromagnetic fields exist only in vacuum. Its properties are characterized by the vacuum permittivity $\epsilon_0$ and vacuum permeability $\mu_0$.
    
    \item \textbf{Physical Interpretation:} The wave equation
    \[
    \nabla^2 \mathbf{E} = \epsilon_0 \mu_0 \frac{\partial^2 \mathbf{E}}{\partial t^2}
    \]
    indicates that any spatial variation (curvature) in the electric field $\mathbf{E}$ leads to changes in its temporal acceleration. This means that a disturbance in the electric field will propagate outward as an electromagnetic wave, carrying energy through space even in the absence of material media.
    
    \item \textbf{Speed of Light:} By comparing the given wave equation with the standard form of a wave equation,
    \[
    \nabla^2 \mathbf{E} = \frac{1}{c^2}\frac{\partial^2 \mathbf{E}}{\partial t^2},
    \]
    we identify
    \[
    \frac{1}{c^2} = \epsilon_0 \mu_0.
    \]
    Therefore, solving for $c$ yields
    \[
    c = \frac{1}{\sqrt{\epsilon_0 \mu_0}}.
    \]
    Substituting the values $\epsilon_0 \approx 8.854 \times 10^{-12}\,\mathrm{F/m}$ and $\mu_0 = 4\pi \times 10^{-7}\,\mathrm{H/m}$, we obtain
    \[
    c \approx 3 \times 10^8 \,\mathrm{m/s}.
    \]
\end{enumerate}

\subsubsection{Solution 3}
\textbf{Answer:} 

In classical electromagnetism, \emph{free space} is defined as a region devoid of any matter (no charges or currents), characterized solely by the constants \(\epsilon_0\) (vacuum permittivity) and \(\mu_0\) (vacuum permeability). 

The given equation 
\[
\nabla^2 \mathbf{E} = \epsilon_0 \mu_0 \frac{\partial^2 \mathbf{E}}{\partial t^2}
\]
is the wave equation for the electric field \(\mathbf{E}\) in free space. Here, the Laplacian \(\nabla^2 \mathbf{E}\) describes the spatial variation of the field, while the second time derivative \(\frac{\partial^2 \mathbf{E}}{\partial t^2}\) indicates how the field changes with time. This formulation shows that a disturbance in the electric field will propagate as a wave through space.

To extract the speed of propagation, note that the standard form of the wave equation is 
\[
\nabla^2 \mathbf{E} = \frac{1}{v^2}\frac{\partial^2 \mathbf{E}}{\partial t^2},
\]
where \(v\) is the wave speed. By equating the coefficients, we obtain 
\[
\frac{1}{v^2} = \epsilon_0 \mu_0 \quad \Rightarrow \quad v = \frac{1}{\sqrt{\epsilon_0 \mu_0}}.
\]
This speed \(v\) is identified with the speed of light \(c\), whose value is approximately 
\[
c \approx 3.00 \times 10^8 \, \text{m/s}.
\]
Thus, the equation not only shows that electric fields propagate as waves but also establishes that these waves travel at the speed of light.


\subsection{Question 3}
\begin{enumerate}
    \item Find the electric field a distance, \( d \), above one end of a straight line segment of length \( L \) that has a constant line charge density \( \lambda \) (as seen in the figure).
    \item In the limit of \( d \gg L \), what does the electric field reduce to and what is its physical interpretation?
\end{enumerate}


\paragraph{Prompt: \\} 
Answer the given question as a top physics student, giving step-by-step reasoning. Format your answer as LaTeX code without using sub-sections. Available marks: [14 marks]

\subsubsection{Solution 1}
\textbf{Solution:}

We consider a line segment of length \(L\) lying along the \(x\)-axis from \(x=0\) to \(x=L\) with constant line charge density \(\lambda\). We wish to find the electric field at the point \(P\) with coordinates \((0,d)\) (i.e. a distance \(d\) vertically above the left end of the segment).

An infinitesimal charge element is \(dq=\lambda\,dx\) at position \(x\). The vector from \(dq\) to \(P\) is 
\[
\vec{r} = -x\,\hat{i} + d\,\hat{j},
\]
and its magnitude is 
\[
r=\sqrt{x^2+d^2}.
\]

Using Coulomb’s law, the differential electric field is
\[
d\vec{E} = \frac{1}{4\pi\epsilon_0}\,\frac{dq}{r^2}\,\frac{\vec{r}}{r} 
= \frac{1}{4\pi\epsilon_0}\,\frac{\lambda\,dx}{(x^2+d^2)}\,\frac{-x\,\hat{i}+d\,\hat{j}}{\sqrt{x^2+d^2}}
= \frac{\lambda}{4\pi\epsilon_0}\,\frac{-x\,\hat{i}+d\,\hat{j}}{(x^2+d^2)^{3/2}}\,dx.
\]

\emph{(i) Components of the Field:}

The \(x\)-component is
\[
E_x = -\frac{\lambda}{4\pi\epsilon_0}\int_{0}^{L} \frac{x\,dx}{(x^2+d^2)^{3/2}}.
\]
Using the substitution (or recognizing the standard integral)
\[
\int \frac{x\,dx}{(x^2+d^2)^{3/2}} = -\frac{1}{\sqrt{x^2+d^2}},
\]
we obtain
\[
E_x = -\frac{\lambda}{4\pi\epsilon_0}\left[-\frac{1}{\sqrt{x^2+d^2}}\right]_{0}^{L}
=\frac{\lambda}{4\pi\epsilon_0}\left(\frac{1}{d}-\frac{1}{\sqrt{L^2+d^2}}\right).
\]

The \(y\)-component is
\[
E_y = \frac{\lambda d}{4\pi\epsilon_0}\int_{0}^{L}\frac{dx}{(x^2+d^2)^{3/2}}.
\]
Recognizing that
\[
\int \frac{dx}{(x^2+d^2)^{3/2}} = \frac{x}{d^2\sqrt{x^2+d^2}},
\]
we have
\[
E_y = \frac{\lambda d}{4\pi\epsilon_0}\left[\frac{x}{d^2\sqrt{x^2+d^2}}\right]_{0}^{L}
=\frac{\lambda L}{4\pi\epsilon_0\,d\sqrt{L^2+d^2}}.
\]

Thus, the net electric field is
\[
\vec{E} = E_x\,\hat{i} + E_y\,\hat{j} 
=\frac{\lambda}{4\pi\epsilon_0}\left(\frac{1}{d}-\frac{1}{\sqrt{L^2+d^2}}\right)\hat{i} 
+\frac{\lambda L}{4\pi\epsilon_0\,d\sqrt{L^2+d^2}}\hat{j}.
\]

\emph{(ii) Far-Field Limit (\(d\gg L\)):}

When \(d\gg L\), we have
\[
\sqrt{L^2+d^2}\approx d\left(1+\frac{L^2}{2d^2}\right)\approx d.
\]
Thus,
\[
E_x \approx \frac{\lambda}{4\pi\epsilon_0}\left(\frac{1}{d}-\frac{1}{d}\right)=0,
\]
and
\[
E_y \approx \frac{\lambda L}{4\pi\epsilon_0\,d\cdot d}
=\frac{\lambda L}{4\pi\epsilon_0\,d^2}.
\]

The resulting field,
\[
\vec{E}\approx \frac{\lambda L}{4\pi\epsilon_0\,d^2}\,\hat{j},
\]
is identical in form to the electric field produced by a point charge \(Q=\lambda L\). This confirms that in the limit \(d\gg L\) the line segment behaves as if all its charge were concentrated at a point.

\textbf{Marks Allocation:} \\
Each major step (setup, integration for \(E_x\) and \(E_y\), and the far-field limit interpretation) is worth approximately \(3-4\) marks, totaling 14 marks.


\subsubsection{Solution 2}
\textbf{Step 1. Setting up the geometry:} \\
Consider a straight line segment extending from \(x=0\) to \(x=L\) along the \(x\)-axis with constant line charge density \(\lambda\). We wish to calculate the electric field at a point \(P(0,d)\) a distance \(d\) above the left end of the segment.

\bigskip
\textbf{Step 2. Expressing the differential field:} \\
An infinitesimal element \(dx\) at position \(x\) carries charge 
\[
dq = \lambda\,dx.
\]
The distance from this element to \(P\) is 
\[
r = \sqrt{x^2 + d^2}.
\]
By Coulomb’s law, the magnitude of the differential electric field is
\[
dE = \frac{1}{4\pi\varepsilon_0}\frac{dq}{r^2} = \frac{1}{4\pi\varepsilon_0}\frac{\lambda\,dx}{(x^2+d^2)}.
\]
The field at \(P\) has both \(x\) (horizontal) and \(y\) (vertical) components. The angles are given by
\[
\cos\theta = \frac{x}{\sqrt{x^2+d^2}},\quad \sin\theta = \frac{d}{\sqrt{x^2+d^2}}.
\]
Thus, the differential components are:
\[
dE_x = dE\cos\theta = \frac{1}{4\pi\varepsilon_0}\frac{\lambda\,x\,dx}{(x^2+d^2)^{3/2}},
\]
\[
dE_y = dE\sin\theta = \frac{1}{4\pi\varepsilon_0}\frac{\lambda\,d\,dx}{(x^2+d^2)^{3/2}}.
\]

\bigskip
\textbf{Step 3. Integration over the line segment:} \\
Integrate \(dE_x\) from \(x=0\) to \(x=L\):
\[
E_x = \frac{\lambda}{4\pi\varepsilon_0}\int_0^L \frac{x\,dx}{(x^2+d^2)^{3/2}}.
\]
Using the substitution \(u=x^2+d^2\) (with \(du=2x\,dx\)), we obtain
\[
\int \frac{x\,dx}{(x^2+d^2)^{3/2}} = -\frac{1}{\sqrt{x^2+d^2}},
\]
so that
\[
E_x = \frac{\lambda}{4\pi\varepsilon_0}\left[\frac{1}{d}-\frac{1}{\sqrt{L^2+d^2}}\right].
\]

Next, integrate \(dE_y\):
\[
E_y = \frac{\lambda d}{4\pi\varepsilon_0}\int_0^L \frac{dx}{(x^2+d^2)^{3/2}}.
\]
A standard integral result is
\[
\int \frac{dx}{(x^2+d^2)^{3/2}} = \frac{x}{d^2\sqrt{x^2+d^2}},
\]
thus
\[
E_y = \frac{\lambda d}{4\pi\varepsilon_0}\left[\frac{L}{d^2\sqrt{L^2+d^2}}-\frac{0}{d^2\sqrt{d^2}}\right] = \frac{\lambda L}{4\pi\varepsilon_0\,d\sqrt{L^2+d^2}}.
\]

\bigskip
\textbf{Step 4. The resultant electric field:} \\
Combining the components, the electric field at the point \(P\) is
\[
\vec{E} = E_x\,\hat{i} + E_y\,\hat{j} = \frac{\lambda}{4\pi\varepsilon_0}\left[\left(\frac{1}{d}-\frac{1}{\sqrt{L^2+d^2}}\right)\hat{i} + \frac{L}{d\sqrt{L^2+d^2}}\hat{j}\right].
\]

\bigskip
\textbf{Step 5. The limit \(d \gg L\):} \\
When \(d \gg L\), we can approximate \(\sqrt{L^2+d^2} \approx d\). Thus:
\[
E_x \approx \frac{\lambda}{4\pi\varepsilon_0}\left(\frac{1}{d}-\frac{1}{d}\right)= 0,
\]
\[
E_y \approx \frac{\lambda L}{4\pi\varepsilon_0\,d^2}.
\]
Therefore, for \(d \gg L\) the electric field reduces to
\[
\vec{E} \approx \frac{\lambda L}{4\pi\varepsilon_0}\frac{1}{d^2}\,\hat{j}.
\]

\bigskip
\textbf{Physical Interpretation:} \\
The expression \(\frac{\lambda L}{4\pi\varepsilon_0}\) represents the total charge \(Q\) on the line segment. Hence, at large distances, the line segment behaves like a point charge \(Q=\lambda L\) located at the origin, producing the familiar inverse-square law field.


\subsubsection{Solution 3}
% Step-by-step solution for Question 3

% (a) Calculation of the electric field

% We consider a uniformly charged line segment of length \( L \) lying along the \( x \)-axis (from \( x=0 \) to \( x=L \)) with constant linear charge density \( \lambda \). The observation point is located at \( (0,d) \) above the end at \( x=0 \).

% An infinitesimal element of charge is \( dq=\lambda\,dx \) at position \( x \) along the \( x \)-axis.
% The distance from this element to the observation point is
\[
r=\sqrt{x^2+d^2}.
\]
% The infinitesimal electric field due to \( dq \) is given by Coulomb’s law:
\[
d\vec{E}=\frac{1}{4\pi\epsilon_0}\frac{dq}{r^2}\hat{r}=\frac{1}{4\pi\epsilon_0}\frac{\lambda\,dx}{(x^2+d^2)}\hat{r},
\]
% where the unit vector from the source to the field point is
\[
\hat{r}=\frac{(-x\,\hat{i}+d\,\hat{j})}{\sqrt{x^2+d^2}}.
\]
% Hence, the components of \( d\vec{E} \) are:
\[
dE_x=-\frac{\lambda\,x}{4\pi\epsilon_0}\frac{dx}{(x^2+d^2)^{3/2}},\quad
dE_y=\frac{\lambda\,d}{4\pi\epsilon_0}\frac{dx}{(x^2+d^2)^{3/2}}.
\]

% We now integrate \( x \) from 0 to \( L \):

\[
E_x=-\frac{\lambda}{4\pi\epsilon_0}\int_{0}^{L}\frac{x\,dx}{(x^2+d^2)^{3/2}},
\]
\[
E_y=\frac{\lambda\,d}{4\pi\epsilon_0}\int_{0}^{L}\frac{dx}{(x^2+d^2)^{3/2}}.
\]

% To evaluate \( E_x \):
Let \( u=x^2+d^2 \), so that \( du=2x\,dx \). Then,
\[
\int\frac{x\,dx}{(x^2+d^2)^{3/2}}=\frac{1}{2}\int u^{-3/2}\,du = \frac{1}{2}\left(-\frac{2}{\sqrt{u}}\right)=-\frac{1}{\sqrt{u}}.
\]
% Evaluating from 0 to \( L \):
\[
\int_{0}^{L}\frac{x\,dx}{(x^2+d^2)^{3/2}} = -\frac{1}{\sqrt{L^2+d^2}} + \frac{1}{d}.
\]
% Therefore,
\[
E_x=-\frac{\lambda}{4\pi\epsilon_0}\left(-\frac{1}{\sqrt{L^2+d^2}}+\frac{1}{d}\right)=\frac{\lambda}{4\pi\epsilon_0}\left(\frac{1}{\sqrt{L^2+d^2}}-\frac{1}{d}\right).
\]

% For \( E_y \), we use the standard integral:
\[
\int \frac{dx}{(x^2+d^2)^{3/2}} = \frac{x}{d^2\sqrt{x^2+d^2}},
\]
% so that
\[
\int_{0}^{L}\frac{dx}{(x^2+d^2)^{3/2}} = \frac{L}{d^2\sqrt{L^2+d^2}}.
\]
% Hence,
\[
E_y=\frac{\lambda\,d}{4\pi\epsilon_0}\frac{L}{d^2\sqrt{L^2+d^2}}=\frac{\lambda\,L}{4\pi\epsilon_0\,d\sqrt{L^2+d^2}}.
\]

% Combining the components, the electric field at the point is:
\[
\boxed{
\vec{E}=\frac{\lambda}{4\pi\epsilon_0}\left[\left(\frac{1}{\sqrt{L^2+d^2}}-\frac{1}{d}\right)\hat{i}+\frac{L}{d\sqrt{L^2+d^2}}\hat{j}\right].
}
\]

% (b) Limit \( d\gg L \)

% When \( d\gg L \), we have \( \sqrt{L^2+d^2}\approx d \) since \( L^2 \) is negligible compared to \( d^2 \).

% Then, the \( x \)-component becomes:
\[
E_x\approx\frac{\lambda}{4\pi\epsilon_0}\left(\frac{1}{d}-\frac{1}{d}\right)=0,
\]
% and the \( y \)-component simplifies to:
\[
E_y\approx\frac{\lambda\,L}{4\pi\epsilon_0\,d^2}.
\]
% Recognizing that the total charge \( Q=\lambda\,L \), we obtain:
\[
\boxed{
\vec{E}\approx\frac{Q}{4\pi\epsilon_0\,d^2}\hat{j}.
}
\]
% This is the electric field of a point charge \( Q \) located at the origin, showing that at large distances the finite charged segment behaves as a point charge.


\subsection{Question 4}
A square loop of wire (sides of length $L$) lies a fixed distance $d$ from a very long straight wire, which carries a constant current $I$ directed downwards (as seen in the figure).

\begin{enumerate}
    \item[(a)] Find the flux of $\mathbf{B}$ through the loop.
    \item[(b)] If the loop is moved directly away from the wire to the right, at speed $v$, what emf is generated? In what direction (clockwise or counterclockwise) does the current flow?
    \item[(c)] What happens if the loop only moves downwards at speed $v$?
\end{enumerate}


\paragraph{Prompt: \\} 
Answer the given question as a top physics student, giving step-by-step reasoning. Format your answer as LaTeX code without using sub-sections. Available marks: a) [5 marks], b) [5 marks], c [1 mark].

\subsubsection{Solution 1}
\[
\textbf{(a)} \quad \text{The magnetic field at a distance } r \text{ from the long straight wire is } 
B(r)=\frac{\mu_0 I}{2\pi r}.
\]
\[
\text{The flux through the square loop (of side } L\text{), with its nearest side at } r=d, \text{ is:}
\]
\[
\Phi = \int_{r=d}^{d+L} B(r)\,L\,dr 
= \frac{\mu_0 I L}{2\pi} \int_{d}^{d+L} \frac{dr}{r} 
= \frac{\mu_0 I L}{2\pi} \ln\frac{d+L}{d}.
\]

\[
\textbf{(b)} \quad \text{If the loop is moved directly away (to the right) at speed } v, \text{ then } d \text{ becomes time-dependent: } d(t)=d+vt.
\]
\[
\text{Differentiating the flux: }
\]
\[
\frac{d\Phi}{dt}=\frac{\mu_0 I L}{2\pi}\frac{d}{dt}\left(\ln\frac{d+L}{d}\right)
=\frac{\mu_0 I L}{2\pi}\left(\frac{1}{d+L}\frac{d(d+L)}{dt}-\frac{1}{d}\frac{dd}{dt}\right).
\]
\[
\text{Since } \frac{d(d+L)}{dt}=\frac{dd}{dt}=v, \text{ this simplifies to:}
\]
\[
\frac{d\Phi}{dt}=\frac{\mu_0 I L}{2\pi}\left(\frac{v}{d+L}-\frac{v}{d}\right)
=\frac{\mu_0 I L v}{2\pi}\left(\frac{1}{d+L}-\frac{1}{d}\right).
\]
\[
\text{The induced emf is given by Faraday's law: }
\]
\[
\mathcal{E}=-\frac{d\Phi}{dt}=\frac{\mu_0 I L v}{2\pi}\left(\frac{1}{d}-\frac{1}{d+L}\right).
\]
\[
\text{Since the loop is moving away, the flux (which is into the page) is decreasing. By Lenz's law, the induced current will act to oppose this change by producing a magnetic field into the page. Using the right-hand rule, this requires a } \textbf{clockwise} \text{ current.}
\]

\[
\textbf{(c)} \quad \text{If the loop moves downwards at speed } v, \text{ its horizontal distance } d \text{ from the wire remains unchanged. Hence, the magnetic field (and thus the flux) through the loop remains constant, and no emf is induced.}
\]

\subsubsection{Solution 2}
\textbf{(a)} The magnetic field at a distance \(x\) from the long straight wire is
\[
B(x)=\frac{\mu_0 I}{2\pi x}.
\]
Since the loop is square with side length \(L\) and extends from \(x=d\) to \(x=d+L\), the magnetic flux through the loop is
\[
\Phi = \int_{x=d}^{d+L} B(x)\, L\,dx = L\int_{d}^{d+L}\frac{\mu_0 I}{2\pi x}\,dx = \frac{\mu_0 I L}{2\pi}\ln\left(\frac{d+L}{d}\right).
\]

\textbf{(b)} When the loop is moved to the right (away from the wire) at speed \(v\), the distance \(d\) becomes time-dependent: \(d(t)=d+vt\). Thus, the flux becomes
\[
\Phi(t)=\frac{\mu_0 I L}{2\pi}\ln\left(\frac{d+vt+L}{d+vt}\right).
\]
Differentiating with respect to time gives
\[
\frac{d\Phi}{dt}=\frac{\mu_0 I L}{2\pi}\left[\frac{v}{d+vt+L}-\frac{v}{d+vt}\right].
\]
By Faraday’s law, the induced emf is
\[
\mathcal{E}=-\frac{d\Phi}{dt}=-\frac{\mu_0 I L v}{2\pi}\left[\frac{1}{d+vt+L}-\frac{1}{d+vt}\right].
\]
At \(t=0\), this simplifies to
\[
\mathcal{E}=\frac{\mu_0 I L v}{2\pi}\left(\frac{1}{d}-\frac{1}{d+L}\right).
\]
The flux through the loop is directed into the page (since the magnetic field from a downward current circles clockwise and points into the page on the right side). As the loop moves away, the flux decreases, so by Lenz’s law the induced current must create a magnetic field into the page. Using the right‐hand rule, a clockwise current produces a magnetic field into the page. Therefore, the induced current flows \textbf{clockwise}.

\textbf{(c)} If the loop moves downward at speed \(v\), the horizontal distance \(d\) from the wire remains unchanged. Hence, the magnetic field and the flux through the loop do not change with time and the induced emf is \(\mathcal{E}=0\).


\subsubsection{Solution 3}
\[
\textbf{(a)} \quad \text{The magnetic field due to the long wire at a distance } x \text{ is} \quad B(x)=\frac{\mu_0 I}{2\pi x}.
\]
\[
\text{Since the loop has side length } L, \text{ its horizontal extent is from } x=d \text{ to } x=d+L. \text{ Thus, the magnetic flux } \Phi \text{ is}
\]
\[
\Phi = \int_{x=d}^{d+L} B(x) \, L\, dx = L\int_{d}^{d+L}\frac{\mu_0 I}{2\pi x}\, dx 
= L\frac{\mu_0 I}{2\pi}\left[\ln x\right]_{d}^{d+L}
= L\frac{\mu_0 I}{2\pi}\ln\left(\frac{d+L}{d}\right).
\]
\[
\textbf{(b)} \quad \text{When the loop moves to the right at speed } v, \text{ the distance } d \text{ becomes time-dependent, } d(t)=d+vt.
\]
\[
\text{Thus, the flux becomes } \Phi(t)=L\frac{\mu_0 I}{2\pi}\ln\left(\frac{d+L+vt}{d+vt}\right).
\]
\[
\text{The induced emf is given by Faraday's law: } \mathcal{E}=-\frac{d\Phi}{dt}.
\]
\[
\frac{d\Phi}{dt}=L\frac{\mu_0 I}{2\pi}\frac{d}{dt}\ln\left(\frac{d+L+vt}{d+vt}\right)
=L\frac{\mu_0 I}{2\pi}\left(\frac{v}{d+L+vt}-\frac{v}{d+vt}\right).
\]
\[
\text{At a given instant (with } d+vt \text{ replaced by } d \text{ for simplicity), we write}
\]
\[
\frac{d\Phi}{dt}=L\frac{\mu_0 I v}{2\pi}\left(\frac{1}{d+L}-\frac{1}{d}\right)=-L\frac{\mu_0 I v}{2\pi}\left(\frac{1}{d}-\frac{1}{d+L}\right).
\]
\[
\text{Thus, the magnitude of the emf is}
\]
\[
\mathcal{E}=L\frac{\mu_0 I v}{2\pi}\left(\frac{1}{d}-\frac{1}{d+L}\right).
\]
\[
\text{Since the flux (directed into the page) decreases as the loop moves away, Lenz's law requires the induced current to produce a magnetic field into the page. Using the right‐hand rule, this corresponds to a clockwise current.}
\]
\[
\textbf{(c)} \quad \text{If the loop moves downward at speed } v, \text{ the horizontal distances from the wire remain unchanged. Hence, the magnetic field at every point in the loop remains constant, and the flux does not change. Therefore, no emf is induced.}
\]

\subsection{Question 5}
Consider the circuit diagram seen in the figure.
\begin{enumerate}
    \item Calculate the current seen by the ammeter.
    \item Calculate the energy delivered by the 12 V battery in 4 seconds.
\end{enumerate}


\paragraph{Prompt: \\} 
Answer the given question as a top physics student, giving step-by-step reasoning. Format your answer as LaTeX code without using sub-sections. Available marks: a) [8 marks], b) [3 marks].

\subsubsection{Solution 1}
\[
\textbf{Solution:}
\]
First, label the negative terminal of the 12 V source as \(0\,\mathrm{V}\) so that its positive terminal is at \(12\,\mathrm{V}\). Let the node after the \(2\,\Omega\) resistor from the \(12\,\mathrm{V}\) battery be at potential \(V_A\). There are then two parallel branches from \(V_A\) down to \(0\,\mathrm{V}\):

\[
\begin{aligned}
&\text{(i) A series combination of a }2\,\Omega\text{ resistor and a }2\,\mathrm{V}\text{ source.}\\
&\text{(ii) Two }1\,\Omega\text{ resistors in series (total }2\,\Omega\text{).}
\end{aligned}
\]

Denote:
\[
I_1 = \text{current through the }2\,\Omega\text{ resistor from the }12\,\mathrm{V}\text{ battery (the same current the ammeter reads),}
\]
\[
I_2 = \text{current in the branch with the }2\,\Omega\text{ resistor and }2\,\mathrm{V}\text{ source,}
\]
\[
I_3 = \text{current in the branch with the two }1\,\Omega\text{ resistors.}
\]

\textbf{Step 1: Voltage drop from the 12 V battery to node \(A\).}

Because the top resistor is \(2\,\Omega\) and carries current \(I_1\),
\[
V_A = 12\,\mathrm{V} - 2\,\Omega \cdot I_1.
\]

\textbf{Step 2: KVL in each parallel branch.}

\[
\text{(a) For the branch with the }2\,\Omega\text{ resistor and }2\,\mathrm{V}\text{ source:}
\]
Traversing from \(V_A\) (top) to \(0\,\mathrm{V}\) (bottom) in the direction of current \(I_2\), the voltage drop is
\[
2\,\Omega \cdot I_2 + 2\,\mathrm{V} = V_A.
\]
Hence,
\[
V_A = 2I_2 + 2.
\]

\[
\text{(b) For the branch with two }1\,\Omega\text{ resistors in series (total }2\,\Omega\text{):}
\]
\[
2\,\Omega \cdot I_3 = V_A.
\]
Hence,
\[
V_A = 2I_3.
\]

\textbf{Step 3: Current conservation at node \(A\).}

\[
I_1 = I_2 + I_3.
\]

\textbf{Step 4: Combine the equations.}

From the two parallel branches, we get
\[
2I_2 + 2 = 2I_3
\quad\Longrightarrow\quad
I_3 = I_2 + 1.
\]
Also, from the top resistor expression,
\[
V_A = 12 - 2I_1,
\]
and we know \(V_A = 2I_2 + 2\). Hence
\[
12 - 2I_1 = 2I_2 + 2
\quad\Longrightarrow\quad
10 = 2I_1 + 2I_2
\quad\Longrightarrow\quad
5 = I_1 + I_2.
\]
But
\[
I_1 = I_2 + I_3 = I_2 + (I_2 + 1) = 2I_2 + 1.
\]
Substitute into \(5 = I_1 + I_2\):
\[
5 = (2I_2 + 1) + I_2 = 3I_2 + 1 
\quad\Longrightarrow\quad
3I_2 = 4
\quad\Longrightarrow\quad
I_2 = \frac{4}{3}\,\mathrm{A}.
\]
Hence
\[
I_3 = I_2 + 1 = \frac{4}{3} + 1 = \frac{7}{3}\,\mathrm{A},
\]
\[
I_1 = 2I_2 + 1 = 2\cdot\frac{4}{3} + 1 = \frac{8}{3} + 1 = \frac{11}{3}\,\mathrm{A}.
\]

\textbf{Answer to (a):} The ammeter reads
\[
I_1 \;=\; \frac{11}{3}\,\mathrm{A}\;\approx\;3.67\,\mathrm{A}.
\]

\textbf{Answer to (b):} The power delivered by the 12 V battery is
\[
P = (12\,\mathrm{V}) \times \left(\frac{11}{3}\,\mathrm{A}\right)
= 44\,\mathrm{W}.
\]
In 4 seconds, the energy delivered is
\[
E = P \times t = 44\,\mathrm{J/s} \times 4\,\mathrm{s} = 176\,\mathrm{J}.
\]

\subsubsection{Solution 2}
\[
\textbf{Solution:}
\]

\[
\textit{Notation: Let the battery’s negative terminal be }0\,\mathrm{V},\,
\text{and its positive terminal be }12\,\mathrm{V}.
\]
From the diagram, label the top node (after the \(2\,\Omega\) resistor from the battery) as \(V_A\) and the node above the ammeter as \(V_B\). Thus:

\[
\begin{aligned}
&\text{Battery current} \quad I_{\text{bat}} = \frac{12 - V_A}{2},\\
&\text{Current through the vertical }2\,\Omega\text{ resistor} \quad I_{AB} = \frac{V_A - V_B}{2},\\
&\text{Current through the }1\,\Omega\text{ resistor from }V_A \text{ to }0\,\mathrm{V} \quad I_A = \frac{V_A}{1},\\
&\text{Current through the }1\,\Omega\text{ resistor from }V_B \text{ to }0\,\mathrm{V} \quad I_B = \frac{V_B}{1}.
\end{aligned}
\]

Because the ammeter is in series with the bottom \(1\,\Omega\) resistor, the ammeter reading is precisely \(I_B = V_B/1\). We apply Kirchhoff’s Current Law (KCL) at each node.

\[
\textbf{KCL at node }A:\quad
I_{\text{bat}} = I_{AB} + I_A
\quad\Longrightarrow\quad
\frac{12 - V_A}{2} \;=\; \frac{V_A - V_B}{2} \;+\; \frac{V_A}{1}.
\]
Multiply both sides by 2:
\[
12 - V_A \;=\; (V_A - V_B) + 2\,V_A \;=\; 3\,V_A - V_B
\quad\Longrightarrow\quad
12 \;=\; 4\,V_A \;-\; V_B.
\tag{1}
\]

\[
\textbf{KCL at node }B:\quad
I_{AB} = I_B
\quad\Longrightarrow\quad
\frac{V_A - V_B}{2} \;=\; \frac{V_B}{1}.
\]
Multiply both sides by 2:
\[
V_A - V_B \;=\; 2\,V_B
\quad\Longrightarrow\quad
V_A \;=\; 3\,V_B.
\tag{2}
\]

Substitute \(V_A = 3\,V_B\) from \((2)\) into \((1)\):
\[
12 \;=\; 4\,(3\,V_B) \;-\; V_B
\;=\; 12\,V_B \;-\; V_B
\;=\; 11\,V_B
\quad\Longrightarrow\quad
V_B \;=\; \frac{12}{11}\,\mathrm{V}.
\]
Hence
\[
V_A = 3\,V_B \;=\; 3\,\times\,\frac{12}{11} \;=\; \frac{36}{11}\,\mathrm{V}.
\]

\[
\textbf{(a) Ammeter current:}
\]
The current through the ammeter is the same as the current through the bottom \(1\,\Omega\) resistor (from \(V_B\) to \(0\,\mathrm{V}\)), namely
\[
I_{\text{ammeter}} \;=\; \frac{V_B}{1} 
\;=\; \frac{12}{11}\,\mathrm{A}
\;\approx\; 1.09\,\mathrm{A}.
\]

\[
\textbf{(b) Energy delivered by the 12\,V battery in 4\,s:}
\]
First find the total current drawn from the battery:
\[
I_{\text{bat}} \;=\; \frac{12 - V_A}{2}
\;=\; \frac{12 - \tfrac{36}{11}}{2}
\;=\; \frac{\tfrac{132}{11} - \tfrac{36}{11}}{2}
\;=\; \frac{96}{11} \,\cdot \frac{1}{2}
\;=\; \frac{48}{11}\,\mathrm{A}.
\]
The power supplied by the battery is
\[
P_{\text{battery}} \;=\; \bigl(12\,\mathrm{V}\bigr)\,\bigl(I_{\text{bat}}\bigr)
\;=\; 12 \,\times\, \frac{48}{11}
\;=\; \frac{576}{11}\,\mathrm{W}
\;\approx\; 52.36\,\mathrm{W}.
\]
Hence, in \(4\,\mathrm{s}\), the energy delivered by the battery is
\[
E \;=\; P_{\text{battery}} \,\times\, 4\,\mathrm{s}
\;=\; \frac{576}{11}\,\mathrm{W} \;\times\; 4\,\mathrm{s}
\;=\; \frac{2304}{11}\,\mathrm{J}
\;\approx\; 209.45\,\mathrm{J}.
\]

\[
\boxed{
\text{(a) }I_{\text{ammeter}} = \frac{12}{11}\,\mathrm{A}\approx1.09\,\mathrm{A}, 
\quad
\text{(b) }E = \frac{2304}{11}\,\mathrm{J}\approx209.45\,\mathrm{J}.
}
\]

\subsubsection{Solution 3}
```latex
\[
\textbf{Answer:}\\
\textbf{(a) Current through the ammeter}\\
\text{First, label the resistors as follows for clarity:}\\
\text{Let }R_1 = 2\,\Omega\text{ be the resistor from the 12\,V battery's positive terminal to the top node.}\\
\text{From that top node, two parallel branches lead to the battery's negative terminal:}\\
\quad \text{Branch A: }R_2 = 1\,\Omega\text{ directly to the negative terminal.}\\
\quad \text{Branch B: }R_3 = 2\,\Omega \rightarrow \text{Ammeter} \rightarrow R_4 = 1\,\Omega\text{ in series to the negative terminal.}\\[6pt]

\text{Let }V_N\text{ be the potential at the top node (relative to the negative terminal).}\\
\text{The total current from the battery is }I,\text{ which flows through }R_1\text{ and then splits.}\\
\text{Hence, across }R_1\text{ we have a voltage drop }(12 - V_N)\text{ and }I = \frac{12 - V_N}{R_1}=\frac{12 - V_N}{2}.\\[6pt]

\text{In the parallel section:}\\
I_A = \frac{V_N}{R_2} = \frac{V_N}{1} = V_N\quad (\text{current through }R_2),\\
I_B = \frac{V_N}{R_3 + R_4} = \frac{V_N}{2 + 1} = \frac{V_N}{3}\quad (\text{current through the ammeter}).\\
\text{By current conservation at the top node,}\\
I = I_A + I_B = V_N + \frac{V_N}{3} = \frac{4V_N}{3}.\\[6pt]

\text{We also have }I = \frac{12 - V_N}{2}.\text{ Equate the two expressions for }I\text{:}\\
\frac{12 - V_N}{2} = \frac{4V_N}{3}\quad\Longrightarrow\quad 12 - V_N = \frac{8V_N}{3}\quad\Longrightarrow\quad 12 = V_N + \frac{8V_N}{3} = \frac{11V_N}{3}.\\
\text{So }V_N = \frac{36}{11}\,\text{V}.\\
\text{The current through the ammeter is }I_B = \frac{V_N}{3} = \frac{36/11}{3} = \frac{36}{33} = \frac{12}{11}\,\text{A}.\\[8pt]

\textbf{(b) Energy delivered by the 12\,V battery in 4\,s}\\
\text{First find the total current }I\text{ from the battery:}\\
I = I_A + I_B = \frac{4V_N}{3} = \frac{4}{3}\times \frac{36}{11} = \frac{144}{33} = \frac{48}{11}\,\text{A}.\\
\text{The power supplied by the battery is }P = (\text{voltage}) \times (\text{current}) = 12 \times \frac{48}{11} = \frac{576}{11}\,\text{W}.\\
\text{Hence, the energy delivered in }4\,\text{s is }E = P \times t = \frac{576}{11}\times 4 = \frac{2304}{11}\,\text{J}\approx 209.45\,\text{J}.\\[6pt]

\textbf{Final answers:}\\
\quad \text{(a) }I_{\text{ammeter}} = \frac{12}{11}\,\text{A}.\\
\quad \text{(b) }E_{\text{battery\ in 4\,s}} = \frac{2304}{11}\,\text{J}\approx 209.45\,\text{J}.
\] 
```

\subsection{Question 6}
Q6. Two infinite parallel planes have equal and opposite uniform surface charge density $-\sigma$ and $+\sigma$, where $\sigma$ is positive. The planes are separated by a distance of 15 cm, and the potential difference between the plates is 120 V.

\begin{enumerate}
    \item[(a)] Determine the electric field, $E$, between the plates.
    \item[(b)] An object with charge $+0.001 \, \text{C}$ and a mass of $23 \, \text{g}$ is held at rest at the positive plate, then let go. Determine the acceleration of the object.
\end{enumerate}


\paragraph{Prompt: \\} 
Answer the given question as a top physics student, giving step-by-step reasoning. Format your answer as LaTeX code without using sub-sections. Available marks: a) [3 marks], b) [2 marks].

\subsubsection{Solution 1}
% Step-by-step solution for Question 6

% (a) Determine the electric field, E, between the plates.
% The potential difference between two points in a uniform electric field is given by:
% \[
% V = E \, d,
% \]
% where \( V \) is the potential difference and \( d \) is the separation between the plates.
% Solving for \( E \):
\[
E = \frac{V}{d} = \frac{120 \, \text{V}}{0.15 \, \text{m}} = 800 \, \text{V/m}.
\]

% (b) Determine the acceleration of an object with charge \( q = +0.001 \, \text{C} \) and mass \( m = 23 \, \text{g} = 0.023 \, \text{kg} \).
% The force on a charge in an electric field is given by:
% \[
% F = qE.
% \]
% Substituting the given values:
\[
F = (0.001 \, \text{C})(800 \, \text{V/m}) = 0.8 \, \text{N}.
\]
% Now, using Newton's second law, \( F = ma \), the acceleration \( a \) is:
\[
a = \frac{F}{m} = \frac{0.8 \, \text{N}}{0.023 \, \text{kg}} \approx 34.78 \, \text{m/s}^2.
\]

 Final answers:
 (a) \( E = 800 \, \text{V/m} \)
 (b) \( a \approx 34.78 \, \text{m/s}^2 \)


\subsubsection{Solution 2}
\[
\textbf{(a)} \quad \text{Since the potential difference } V \text{ between two plates separated by distance } d \text{ is given by } V = Ed, \text{ we have:}
\]
\[
E = \frac{V}{d} = \frac{120 \, \text{V}}{0.15 \, \text{m}} = 800 \, \text{V/m}.
\]
\[
\textbf{(b)} \quad \text{The force on an object with charge } q \text{ in an electric field } E \text{ is } F = qE.
\]
\[
F = (0.001 \, \text{C})(800 \, \text{V/m}) = 0.8 \, \text{N}.
\]
\[
\text{The acceleration } a \text{ is then given by Newton's second law, } a = \frac{F}{m}.
\]
\[
m = 23 \, \text{g} = 0.023 \, \text{kg}, \quad \text{so:} \quad a = \frac{0.8 \, \text{N}}{0.023 \, \text{kg}} \approx 34.78 \, \text{m/s}^2.
\]


\begin{enumerate}
    \item[(a)] Determine the electric field, $E$, between the plates.
    \item[(b)] An object with charge $+0.001 \, \text{C}$ and a mass of $23 \, \text{g}$ is held at rest at the positive plate, then let go. Determine the acceleration of the object.
\end{enumerate}

\subsubsection{Solution 3}
\[
\textbf{(a)}\quad E = \frac{V}{d} = \frac{120\,\text{V}}{0.15\,\text{m}} = 800\,\text{V/m}.
\]

\[
\textbf{(b)}\quad \text{The force on the object is } F = qE = (0.001\,\text{C})(800\,\text{V/m}) = 0.8\,\text{N}.
\]
\[
\text{Then, using } F = ma, \text{ the acceleration is }
\]
\[
a = \frac{F}{m} = \frac{0.8\,\text{N}}{0.023\,\text{kg}} \approx 34.8\,\text{m/s}^2.
\]

\subsection{Question 7}
An RLC circuit has a resistor with resistance $R = 600\,\Omega$, capacitor with capacitance $C = 1500\,\mathrm{pF}$, inductor with inductance $L = 20\,\mathrm{mH}$.  

\begin{enumerate}
    \item[(a)] What is the resonant frequency, $\omega_0$, of the circuit?  

    \item[(b)] The circuit is driven with an e.m.f source of the form  
    \[
    \mathcal{E} = \mathcal{E}_0 \cos(\omega t)
    \]  
    Find an expression for the voltage drop across the inductor, $\Delta V_L$, as a function of $\mathcal{E}_0$, the impedance $Z$, inductive reactance $X_L$, angular frequency $\omega$, and some phase shift $\phi$.  

    \item[(c)] Describe the phase relationship between the source e.m.f and voltage across the inductor in the limits of low frequency ($\omega \ll \omega_0$) and high frequency ($\omega \gg \omega_0$).
\end{enumerate}


\paragraph{Prompt: \\} 
Answer the given question as a top physics student, giving step-by-step reasoning. Format your answer as LaTeX code without using sub-sections. Available marks: a) [1 mark], b) [3 marks], c [2 marks].

\subsubsection{Solution 1}
\textbf{(a)} \quad The resonant (angular) frequency is given by 
\[
\omega_0 = \frac{1}{\sqrt{LC}}.
\]
Substituting \( L = 20\,\mathrm{mH} = 20\times10^{-3}\,\mathrm{H} \) and \( C = 1500\,\mathrm{pF} = 1500\times10^{-12}\,\mathrm{F} = 1.5\times10^{-9}\,\mathrm{F} \),
\[
\omega_0 = \frac{1}{\sqrt{(20\times10^{-3})(1.5\times10^{-9})}} 
= \frac{1}{\sqrt{30\times10^{-12}}} 
= \frac{1}{\sqrt{30}\times10^{-6}}.
\]
Since \(\sqrt{30} \approx 5.477\), we have 
\[
\omega_0 \approx \frac{1}{5.477\times10^{-6}} \approx 1.83\times10^5\,\mathrm{rad/s}.
\]

\textbf{(b)} \quad For a series RLC circuit driven by 
\[
\mathcal{E} = \mathcal{E}_0\cos(\omega t),
\]
the current is 
\[
I = \frac{\mathcal{E}_0}{Z}\cos(\omega t - \phi),
\]
where \(Z\) is the impedance and \(\phi\) is the phase shift between the source voltage and the current. The voltage drop across the inductor is given by 
\[
\Delta V_L = L\frac{dI}{dt}.
\]
Differentiating \(I\) with respect to time gives
\[
\frac{dI}{dt} = -\frac{\mathcal{E}_0}{Z}\omega \sin(\omega t - \phi).
\]
Thus,
\[
\Delta V_L = -L\frac{\mathcal{E}_0\omega}{Z}\sin(\omega t - \phi).
\]
Recognizing that \(\sin(\omega t - \phi) = \cos\left(\omega t - \phi - \frac{\pi}{2}\right)\), we can write
\[
\Delta V_L = \frac{\mathcal{E}_0\,X_L}{Z}\cos\left(\omega t - \phi - \frac{\pi}{2}\right),
\]
where the inductive reactance is \(X_L = \omega L\).

\textbf{(c)} \quad The phase relationship between the source e.m.f. and the voltage across the inductor depends on the frequency:

- \(\omega \ll \omega_0\): The circuit behaves as \emph{capacitive} (the capacitive reactance dominates), so the overall phase shift \(\phi\) is negative, approximately \(\phi \approx -\frac{\pi}{2}\). Inserting this into the expression,
  \[
  \Delta V_L \approx \frac{\mathcal{E}_0\,X_L}{Z}\cos\left(\omega t + \frac{\pi}{2} - \frac{\pi}{2}\right)
  = \frac{\mathcal{E}_0\,X_L}{Z}\cos(\omega t + \pi),
  \]
  which indicates that the inductor’s voltage is nearly \(180^\circ\) (or in anti-phase) relative to the source.

- \(\omega \gg \omega_0\): The circuit behaves as \emph{inductive} (the inductive reactance dominates), so the overall phase shift is positive, approximately \(\phi \approx \frac{\pi}{2}\). Then,
  \[
  \Delta V_L \approx \frac{\mathcal{E}_0\,X_L}{Z}\cos\left(\omega t - \frac{\pi}{2} - \frac{\pi}{2}\right)
  = \frac{\mathcal{E}_0\,X_L}{Z}\cos(\omega t),
  \]
  meaning that the inductor’s voltage is in phase with the source e.m.f.

Thus, at low frequencies the voltage drop across the inductor is nearly \(180^\circ\) out-of-phase with the source, and at high frequencies it is in phase with the source.

\subsubsection{Solution 2}
\textbf{(a) [1 mark]}\\
The resonant angular frequency is given by
\[
\omega_0=\frac{1}{\sqrt{LC}}.
\]
Inserting the given values, with 
\[
L=20\,\mathrm{mH}=20\times10^{-3}\,\mathrm{H} \quad\text{and}\quad C=1500\,\mathrm{pF}=1500\times10^{-12}\,\mathrm{F},
\]
we have
\[
LC=20\times10^{-3}\times1500\times10^{-12}=30\times10^{-12}=3.0\times10^{-11}\,\mathrm{H\cdot F}.
\]
Thus,
\[
\omega_0=\frac{1}{\sqrt{3.0\times10^{-11}}}\approx\frac{1}{5.48\times10^{-6}}\approx1.83\times10^5\,\mathrm{rad/s}.
\]

\bigskip
\textbf{(b) [3 marks]}\\
In an RLC circuit driven by 
\[
\mathcal{E}(t)=\mathcal{E}_0\cos(\omega t),
\]
the phasor form of the source is 
\[
\tilde{\mathcal{E}}=\mathcal{E}_0 e^{j0}.
\]
The current phasor is 
\[
\tilde{I}=\frac{\tilde{\mathcal{E}}}{Z}=\frac{\mathcal{E}_0}{Z}e^{-j\phi},
\]
where the impedance is 
\[
Z=\sqrt{R^2+(X_L-X_C)^2},
\]
with the inductive reactance \(X_L=\omega L\) and capacitive reactance \(X_C=1/(\omega C)\); \(\phi=\arctan\left(\frac{X_L-X_C}{R}\right)\) is the phase angle of the circuit.\\[1mm]
The voltage drop across the inductor in the phasor domain is given by
\[
\Delta \tilde{V}_L=jX_L\tilde{I}=\frac{jX_L\,\mathcal{E}_0}{Z}e^{-j\phi}.
\]
Converting back to the time domain (noting that multiplication by \(j\) corresponds to a phase shift of \(+\pi/2\)), we write
\[
\Delta V_L(t)=\frac{\mathcal{E}_0\,X_L}{Z}\cos\Bigl(\omega t-\phi+\frac{\pi}{2}\Bigr).
\]
This is the required expression in terms of \(\mathcal{E}_0\), \(Z\), \(X_L\), \(\omega\) and a phase shift \(\phi\).

\bigskip
\textbf{(c) [2 marks]}\\
The phase difference between the source e.m.f. and the inductor voltage is determined by the phase of
\[
\frac{jX_L}{R+j(X_L-X_C)},
\]
which has an overall phase
\[
\Delta\theta=\frac{\pi}{2}-\arctan\left(\frac{X_L-X_C}{R}\right).
\]
\begin{itemize}
    \item \(\omega\ll\omega_0\): At low frequencies, \(X_L=\omega L\) is very small and \(X_C=1/(\omega C)\) is very large so that \(X_L-X_C\approx -X_C\). Then,
    \[
    \arctan\left(\frac{X_L-X_C}{R}\right)\approx\arctan\left(-\frac{X_C}{R}\right)\approx -\frac{\pi}{2},
    \]
    giving
    \[
    \Delta\theta\approx\frac{\pi}{2}-\left(-\frac{\pi}{2}\right)=\pi.
    \]
    Thus, the inductor voltage is approximately 180° out of phase (inverted) relative to the source.
    
    \item \(\omega\gg\omega_0\): At high frequencies, \(X_L=\omega L\) is very large and \(X_C=1/(\omega C)\) is very small so that \(X_L-X_C\approx X_L\). Then,
    \[
    \arctan\left(\frac{X_L-X_C}{R}\right)\approx\arctan\left(\frac{X_L}{R}\right)\approx\frac{\pi}{2}\quad (\text{if }X_L\gg R),
    \]
    yielding
    \[
    \Delta\theta\approx\frac{\pi}{2}-\frac{\pi}{2}=0.
    \]
    Hence, the inductor voltage is in phase with the source.
\end{itemize}

\subsubsection{Solution 3}
\[
\textbf{(a)}\quad \omega_0=\frac{1}{\sqrt{LC}}=\frac{1}{\sqrt{(20\times10^{-3}\,\mathrm{H})(1500\times10^{-12}\,\mathrm{F})}}=\frac{1}{\sqrt{3\times10^{-11}}}\approx 1.83\times10^5\,\mathrm{rad/s}.
\]

\[
\textbf{(b)}\quad \text{If } \mathcal{E}(t)=\mathcal{E}_0\cos(\omega t) \text{, then the current amplitude is } I_0=\frac{\mathcal{E}_0}{Z}\text{. Since } X_L=\omega L, \text{ the inductor’s voltage is given by}
\]
\[
\Delta V_L(t)=L\frac{dI}{dt}=-\frac{\mathcal{E}_0\,\omega L}{Z}\sin(\omega t-\phi)=\frac{\mathcal{E}_0\,X_L}{Z}\cos\Bigl(\omega t-\phi+\frac{\pi}{2}\Bigr).
\]

\[
\textbf{(c)}\quad \text{For } \omega\ll\omega_0,\; X_L\ll X_C\text{ and the overall phase shift } \phi\approx -\tan^{-1}\Bigl(\frac{X_C}{R}\Bigr)\text{ is small and negative. The inductor’s voltage, being }90^\circ\text{ ahead of the current, is approximately }90^\circ\text{ ahead of the source e.m.f.}\\[6pt]
\quad \text{For } \omega\gg\omega_0,\; X_L\gg X_C\text{ so that } \phi\approx \tan^{-1}\Bigl(\frac{X_L}{R}\Bigr)\to 90^\circ. \text{ Now, the inductor’s }90^\circ\text{ lead on the current cancels the nearly }90^\circ\text{ lag of the current relative to the source, so } \Delta V_L(t) \text{ becomes nearly in phase with } \mathcal{E}(t).
\]


\subsection{Question 8}
 Two linear magnetic media occupy the half-spaces above and below the $xy$ plane. The space $z > 0$ is occupied by Material 1, with relative permeability $\mu_{r1} = 1.493$. The magnetic field $\mathbf{B}_1$ in this region is spatially uniform and static, with positive $x$ and $z$ components, and zero $y$ component. It is directed at an angle $\alpha_1$ with respect to the positive $z$ direction. The space $z < 0$ is occupied by Material 2, with relative permeability $\mu_{r2} = 3.012$. The magnetic field $\mathbf{B}_2$ in this region is spatially uniform and static, with positive $x$ and $z$ components, and zero $y$ component. It is directed at an angle of $\alpha_2 = 45^\circ$ with respect to the positive $z$ direction. There is no free current flowing anywhere in the system. Using the information above, calculate the value of the angle $\alpha_1$.

\paragraph{Prompt: \\} 
Answer the given question as a top physics student, giving step-by-step reasoning. Format your answer as LaTeX code without using sub-sections. Available marks: [6 marks]

\subsubsection{Solution 1}
\[
\textbf{Step 1: Boundary Conditions}
\]
Since there is no free current, the boundary conditions at the interface require that the normal component of \(\mathbf{B}\) and the tangential component of \(\mathbf{H}\) are continuous. Thus, we have:
\[
B_{1z} = B_{2z} \quad \text{and} \quad H_{1x} = H_{2x}
\]

\[
\textbf{Step 2: Expressing Field Components}
\]
For Material 1 (with permeability \(\mu_1 = \mu_{r1}\mu_0\)):
\[
B_{1z} = B_1 \cos \alpha_1, \quad B_{1x} = B_1 \sin \alpha_1
\]
For Material 2 (with permeability \(\mu_2 = \mu_{r2}\mu_0\)) and given \(\alpha_2 = 45^\circ\):
\[
B_{2z} = B_2 \cos 45^\circ, \quad B_{2x} = B_2 \sin 45^\circ
\]

\[
\textbf{Step 3: Applying the Normal Component Continuity}
\]
From \(B_{1z} = B_{2z}\):
\[
B_1 \cos \alpha_1 = B_2 \cos 45^\circ \quad \Longrightarrow \quad B_2 = \frac{B_1 \cos \alpha_1}{\cos 45^\circ}
\]

\[
\textbf{Step 4: Applying the Tangential Component Continuity}
\]
Since \(\mathbf{H} = \mathbf{B}/\mu\), the continuity of \(H_{x}\) gives:
\[
\frac{B_{1x}}{\mu_1} = \frac{B_{2x}}{\mu_2}
\]
Substitute the components:
\[
\frac{B_1 \sin \alpha_1}{\mu_1} = \frac{B_2 \sin 45^\circ}{\mu_2}
\]
Now substitute \(B_2\) from Step 3:
\[
\frac{B_1 \sin \alpha_1}{\mu_1} = \frac{\left(\frac{B_1 \cos \alpha_1}{\cos 45^\circ}\right) \sin 45^\circ}{\mu_2}
\]
Cancel \(B_1\) (assuming \(B_1 \neq 0\)):
\[
\frac{\sin \alpha_1}{\mu_1} = \frac{\cos \alpha_1}{\mu_2} \cdot \frac{\sin 45^\circ}{\cos 45^\circ}
\]
Since \(\frac{\sin 45^\circ}{\cos 45^\circ} = \tan 45^\circ = 1\), the equation simplifies to:
\[
\frac{\sin \alpha_1}{\mu_1} = \frac{\cos \alpha_1}{\mu_2}
\]

\[
\textbf{Step 5: Solving for \(\alpha_1\)}
\]
Rearrange the above expression:
\[
\tan \alpha_1 = \frac{\mu_1}{\mu_2}
\]
Substitute the given relative permeabilities:
\[
\tan \alpha_1 = \frac{1.493}{3.012} \approx 0.495
\]
Thus,
\[
\alpha_1 = \arctan(0.495) \approx 26.4^\circ
\]

\[
\textbf{Final Answer: } \alpha_1 \approx 26.4^\circ
\]

\subsubsection{Solution 2}
\[
\begin{aligned}
\text{Since there is no free current, the boundary conditions are:} \\
\quad & B_{1z} = B_{2z} \quad \text{(continuity of normal component)} \\
\quad & H_{1x} = H_{2x} \quad \text{(continuity of tangential component)} \\
\\
\text{Write the fields in components:} \\
& B_{1x} = B_1 \sin \alpha_1,\quad B_{1z} = B_1 \cos \alpha_1, \\
& B_{2x} = B_2 \sin 45^\circ,\quad B_{2z} = B_2 \cos 45^\circ. \\
\\
\text{Using } B_{1z} = B_{2z}: \\
& B_1 \cos \alpha_1 = B_2 \cos 45^\circ \quad \Rightarrow \quad B_1 = \frac{B_2 \cos 45^\circ}{\cos \alpha_1}. \\
\\
\text{The auxiliary fields are given by } \mathbf{H} = \frac{\mathbf{B}}{\mu_0 \mu_r}. \\
\text{Thus, the tangential components are:} \\
& H_{1x} = \frac{B_1 \sin \alpha_1}{\mu_0 \mu_{r1}},\quad H_{2x} = \frac{B_2 \sin 45^\circ}{\mu_0 \mu_{r2}}. \\
\\
\text{Setting } H_{1x} = H_{2x} \text{ yields:} \\
& \frac{B_1 \sin \alpha_1}{\mu_{r1}} = \frac{B_2 \sin 45^\circ}{\mu_{r2}}. \\
\\
\text{Substitute } B_1 = \frac{B_2 \cos 45^\circ}{\cos \alpha_1}: \\
& \frac{B_2 \cos 45^\circ \sin \alpha_1}{\mu_{r1} \cos \alpha_1} = \frac{B_2 \sin 45^\circ}{\mu_{r2}}. \\
\\
\text{Cancel } B_2 \text{ (assuming } B_2 \neq 0 \text{) and note that } \cos 45^\circ = \sin 45^\circ: \\
& \frac{\sin \alpha_1}{\cos \alpha_1 \mu_{r1}} = \frac{1}{\mu_{r2}}, \quad \text{or} \quad \frac{\tan \alpha_1}{\mu_{r1}} = \frac{1}{\mu_{r2}}. \\
\\
\text{Thus,} \\
& \tan \alpha_1 = \frac{\mu_{r1}}{\mu_{r2}} = \frac{1.493}{3.012} \approx 0.495. \\
\\
\text{Taking the arctan:} \\
& \alpha_1 \approx \arctan(0.495) \approx 26.4^\circ.
\end{aligned}
\]

\[
\boxed{\alpha_1 \approx 26.4^\circ.}
\]

\subsubsection{Solution 3}
```latex
We begin by applying the boundary conditions at the \(xy\) plane. Since there is no free current, the tangential component of the magnetic field intensity \(\mathbf{H}\) is continuous. Recall that
\[
\mathbf{H} = \frac{\mathbf{B}}{\mu},
\]
so the tangential component (along \(x\)) must satisfy
\[
\frac{B_{1x}}{\mu_1} = \frac{B_{2x}}{\mu_2}.
\]
In Material 1 (\(z>0\)), the field is given by
\[
B_{1x} = B_1 \sin\alpha_1, \quad B_{1z} = B_1 \cos\alpha_1,
\]
and in Material 2 (\(z<0\)), with \(\alpha_2 = 45^\circ\),
\[
B_{2x} = B_2 \sin 45^\circ, \quad B_{2z} = B_2 \cos 45^\circ.
\]

The normal component of \(\mathbf{B}\) is also continuous:
\[
B_{1z} = B_{2z} \quad \Longrightarrow \quad B_1 \cos\alpha_1 = B_2 \cos 45^\circ.
\]
However, for our purpose, the key condition is the tangential one. Substituting the \(x\)-components,
\[
\frac{B_1 \sin\alpha_1}{\mu_1} = \frac{B_2 \sin 45^\circ}{\mu_2}.
\]
Dividing this equation by the normal component condition (or alternatively eliminating \(B_1\) and \(B_2\)) gives:
\[
\frac{\sin\alpha_1}{\mu_1 \cos\alpha_1} = \frac{\sin 45^\circ}{\mu_2 \cos 45^\circ}.
\]
Since \(\sin 45^\circ = \cos 45^\circ = \frac{1}{\sqrt{2}}\), we have:
\[
\frac{\tan\alpha_1}{\mu_1} = \frac{1}{\mu_2} \quad \Longrightarrow \quad \tan\alpha_1 = \frac{\mu_1}{\mu_2}.
\]
Using the fact that \(\mu_1 = \mu_0 \mu_{r1}\) and \(\mu_2 = \mu_0 \mu_{r2}\), the ratio simplifies to:
\[
\tan\alpha_1 = \frac{\mu_{r1}}{\mu_{r2}} = \frac{1.493}{3.012}.
\]
Numerically,
\[
\tan\alpha_1 \approx 0.495.
\]
Thus, the angle \(\alpha_1\) is:
\[
\alpha_1 = \arctan(0.495) \approx 26.3^\circ.
\]
This is the required angle of the magnetic field in Material 1.
```

\subsection{Question 9}
A transverse electromagnetic wave propagating in vacuum has an electric field which has complex representation:
\[
E(r, t) = E_0 \exp(i (k \cdot r - \omega t))
\]
With real valued vectors \(E_0, k\).

\begin{enumerate}
    \item[(a)] Apply Faraday’s Law and show magnetic intensity of the wave has magnitude given by
    \[
    H(r, t) = \sqrt{\frac{\epsilon_0}{\mu_0}} E(r, t)
    \]

    \item[(b)] Show the Poynting flux averaged over one period of oscillation is given by
    \[
    \langle N \rangle = \frac{1}{2} \sqrt{\frac{\mu_0}{\epsilon_0}} (H_0)^2 \hat{k}
    \]
    where \(H_0\) denotes the amplitude of the magnetic intensity.

    \item[(c)] The pressure exerted by radiation has intensity \(I\) on a perfect planar reflector and has the value
    \[
    \frac{2I}{c},
    \]
    where \(c\) is the speed of light. Show that when the angle of incidence of radiation is \(\alpha\), the radiation pressure becomes
    \[
    2I \cos^2 \alpha / c.
    \]
\end{enumerate}


\paragraph{Prompt: \\} 
Answer the given question as a top physics student, giving step-by-step reasoning. Format your answer as LaTeX code without using sub-sections. Available marks: a) [7 marks], b) [6 marks], c [2 marks].

\subsubsection{Solution 1}
% (a) [7 marks]
We start from Faraday’s law in differential form:
\[
\nabla \times E(r,t) = -\frac{\partial B(r,t)}{\partial t},
\]
where in vacuum \(B = \mu_0 H\).

Given the complex representation of the electric field
\[
E(r,t) = E_0 \exp[i(k\cdot r - \omega t)],
\]
its curl is
\[
\nabla \times E = i(k \times E_0) \exp[i(k\cdot r - \omega t)].
\]
On the other hand, differentiating the magnetic flux density \(B\) with respect to time, we have
\[
-\frac{\partial B}{\partial t} = -\frac{\partial (\mu_0 H)}{\partial t} = i\omega \mu_0 H \quad (\text{since } \partial_t \exp[i(k\cdot r-\omega t)] = -i\omega \exp[i(k\cdot r-\omega t)]).
\]
Equating both expressions gives:
\[
i(k \times E_0) \exp[i(k\cdot r - \omega t)] = i\omega \mu_0 H \exp[i(k\cdot r - \omega t)].
\]
Canceling the common factors (including \(i\) and the exponential), we obtain:
\[
k \times E_0 = \omega \mu_0 H.
\]
Taking magnitudes and noting that for a transverse electromagnetic wave the fields are perpendicular to the propagation direction (i.e., \(E_0 \perp k\)), we have
\[
|k \times E_0| = |k|\,|E_0| = \omega \mu_0 |H|.
\]
In vacuum the dispersion relation is
\[
\omega = c|k|, \quad \text{with } c=\frac{1}{\sqrt{\epsilon_0\mu_0}},
\]
so that
\[
|H| = \frac{|k|}{\omega \mu_0} |E_0| = \frac{|E_0|}{\mu_0 c|k|/|k|} = \frac{|E_0|}{\mu_0 c}.
\]
Since
\[
\frac{1}{\mu_0 c} = \sqrt{\frac{\epsilon_0}{\mu_0}},
\]
we conclude that
\[
H(r,t) = \sqrt{\frac{\epsilon_0}{\mu_0}}\,E(r,t).
\]

% (b) [6 marks]
The instantaneous Poynting vector is defined as
\[
N(r,t) = E(r,t) \times H(r,t).
\]
In a plane wave the fields are in phase and perpendicular to each other, and the time-average over one period \(T\) is given by
\[
\langle N \rangle = \frac{1}{2} \Re\{E_0 \times H_0^*\} \quad \text{(with } E_0,\,H_0 \text{ representing the complex amplitudes)}.
\]
From part (a) we have the relation between the amplitudes:
\[
H_0 = \sqrt{\frac{\epsilon_0}{\mu_0}}\,E_0.
\]
Thus, the cross product becomes
\[
E_0 \times H_0^* = E_0 \times \left(\sqrt{\frac{\epsilon_0}{\mu_0}}\,E_0\right)^* = \sqrt{\frac{\epsilon_0}{\mu_0}}\,E_0 \times E_0^*.
\]
Since \(E_0\) is real (as given in the problem) we have \(E_0^* = E_0\), and therefore,
\[
E_0 \times E_0 = 0 \quad \text{unless we explicitly take into account the direction of propagation } \hat{k}.
\]
More directly, we can note that for a plane wave propagating in the \(\hat{k}\) direction, the time-averaged magnitude of the Poynting vector is given by
\[
\langle N \rangle = \frac{1}{2}\,|E_0||H_0|\,\hat{k}.
\]
Substituting \(H_0 = \sqrt{\frac{\epsilon_0}{\mu_0}}\,E_0\) yields
\[
\langle N \rangle = \frac{1}{2}\, \sqrt{\frac{\epsilon_0}{\mu_0}}\, E_0^2\,\hat{k}.
\]
Expressing the result in terms of \(H_0\) instead, since \(E_0 = \sqrt{\frac{\mu_0}{\epsilon_0}}\,H_0\), we find:
\[
\langle N \rangle = \frac{1}{2}\, \sqrt{\frac{\epsilon_0}{\mu_0}} \left(\sqrt{\frac{\mu_0}{\epsilon_0}}\,H_0\right)^2\,\hat{k} = \frac{1}{2}\, \sqrt{\frac{\mu_0}{\epsilon_0}}\, H_0^2\,\hat{k}.
\]

% (c) [2 marks]
For a perfectly reflecting surface, the change in momentum flux doubles compared to absorption. When the radiation is incident at an angle \(\alpha\) relative to the normal, only the normal component of the incident intensity \(I\) contributes to the pressure. The effective intensity normal to the surface is \(I \cos\alpha\). The pressure due to reflection is then given by:
\[
p = \frac{2(I\cos\alpha)}{c}\cos\alpha = \frac{2I\cos^2\alpha}{c},
\]
which is the desired result.


\subsubsection{Solution 2}
% (a) [7 marks]
We begin with Faraday’s law in vacuum:
\[
\nabla \times \mathbf{E} = -\mu_0 \frac{\partial \mathbf{H}}{\partial t}.
\]
For a plane wave with complex representation
\[
\mathbf{E}(\mathbf{r},t) = \mathbf{E}_0\,e^{i(\mathbf{k}\cdot\mathbf{r}-\omega t)},
\]
the spatial derivative is
\[
\nabla \times \mathbf{E} = i\,\mathbf{k} \times \mathbf{E}.
\]
Likewise, the time derivative of the magnetic field, assumed to have the form
\[
\mathbf{H}(\mathbf{r},t) = \mathbf{H}_0\,e^{i(\mathbf{k}\cdot\mathbf{r}-\omega t)},
\]
is
\[
\frac{\partial \mathbf{H}}{\partial t} = -i\omega\,\mathbf{H}.
\]
Substituting these into Faraday’s law gives
\[
i\,\mathbf{k} \times \mathbf{E} = -\mu_0(-i\omega\,\mathbf{H}) \quad \Longrightarrow \quad \mathbf{k} \times \mathbf{E} = \omega\mu_0\,\mathbf{H}.
\]
Taking magnitudes and noting that for a transverse wave \(\mathbf{E}\) is perpendicular to \(\mathbf{k}\) (so \(|\mathbf{k}\times \mathbf{E}| = |\mathbf{k}|\,|\mathbf{E}|\)), we obtain
\[
|\mathbf{k}|\,|\mathbf{E}| = \omega\mu_0\,|\mathbf{H}|.
\]
In vacuum the dispersion relation is
\[
|\mathbf{k}| = \frac{\omega}{c} = \omega\sqrt{\mu_0\epsilon_0},
\]
so that
\[
\omega\sqrt{\mu_0\epsilon_0}\,|\mathbf{E}| = \omega\mu_0\,|\mathbf{H}|.
\]
Canceling \(\omega\) (assuming \(\omega\neq0\)) yields
\[
|\mathbf{H}| = \sqrt{\frac{\epsilon_0}{\mu_0}}\,|\mathbf{E}|.
\]
Since the spatial and temporal dependence is identical for both fields, we may write
\[
\mathbf{H}(\mathbf{r},t) = \sqrt{\frac{\epsilon_0}{\mu_0}}\,\mathbf{E}(\mathbf{r},t).
\]

% (b) [6 marks]
The instantaneous Poynting vector is given by
\[
\mathbf{N} = \mathbf{E} \times \mathbf{H}.
\]
For harmonic fields, the time-averaged Poynting flux over one period is
\[
\langle \mathbf{N} \rangle = \frac{1}{2}\,\Re\left[\mathbf{E} \times \mathbf{H}^*\right],
\]
where the asterisk denotes complex conjugation. For a plane wave, the fields are in phase and perpendicular to each other, and using the result from part (a) we have the relation in magnitudes:
\[
|E_0| = \sqrt{\frac{\mu_0}{\epsilon_0}}\,|H_0|.
\]
Thus, the magnitude of the time-averaged Poynting vector becomes
\[
\langle N \rangle = \frac{1}{2}\,E_0\,H_0 = \frac{1}{2}\,\sqrt{\frac{\mu_0}{\epsilon_0}}\,(H_0)^2.
\]
Since the energy propagates in the direction of the wave vector \(\hat{k}\), the full expression is
\[
\langle \mathbf{N} \rangle = \frac{1}{2}\,\sqrt{\frac{\mu_0}{\epsilon_0}}\,(H_0)^2\,\hat{k}.
\]

% (c) [2 marks]
For a perfect reflector at normal incidence, the radiation pressure is given by
\[
P = \frac{2I}{c},
\]
since the momentum change doubles upon reflection. For an incidence at an angle \(\alpha\) from the normal, only the normal component of the momentum flux contributes. The normal component of the intensity is \(I\cos\alpha\), and hence the pressure is
\[
P = \frac{2(I\cos\alpha)}{c}\cos\alpha = \frac{2I\cos^2\alpha}{c}.
\]
Thus, the radiation pressure at an angle \(\alpha\) is
\[
\frac{2I\cos^2\alpha}{c}.
\]

\subsubsection{Solution 3}
\[
\textbf{(a) }
\]
We start from Faraday’s law in differential form,
\[
\nabla \times \mathbf{E} = -\mu_0 \frac{\partial \mathbf{H}}{\partial t}.
\]
For a plane wave with 
\[
\mathbf{E}(\mathbf{r},t)=\mathbf{E}_0\, e^{i(\mathbf{k}\cdot\mathbf{r}-\omega t)},
\]
we compute the curl:
\[
\nabla \times \mathbf{E} = i\,\mathbf{k}\times \mathbf{E}_0\, e^{i(\mathbf{k}\cdot\mathbf{r}-\omega t)}.
\]
Similarly, the time derivative gives
\[
\frac{\partial \mathbf{H}}{\partial t} = -i\omega\, \mathbf{H}_0\, e^{i(\mathbf{k}\cdot\mathbf{r}-\omega t)},
\]
assuming a similar harmonic dependence for \(\mathbf{H}\) with amplitude \(\mathbf{H}_0\). Inserting these into Faraday’s law we have:
\[
i\,\mathbf{k}\times \mathbf{E}_0\, e^{i(\mathbf{k}\cdot\mathbf{r}-\omega t)} = -\mu_0 \Bigl(-i\omega\, \mathbf{H}_0\, e^{i(\mathbf{k}\cdot\mathbf{r}-\omega t)}\Bigr).
\]
Cancelling common factors and \(i\), we obtain:
\[
\mathbf{k}\times \mathbf{E}_0 = \mu_0 \omega\, \mathbf{H}_0.
\]
For a transverse electromagnetic wave \(\mathbf{E}_0 \perp \mathbf{k}\), taking magnitudes yields:
\[
kE_0 = \mu_0 \omega H_0.
\]
But in vacuum, the dispersion relation is \( \omega = ck\) with \( c=\frac{1}{\sqrt{\mu_0\epsilon_0}}\). Therefore,
\[
kE_0 = \mu_0 c k H_0 \quad \Longrightarrow \quad H_0 = \frac{E_0}{\mu_0 c}.
\]
Substituting \( c=\frac{1}{\sqrt{\mu_0\epsilon_0}} \) we get:
\[
H_0 = \frac{E_0}{\mu_0\,(1/\sqrt{\mu_0\epsilon_0})} = \sqrt{\frac{\epsilon_0}{\mu_0}}\, E_0.
\]
Thus, the complex magnetic intensity is:
\[
\mathbf{H}(\mathbf{r},t)= \sqrt{\frac{\epsilon_0}{\mu_0}}\, \mathbf{E}(\mathbf{r},t).
\]

\[
\textbf{(b) }
\]
The instantaneous Poynting vector is given by
\[
\mathbf{S} = \mathbf{E} \times \mathbf{H}.
\]
For time-harmonic fields we take the time average over one period, so that
\[
\langle \mathbf{N} \rangle = \frac{1}{2} \Re\left\{\mathbf{E} \times \mathbf{H}^*\right\}.
\]
From part (a) we have the relation between the amplitudes,
\[
\mathbf{H}_0 = \sqrt{\frac{\epsilon_0}{\mu_0}}\, \mathbf{E}_0,
\]
and since the wave propagates in the \(\hat{\mathbf{k}}\) direction we have
\[
\mathbf{E}_0 \times \mathbf{H}_0^* = E_0 H_0\, \hat{\mathbf{k}}.
\]
Thus, the time-averaged Poynting flux becomes
\[
\langle \mathbf{N} \rangle = \frac{1}{2}\,E_0H_0\,\hat{\mathbf{k}}.
\]
Expressing \(E_0\) in terms of \(H_0\) using the relation from part (a),
\[
E_0 = \sqrt{\frac{\mu_0}{\epsilon_0}}\,H_0,
\]
we obtain:
\[
\langle \mathbf{N} \rangle = \frac{1}{2} \sqrt{\frac{\mu_0}{\epsilon_0}}\,H_0^2\,\hat{\mathbf{k}}.
\]

\[
\textbf{(c) }
\]
For radiation incident normally on a perfect reflector, the radiation pressure is given by
\[
P = \frac{2I}{c},
\]
where \(I\) is the intensity and \(c\) is the speed of light. For an angle of incidence \(\alpha\), the normal component of the momentum flux is reduced by a factor \(\cos\alpha\). Moreover, upon reflection, the momentum change is doubled, but only the normal component contributes. Therefore, the effective radiation pressure becomes:
\[
P = \frac{2I\cos\alpha}{c}\cos\alpha = \frac{2I\cos^2\alpha}{c}.
\]
Thus, when the radiation is incident at an angle \(\alpha\), the pressure is given by
\[
\frac{2I\cos^2\alpha}{c}.
\]


\subsection{Question 10}
Q10. Consider 2 inertial reference frames \( S \) and \( S' \). The frames are aligned in such a way that origins of both frames coincide at time zero within both frames (\( t = t' = 0 \)). The frame \( S' \) moves with velocity \( v \) in the \( x \)-direction as seen by \( S \). The transformation of electric and magnetic fields from frame \( S \) to \( S' \) is given by 

\[
E'_x = E_x, \quad E'_y = \gamma (E_y - v B_z), \quad E'_z = \gamma (E_z + v B_y)
\]
\[
B'_x = B_x, \quad B'_y = \gamma \left( B_y + \frac{v}{c^2} E_z \right), \quad B'_z = \gamma \left( B_z - \frac{v}{c^2} E_y \right)
\]

\begin{enumerate}
    \item[(a)] Using the transformations given, show that the scalar product, \( (\mathbf{E} \cdot \mathbf{B}) \), is invariant under transformation.
    
    \item[(b)] A plane electromagnetic wave observed in the reference frame \( S \) propagates in a vacuum along the \( x \)-direction. In frame \( S \), it is represented by the Cartesian representation:
    \[
    \mathbf{E} = E_0 \hat{y} \exp \left( i (k x - \omega t) \right)
    \]
    with \( E_0 \), a real value denoting the amplitude, \( k \) is the wavevector, and \( \omega \) the angular frequency. \( (kx - \omega t) \) defines the phase of the wave in frame \( S \). Show that this phase in reference frame \( S' \) is written \( (k' x' - \omega' t') \), where primed coordinates correspond to the frame \( S' \) and 
    \[
    \omega' = \gamma (\omega - k v), \quad k' = \gamma \left( k - \frac{\omega v}{c^2} \right)
    \]

    \item[(c)] Examine the relationship between the wave frequencies \( \omega \) and \( \omega' \) in the limit \( v / c \to 0 \). What physical phenomenon does this represent? Justify your answer.
\end{enumerate}


\paragraph{Prompt: \\} 
Answer the given question as a top physics student, giving step-by-step reasoning. Format your answer as LaTeX code without using sub-sections. Available marks: a) [4 marks], b) [2 marks], c [1 mark].

\subsubsection{Solution 1}
\textbf{(a)} We begin by writing the scalar product in the unprimed frame:
\[
\mathbf{E} \cdot \mathbf{B} = E_x B_x + E_y B_y + E_z B_z.
\]
Under a Lorentz transformation in the \(x\)-direction the fields transform as
\[
E'_x = E_x,\quad E'_y = \gamma (E_y - vB_z),\quad E'_z = \gamma (E_z + vB_y),
\]
\[
B'_x = B_x,\quad B'_y = \gamma \left( B_y + \frac{v}{c^2} E_z \right),\quad B'_z = \gamma \left( B_z - \frac{v}{c^2} E_y \right).
\]
Thus, the scalar product in the primed frame is
\[
\mathbf{E}' \cdot \mathbf{B}' = E'_x B'_x + E'_y B'_y + E'_z B'_z.
\]
Since \(E'_x B'_x = E_x B_x\), we compute the remaining terms:
\[
E'_y B'_y = \gamma (E_y - vB_z)\, \gamma \left(B_y + \frac{v}{c^2}E_z\right) = \gamma^2\left( E_yB_y + \frac{v}{c^2}E_yE_z - vB_zB_y - \frac{v^2}{c^2}B_zE_z\right),
\]
\[
E'_z B'_z = \gamma (E_z + vB_y)\, \gamma \left(B_z - \frac{v}{c^2}E_y\right) = \gamma^2\left( E_zB_z - \frac{v}{c^2}E_zE_y + vB_yB_z - \frac{v^2}{c^2}B_yE_y\right).
\]
Adding these results yields:
\[
E'_y B'_y + E'_z B'_z = \gamma^2 \left[ E_yB_y + E_zB_z - \frac{v^2}{c^2}(B_zE_z+B_yE_y) \right].
\]
But recall that
\[
\gamma^2\left(1-\frac{v^2}{c^2}\right)=1.
\]
Thus,
\[
E'_y B'_y + E'_z B'_z = E_yB_y + E_zB_z.
\]
Therefore,
\[
\mathbf{E}' \cdot \mathbf{B}' = E_xB_x + E_yB_y + E_zB_z = \mathbf{E} \cdot \mathbf{B},
\]
showing the invariance of the scalar product under Lorentz transformations. \hfill [4 marks]

\bigskip

\textbf{(b)} For a plane electromagnetic wave in \(S\) with phase
\[
\phi = kx - \omega t,
\]
the invariance of phase under Lorentz transformation implies that in \(S'\) the phase must be
\[
\phi' = k'x' - \omega' t' = \phi.
\]
Using the Lorentz transformations for the coordinates:
\[
x' = \gamma (x - vt),\quad t' = \gamma \left(t - \frac{v}{c^2}x\right),
\]
we have
\[
k'x' - \omega' t' = k'\gamma (x-vt) - \omega'\gamma \left(t-\frac{v}{c^2}x\right).
\]
Expanding this expression gives:
\[
\gamma \left[ k'x - k'vt - \omega't + \frac{v\omega'}{c^2}x \right] = \gamma \left[\left(k' + \frac{v\omega'}{c^2}\right)x - (\omega' + k'v)t \right].
\]
For this to equal \(kx - \omega t\) for all \(x\) and \(t\), the coefficients must match:
\[
\gamma \left(k' + \frac{v\omega'}{c^2}\right) = k,\quad \gamma (\omega' + k'v) = \omega.
\]
Solving these equations for \(\omega'\) and \(k'\) yields:
\[
\omega' = \gamma (\omega - k v),\quad k' = \gamma \left(k - \frac{\omega v}{c^2}\right).
\]
Thus, the phase in \(S'\) is indeed written as \(k' x' - \omega' t'\) with the given transformations. \hfill [2 marks]

\bigskip

\textbf{(c)} In the limit \( \frac{v}{c} \to 0\), the Lorentz factor \(\gamma \to 1\). Therefore, the transformed frequency becomes
\[
\omega' \approx \omega - k v.
\]
For a wave propagating in the \(x\)-direction in vacuum, we have \(k = \frac{\omega}{c}\), so
\[
\omega' \approx \omega \left(1 - \frac{v}{c}\right).
\]
This is the classical Doppler shift for a source (or observer) moving with a speed \(v\) relative to the medium, representing a decrease in frequency (redshift) when the observer moves away from the source. \hfill [1 mark]


\subsubsection{Solution 2}
\textbf{(a) [4 marks]} \\
We begin by writing the transformed fields:
\[
E'_x = E_x, \quad E'_y = \gamma (E_y - v B_z), \quad E'_z = \gamma (E_z + v B_y),
\]
\[
B'_x = B_x, \quad B'_y = \gamma \left( B_y + \frac{v}{c^2} E_z \right), \quad B'_z = \gamma \left( B_z - \frac{v}{c^2} E_y \right).
\]
The scalar product in the primed frame is
\[
\mathbf{E}'\cdot\mathbf{B}' = E'_x B'_x + E'_y B'_y + E'_z B'_z.
\]
Substitute the transformed components:
\[
\begin{split}
\mathbf{E}'\cdot\mathbf{B}' &= E_x B_x \\
&\quad + \gamma (E_y - v B_z) \, \gamma \left(B_y + \frac{v}{c^2} E_z\right) \\
&\quad + \gamma (E_z + v B_y) \, \gamma \left(B_z - \frac{v}{c^2} E_y\right) \\
&= E_x B_x + \gamma^2 \Bigl[ E_yB_y + \frac{v}{c^2}E_yE_z - vB_zB_y - \frac{v^2}{c^2}B_zE_z \\
&\quad + E_zB_z - \frac{v}{c^2}E_zE_y + vB_yB_z - \frac{v^2}{c^2}B_yE_y \Bigr].
\end{split}
\]
Notice that the cross terms cancel since
\[
\frac{v}{c^2}E_yE_z - \frac{v}{c^2}E_zE_y = 0,\quad -vB_zB_y + vB_yB_z = 0.
\]
Thus,
\[
\mathbf{E}'\cdot\mathbf{B}' = E_xB_x + \gamma^2\left[E_yB_y + E_zB_z - \frac{v^2}{c^2}(B_zE_z+B_yE_y)\right].
\]
Factor the common terms:
\[
\mathbf{E}'\cdot\mathbf{B}' = E_xB_x + \gamma^2\left(1 - \frac{v^2}{c^2}\right)(E_yB_y+E_zB_z).
\]
Since by definition \(\gamma^2\left(1-\frac{v^2}{c^2}\right)=1\), we obtain
\[
\mathbf{E}'\cdot\mathbf{B}' = E_xB_x+E_yB_y+E_zB_z = \mathbf{E}\cdot\mathbf{B}.
\]
Thus, the scalar product \(\mathbf{E}\cdot\mathbf{B}\) is invariant under Lorentz transformation.

\bigskip

\textbf{(b) [2 marks]} \\
For a plane wave in \(S\) with electric field
\[
\mathbf{E} = E_0\,\hat{y}\,\exp\left[i(kx-\omega t)\right],
\]
the phase is given by \(\phi = kx - \omega t\). Because phase is a Lorentz invariant quantity, we have in the primed frame:
\[
\phi' = k'x' - \omega't' = \phi.
\]
The Lorentz transformations for the coordinates are:
\[
x = \gamma (x' + vt'), \quad t = \gamma \left(t' + \frac{v}{c^2} x'\right).
\]
Substitute these into the phase:
\[
\begin{split}
\phi &= k \gamma (x' + vt') - \omega \gamma \left(t' + \frac{v}{c^2} x'\right) \\
&= \gamma \left[ (k - \frac{\omega v}{c^2})x' + (kv-\omega)t'\right].
\end{split}
\]
Comparing with \(\phi' = k'x' - \omega't'\), we identify:
\[
k' = \gamma \left(k - \frac{\omega v}{c^2}\right), \quad \omega' = \gamma (\omega - kv).
\]

\bigskip

\textbf{(c) [1 mark]} \\
In the limit \(v/c\to 0\), we have \(\gamma\approx 1\). Therefore,
\[
\omega' \approx \omega - kv, \quad k' \approx k - \frac{\omega v}{c^2}.
\]
For an electromagnetic wave in vacuum, \(\omega = ck\), so
\[
\omega' \approx \omega\left(1 - \frac{v}{c}\right).
\]
This linear shift in the frequency corresponds to the classical Doppler effect (a redshift if \(v>0\), i.e. when the observer is moving in the direction of wave propagation). Thus, the result represents the Doppler shift in the non-relativistic limit.


\subsubsection{Solution 3}
\textbf{(a) [4 marks]}\\
We start by computing the dot product in the primed frame:
\[
\mathbf{E}'\cdot\mathbf{B}' = E'_xB'_x + E'_yB'_y + E'_zB'_z.
\]
Since
\[
E'_x = E_x,\quad B'_x = B_x,
\]
and
\[
E'_y = \gamma (E_y - vB_z),\quad B'_y = \gamma \left( B_y + \frac{v}{c^2}E_z \right),
\]
\[
E'_z = \gamma (E_z + vB_y),\quad B'_z = \gamma \left( B_z - \frac{v}{c^2}E_y \right),
\]
we have
\[
\begin{aligned}
\mathbf{E}'\cdot\mathbf{B}' &= E_xB_x \\
&\quad + \gamma^2\left[(E_y-vB_z)\left(B_y+\frac{v}{c^2}E_z\right) + (E_z+vB_y)\left(B_z-\frac{v}{c^2}E_y\right)\right].
\end{aligned}
\]
Expanding the products gives:
\[
\begin{aligned}
(E_y - vB_z)\left(B_y+\frac{v}{c^2}E_z\right) &= E_yB_y + \frac{v}{c^2}E_yE_z - vB_zB_y - \frac{v^2}{c^2}B_zE_z, \\
(E_z + vB_y)\left(B_z-\frac{v}{c^2}E_y\right) &= E_zB_z - \frac{v}{c^2}E_zE_y + vB_yB_z - \frac{v^2}{c^2}E_yB_y.
\end{aligned}
\]
Notice that the cross terms cancel:
\[
\frac{v}{c^2}E_yE_z - \frac{v}{c^2}E_zE_y = 0 \quad \text{and} \quad -vB_zB_y + vB_yB_z = 0.
\]
Thus,
\[
\mathbf{E}'\cdot\mathbf{B}' = E_xB_x + \gamma^2\left[E_yB_y + E_zB_z - \frac{v^2}{c^2}(E_yB_y + E_zB_z)\right].
\]
Factor out \((E_yB_y+E_zB_z)\):
\[
\mathbf{E}'\cdot\mathbf{B}' = E_xB_x + \gamma^2\left(1-\frac{v^2}{c^2}\right)(E_yB_y+E_zB_z).
\]
Since \(\gamma^2(1-\frac{v^2}{c^2})=1\), it follows that
\[
\mathbf{E}'\cdot\mathbf{B}' = E_xB_x+E_yB_y+E_zB_z = \mathbf{E}\cdot\mathbf{B}.
\]
Thus, the scalar product \(\mathbf{E}\cdot\mathbf{B}\) is invariant under the transformation.\\

\textbf{(b) [2 marks]}\\
The phase of the wave in frame \(S\) is given by
\[
\phi = kx-\omega t.
\]
Under a Lorentz transformation, the coordinates transform as:
\[
x'=\gamma (x-vt), \quad t'=\gamma \left(t-\frac{vx}{c^2}\right).
\]
Substituting the inverse relations \(x = \gamma (x'+vt')\) and \(t = \gamma \left(t'+\frac{vx'}{c^2}\right)\) into the phase, we obtain:
\[
\begin{aligned}
\phi &= k\gamma (x'+vt') - \omega\gamma \left(t'+\frac{vx'}{c^2}\right) \\
&=\gamma\left[\left(k-\frac{\omega v}{c^2}\right)x' + (kv-\omega)t'\right].
\end{aligned}
\]
Identifying the primed wavevector and angular frequency as:
\[
k'=\gamma\left(k-\frac{\omega v}{c^2}\right), \quad \omega'=\gamma (\omega-kv),
\]
we see that the phase in \(S'\) becomes
\[
\phi = k'x' - \omega' t',
\]
which confirms the required result.\\

\textbf{(c) [1 mark]}\\
In the limit \(v/c\to0\), we have \(\gamma\approx 1\), so the transformed frequency becomes
\[
\omega'\approx\omega - kv.
\]
Since for electromagnetic waves in vacuum \(k=\omega/c\), this reduces to
\[
\omega'\approx\omega\left(1-\frac{v}{c}\right).
\]
This represents the \emph{Doppler shift} for waves in the non-relativistic limit, where the observed frequency decreases (redshift) if the observer moves away from the source (or increases, blueshift, if moving towards it).



\end{document}


